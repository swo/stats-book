\documentclass{book}

\usepackage{amsmath}
\usepackage{amsfonts}
\usepackage{bm}
\usepackage{hyperref}
\usepackage{longtable}
\usepackage{booktabs}

\providecommand{\tightlist}{%
  \setlength{\itemsep}{0pt}\setlength{\parskip}{0pt}}

\title{Statistic for people who do numbers and science}
\author{Scott W. Olesen}

\begin{document}
\maketitle

\frontmatter

%!TEX root=main.tex

\chapter{Preface}

\section*{Who this book is for}

I finished my PhD at MIT in Biological Engineering in 2016. While I was a
graduate student, and afterward when I was a postdoc, I found that graduate
students, especially in the life sciences, have a fair amount of quantitative
training, often including statistics, and are capable of hacking together
fairly sensible quantitative methods that answer their scientific questions. I
found, though, that there was often a steep drop-off when it came time to apply
statistical rigor to these hacky methods. I say this, in part, because it is my
own story too.

This book is for people who are experienced quantitative and scientific
thinkers, have decent algebra and maybesome rusty stats, and who want to learn
about how to use statistical thinking to improve their scientific thinking.

I also tried to have fun when I wrote this book, so there are lots of things
in it that I think are fun, like history\footnote{I think history is actually
a useful part of the study of statistics, because it gets you away from
thinking about statistics as some monolithic thing and gets you toward
thinking about it as a work-in-progress whose shape has been determined by the
abilities and prejudices of its makers. Ronald Fisher's personal tastes will
come up a lot; I have a pet theory that a lot of the flavor of contemporary
statistics is due to his tastes.} and footnotes.

\section*{What this book is for}

As I tried to fill gaps in my statistics knowedge, I found that books about
statistics tended to come in two varities: utterly simplistic and
overwhelmingly complex. It was a similar feeling as trying to learn a new
language as an adult: I have the mind of an adult, and am interested in
grown-up things, but I can only read my target language at the level of an
elementary schooler. Books about grown-up things have overwhelming vocabulary,
while books are childish things are way below my level.

I intend this book to be conceptually advanced but technically manageable. Very
often there is sophisticated mathematical machinery behind something that can
be summarized, from a scientific view, very quickly. On the other hand, there
are some times that a little algebra goes a long way to illustrating a
statistical concept.

For example, "maximum likelihood" is a key concept in statistics. Rather than
show you how you can use linear algebra to compute maximum likelihood values
for linear regression, I will just tell you that there is such software as an
optimizer that finds maxima. On the other hand, I won't shy away from
explaining to you why you divide by $n$ when you compute the mean but you
divide by $n-1$ when you compute the standard deviation, which is pretty easy
to derive algebraically.

\section*{What this book is not}

Maybe it's just as easy to say that this \emph{is} a book for people who want
to \emph{use} rigorous statistics to do science, and maybe even to develop some
new statistical methods, but it is \emph{not} a book for people who want to
push the boundaries of what statisticians think are interesting mathematical
problems.

This is not a cookbook that tells you what steps to do to get a $p$-value. This
is more like Harold McGee's \textit{On Food and Cooking}, that tries to help
give you the principles by which you can understand what the tests are doing.

I also do not intend to fill this book with examples, because I find that it's
very easy to think that your particular scientific question is very
itneresting, and that the statistics that motivate are interesting, but that
it's hard to get motivated by hearing about someone \emph{else's} statistical
woes. I try to keep it general so you can imagine you are the hero of the story.

\section*{What you should be able to do after reading this book}

I basically want you to be able to articulate you own statistical tests, modify
and critique existing methodologies, and develop a healthy skepticism of the
idea of statistical inference in general.

\section*{What else should I read}

My principal sources of motivation for this book were Allen Downey's
\textit{Think Stats} and \textit{Think Bayes}, Miran Lipva\v{c}a's
\textit{Learn You a Haskell for Great Good}, and Stephen Stigler's
\textit{History of Statistics}. That order is probably the most profitable.

\section*{Notes on notation}

\paragraph{Functions} When introducing a function $f$ that maps things in a domain
$D$ to things in a range $R$, I write that as
$$
f : D \to R.
$$
When writing a function $f$ that takes numbers, I use normal parentheses:
$f(x)$. When writing a function that takes things that aren't numbers, I use
square brackets: $\prob{A}$. My idea is to emphasize that which functions
are number-taking and which are something-else-taking.

When a function takes multiple inputs, I use a semicolon to distinguish
between the input(s) that, in that context, we think of as changing and the
input(s) that we think of as fixed. I put the ``variables'' before the
semicolon and the ``parameters'' after. So I would write the probability of
getting $x$ successes out of $n$ trials each with probability $p$ of success
as $f(x; n, p)$. Other people might write this as $f(x | n, p)$, especially
for likelihood, but I like to reserve the bar specifically for conditional
probabilities $\prob{A | B}$.

\paragraph{Probability} When probability is a function, I use the
``blackboard'' letter $\mathbb{P}$. When probability is a number---that is, an
output of the function $\mathbb{P}$---I write $p$. I try to avoid capital
$P$, because I think it's not clear whether that's a function or a number.

\paragraph{Expected value} When expected value is a function, I use the
``blackboard'' letter $\mathbb{E}$. I use an ``overline'' to write the arithmetic
mean of a group of numbers: $\overline{x} = (1/n)\sum_i x_i$. I try to avoid
using the overline with random variables and use, say, $\mu$ as the true mean
and, as per the note below, $\hat{\mu}$ as an estimator of the mean (which,
incidentally, $\bar{x}$ is).

\paragraph{Variance} When variance is a function, I use the ``blackboard''
letter $\mathbb{V}$. As per the note below, I write the estimated variance
as $\hat{\mathbb{V}}$ but the sample variance as $\sigma^2$. I do this in
part to clarify why we use $n$ in the denominator of $\sigma^2$ and $n-1$
in the denominator of $\hat{\mathbb{V}}$.

\paragraph{Estimators} I use ``hats'' for estimators, so $\hat{X}$ is an
estimator for $X$. When it's important to distinguish between estimators,
I use subscripts, so maybe $\hat{X}_\mathrm{ML}$ is the maximum likelihood
estimator for $X$.


\mainmatter

\chapter{What is statistics?}

"Statistics" is confusing because it means two things. It's a plural noun that
refers to multiple things, each of which is called a "statistic". It's also a
singular noun that refers to the study of these mathematical objects.

Statistics is a tool.\footnote{I really enjoyed reading this anecdote from a
paper by Gene Glass, the inventor of meta-analysis (doi:10.1002/jrsm.1133).
Glass. He ends up sitting on the plane next to the famous statistician
Tukey. Tukey asks, "Along which dimension do you see the greatest
variability in effects?" By investigator. "Then jackknife on investigator."
The point is that there's a wide gap between being able to understand what
the jackknife is in a mathematical sense (i.e., how to compute it, whether
it's an unbiased way to determine the variance in the median) and how to apply
it. It also goes to show that "real" statisticians can be pretty hacky!}

A \emph{statistic} is some function of the sample data. Although the field of
statistics is often described as consisting of "descriptive statistics" and
"inferential statistics"---that is, the study of mathematical objects used to
describe data and to make inferences about the populations the data were taken
from---in this book I want to emphasize that, for both purposes, we are
interested in the properties of statistics.

The most interesting statistics are the ones that are used to estimate some
property of the population. These kinds of statistics are sensibly called
\emph{estimators}. Most of the study of statistics comes down to figuring out things
about estimators. One of the most critical is understanding the variance of
estimators, which is essential for both descriptive statistics---so that you
can put error bars on your measurements---and inferential statistics---so you
can make a guess about how probable it is that your data arose under some null
hypothesis. All of this is about estimators and their variance.

\chapter{Probability: statistics's toolkit}

To understand the stuff in this book, you'll need three major toolkits:
algebra, probability, and optimization. These aren't what statistical thinking
is about, but they are really handy ways to talk about what we'll learn.

In a mathematical introduction, a random variable is a function that maps
events to some measurable space. This mathematical approach is powerful but not
intuitive. I think the contructivist approach is better: if you can say how
this thing would arise, I think you are in a better place as a scientist and
modeler (but not necessarily as a mathematician).

This part will have some math jargon, but you can handle it.

In probability, we're interested in a \emph{sample space} of
\emph{outcomes}. If I'm drawing a card from a shuffled deck, the
outcomes are each of the 52 cards I could draw. Outcomes get combined
into \emph{events}, like ``drew a red card''. Every outcome has an
associated \emph{probability}, usually written $P$, always in the
range $[0, 1]$, and each event has a probability that is the sum of
the probabilities of its constituent outcomes. Of the 52 cards in the
deck, 26 are red, and the probability of drawing any particular card is
$\tfrac{1}{52}$, so the probability of drawing a red card is
$\tfrac{26}{52} = \tfrac{1}{2}$.

There is a sophisticated mathematical theory about outcomes, events, and
probabilities called \emph{measure theory}. The simple definitions I
gave here are not sufficient for a complete theory of
probability\footnote{For example, what's the probability of flipping
  heads on a fair an infinite number of times? To say ``zero'' makes it
  sound like it's impossible, but to say anything other than zero is
  even more confusing.}, but they will work well enough for our
purposes.

The really interesting quantity here is ``probability''. In math world,
probability is just an abstract concept. In the real world, it's
something that we think about every day. For the first part of the book,
we'll use the \emph{frequentist} definition of probability: if an event
has probability $p$, it means that, if we repeated whatever situation
led to the possibility of that event very many times, the proportion of
situations leading to that event would approach $p$. For example, if I
flip a fair coin very many times, I expect the proportion of heads to
approach $\tfrac{1}{2}$, so the ``probability'' of heads is
$\tfrac{1}{2}$.

This frequentist definition will lead us into confusing problems later
on, which will lead to the other major definition of probability,
\emph{Bayesian} probability\footnote{The fundamental notion in Bayesian
probability is that ``probability'' should mean something about
confidence. Intuitively, we say that ``probability'' of heads is
$\tfrac{1}{2}$ not because we're thinking of an infinite number of
flips, but because we think that heads is 50\% ``likely''. (Confusing,
``likelihood'' has a specific technical definition that we will get to
later.) The problem in frequentist probability is that some things
cannot be expressed as frequencies of many repeated trials. See the
Bayesian chapter for more.}, which is a subject of a later chapter.

\subsection{Manipulating probabilities}

A core concept that presented a lot of theoretical confusion in the
development of statistics is \emph{conditional probability}: what is the
probability of \(A\) ``given that'' \(B\) is true?

This feels like a natural concept: given that I drew a Jack from the
deck, what's the probability I drew the Jack of Hearts? Clearly
\(\tfrac{1}{4}\). What's the chance I drew the Queen of Spades? Clearly
zero.\footnote{Those who took note of the frequentist definition will
  say that I'm not being accurate with my language, and they'd be right.
  I should say, ``Among trials in which a Jack is drawn, in what
  proportion was the Jack of Hearts drawn?'' This idea of ``the chance
  that'' is clearly more in the Bayesian vein.}

Mathematically, we write \(A | B\) to mean ``\(A\) given that \(B\)''.
The definition of conditional probability is \[
\mathbb{P}[A | B] \equiv \frac{\mathbb{P}[A \cap B]}{\mathbb{P}[B]}.
\] I think the best way to read this is: rather than considering the
entire universe of possible events, go into the smaller universe of
events \(B\), and calculate probabilities there. Then the denominator is
just the ``size'' of the universe, and the numerator is the ``size'' of
the event of interest within the scope of the universe I'm thinking
about.

\textbf{independence}

\subsection{Random variables}\label{random-variables}

We can associated this space of outcomes and events with \emph{random
variables}, which are functions of the event space that return numbers.
For the coin flip, the only interesting random variable is: if the flip
is heads, the value is one; if tails, then zero. For a dice roll, the
easy-to-think-of random variable is the number of dots that came up: if
the dice rolls one, the value is 1; if two, then 2; etc. You could think
up other variables: maybe you get 1 on even dice rolls and 0 on odds.

To a scientist, this sounds really abstract, which it is, but it lets us
write something that feels more natural: for a random variable \(X\),
what's the probability that \(X\) is less than some particular value
\(x\)? I'll write this \(\mathbb{P}[X < x]\). This is called the
\emph{cumulative distribution function}, of cdf, of \(X\). Traditionally
it's written as \(F_X\): \[
F_X(x) \equiv \mathbb{P}[X < x].
\]

If \(X\) is a finite, discrete function (that can take on only
particular values), then \(F_X\) is zero for the smallest \(x\)
supported by \(X\) (although it's less than 1 for the highest \(x\)).
For a function that can take on arbitrarily large or small values, we
can see that \[
\lim_{x \to -\infty} F_X(x) = 0 \text{ and } \lim_{x \to \infty} F_X(x) = 1.
\]

In the finite, discrete case, like a dice roll, it's straightforward to
compute the cumulative distribution function from the \emph{probability
mass function}, which is just the probability that \(X\) takes on any
particular value. For other kinds of random variables, the corresponding
concept is the \emph{probability distribution function}, or pdf,
traditionally written \(f_X\), which is actually derived from the
cumulative distribution function: \[
f_X(x) \equiv \frac{d}{dx} F_X(x), \text{ or equivalently } F_X(x) = \int_{-\infty}^x f_X(x') dx'.
\]

\subsection{Detritus}\label{detritus}

I'll need an introduction to some basic ideas: events, unions,
conditional probability, expected values, variance, pdf, cdf. I think
that's most of what you need to get the basic gist. (As an aside, you
might need some really complicated stuff to answer the finer point
questions, like what's the space of rational ``coin flips''?)

\chapter{copied material}

\section{Things to include}\label{things-to-include}

\begin{itemize}
\item
  LMM as model for when biology becomes stats: you're doing GWAS.
\item
  Multiple hypothesis correction. Start by looking at the distribution
  of \(p\)-values.
\item
  Likelihood of data vs. \emph{probability} of data.
\item
  Start with Bayesian as a conceptual thing, then go to frequentist for
  most of the stats you will encounter, and then go back to Bayesian for
  the more hardcore stuff.

  \begin{itemize}
  \item
    Introduce priors with the binomial and beta prior. I think that's
    really intuitive.
  \item
    Confidence intervals as random variables with the parameter fixed,
    versus credible interval as a fixed thing that arises from the
    posterior of the random variable of the parameter.
  \end{itemize}
\item
  Visualizing pdf as equiprobable bins. Uniform distribution as dice
  rolls on a d10, for why the probability of rolling exactly
  0.5000\ldots{} is (nearly) zero.
\item
  Math on credible intervals: they align with confidence intervals in
  very particular cases. Even the normal distribution with unknown
  variance breaks that.
\item
  Normal approximation to the binomial, but then also to the Poisson, or
  whatever you want.
\item
  ``Regression to the mean'' means a slope less than one
\item
  MCMC for complex SIR-type models: you know the observed outcomes
  (e.g., occupancy of each compartment) comes from the input data, but
  it's not clear how the outcomes relate to the inputs
\item
  ABC for something similar, where we're interested in the parameters
  that generate the data, but there's no obvious mathematical
  relationship between the parameters and the observed data.
\item
  Poisson as horse kicks. Or as drops with cells (a la Weitz, etc.)
\item
  Simulation in general
\item
  Bayesian as the mystical: the only way to say the probability about
  the truth is to assert what you think is the truth
\item
  Frequentist as a weird subset of the Bayesian
\item
  Variance (shape) as more important than mean (location). If you have
  variance, it's pretty easy to measure the mean and get a sense of
  things. If you have the other way, it's a lot more murky.
\item
  Inference vs.~decision-making under uncertainty. How are these two
  things different? (e.g., what's the ``cost'' of making an incorrect
  scientific conclusion?)
\item
  Simpson's paradox: why two-way associations are confusing
\item
  Random number generation
\item
  Most textbooks are boring, and they just present a parade of stuff. I
  want to use those things that are probably familiar as entry points
  for learning. E.g., the difference between descriptive statistics and
  inferential statistics is what those statistics are used for. In one,
  they are used a estimators of population parameters. In the other,
  they are summed over the ask about the likelihood of data.
\item
  ABC as super hacky thing!
\item
  Finite population correction:
  \(\sigma_{\overline{X}} = \frac{\sigma}{\sqrt{n}} \sqrt{\frac{N - n}{n - 1}}\)
\item
  Wilk's theorem for likelihood ratio tests!
\item
  Cramer-Rao bound: inverse of Fisher information matrix is the lower
  bound for the variance of unbiased estimators. An estimator that hits
  this bound is minimum variance unbiased (MVU). (BLUE as a weird name,
  since B and MV mean the same thing?) Not \emph{a priori} clear that
  some estimator is an MVU.
\item
  Start with Bayesian, because that's how scientists are used to
  thinking, and then go onto frequentist, because that's how it's easier
  to do the math, and then return to Bayes.
\item
  Gauss: a minimum-variance, mean-unbiased estimator minimizes the
  squared-error loss function. Laplace: among median-unbiased
  estimators, a minimum-average-absolute-deviation estimator minimizes
  the absolute loss function. Maybe it's better to allow some bias so
  you can get less variance. That's the domain of statistical theory.
\item
  Linear mixed models using relationship-matrix?
\item
  Fisher's crazy sum test is the same thing as is used in \emph{TRANSIT}
  (DeJesus \emph{et al}.): they treat TA sites in the same gene as
  independent; the statistic is the difference in the sum of the
  (normalized) number of insertions in two treatments; the null
  distribution is generated by shuffling the values across the two
  datasets. OK, it's not \emph{exactly} like Fisher's test, since it's
  not paired, but it's pretty close. Fisher probably wouldn't have
  wanted to to the \(\binom{n}{2}\) options, compared to the \(2^n\)
  that he did.
\end{itemize}

\section{GEE notes}\label{gee-notes}

The estimator for the covariance matrix of the estimator for the
parameters \(\bm{\theta}\) has typical element \[
\widehat{\mathrm{Cov}}[\hat{\bm{\theta}}]_{i,j} =
  \left[ \left( -\frac{\partial^2 \mathcal{L}}{\partial \theta_u \, \partial \theta_v} \right) \right]^{-1}_{i,j}.
\] The inverse is a matrix inverse. This is weird notation.

For example, consider the easy case with just one parameter, so that \[
\widehat{\mathrm{Var}}[\hat{\theta}] =
  -\left( \frac{\partial^2 \mathcal{L}}{\partial \theta^2} \right)^{-1}.
\]

Then try the easy example where we're looking at the sample mean, which
is the estimator \(\hat{\mu}\) for some population with unknown mean
\(\mu\) and known variance \(\sigma^2\). Then the log likelihood is \[
\mathcal{L}(\theta) = \sum_i \left\{ -\frac{1}{2} \log (2 \pi \sigma^2) -
  \frac{(x_i - \mu)^2}{2 \sigma^2} \right\}
\] and the second derivative is just \[
\frac{\partial^2 \mathcal{L}}{\partial \mu^2} = -\frac{n}{\sigma^2},
\] from which it's clear that
\(\widehat{\mathrm{Var}}[\hat{\mu}] = \sigma^2/n\). This is just the
result that we got from our more straightforward approach, where we
asked about the variance of the estimator, before we had called it that.

Note, however, that this estimate depends on \(\sigma^2\), the
\emph{true} population variance. In an exercise, you'll compute the
estimator covariance matrix for the estimator of
\(\bm{\theta} = (\mu, \sigma^2)\). That will show that there is some
estimated covariance between \(\hat{\mu}\) and \(\hat{\sigma}^2\), but
you find that each of \(\widehat{\mathrm{Var}}[\hat{\mu}]\),
\(\widehat{\mathrm{Var}}[\hat{\sigma}^2]\), and
\(\widehat{\mathrm{Cov}}[\hat{\mu}, \hat{\sigma}^2]\) depend on the true
values, not on the estimates themselves. This isn't unreasonable: each
of these thing is a well-defined random variable that will have a real,
honest, true distribution, and we're deriving its properties. Clearly
the true properties of these random variables should depend on the true
values, but it leaves us in a pickle when we're doing estimation: when
computing the standard errors on our estimators, we need to use our
points estimates as the true values!

\textbf{Relationship between ``sampling distribution'' and estimator.
``Standard error'' as SD of sampling distribution OR estimate of
standard deviation.}

\section{Summary statistics}\label{summary-statistics}

The mean is also called the \emph{expectation value}, which gives a
contemporary English speaker a strange idea of what it is. The
expectation value is \emph{not} ``a value that you might expect''. That
is the \emph{mode}. In certain cases, the mean and the mode are the
same, but this is often not true. This is why it is so critical to look
at the distribution of your data \emph{before} selecting summary
statistics.

\subsection{The (arithmetic) mean}\label{the-arithmetic-mean}

Everyone knows the average: take your numbers, sum them, and divide by
the number of numbers.

Why is this a good idea? Early thinkers in statistics---who thought
about it before it had a name---were curious about the same thing. The
arithmetic mean was computationally simple (i.e., you could do it will
quill and parchment, OK, pen and paper) and it seemed to ``work'', but
why?

It turns out that the arithmetic mean has some nice properties for, you
guessed it, the normal distribution. In fact, we'll show that it is a
certain kind of ``best way'' to guess the true, population mean from our
sample mean. (It's the maximum likelihood estimator.)

\subsection{The ``average''}

In typical speech, we say something like, ``The average family has two
children'', by which we mean that a typical family has two children.
This is confusing because we also use the word ``average'' to refer to
the arithmetic mean. These two things are similar only in special cases,
including (you guessed it!) the normal distribution.

The arithmetic mean of number of children in families is somewhere
between 2 and 3 (let's say 2.5), but it's clear that no ``average
family'' has 2.5 children. Similarly, it's confusing that the mean
salary in the US is whatever, even thought that is above what whatever
percent of people get paid.

\subsection{Mode, median, and
percentiles}\label{mode-median-and-percentiles}

Typical can mean ``common'', so we might say, in a sense, that an
``average'' family has two children. Of the number of children you can
have (zero, one, two, etc.), two is the most common, so that's
``average''. Technically, the most common value is the \emph{mode}. No
one really talks about the mode except in stats textbooks.

A more interesting number is the middle point. We might say that an
average American makes whatever because half of people make more and
half of people make less. This is the \emph{median} (from Latin
\emph{medius}, meaning ``middle'').

It even feels natural to combine the median and mode ideas (have our
median cake and eat it a la mode?). We want a sense of what are
``middling'' numbers, and we want a sense of what are ``common''
numbers. So you might ask what range is covered by the most middle half
of numbers.

Children are often measured against a growth chart, which shows
\emph{percentiles}: if you are in the 25th percentile, you are taller
than 50\% of people. (The median is just another name for the 50th
percentile.) The range between the 25th and 75th percentiles, which
covers the middle half of people, is the \emph{interquartile range},
because the 25th, 50th, and 75th percentiles are also called the
\emph{quartiles}, because they divide the numbers into four quarters
(bottom quarter, bottom-middle, top-middle, and top).

\subsection{Mean, median, and mode are usually
different}\label{mean-median-and-mode-are-usually-different}

Roll a dice many times. What are the mean, median, and mode? Mean is
easy: each number is equally likely to come up, so we just take the mean
of 1, 2, 3, 4, 5, 6, which is 3.5. The median is a little tricky here:
it's between 3 and 4, and when we hit this situation we normally define
the median as the arithmetic mean of the two middle values. So the
median is also 3.5 here, but only because of some convention. The mode
is also confusing, since all numbers are equally likely, which means
that they are all equally the mode.

For the normal distribution, the mean, median, and mode are all the
same. So it's only in this very potentially unusual case that our
intuition is correct that that the ``average'' value, in the sense of
something common (i.e., the mode), is the same as the ``average'' value,
in the sense of something middling (i.e., the median), is the same as
the arithmetic mean.

Historical box for Quetelet: the idea of an ``average person'' is
directly traceable to a particular proto-statistician, who is remarkable
for having believed that almost \emph{everything} was normally
distributed. At the time, they only had very rudimentary methods for
determining if something was normally distributed (basically, eyeballing
it), and there were very appealing aesthetic/philosophical reasons to
believe that almost everything was normally distributed, so he believed
that too.

\section{$t$-test}

\subsection{Equal variance}\label{equal-variance}

The (old school) \emph{t}-test is two sample, assuming equal variances.
We're interested in the difference in the means between the two
populations.

The null hypothesis is that we're drawing \(n_1 + n_2\) samples from a
population that has this equal variance, and that the labels on the two
``populations'' are just fictitious.

Our estimator \(s_p^2\) for the pooled variance is just the average of
the variances of the two ``populations'', weighted by \(n_i - 1\) (which
is a better estimator than weighting by just \(n_i\)): \[
s_p^2 = \frac{(n_1 - 1) s_1^2 + (n_2 - 1) s_2^2}{n_1 + n_2 - 2}.
\]

The thing we're observing is the difference between the mean of \(n_1\)
samples from a (potentially ficitious) variable \(X_1\) and \(n_2\) from
\(X_2\): \[
\overline{X}_1 - \overline{X}_2 = \frac{1}{n_1} \sum_{i=1}^{n_1} X_{1i} - \frac{1}{n_2} \sum_{i=1}^{n_2} X_{2i}.
\] It would be nice if our statistic was distributed like
\(\mathcal{N}(0, 1)\), so we compute the variance of this observation:
\[
\begin{aligned}
\mathrm{Var}\left[ \overline{X}_1 - \overline{X}_2 \right]
  &= \frac{1}{n_1^2} \sum_i \mathrm{Var}[X_1] + \frac{1}{n_2^2} \sum_i \mathrm{Var}[X_2] \\
  &= \frac{1}{n_1^2} n_1 s_p^2 + \frac{1}{n_2^2} n_2 s_p^2 \\
  &= \left( \frac{1}{n_1} + \frac{1}{n_2} \right) s_p^2.
\end{aligned}
\]

So the statistic for this test is just the observation over its
variance: \[
t = \frac{\overline{X}_1 - \overline{X}_2}{s_p \sqrt{\frac{1}{n_1} + \frac{1}{n_2}}}.
\]

The confusing thing is that \(\overline{X}_1\), \(\overline{X}_2\), and
\(s_p\) are all random variables. We know how to take the sum (or
difference) of two random variables (i.e., how to figure out the
distribution of the numerator), but it's not immediately obvious how to
find the distribution of the whole thing, which has a different random
variable in its denominator.

\subsubsection{Computational approach}\label{computational-approach}

\begin{itemize}
\tightlist
\item
  Compute the observed \(t\) statistic
\item
  Compute the observed sizes, means, and standard deviations for the two
  sample populations
\item
  Many times, generate two sets of random variates. One set of variates
  is drawn from a normal distribution with the first sample mean and
  variance.
\item
  For each iteration, compute the simulated \(t\) statistic.
\item
  The empirical \(p\)-value is the fraction (not true! need \(r+1/n+1\))
  of simulated statistics that are greater than the observed statistic.
\end{itemize}

\subsection{Unequal variance (Welch's)}\label{unequal-variance-welchs}

This is the Behrens-Fisher problem. It stumped Fisher! He came up with a
weird statistic with a weird distribution (Behrens-Fisher), but it
didn't really stick, since he couldn't calculate confidence intervals
(?).

Instead, people went for the Welch-Satterthwaite equation, which
approximates the interesting distribution using a more handy one by
matching the first and second moments. (Maybe worth discussing those? Or
just say mean and variance?)

\section{anova}\label{anova}

Say you have some (equally-sized) groups. Each group was drawn from a
normal distribution (all with the same variance). Are the data
consistent with the model in which all those groups have the same mean?

The statistic is \(F \equiv \frac{\mathrm{MS}_B}{\mathrm{MS}_W}\), where
\(MS_B\) is the mean of the squares of the residuals(?) between the
group means and the grand mean (``between'') and \(MS_W\) is the mean of
the squares of the residuals between the data points and the group means
(``within'').

Again, focus on what's the population we're sampling from. It's easy to
think about a finite population, where you can just do all the possible
combinations and compare their \(F\) statistics. Then move on to say
that, if you believe that the particular variances and means you
measured are the exact, true distribution that you're sampling from, ask
what happens when you sample from that infinite population.

\subsection{z-test example}\label{z-test-example}

What's the nonparametric equivalent of this? It's just saying what the
empirical cdf is! Then say, if you really truly believe your mean and
standard deviation, then you can do that.

In other words, you say you are absolutely sure what population you are
drawing from. The same is actually true of the \emph{t}-test, except
that the \emph{z}-test is asking about a single value, distributed like
\(N(\mu, \sigma^2)\), while the \emph{t}-test is about the mean of the
\(n\) points, which is distributed like \(N(\mu, \sigma^2/n)\).

\section{Paired differences}\label{paired-differences}

\subsection{Historical example and
motivation}\label{historical-example-and-motivation}

Darwin's thing with the pots, as described in Fisher's \emph{Design of
Experiments}

We'll make a tower of the kinds of assumptions made to test Darwin's
hypothesis.

\subsection{Sign test}\label{sign-test}

Assume that it's meaningful if a hybrid plant is taller than a
self-fertilized plant, but don't assign any meaning beyond that. Then
the data are effectively dichotomized: you get some number of cases in
which one is taller and some number of cases in which the other is
taller.

Better to say that we're sampling from any distribution that has zero
median. You could even say you're sampling from \emph{all}
distributions. That's confusing, mathematically, because there are
infinitely many distributions, and it's not obvious how you should
sample from that functional space, but it works out, because all those
distributions will have the same distribution of pluses and minuses.

This is now just a binomial test.

\subsection{Rank test (Mann-Whitney $U$)}

Assume that the \emph{ranks} of the differences are meaningful.

Now you're sampling from any distribution that is symmetric about zero.
That means it has zero median also.

\subsection{Fisher's weird sum test}\label{fishers-weird-sum-test}

Not sure if there's any name for this. Assume that the actual values of
the differences are meaningful.

\subsection{Welch's $t$-test}

Assume that the two populations are normally distributed and, and
therefore that that the variances of the populations are meaningful.
Then you can infer where this set of differences would stand in an
infinite set of such differences.

For early statisticians, this was really appealing, mostly from a
computational point of view: you could actually compute the mean and
standard deviation with pen and paper in a reasonable amount of time,
but you definitely couldn't do all \(2^n\) different ways of taking
sums. Fisher does it for one example in his book, and I'm sure it was
pretty crazy. He makes it clear that he went to great lengths to do it,
and his conclusion is that the results are basically the same, so you
should probably be doing the easier thing and not worry about it.
Nowadays it's gotten pretty easy to do the other thing!, so it's

\section{Wilcox test and Mann-Whitney
test}\label{wilcox-test-and-mann-whitney-test}

\textbf{Walsh averages and confidence intervals}, from
\href{http://www.stat.umn.edu/geyer/old03/5102/notes/rank.pdf}{here}

There a few different names for these things:

\begin{itemize}
\tightlist
\item
  One-sample test: is this distribution symmetric about zero (or
  whatever)?
\item
  Two-sample unpaired (independent; Mann-Whitney): does one of these
  distributions ``stochastically dominate'' the other (i.e., is it that
  a random value drawn from population \(A\) is more than 50\% probable
  to be greater than a random value from \(B\))?
\item
  Two-sample paired (dependent): are the differences between paired data
  points symmetric about zero?
\end{itemize}

\subsection{Wilcoxon}\label{wilcoxon}

\begin{enumerate}
\def\labelenumi{\arabic{enumi}.}
\tightlist
\item
  For each pair \(i\), compute the magnitude and sign \(s_i\) of the
  difference. Exclude tied pairs.
\item
  Order the pairs by the magnitude of their difference: \(i=1\) is the
  pair with the smallest magnitude. Now \(i\) is the rank.
\item
  Compute the \(W = \sum_i i s_i\).
\end{enumerate}

Thus, the bigger differences get more weight.

(There might be a way to do a visualization of this: as you walk along
the data points, you get a good bump for every rank that is in the
``high'' data set, and you get a bad hit for every rank that is not.
Then it settles out pretty quickly, and you want to know the meaning of
the final intercept.)

For small \(W\), the distribution has to be computed for each integer
\(W\). For larger values (\(\geq 50\)), a normal approximation works.

Compare the sign test, which does not use ranks, and which assumes the
median is zero, but not that the distributions are symmetric. That's
just a binomial test of the number of pluses or minuses you get. It's
like setting the weights, which in \(W\) are the ranks, all equal.

\subsection{Mann-Whitney}\label{mann-whitney}

\begin{enumerate}
\def\labelenumi{\arabic{enumi}.}
\tightlist
\item
  Assign ranks to every observation.
\item
  Compute \(R_1\), the sum ranks that belong to points for sample 1.
  Note that \(R_1 + R_2 = \sum_{i=1}^N = N(N+1)/2\).
\item
  Compute \(U_1 = R_1 - n_1(n_1+1)/2\) and \(U_2\). Use the smaller of
  \(U_1\) or \(U_2\) when looking at a table.
\end{enumerate}

At minimum \(U_1 = 0\), which means that sample 1 had ranks
\(1,2,\ldots,n_1\). Note that \(U_1 + U_2 = n_1 n_2\).

For large \(U\), there is a normal distribution approximation.

\subsubsection{Generation}\label{generation}

Say you have \(N\) total points and \(n_1\) in sample 1. Find all the
ways to draw \(n_1\) numbers from the sequence \(1, 2, \ldots N\).
Compute \(U\) for each of those. Voila.

Note that, if you fix \(n_1\), then you don't have to subtract the
\(n_1(n_1+1)/2\) to get the right \(p\)-value.

\section{Statistical power: Cochrane-Armitage
test}\label{statistical-power-cochrane-armitage-test}

We never want to run just any test: we want to use the test that is most
capable of distinguishing between the scenarios we're interested in.
Usually this is a matter of choosing the test that has the right
assumptions: the one-sample \emph{t}-test is more powerful than the
Wilcoxon test if the data come from a truly normally-distributed
population.

In other cases, you might have more flexibility. There's a somewhat
obscure test that is, I think, a great illustration of this.

Imagine that you have some data with a dichotomous outcome for some
categorical predictor value. One classic example is drug dosing: you
think that, as the dosage of the drug goes up, you have more good
outcomes than bad outcomes. Did a greater proportion of people on board
the Titanic survive as you go up from crew to Third Class to Second to
First? Did the proportion of some kind of event increase over years?
Technically, this means you have a \(2 \times k\) table of counts, with
two outcomes and \(k\) predictor categories.

\begin{longtable}[]{@{}llll@{}}
\toprule
Outcome & Dose 1 & Dose 2 & Dose 3\tabularnewline
\midrule
\endhead
Good & 1 & 5 & 9\tabularnewline
Bad & 9 & 4 & 1\tabularnewline
\bottomrule
\end{longtable}

You could use a \(\chi^2\) test with equal expected frequencies across
the columns. In other words, there might be more ``good'' than ``bad''
outcomes, but you don't expect that proportion to differ meaningfully
across categories. You would pool the data across categories, use the
observed proportion of good outcomes as you best guess of the true
proportion \(f\), and compare the actual data with you expectation that
a fraction \(f\) of the counts in each column are ``good''.

In our examples, we think the data have some \emph{particular} kind of
pattern. The \(\chi^2\) test doesn't look for any particular pattern; it
just looks for any deviation from the null. The test statistic for the
\(\chi^2\) distribution is based around the sum of the square deviations
from the expected values, usually written \(\sum_i (O_i - E_i)^2\), with
some stuff in the denominator to make the distribution of the statistic
easier to work with. If the sum of the squared deviations is too large,
then we have evidence that the observed values are not ``sticking to''
the expected frequencies.

The trick I'm going to show you is to keep the same null
hypothesis---that outcome doesn't depend on dose---but adjust the test
so that it's more sensitive to particular kinds of dependencies.

This is a fair approach because we're still just trying to say, ``OK,
say you (the nameless antagonist) were right, and there really was no
pattern in the data. Then I'm free to make up any test statistic, so
long as, if you're right, we can show that the observed data were likely
to have arisen by chance.''

To start constructing the test, think about each flip as a weighted
binomial trial. We'll use these weights to adjust the test statistic to
be more sensitive to what we suspect the true pattern in the data is,
but we'll need to derive the distribution of the test statistic so that
we can satisfy the nameless antagonist.

Say each flip \(y_i\), which is in some category \(x_i\), gets some
associated weight \(w_i\). A really simple statistic would be
\(\sum_i w_i y_i\), the sum of the weights of the ``successful'' trials.
It would be nice to have this be zero-centered: \[
\sum_i \left\{ w_i y_i - \mathbb{E}\left[ w_i y_i \right] \right\} = \sum_i w_i (y_i - \overline{p}),
\] where \(\overline{p} = (1/N)\sum_i y_i\).

It would also be nice for this to have variance 1, so we can divide by
the square root of \[
\mathrm{Var}\left[ \sum_i w_i (y_i - \overline{p}) \right] = \overline{p} (1-\overline{p}) \sum_i w_i^2
\] to produce the statistic \[
T = \frac{\sum_i w_i (y_i - \overline{p})}{\sqrt{\overline{p} (1-\overline{p}) \sum_i w_i^2}}.
\]

You could also conceive of this being a table with two rows and some
number of columns. We bin the trials by their weights: all trials with
the same weight are in the same column. Successes go in the top row;
failures in the bottom. Now write \(t_c\) as the weight of the trials in
the \(c\)-th column, \(n_{1c}\) is the number of successful trials with
weight \(t_c\) (i.e., in column \(c\)), and \(n_{2c}\) is the number of
failures. Then some math shows that you can rewrite \(T\) as \[
T = \frac{\sum_c w_c (n_{1c} n_{2\bullet} - n_{2c} n_{1\bullet})}{\sqrt{(n_{1\bullet} n_{2\bullet} / n_{\bullet\bullet}^2) \sum_c n_{\bullet c} w_c^2}}
\] where \(n_{r\bullet}\) are the row margins, \(n_{\bullet c}\) are the
column margins, and \(n_{\bullet\bullet}\) is the total number of
trials.

\emph{N.B.}: The wiki page gives a different answer, but I don't trust
it, since the variance formula doesn't assume independence of the
\(y_i\). A fact sheet about the PASS software that shows the formula in
terms of the \(y_i\) seems to make a mistake by using \(i\) as an index
both for individual trials and for the weight categories.

The confusing thing here is how to pick the weights. This test is mostly
used to look for linear trends: imagine that each \(y_i\) is associated
with some \(x_i\), so that the weights would be \(x_i\) or
\(x_i - \overline{x}\). Why you pick these exact weights has to do with
the \emph{sensitivity} of the test. There could, of course, be a
nonlinear trend, like a U-shape, that would lead to a zero expectation
for this statistic. The \(\chi^2\) test can find that, but
Cochrane-Armitage with these weights cannot.

To see why you use those weights for a linear test, imagine that
\(p_i \propto x_i\), and zero-center the \(x_i\) such that
\(p_i = m x_i + \overline{p}\).

Then the question is what \(w_i\) maximize \(\mathbb{E}[T]\)? You can
quickly see that this is equivalent to maximizing
\(\sum_i w_i x_i / \sqrt{\sum_i w_i^2}\), and taking a derivative with
respect to \(w_j\) shows that, at the extremum,
\(x_j \sum_i w_i^2 = w_j \sum_i w_i x_i\), which \(w_i = x_i\) for all
\(i\) satisfies. So those weights are the best way to get a large
statistic if you think that there actually is a linear test.

\section{Estimators}\label{estimators}

\subsection{Motivating example: uniform
distribution}\label{motivating-example-uniform-distribution}

Consider a uniform distribution from zero up to some number \(B\).
What's the maximum of this distribution? Well, if you know it, it's
easy: it's just \(B\). It's also easy if you can sample a zillion times
from the distribution: you are very likely at some point to get a value
very close to \(B\), so you can just use that as your estimate.

But what if you only drew one point? Then clearly the maximum of the
drawn values, just that point, is a pretty bad estimate for the upper
limit. If you draw 3 points, then the maximum is a little better, but
you can feel pretty sure that estimating the \emph{population} maximum
using your \emph{sample} maximum is biased: you're always going to
underestimate the population maximum.

When we use our data to make a guess about a population parameter,
that's an \emph{estimator}. (Good name.) We say that an estimator that
we expect to not be ``systematically''\footnote{Why the scare quotes?
  Like in all frequentist statistics, we're stuck with assuming that we
  know the truth, finding the best answer knowing the full truth, and
  then pretending we didn't know the truth in the first place. So
  ``systematically'' is going to mean: assume you know the true value,
  evaluate the behavior of the estimator knowing the truth, and then
  forget you know the truth.} above or below the true value is
\emph{unbiased}. Estimators are typically written with a hat (because
you use a cone hat to communicate with the mothership?), so \(\hat{B}\)
is an estimator for \(B\). Of course, there isn't just one estimator for
a value. I know plenty of people who are perfectly happy to guess that 3
is the right answer in statistics, so for them we might write
\(\hat{B} = 3\). This is clearly a bad estimator, but what does ``bad''
mean?

To answer that, let's consider some of the properties of different
estimators. To keep the different estimators straight, let's put
subscripts on them. I'll call the clearly bad estimator
\(\hat{B}_\mathrm{bad} = 3\) and I'll call our naive estimator \[
\hat{B}_1 = \max_i X_i,
\] where I called it ``1'' because it was our first guess. I also called
the variables we're drawing from the uniform distribution as \(X_i\).

\subsubsection{Consistent estimators}\label{consistent-estimators}

One reason that \(\hat{B}_\mathrm{bad}\) seems bad is that it doesn't
change. It won't ever get more ``right'', no matter how much data we
collect. On the other hand, \(\hat{B}_1\) will get better. In fact, the
more data we take, the closer we expect \(\hat{B}_1\) to get to the true
maximum.

Estimators whose expected value approach the true value as the number of
points approaches infinity is called \emph{consistent}. Call \(M\) the
true value and \(n\) the number of points we drew. Then \(\hat{B}_1\) is
consistent because \[
\lim_{n \to \infty} \hat{B}_1 = M.
\] In other words, if you specify an error threshold (e.g., you want an
estimate within 10\% of the true value), then I can tell you what \(n\)
you need to pick in order to get an estimate within that threshold (with
some finite probability). In contrast, with \(\hat{B}_\mathrm{bad}\),
there's no \(n\) you can pick that will get you arbitrarily close to the
true value.

\subsubsection{Unbiased estimators}\label{unbiased-estimators}

Clearly, though, \(\hat{B}_1\) is not perfect. No matter how many \(n\)
we draw, \(\hat{B}_1\) will always be less than \(M\). We therefore say
that \(\hat{B}_1\) is \emph{biased}. Mathematically, we might write
\(\mathbb{E}[\hat{B}_1] < B\). Because it feels natural to expect that
the expected value of your estimator will be the true value, an
estimator \(\hat{B}\) for which \(\mathbb{E}[\hat{B}] = B\) is called
\emph{unbiased}.\footnote{Another standard for ``good'' estimators is
  the \emph{best linear unbiased} estimator (BLUE). ``Linear'' means
  that the estimator is a linear function of the data. (You could maybe
  imagine that there is some wacky, convoluted function that's a better
  estimator, but we stick with linear.) The final qualifier, ``best'',
  means that the variance of the estimator is the minimum of all other
  linear unbiased estimators. Here ``variance of the estimator'' is
  computed in the same sense that the expected value of the estimator is
  computed for determining bias. In other words, ``best unbiased'' means
  its centered on the true value with the smallest variance possible.}

Let's try to salvage our naive estimator \(\hat{B}_1\) and make an
unbiased estimator \(\hat{B}_\mathrm{ub}\) instead.

Where to start? It's likely that there are many unbiased operators, so
we need to pick the form of the operator we want to look for. It seems
likely that all our observed \(X_i\) only have two interesting things
about them: their maximum, and how many of them there are. For example,
if my first draw was \(1.0\), then I don't learn that much if I draw
\(0.1\) the next five times, because I know that \(M > 1.0\), and,
because it's a uniform distribution, I was just as likely to draw
\(0.1\) as \(1.0\). If I draw many values below \(1.0\), I only learn
that the maximum is probably not far above \(1.0\), which is already
encoded in the number of points \(n\).

So I want \(\hat{B}_\mathrm{ub}\) to be some function of \(n\), which is
fixed, and \(M\), which is a random variable{[}\^{}wtf{]}, such that
\(\mathbb{E}[\hat{B}_\mathrm{ub}] = M\). That will require knowing the
distribution of \(M\).

\textbf{swo} I mixed a lot of this up. I called \(B\) the true value,
which got confusing. I should just use \(M\) and \(\hat{M}\).

The cumulative distribution function, the probability that the observed
maximum is less than some \(m\), is just the probability that there is
no point that is above \(m\): \[
F_M(m) = \mathbb{P}[M \leq m] = \mathbb{P}[X_i \leq m \text{ for all } i] = \left( \frac{m}{B} \right)^n.
\] Then we compute the probability density function: \[
f_M(m) = \frac{d}{dm} F_M(m) = \frac{n}{B} \left( \frac{m}{B} \right)^{n-1},
\] from which we compute the expected value of \(M\): \[
\mathbb{E}[M] = \int_0^B m \, f_M(m) \, \mathrm{d}m = \frac{n}{n+1} B.
\]

Interesting, the expected value of \(M\) is just a multiple of \(B\).
Consider that if \(n=1\), you just draw one point, the expected value of
the maximum, which is just that value you drew, is \(\tfrac{1}{2} B\).
Makes sense: you're likely to get something right in the middle.

If you draw 2 points, the expected maximum if \(\tfrac{2}{3}B\),
somewhat closer to \(B\). It's easy to see that as \(n\) increases,
\(\mathbb{E}[B]\) approaches \(B\), just as we reasoned about before
doing any math.

Now, by linearity of expectation, I know that
\(\mathbb{E}\left[\tfrac{n+1}{n} M \right] = B\), so lo! I have found an
unbiased estimator: \[
\hat{B}_\mathrm{ub} = \frac{n+1}{n} M.
\]

\subsubsection{Maximum likelihood
estimator}\label{maximum-likelihood-estimator}

It's nice that we found an unbiased estimator, which we think is
``good'' in the sense that we're not systematically off in one direction
or the other. In one way this is comforting, but it isn't foolproof.
Imagine you're working with a bimodal distribution: it tends to give
small numbers or big numbers. Then an unbiased estimator splits the
difference, potentially estimating values that aren't even possible!

For this reason, it's useful to think about \emph{maximum likelihood}
estimators (or MLEs, if you're into that sort of acronym). ``Maximum
likelihood'' means that, given the \emph{observed} value, you find the
\emph{true} value that, if it were true, would be most likely, out of
all the other true values, to give rise to your data.\footnote{Did that
  sound crazy? If so, get excited about the Bayesian section of this
  book. In there we'll show that there's a mathematically precise way to
  show that this kind of thinking---saying that the most likely state of
  nature given our data is the state of nature that, of all states of
  nature, was most likely to produce the data we observed---is coherent
  only under some pretty strong assumptions about the universe.}

In other words, given the observed \(M\), what \(B\) would have given
rise to that \(M\) with the greatest probability? Mathematically we
write \[
\hat{B}_\mathrm{ML} = \max_B \mathbb{P}[M | B],
\] where ``MLE'' stands for ``maximum likelihood''.\footnote{The
  promised Bayesian result is that the \(B\) that maximizes
  \(\mathbb{P}[M | B]\), the probability of the data given the state of
  nature, is also the \(B\) for which \(\mathbb{P}[B | M]\), the
  ``probability of the state of nature'' given the data, is maximum,
  although that's only true under a specific, and pretty
  hard-to-believe, criterion. I put more scare quotes here because, as
  mentioned before, if probability is a frequency, it doesn't make sense
  to talk about the probability of a state of nature.} This should be a
piece of cake, since we already found the probability density function
for \(M\), which is a function of \(B\). It's easy to see, just by
inspection, that the value between zero and \(B\) for which \(m\)
maximum \(f_M(m)\) is just \(B\). In other words,
\(\hat{B}_\mathrm{ML} = M\).

You may have noticed that this is the same estimator that I previous
denigrated as ``naive'' and ``biased''. To be clear, this is the maximum
likelihood estimator: you know that \(\hat{B}_\mathrm{ML}\) can't be any
less than \(M\), since the upper bound must be at least as high as the
biggest data point we observed, and some careful thought shows that
\(\hat{B}_\mathrm{ML}\) can't be any greater than \(M\), since
hypothetically increasing \(B\) above \(M\) just decreases the
probability that you would have observed \(M\) as the maximum
value.\footnote{Imagine hypothetically increasing \(B\) to a zillion
  times \(M\). Clearly, observing that small of an \(M\) is given that
  large of a \(B\) is unlikely, and there's no ``cut point'' so that an
  infinitesimal increase is reasonable but a zillion-fold increase is
  not.}

So this example tells an interesting lesson: it's not the case that an
estimator that is ``good'' in one sense (e.g., unbiased) will be
``good'' in another sense (e.g., maximum likelihood). We'll show that
there's a special case in which that's true. (Can you guess what it is?
No really, try. Or, don't try, and just say the first probability
distribution that comes to mind.)

\textbf{German tank problem} as an exercise.

\subsection{The tyranny of the normal, part
XLIV}\label{the-tyranny-of-the-normal-part-xliv}

You get some data points from a normal distribution (I hope you guessed
it), and you're interested in the mean and standard deviation of that
distribution (because those are the only things to know about).

What is the maximum likehood estimator for the mean? In other words,
given your data points \(x_i\), what \(\mu\), among all the other
\(\mu\), would have given rise to these particular data with the
greatest probability? \[
\begin{aligned}
\hat{\mu}_\mathrm{ML}
  &= \mathrm{argmax}_\mu \, \prod_i \mathbb{P}[X_i = x_i] \\
  &= \mathrm{argmax}_\mu \, \prod_i \frac{1}{\sqrt{2 \pi \sigma^2}} \exp\left\{ -\frac{(x_i - \mu)}{2\sigma^2} \right\} \\
  &= \mathrm{argmax}_\mu \, \left(2\pi\sigma^2\right)^{-\frac{n}{2}} \exp\left\{ \sum_i -\frac{(x_i-\mu)^2}{2\sigma^2} \right\}
\end{aligned}
\]

At this point, it's nice to pull a little trick: the value \(x\) that
maximum a function \(f(x)\) is the same value that maximizes
\(\log f(x)\). In other words: \[
\mathrm{argmax}_x f(x) = \mathrm{argmax}_x \log f(x).
\] This turns the nasty exponent into a sum: \[
\hat{\mu}_\mathrm{ML} = \mathrm{argmax}_\mu \left\{ -\frac{n}{2} \left(2 \pi \sigma^2\right) - \frac{1}{2\sigma^2} \sum_i (x_i-\mu)^2 \right\}.
\] It's pretty clear that we'll need to do something about \(\sigma\) at
some point: if \(\sigma\) is too big, then the first term blows up, and
the whole thing goes negative, which is not great for the argmax. If
\(\sigma\) is too small, then second term blows up. Whatever we pick
\(\sigma\) to be, it's clear that that choice is independent of the one
we should make for \(\mu\), since we should just pick a \(\mu\) that
minimizes \(\sum_i (x_i - \mu)^2\).

Is this starting to look familiar? The \(\mu\) that minimizes this sum
is the one that solves: \[
0 = \frac{d}{d\mu} \sum_i (x_i - \mu)^2 = \sum_i -2 (x_i - \mu) = -2 \left( \sum_i x_i - \sum_i \mu \right),
\] which implies that \(\sum_i x_i = n\mu\), that is,
\(\mu = \tfrac{1}{n} \sum_i x_i\). It's just our old, dear friend, the
arithmetic mean.

\subsubsection{Gauss-Laplace synthesis}\label{gauss-laplace-synthesis}

If you're not shocked by what just happened, you have no heart. If you
\emph{are} shocked, then you're in good company: the realization that
the normal distribution is the distribution for which the arithmetic
mean is the maximum likelihood estimator was part of the ``Gauss-Laplace
synthesis''. In 1809, Gauss published a book about ``least squares
estimation'', a method for picking a ``best fit'' estimate to data by
minimizing the squares of the deviations from the arithmetic mean (i.e.,
minimizing \(\sum_i (x_i - \mu)^2\)), was a ``good'' guess for the
``true'' value of the mean, where ``good'' meant that it was relatively
easy to do my hand and seemed to work right.

In the same book, Gauss worked with the normal distribution, which had
previously been cooked up by Laplace, and showed how you could use it as
a model for measurement errors. (Gauss's work with the normal
distribution so overshadowed Laplace's original description that we now
call the normal distribution ``Gaussian'', and ``Laplacian'' refers to a
different distribution.)

In 1810, Laplace showed that the normal distribution arises as the sum
of many random variables \textbf{see somewhere}. Later that year, he
read Gauss's book from 1809, and he was \textbf{get the quote}. In that
moment, Laplace realized that the reason that the arithmetic mean always
seemed so nice, and the reason that least squares seemed to work so
well, was because the normal distribution's fundamental property causes
it to arise in so many places, and \textbf{clean up this section}.

\subsubsection{End of historical note}\label{end-of-historical-note}

As you might guess, the arithmetic mean is also the unbiased estimator
for the population mean: \[
\mathbb{E}[\mathrm{\mu}] = \mathbb{E}\left[\frac{1}{n} \sum_i X_i \right] = \frac{1}{n} \sum_i \mathbb{E}[X_i] = \mu.
\]

\textbf{Is this really all that special, then?}

\subsection{Variance}\label{variance}

One estimator is used so often that it's often not even explained as an
estimator, or even called an estimator, which is very confusing. Imagine
if someone said that the definition of a maximum of a set of points was
\(\tfrac{n+1}{n}\) times the biggest value. Crazy, right? If you agree,
then you're well prepared.

The definition of sample variance is the average square deviation from
the mean: \[
s^2 \equiv \frac{1}{n} \sum_i (x_i - \overline{x})^2,
\] where \(\overline{x}\) is just the normal arithmetic mean. Because
we're used to estimators, it's a nice question to ask, is this a
``good'' estimator of the true variance \(\sigma^2\)?

\textbf{Some math} will show that
\(\mathbb{E}[s^2] = \tfrac{n-1}{n} \sigma^2\), that is, that the sample
variance is a biased (underestimating) estimator for the true variance.
Happily, it requires merely multiplying by \(\tfrac{n}{n-1}\) (called
``Bessel's correction''), so that the unbiased estimator is: \[
\hat{\sigma^2} = \frac{1}{n-1} \sum_i (x_i - \overline{x})^2.
\]

This estimator is so commonly used that it is often simply referred to
as ``sample variance'', leaving students to wonder why the variance is
one thing for ``true distributions'' and another thing for samples. To
be clear, the sum of square deviations divided by \(n-1\) is an unbiased
\emph{estimator} for the population variance.

Unfortunately, the words ``sample variance'' are used to refer to the
actual sample variance (sum of square deviations divided by \(n\)) as
well as to this estimator. There's often no way to tell. I'm sorry. You
should probably guess that people mean the \(n-1\) denominator
\emph{estimator}, and you should probably guess that most people won't
know the difference.\footnote{Comfortingly, as \(n\) grows, the
  difference between the sample variance and the ``sample variance''
  becomes negligible.}

\subsection{Estimators about
estimators}\label{estimators-about-estimators}

\subsubsection{Jackknife}\label{jackknife}

You have \(n\) data points and compute an estimator \(\hat{\theta}\) for
some population parameter \(\theta\). If you don't know how the
population is structured, then it's not clear what you expect the
variance of \(\hat{\theta}\) to be. How sure can you be of this value?
In terms of inference, can you make any inference with it?

Compute the \emph{jackknife replicates}\footnote{The ``jackknife''
  method is so called because Tukey compared the method, which is
  ``rough-and-ready'', to another rough-and-ready tool, the pocket
  knife, also known as a jackknife. Although this name has the
  disadvantage of giving you no clue what it is about, it had the
  advantage of having more brevity and vivacity than ``delete-1
  resampling'', which is probably the more accurate name.}
\(\hat{\theta}_j\), which are the estimators computed using all the data
points except the \(j\)-th one.

That seems like a weird thing to have done, but you can use them to
compute two handy things:

\begin{enumerate}
\def\labelenumi{\arabic{enumi}.}
\tightlist
\item
  An estimate of the variance of the estimator. This can help you for
  description---by giving a confidence interval(?)---and for
  inference---by giving you a sense of the ``random'' ranges you would
  expect from two samples.
\item
  An estimate of the bias in the estimator. This is helpful if you don't
  want want your estimator to be biased but you don't know how to fix
  it.
\end{enumerate}

\paragraph{Jackknife variance
estimator}\label{jackknife-variance-estimator}

The variance estimator is \[
\widehat{\mathrm{Var}}_\mathrm{jk}[\hat{\theta}] := \frac{n-1}{n}  \sum_j \left( \hat{\theta}_j - \hat{\theta}_{(\cdot)} \right)^2,
\] where \(\hat{\theta}_{(\cdot)}\) is the average of the jackknife
replicates: \[
\hat{\theta}_{(\cdot)} := \frac{1}{n} \sum_j \hat{\theta}_j.
\] In other words, it's the variance of the jackknife replicates with
some rescaling: \[
\mathrm{Var}[\hat{\theta}_j] = \frac{1}{n-1} \sum_j \left( \hat{\theta}_j - \hat{\theta}_{(\cdot)} \right)^2 \implies
  \widehat{\mathrm{Var}}_\mathrm{jk}[\hat{\theta}] = \frac{(n-1)^2}{n} \mathrm{Var}[\hat{\theta}_j].
\]

The reason for that scaling factor is beyond the scope of this book
(Efron \& Stein 1981?), but the exercise gives you a sense of why it has
to be true for a specific case.

Some other work, also beyond the scope of this book, shows that the
jackknife estimate of variance is biased: it tends to overestimate the
true variance. This makes the jackknife a conservative tool.

\textbf{Exercise}. Let \(\theta\) be the mean. Show that the scaling
factor is what we think. Hints:

\begin{itemize}
\tightlist
\item
  Show that \(\hat{\theta}_{(\cdot)}\) is the sample mean.
\item
  Show that
  \(\hat{\theta}_j - \hat{\theta}_{(\cdot)} = (n \overline{x} - x_j) / (n - 1)\).
\item
  Show that that value is equal to \((\overline{x} - x_j) / (n - 1)\).
\end{itemize}

That exercise is from McIntosh's bioRxiv about jackknife resampling.

\paragraph{Jackknife bias estimator}\label{jackknife-bias-estimator}

The jackknife estimate of bias is
\((n-1) \left( \hat{\theta}_{(\cdot)} - \theta \right)\). This is the
sum of the deviations of the jackknife replicates from the observed
value \(\hat{\theta}\). Again, the reason that you would take the
average deviation and scale it up to the sum is beyond the scope.

However, if you have an expectation about the bias in an estimator, you
can make an unbiased estimator by subtracting out that bias: \[
\hat{\theta}_\mathrm{jk} := \hat{\theta} - \widehat{\mathrm{Bias}}_\mathrm{jk}[\theta].
\]

\textbf{Exercise}. Show that the jackknife estimate of bias for the
variance gives you the familiar unbiased variance estimator.

\textbf{Exercise}. Something about the maximum estimator?

\paragraph{Pros and cons of the
jackknife}\label{pros-and-cons-of-the-jackknife}

It's a piece of cake to implement. There are only \(n\) replicates to
do, so it's tractable. Those replicates are deterministic, so you only
run it once.

The cons are that it doesn't always work. For example, a jackknife
estimate of the variance of a median (\textbf{swo check Knight}) is not
consistent. It's also overly conservative: it's biased toward higher
variances. You can rescue some properties if you move to a delete-\(d\)
resampling and pick \(d\) from the correct range.

\subsubsection{Bootstrap}\label{bootstrap}

What do you do when you want to compute the variance of some statistic
that's not easy to compute? Or you don't know what distribution you're
sampling from? Then you permute your own data. How do we relate:

\begin{itemize}
\tightlist
\item
  permutation tests
\item
  bootstrapping (and jack-knifing, etc.)
\item
  nonparametric (which I know is different)
\end{itemize}

\section{\texorpdfstring{What does it mean to
``sample''?}{What does it mean to sample?}}\label{what-does-it-mean-to-sample}

Does it make sense to compute a confidence interval when you're sampled
all the 50 United States?

\textbf{Finite correction factor} to point out that there's a difference
between simple random sampling and something else. Then need to explain
what simple random sampling is!

\section{Confidence intervals}\label{confidence-intervals}

Confidence intervals are very slippery things. It's tempting to say that
``I am 95\% confident'' that the true value of a quantity lies within
the 95\% confidence interval. In frequentist statistics, ``I am
\emph{X}\% confident'' has no meaning. The probability that the true
value lies within an interval is either 0 or 1, since if you repeat the
world-experiment many times, the true value will always be the same. In
other words, ``confidence'' is a Bayesian notion of probability. The
interval that you are 95\% confident that something falls in is
therefore a Bayesian concept (and it gets the confusing name of
``credible interval''). So forget that ``confidence interval'' has
anything to do with confidence.

Here's how things actually work: before you collect any data, you
develop a \emph{method} for generating the upper and lower confidence
intervals, which are a pair of statistics, that is, functions of the
data. This method has the property that the statistics it generates an
interval that, in 95\% of cases, contains the true value.

Here's the slippery part: the confidence interval is generated so that,
in 95\% of cases, they are ``correct'' in that they contain the true
value. In the other 5\% of cases, they don't include the true value.
Strictly speaking, they can't tell you anything about the probability
that the value is in the range.

\textbf{What's the easy way to explain the Bayesian link?} The typical
frequentist answer is really pedantic.

\subsection{Binomial test example}\label{binomial-test-example}

We set \(N\) and observe \(X\) to guess \(p\): \[
P[p \in \mathrm{CI} | X] \propto P[X | p \in \mathrm{CI}] \times P[p \in \mathrm{CI}].
\]

Confidence intervals don't actually do any of these. E.g.,
Clopper-Pearson guarantees that, given any \(p\), 95\% of the outcomes
will include the true value. The actual ranges included depend on \(p\)!
This isn't the case for the normal distribution, because it has all
these amazing properties about scaling and so forth. I think this is the
best example.

For example, start with the super dorpy confidence interval \([0, 1]\).
This is always true, but it's way too conservative. Then say something
else dorpy like a constant range, and show that this won't work if \(N\)
is too small.

Let's look at a real example. The Clopper-Pearson interval are the
limits of the range of values of \(p\) such that
\(P[x < X | p] > 2.5\%\) and \(P[x > X | p] > 2.5\%\). The first
inequality is fulfilled by smaller values of \(p\) (e.g., if \(p\) is
zero, then \(x\) has to be zero), so the \emph{upper} confidence limit
is determined by the greatest \(p\) that satisfies that inequality. The
second inequality is fulfilled by larger \(p\) (e.g., if \(p\) is 1 then
\(x=N\)), so the \emph{lower} confidence limit is determined by the
smallest \(p\) that satisfies this inequality.

How can we tell that this is a confidence interval? Given any true \(p\)
and \(N\), if I make draws \(x\) from \(\mathrm{Bin}(p, N)\), will this
confidence interval include \(p\) in 95\% of cases?

\subsection{\texorpdfstring{Comparison to
\(p\)-value}{Comparison to p-value}}\label{comparison-to-p-value}

Given \(N\), and having observed \(X\), we want to test the hypothesis
that \(p\) equals some \(p_0\) (typically \(\tfrac{1}{2}\)).

The nonrejection interval is the range of \(x\) such that \(x N\) is
close enough to \(p_0\) that the probability that \(x\) arose from
\(p_0\) is above some threshold. The lower limit is the lowest \(x\)
that would have arisen from \(p_0\) with some probability \(\alpha/2\),
that is, the smallest \(x\) such that \(P[x | p_0] > (1-\alpha)/2\).
This means that we take the state of nature \(p_0\) as given and scan
over the possible data values, comparing our data to those values.

In contrast, in a confidence interval, we take the data as given and
scan over the possible states of nature \textbf{in some way}. For the
binomial, it's obvious that the confidence intervals and the
nonrejection intervals are not the same thing, since it the one is
discrete and the other is continuous. Only in the normal (\textbf{???})
can we assume they are the same thing.

\subsection{Returning}\label{returning}

\begin{enumerate}
\def\labelenumi{\arabic{enumi}.}
\tightlist
\item
  You get some data that has some parameters (e.g., a sample mean and
  sample variance).
\item
  You \emph{guess} that your observed sample mean and variance are the
  \emph{true} mean and variance.
\item
  You ask, ``If someone else sampled from this true distribution many
  times, and they got all kinds of sample means and variances, what
  method could they use to construct, from the values they observed, an
  interval that would, in 95\% of cases, include the true value?''
\item
  Then you forget that you are omniscient and know the true value and
  instead use the methodology that these ignorant people would use, but
  you use it on your own data.
\end{enumerate}

Now the crazy Bayesian switch comes in: you conflate the frequency of
cases with your confidence that you are in the 95\% of cases.

\emph{N.B.}: For the \emph{t}-distribution, there's this ``pivotal
quantity'' thing, which means that the true \(\mu\) and \(\sigma\) drop
out, which is very luck, and it means that we \emph{don't} need to make
a parametric assumption about how things work.

\subsection{\texorpdfstring{\emph{t}-distribution}{t-distribution}}\label{t-distribution}

Let's think about how to construct that method. Say you knew the true
variance \(\sigma^2\). Then we know that the sample means are drawn from
\(\mathcal{N}(0, \sigma^2/n)\). So it's pretty easy to see that
\((\overline{x} - \mu) / (\sigma^2) \sim \mathcal{N}(0, 1)\), from which
the familiar \(1.96\), etc. come.

What if you \emph{don't} know the true variance? The means are still
drawn from \(\mathcal{N}(0, \sigma^2/n)\), but now the sample variance
is also a random variable.

We know the confidence interval is some function of the sample mean and
variance, and let's guess that it's symmetric about the sample mean and
is some linear function of sample variance: \[
\mathrm{CI}_\pm(\overline{x}, s) = \overline{x} \pm A s.
\] We want to find \(A\) such that \[
\mathbb{P}\left[ \mathrm{CI}_- < \mu < \mathrm{CI}_+ \right] = 95\%,
\] or, if we're willing to trust in symmetry, \[
2.5\% = \mathbb{P}\left[ \mathrm{CI}_- > \mu \right] = \mathbb{P}\left[ \frac{\overline{x} - \mu}{A} - s > 0 \right].
\] We know the distribution of the first thing: \[
(\overline{x}-\mu)/A \sim \mathcal{N}\left(0, \frac{\sigma^2}{n A^2}\right).
\] Some math shows that \[
\frac{(n-1) s^2}{\sigma^2} \sim \chi^2(n-1).
\]

Call the first thing \(K\) and the second \(L\). We're interested in the
distribution of \(M \equiv K - L\): \[
f_M(m) = \int_0^\infty f_K(m + l) f_L(l) \,\mathrm{d}l,
\]

where the limits come from the fact that variance is positive. You're
probably not excited to do this integral, which was considered a major
achievement (well, it was the thought leading up to the integral, which
we've just outlined, but whatever). This major achievement was made by
William Sealy Gosset, who made it while he was a researcher for Guinness
ensuring the quality of their beer. Guinness had a policy of not
allowing its employee to publish their results, so Gosset signed his
paper ``a student'', so the result of that integral is now called
Student's \emph{t}-distribution: \[
f_t(x; \nu) = \frac{\Gamma(\frac{\nu+1}{2})}{\sqrt{\nu\pi} \Gamma\left(\frac{\nu}{2}\right)}
  \left(1+ \frac{x^2}{\nu}\right)^{-\frac{\nu+1}{2}},
\] where the (badly named) ``degrees of freedom'' \(\nu\) is \(n-1\) for
our purposes. I write this out fully because it is one of the things we
will \emph{not} derive in this book.

\section{Contingency tables}\label{contingency-tables}

These are nice examples for how to do statistical thinking.

\subsection{Barnard's test}\label{barnards-test}

The classic example is whether a certain treatment causes more of the
outcome of interest than just doing nothing. In medicine, that means
splitting your participants into a placebo group and a treatment group
and asking what fraction of each gets well. In a biology experiment, you
might split your mice into a treatment group and a control group and ask
what proportion of the mice in each group get cancer.

In statistics jargon, this is called a \(2 \times 2\) contingency table:

\begin{longtable}[]{@{}llll@{}}
\toprule
Group & Outcome \(p\) & Outcome not-\(p\) & Row sums\tabularnewline
\midrule
\endhead
A & \(a\) & \(c\) & \(m\)\tabularnewline
B & \(b\) & \(d\) & \(n\)\tabularnewline
Column sums & \(r\) & \(s\) & \(N\)\tabularnewline
\bottomrule
\end{longtable}

Because we picked \(m\) and \(n\), the sizes of the two groups, those
are fixed parameters. The question is whether the way that \(m\) gets
distributed into \(a\) and \(c\) (and that way that the \(n\) get put
into the \(b\) and \(d\)) is consistent with there being a common
probability \(p\) of the outcome of interest.

So we might say that \(a\) is distributed like a binomial distribution
with \(m\) draws and probability \(p_a\) of success, and \(b\) is
distributed like a binomial with \(n\) draws and a probability of
\(p_b\) of success. The null hypothesis is that \(p_a = p_b\). What's
the likelihood of the data given the null?

If we didn't assume the null, and gave the two binomials their own
probabilities, the likelihood of the data would be: \[
P(a, b | p_a, p_b) = \mathrm{Bin}(a; m, p_a) \times \mathrm{Bin}(b; n, p_b).
\] But, given that the probabilities are the same, we can collapse it:
\[
\begin{aligned}
\mathcal{P}[a, b | p_a = p_b = p] &= \mathrm{Bin}(a; m, p) \times \mathrm{Bin}(b; n, p) \\
  &= \binom{m}{a} p^a (1-p)^{m-a} \times \binom{n}{b} p^b (1-p)^{n-b} \\
  &= \binom{m}{a} \binom{n}{b} p^{a+b} (1-p)^{m+n-(a+b)} \\
  &= \frac{m! \, n!}{a! \, b! \, c! \, d!} p^r (1-p)^s.
\end{aligned}
\]

This result is a little confusing\footnote{I trotted out this test
  because these two confusions are actually great learning
  opportunities.}, for two reasons:

\begin{enumerate}
\def\labelenumi{\arabic{enumi}.}
\tightlist
\item
  The probability \(p\) of the outcome of interest might be interesting
  to design a later experiment, but it's \emph{not} interesting for
  designing a test. We certainly don't want to deliver a result like,
  ``Well, if the null hypothesis is true, \emph{and} \(p\) happens to be
  exactly such-and-such, then your \(p\)-value is so-and-so.'' The value
  \(p\) is called a \emph{nuisance parameter} since we don't actually
  care about its value.
\item
  We're usually not interested in the likelihood of exactly this data,
  but rather in the likelihood of data \emph{at least this extreme}. We
  usually measure ``extremeness'' using a statistic---a single
  number---so it's clear that ``more extreme'' means ``bigger'' (or
  ``smaller'' or ``bigger or smaller'', depending on if it's a one-sided
  or two-sided test). Here, we have two numbers, \(a\) and \(b\), so
  there aren't two ``sides'' to the distribution: there are four!
\end{enumerate}

To resolve the first point, we say that the null hypothesis
\(p_a = p_b = p\) doesn't restrict us to a particular value of \(p\). In
other words, the null hypothesis, which functions as a sort of Annoying
Skeptic, is free to pick \(p\) to make our results as uninteresting as
possible. Mathematically, this means that, when computing the
\(p\)-value, we should optimize over all values of \(p\), choosing the
one that makes our results as uninteresting as possible (i.e., which
maximizes the \(p\)-value).

We can't really ``resolve'' the second point, since it demonstrates that
our previous way of thinking about extremeness was not sufficient for
all cases. As Barnard notes in his original paper\footnote{Barnard
  conceived of the \((a, b)\) as points ``in a plane lattice diagram of
  points with integer co-ordinates'', that is, that \(a\) is like the
  \(x\)-axis and \(b\) is like the \(y\)-axis. Then the possible
  outcomes of the experiment are the points in the rectangle bounded by
  the horizontal lines \(a = 0\) and \(a = m\) and the vertical lines
  \(b = 0\) and \(b = n\). He then said that you should pick the
  non-extremal points (i.e., the values of \((a, b)\) for which you
  would not reject the null) such that they ``consist of as many points
  as possible, and should like away from that diagonal of the rectangle
  which passes through the origin. Formulated mathematically, these
  latter requirements mean that the {[}points for which you would reject
  the null{]} must in a certain sense be convex, symmetrical and
  minimal.''}, there are actually many ways to choose the pairs
\((a, b)\) that produce a \(p\)-value more than our threshold. This gets
into some fancy footwork to articulate exactly how you should pick this
area, but the basic results are pretty intuitive: when \(a/m\) and
\(b/n\) are similar, you tend to be under the rejection threshold; when
they are far apart, you tend to be over.

The interesting point here is that, whatever fancy footwork you pick to
choose that region, and no matter how ``reasonable'' your footwork is,
it's still footwork that doesn't obviously follow from the simple
definition of a hypothesis test. We'll encounter this problem again in
Bayesian statistics, when we find that the Bayesian analog of a
confidence interval is not unique: there are many ranges of values that
are compatible with our ignorance.

\subsection{Fisher's test to the
rescue(?)}\label{fishers-test-to-the-rescue}

If you've worked with contingency tables, you're probably saying, ``I've
never heard of this crazy Bernard's test, with its weird multi-sided
rejection space and its requirement to maximize over \(p\). We have
Fisher's exact test, which is the exactly right test to use here!''

Looking at the same contingency table, Fisher's test asks, given the row
marginals \(m\) and \(n\), the first column marginal \(r\), and the
grand total \(N\), what is the probability of a table at least this
extreme?

This is just a combinatoric problem: if you're as likely to assign items
in \(m\) to \(a\) as to \(c\) (and, analogously, to assign items from
\(n\) to \(b\) or \(d\)), then ``what's the probability of this table''
is equivalent to asking ``given the marginals, how many ways are there
to choose this table?''. More specifically, how many ways are there to
choose \(a\) items from a bank of \(m\) items and \(b\) items from a
bank of \(n\), given that we chose \(r = a + b\) items from the total
\(N\)? Mathematically: \[
\mathbb{P}[a | m, n, r, s] = \frac{\binom{m}{a} \binom{n}{b}}{\binom{N}{r}} = \frac{m! \, n! \, r! \, s!}{N! \, a! \, b! \, c! \, d!}.
\]

Computing the \(p\)-value is easier here than with Barnard's test
because we need to keep the row \emph{and column} marginals the same. In
Barnard's test, we just kept the row marginals constant, because we
considered those as fixed parameters, corresponding to things like the
number of patients we assigned to each of the placebo and treatment
groups. It doesn't make sense to allow the Annoying Skeptic to fiddle
with those values.

In Banard's test, we \emph{did} allow the Annoying Skeptic to fiddle
with the column marginals, since it wasn't clear, before the experiment
began, that \(r\) would have the outcome of interest. In other words, we
didn't know that \(r\) people in both the placebo and treatment groups
would get well.

Fisher's test, however, \emph{does} keep the column marginal constant.
This makes it a lot easier to compute the \(p\)-value. First, the
nuisance parameter \(p\) doesn't appear in the likelihood, so we don't
need to do the weird maximization. Second, we only need to vary one
value, \(a\) (or, equivalently, \(b\)), since, if you know the
marginals, there is only one axis along which to change the values in
the table. In other words, if you know \[
\begin{aligned}
a + c &= m \\
b + d &= n \\
a + b &= r,
\end{aligned}
\] then that's three equations with four unknowns (\(a\), \(b\), \(c\),
\(d\)), so specifying any one of \(a\), \(b\), \(c\), or \(d\) specifies
all the others. (You might be looking for a fourth equation
\(c + d = s\), but you can get that by adding the first two equations
and subtracting the third.)

Here's an example:

\begin{longtable}[]{@{}llll@{}}
\toprule
Group & Success & Failure & Row sums\tabularnewline
\midrule
\endhead
A & 1 & 9 & 10\tabularnewline
B & 11 & 3 & 14\tabularnewline
Column sums & 12 & 12 & 14\tabularnewline
\bottomrule
\end{longtable}

There's only one way to make this table more ``extreme'' without
changing the marginals: you can take the one group A success and make it
a group A failure and simultaneously make a group B failure into a group
B success. Similarly, there's only one way to make this table less
extreme: turn a group A failure into success, and turn a group B success
into failure.

So keeping the column sums constant made it way easier to compute the
\(p\)-value: count this table and all the tables with a more extreme
upper-left or bottom-right and see if your summed probability hits the
rejection threshold.

However, this simplicity came at a cost, which you may have noticed:
does it make sense to keep the columns constant? Experimentally, this
means that you're restricting the Annoying Skeptic to only consider
cases in which, say, the number of patients who got well \emph{in both
groups} is equal to the experimentally observed value. This is a little
weird. It suggest that your experimental design was like this:

\begin{enumerate}
\def\labelenumi{\arabic{enumi}.}
\tightlist
\item
  Pick \(m\), \(n\), and \(r\).
\item
  Assign \(m\) patients to placebo and \(n\) to treatment.
\item
  Wait until \(r\) patients \emph{across both groups} have gotten well.
\item
  Stop the experiment.
\end{enumerate}

This is almost certainly not reflective of how typical experiments are
run\footnote{It is, however, the way the famous ``lady tea tasting''
  experiment was designed. The myth is that Fisher didn't believe it
  when a high-class lady told him that she could detect whether tea was
  added to a cup with milk in it or whether the milk was added to the
  tea. He designed an experiment with \(m\) cups prepared one way, \(n\)
  prepared the other, and told her to detect the \(r = m\) cups that
  were prepared the first way. A Barnard-style experiment, in which the
  same \(m\) and \(n\) cups}.

\textbf{Fisherian small data}

\textbf{What happens if I use the ``wrong'' test? Chi-square as an
example of wrongness}

\section{Regression}\label{regression}

Hardin pages 56-57 talks about the difference between the normal
(subject-specific) and the generalized (population-average) estimating
equations. E.g., second-hand smoking: what's the odds of a kid having a
respiratory illness given that mom smokes? SS parameters give the OR for
\emph{each individual child} having the illness, so it's what we would
expect would happen if particular moms stopped smoking. PA parameters
give the OR \emph{across the population}, so it explains the difference
in prevalence we expect in the two groups. The first one is: \[
\mathrm{OR}^\mathrm{SS} = \frac{P(Y_{it}=1 | X_{it}=1, \nu_i) / P(Y_{it}=0 | X_{it}=1, \nu_i)}{P(Y_{it}=1 | X_{it}=0, \nu_i) / P(Y_{it}=0 | X_{it}=0, \nu_i)}
\] and the second is \[
\mathrm{OR}^\mathrm{PA} = \frac{P(Y_{it}=1 | X_{it}=1) / P(Y_{it}=0 | X_{it}=1)}{P(Y_{it}=1 | X_{it}=0) / P(Y_{it}=0 | X_{it}=0)}.
\] The difference, in case you didn't catch it, is whether you condition
on the random effect \(\nu_i\). The SS estimate is for these particular
people; the PA estimate marginalizes over the random effects.

\section{Bayesian}\label{bayesian}

One of the biggest sticking points about a Bayesian analysis is that it
requires specification of a prior. It can be thought of as an advantage
or a disadvantage, but I think it's better to think of it as a
responsibility. Let me tell you an allegory.

Once, a young statistician lived in her parents' house. She paid no rent
and simply never considered her orientation in the world. Some year
later, she left the home and had to do statistics in the wide world.
Where would she live? How would she pay rent? These decisions brought
power, since she was free to do things she couldn't do at home. She
could live a life that was more accurate to the real world. This
allegory is too long and rambly. But there's something in here.

Frequentist statistics is correct so long as you use it exactly for what
it is designed to do. The trouble is that we \emph{want} statistics to
answer the kinds of questions that \emph{only} Bayesian statistics can
answer. For example, how likely is it that this hypothesis is true? If
you perfectly adhere to the frequentist interpretation, then you are in
good shape. But if you deviate, if you start to say, ``Oh, the p-value
is kind-of like the probability my hypothesis is false.'' Then you have
SCREWED UP son.

\section{Appendix}\label{appendix}

\subsection{Random number generation}\label{random-number-generation}

In many places in this book, we've relied on the ability to generate
``random'' numbers. However, computers (in the sense of logical
machines) have no way to generate truly random numbers. Instead, we have
clever methods that get us something that's a pretty good approximation
of random numbers.

It's worth noting that a lot of the historical, conceptual directions in
statistics are due to the fact that doing any kind of Monte Carlo
methodology without computers is really onerous. Before we had today's
technology (pseudorandom number generators, to be discussed below), we
had tables of random numbers, the most notable being the hefty \emph{A
Million Random Digits with 100,000 Normal Deviates} (where ``normal
deviate'' means ``random number drawn from a standard normal
distribution), published by the (coincidentally-named) RAND
Corporation.\footnote{You can still get this book
  \href{https://www.rand.org/pubs/monograph_reports/MR1418.html}{on
  RAND's website} or
  \href{https://www.amazon.com/Million-Random-Digits-Normal-Deviates/dp/0833030477/}{in
  paperback}. The numbers were generated using an electronic device,
  specifically designed to shuffle a sort-of random table of numbers,
  attached to a computer. See the text about hardware random number
  generators.} RAND published this book because they had a lot of
engineers and researchers using Monte Carlo methods. Before the tables
of random numbers, you had to generate the random numbers yourself,
which was basically infeasible.\footnote{In 1777, Georges-Louis Leclerc,
  Comte de Buffon posed a math problem that included probability and
  geometry: if you throw needles (or matchsticks) onto a surface with
  parallel stripes whose widths are equal to the length of the needles,
  what fraction of the needles touch two stripes? It turns out that the
  answer has \(\pi\) in it. So in 1901, Mario Lazzarini published that
  he had tossed a needle 3,408 times and, using the analytical solution,
  back-calculated \(\pi\) as \(\tfrac{355}{113}\). This estimate of
  \(\pi\) was already known, and the fact that Lazzarini came up with
  exactly that value is taken as strong evidence that he faked the
  experiment. In other words, we have a pretty strong prior against
  believing that someone in 1901 even bothered to throw a needle 3,000
  times, much less the many more times than that that would be required
  for random number generation for more interesting Monte Carlo
  problems!}

As mentioned, now we have \emph{pseudorandom number
generators}\footnote{There is such a thing as a ``hardware'' random
  number generator, which is some kind of device that measures something
  that we think is truly noisy in the real world, like thermal noise or
  (what we believe are truly random) quantum phenomena like
  beamsplitting.}. These rely on some input \emph{seed}, which is the
(hopefully) truly random thing, and from that seed it generates a
deterministic list of numbers that, in the absence of knowing the seed,
appear random.\footnote{It may seem weird that, given a seed, you get a
  deterministic set of numbers. Most software with (pseudo)random number
  generators pick a seed using whatever entropy they have access to when
  you boot up the program, so you never notice that the seed is
  different each time you run a simulation. You can, however, always
  \emph{pick} the seed, which is nice, because it lets you repeat code
  with a Monte Carlo method in it and always get the same result, which
  is nice for testing and debugging.} Usually this seed comes from one
of the many ``entropy sources'' that a computer has access to, things
like the time between keystrokes, the time at which a process was
started, time between network pings, etc.

The pseudorandom number generator in most software now is the Mersenne
Twister. This algorithm is remarkable for having a long \emph{period} of
\(2^{19937} - 1\). (All pseudorandom number generators, started with
with some seed, will eventually end up repeating their output. The
period is the number of outputs you get before closing the loop.) The
random things produced by the generators are typically mapped into
uniformly distribution varibles over \([0, 1]\).

\subsubsection{Generating non-uniformly-distributed
numbers}\label{generating-non-uniformly-distributed-numbers}

Drawing numbers from \([0, 1]\) usually isn't that interesting. We want
to draw numbers from other distributions. There are two main approaches:

\begin{enumerate}
\def\labelenumi{\arabic{enumi}.}
\tightlist
\item
  Clever transformations
\item
  Various forms of \emph{rejection sampling}
\end{enumerate}

The idea with clever transformations is to generate random numbers from
the uniform distribution and somehow turn them into random numbers
distributed according to some other distribution.

Rejection sampling is a big class of approaches, including such notables
as ``Metropolis-Hastings'' and ``Markov chain Monte Carlo'' (MCMC). They
are very useful for the practical scientist.

\textbf{inverse transform, ziggurat, rejection, Metropolis-Hastings and
other Monte carlo mcmc stuff}

Clever transformations are nice when you can do them, but it's unlikely
you'll derive one for yourself. Basically, if a run-of-the-mill random
number generation function in some software purports to be able to
sample numbers from some distribution, it's doing this transformation. I
don't think there's anything really practical to be gained from knowing
these transformations, but they're fun, so I put them here.

\paragraph{Clever transformations}\label{clever-transformations}

If you can write the cdf of your distribution of interest, say
\(F_X(x)\) and you invert it (i.e., solve for \(x\) in terms of
\(F_X\)), then you can use a nice trick called \emph{inverse transform
sampling}.

\subparagraph{Normal distribution: Box-Muller
transformation}\label{normal-distribution-box-muller-transformation}

To generate normally-distributed numbers from uniformly distributed
numbers, consider this trick.

Think about a pair of independent, normally-distributed variables
\(Z_1\) and \(Z_2\). Their joint pdf will be \[
\begin{aligned}
f_{Z_1,Z_2}(z_1, z_2) &=
  \frac{1}{\sqrt{2\pi}} \exp\left\{ -\frac{z_1^2}{2} \right\} \times
  \text{same thing for $z_2$} \\
  &= \frac{1}{2\pi} \exp\left\{ -\frac{1}{2} \left( z_1^2 + z_2^2 \right) \right\}.
\end{aligned}
\] The trick is to think of \(z_1\) and \(z_2\) as Cartesian coordinates
like \(x\) and \(y\), from which it's very natural to replace
\(z_1^2 + z_2^2\) with \(r^2\) and define some \(\theta\) such that
\(z_1 = r \sin \theta\) and \(z_2 = r \cos \theta\). My claim is that
we'll be able to generate \(r\) and \(\theta\) from independent, uniform
random variables.

Because \(f_{Z_1,Z_2}\) is symmetric with respect to \(z_1\) and \(z_2\)
(i.e., you could swap that subscripts and come out with the same
expression), it must be that there isn't anything special about having
sine versus cosine. In other words, there mustn't be anything special
about \(\theta = 0\) versus \(\theta = \pi\). The origin can't matter.
Thus, it must be that \(\theta\) is uniformly distributed over
\([0, 2\pi]\). Any other distribution would end up treating \(z_1\) and
\(z_2\) differently, which would break their independence.

Generating \(r\) is a little more tricky. Let's look at the cumulative
distribution function of the random variable \(R\): \[
\begin{aligned}
\mathbb{P}[R < r] &= \int_0^{2\pi} \int_0^r f_{Z_1, Z_2}
    \,\mathrm{d}z_1 \, \mathrm{d}z_2 \\
  &= \int_0^{2\pi} \int_0^r \frac{1}{2\pi} \exp\left\{-\frac{1}{2} r'^2\right\}
    r' \,\mathrm{d}{r'} \,\mathrm{d}\theta \\
  &= \int_0^r \exp\left\{-\frac{1}{2} r'^2\right\}
    r' \,\mathrm{d}{r'} \,\mathrm{d}\theta \\
  &= \int_0^{\tfrac{1}{2} r^2} \exp\left\{-s\right\} \,\mathrm{d}s,
    \text{where $s = \tfrac{1}{2} r'^2$} \\
  &= 1 - \exp\left\{ -\frac{1}{2} r^2 \right\}.
\end{aligned}
\] If we define \(r = \sqrt{-2 \log u}\), then
\(\mathbb{P}[R < r] = 1 - u\), which is just the cdf for a uniformly
distributed variable \(U\) on \([0, 1]\). So we generate \(r\) using
that formula.

It may seem a little strange that we can generate independent \(z_1\)
and \(z_2\) using \(r\) and \(\theta\). You might think that if I know
\(z_1\), then I can guess something about \(r\) or \(\theta\) and use
that information to make a guess about the value of \(z_2\). However,
because the Cartesian and polar coordinate systems encode exactly the
same information, that argument is like saying that, because I told you
\(x\), you might be able to guess \(y\), which is clearly impossible.

\section{Unplaced}\label{unplaced}

\subsection{Chebyshev's inequality}\label{chebyshevs-inequality}

What's the relationship between confidence intervals and variance? We
all know the relationship for the normal distribution.

As a lemma, consider a random variable \(X\) that only takes on
nonnegative values. Then \[
\mathbb{E}[X] = \sum_{k=0}^\infty k \, f_X(k) \geq \sum_{k=1}^\infty f_X(k) = \mathbb{P}[X \geq 1].
\] (To see the middle inequality, note that you can drop \(k=0\), and
then, for all the \(k \geq 1\), you can replace \(k\) with \(1\), which
makes that term in the sum smaller than it might be.) We'll use the
reversed version: \(\mathbb{P}[X \geq 1] \leq \mathbb{E}[X]\).

Now, Chebyshev's inequality is easy. Use the \[
\begin{aligned}
\mathbb{P}\left[|X - \mu| \geq k \sigma\right]
  &= \mathbb{P}\left[ \frac{(X - \mu)^2}{k^2 \sigma^2} \geq 1 \right] \\
  &\leq \mathbb{E}\left[ \frac{(X - \mu)^2}{k^2 \sigma^2} \right]
    \quad \text{(by the lemma)} \\
  &= \frac{1}{k^2 \sigma^2} \mathbb{E}\left[(X-\mu)^2\right] \\
  &= \frac{1}{k^2}. \quad \text{(since that expected value is $\sigma^2$ by definition)}
\end{aligned}
\]

For \(k=1\), we result is trivial: at most 100\% of values fall outside
1 standard deviation from the mean. (An upper bound of 100\% tells us
nothing.) For \(k=2\), at most \(1/4 = 25\%\) of values fall outside 2
standard deviations. That is, 75\% fall inside. For \(k=3\), 89\% fall
inside. These are much more conservative results than for the normal
distribution, for which 95\% of values fall within 2 standard deviations
and 99.7\% fall within 3.

I'm not sure of the practical utility of this inequality. It requires
knowing the true variance, which already requires a whole bunch of data.

\subsection{Pearson's correlation}\label{pearsons-correlation}

Take the original variables and standardize them to make \(X\) and
\(Y\). A way to measure how they ``go together'' is \(\rho = E[XY]\).
This might be more familiar to you as \[
\rho = \frac{1}{n} \frac{\sum_i (x_i - \overline{x})(y_i - \overline{y})}{\sigma_X \sigma_Y}.
\] To show that this is in the range \([-1, 1]\), consider a new
variable \(Z \equiv X - \rho Y\). The variance of \(Z\) is \[
\begin{aligned}
\mathbb{V}\left[ (X - \rho Y)^2 \right]
  &= \mathbb{E}[X^2] - 2 \rho \mathbb{E}[XY] + \rho^2 \mathbb{E}[Y^2] \\
  &= 1 - 2 \rho^2 + \rho^2 \\
  &= 1 - \rho^2.
\end{aligned}
\] But the expectation of a square is at least zero, so
\(0 \leq 1 - \rho^2\), which implies that \(-1 \leq \rho \leq 1\).

This arises naturally if you're thinking that \(X\) and \(Y\) are both
normally distributed.

And this connects to the regression coefficient. This is why the
``fraction of variance explained'' is related to a regression.

How to show that \(\mathrm{max} \sum x_i y_i\) subject to \(Ex=0\),
\(Ey=0\), \(Vx=1\), and \(Vy=1\) is 1 and occurs for \(Y = \rho X\),
i.e., the two are linearly related?

\subsection{\texorpdfstring{Visualiztion of how \(\hat{p}\) changes with
\(n\)}{Visualiztion of how \textbackslash{}hat\{p\} changes with n}}\label{visualiztion-of-how-hatp-changes-with-n}

\begin{verbatim}
n = 1000
p = 0.3
x = rbinom(n, 1, p)
cumx = cumsum(x)
cumn = cumsum(rep(1, n))
phat = cumx / cumn
cil = mapply(function(x, n) binom.test(x, n)$conf.int[1], cumx, cumn)
ciu = mapply(function(x, n) binom.test(x, n)$conf.int[2], cumx, cumn)

data_frame(flip=x, phat, cil, ciu) %>%
  mutate(x=1:n()) %>%
  ggplot(aes(x, phat)) +
  geom_ribbon(aes(ymin=cil, ymax=ciu), fill='grey80') +
  geom_line() +
  geom_hline(yintercept=p, linetype=2)
\end{verbatim}

\end{document}
