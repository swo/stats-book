%!TEX root = main.tex

\chapter{Functions}
\label{chapter:functions}

It may seem strange to start a book about statistics with something as abstract
as functions, but many of the important objects in statistics are actually
functions. If you are not entirely comfortable with the formalism of mathematical
functions, fundamental concepts in statistics may become very confusing.

As an example, in statistics, you often see notation like $X=x$. You might be
tempted to think that $X$ is a variable representing a number, and $x$ is another
variable representing a number, and the equal sign means that the two variables
represent the same number. However, in probability and statistics, the notation
$X=x$ refers to something entirely different: $X$ represents a kind of function
called a \emph{random variable}, $x$ is a number, and the whole statement is not an
assertion that the two sides of the equation are equal but instead a reference
to the set of \emph{events} that the function $X$ maps to the value $x$. If this
sounds incredibly bewildering, I agree, and helping unpack that is the purpose
of the next few chapters.

\section{Functions map things to other things}

In mathematical terms, a \emph{function} is relationship between sets, linking
each element of one set to an element of another set. For example, say I have a
function $f$ that links each number to that number plus one:
\begin{equation*}
f(1) = 2, \quad f(2) = 3, \quad \text{etc.}
\end{equation*}
Take careful note: the equation $f(1) = 2$, when translated into words, means that the function
$f$ maps the input $1$ to the output $2$. It does not mean that the function equals
$2$. A function is a relationship between sets, not a number.

In computer programming, a ``function'' refers to a sort of computational factory
that takes some input and returns some output. This kind of ``function'' is not
necessarily a mapping like a mathematical function is. For example, a random number
generator ``function'' is designed to return a different, unpredictable number
every time is it called. Some programming languages define \emph{pure} functions,
which are analogous to mathematical functions in that they are guaranteed to deliver
the same output whenever given the same input, but this is unusual. Do not confuse
mathematical functions with the conceptually distinct computer programming concept
that happens to have the same name.

\section{Functions can map different kinds of things, including functions}

The function $f$ above is relatively simple: it takes a single real number as input
and returns another real number. We characterize the function $f$ like this:
\begin{equation*}
    f : \mathbb{R} \to \mathbb{R},
\end{equation*}
where $\mathbb{R}$ represents the set of real numbers. A slightly more complex function
adds two numbers:
\begin{equation*}
    \mathrm{add}(1, 2) = 3
\end{equation*}
Technically, this function still takes a single thing as input, only now that ``thing''
is an ordered pair of numbers. Ordered pairs of real numbers are also called the
\emph{Cartesian product} of the set of real numbers with itself, that is,
$\mathbb{R} \times \mathbb{R}$, or just $\mathbb{R}^2$. We characterize this function
like:
\begin{equation*}
    \mathrm{add} : \mathbb{R}^2 \to \mathbb{R}
\end{equation*}

Functions can map from things that aren't numbers. In linear algebra, every square matrix has a determinant, which is a real number. Thus, ``determinant'' is a function:
\begin{equation*}
    \mathrm{det} : \{\text{square matrices}\} \to \mathbb{R}
\end{equation*}
Functions can also map to things that aren't numbers. Again in linear algebra, one
matrix has multiple eigenvectors. Thus, ``eigenvectors'' is a function:
\begin{equation*}
    \mathrm{eigenvectors} : \{\text{matrices}\} \to \{\text{multiple vectors}\}
\end{equation*}

Importantly, functions can map from functions and to functions. For example,
the minimum of the function $f(x) = x^2$ is zero. We characterize the ``minimum''
function like:
\begin{equation*}
    \mathrm{min} : (\mathbb{R} \to \mathbb{R}) \to \mathbb{R},
\end{equation*}
where $(\mathbb{R} \to \mathbb{R})$ means the set of functions that map from real
number to real numbers. Or, note that the mirror of some function $f(x)$ around
$x=0$ is a new function $g(x) = f(-x)$. Thus, ``mirror around $x=0$'' is a function:
\begin{equation*}
    \mathrm{mirror} : (\mathbb{R} \to \mathbb{R}) \to (\mathbb{R} \to \mathbb{R})
\end{equation*}


\section{Functions are written in different ways}

We are used to writing functions using the \emph{prefix} (or ``Eulerian'') style,
like $f(1) = 2$. In a few cases with well-known functions, we drop the parentheses,
such as in $\sin \pi = 1$.

However, there are other styles of writing functions. In \emph{infix} notation, the
function (or ``operator'') goes between the two inputs, such as in $1 + 2$, rather
than $+(1, 2)$. We use \emph{around-fix} notation for the absolute value, like in $|x|$,
and we use \emph{postfix} notation for the factorial function, such as $4!$. And
there are plenty more. The binomial ``choose'' function $\binom{n}{k}$ maps a pair of
integers to an integer. Summation, such as $\sum_{x=0}^n f(x)$, can be thought of as
mapping a function and two integers (the lower and upper limits) to a real number (i.e.,
the value of the sum).

The important thing is to understand what each symbol on the page is. If it not a
variable, then it is likely a function, or a part of the notation needed to encode
a function. If it is a function, what is it mapping from and to?


\section{Different functions are written in the same way}

In computer science, \emph{polymorphism} refers to the fact that the same ``function''
(in the computer science sense) will, when handed inputs of different types, perform
different computations. For example, in many programming languages, \texttt{+} is used
to signal numeric addition as well as string concatenation: \texttt{1 + 2} returns \texttt{3},
but \texttt{"Cui " + "bono?"} gives \texttt{"Cui bono?"}. Somehow the plus
``function'' knew to add in one case and to concatenate in the
other.

In mathematics, polymorphism is mostly called ``different notation''. For example,
the plus symbol ($+$) can be used to add two real numbers, but it can also be used
to ``add'' two matrices. Adding two numbers and adding two matrices are analogous
operations, but they are not identical. In statistics, we will encounter this situation
often, where a series of symbols on the page could make perfect sense as an operation
on numbers, but when in fact it means something very different.

The most important example of this confusion comes from the meaning of the equal sign
($=$). If I write $x=2$, it is most likely a definition: I'm asserting that the value
of $x$ is $2$. But if I write $1=2$, and ask you, ``Is this true?'', you will say no.
Many programming languages distinguish between the assignment operator \texttt{=} and the
equality test operator \texttt{==}, so that \texttt{x=1} is a command ``set $x$ to $1$!'',
while \texttt{x==2} is a question ``is $x$ equal to 2?'', to which the program will
response \texttt{TRUE} or \texttt{FALSE}.

To help ease some confusion, I will put little symbols on top of the equal signs to
clarify what the equal sign means. For example, $f(x) \defeq x^2$ means that I'm defining
the function $f$, while $x^2 = x \cdot x$ means that I'm showing you the steps in a proof, and $x \stackrel{?}{=} 2$ means that I'm asking a question.
