%!TEX root=main.tex

\chapter{Introduction to statistical testing}

\section{Statistical testing is young}

Statistical hypothesis testing, or ``significance testing'', is one of the most
widespread methods in science. If you pick up an article of \textit{Science} or
\textit{Nature} and read any scientific article, there will almost certainly
be a $p$-value in
it. I put statistical tests after the incredibly fundamental concepts of
observation, quantitation, and experimentation as the building blocks of
contemporary science.

Of these fundamental scientific concepts, statistical testing is by far the youngest.
In the
Western scientific tradition, observation as a method goes back to Aristotle
(b. 384 BC). Other ancient Greeks were using quantitative measurements and
geometry to
determine things like the diameter of the Earth. Experimentation may as we now
think of it was championed by Francis Bacon (b. 1561 AD) but probably has much
older roots. Our earliest examples of statistical hypothesis testing date to
the 1700s ---one of the early examples will be explored in a later chapter---
but testing as we now understand it was formalized by Ronald Fisher (b. 1890),
Jerzy Neyman (b. 1894), and Egon Pearson (b. 1895).

The fact that a young method, whose formalization is less than 100 years old,
has permeated nearly all of science speaks to its intellectual appeal. I think
it also clarifies why statistical testing and interpretation of $p$-values is
such a controversial issue in contemporary science: we as a scientific
community simply have not had enough time to fully digest the idea. Not only is
the conceptual basis of statistical testing not fully refined, we have also
not found the best ways to explain this concept to one another.

\section{Statistical inference is philosophically confusing}

Not only is statistical testing a relatively young idea, but statistical
inference as a conceptual approach is deeply entwined with some of the
fundamental philosophical questions that underly probability theory.  In my
experience, the students and practitioners that are my target audience for this
book are challenged by these philosophical problems of statistical inference
just as much as they stuggle with the mathematical aspects of statistical
testing!

Rather than sweep those philosophical questions under the rug, I will lay them
out, to avoid confusion later.
