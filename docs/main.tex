\documentclass{book}

\usepackage{amsmath,amsfonts,amssymb}
\usepackage{bm}
\usepackage{mathtools} % for dcases
\usepackage{hyperref}
\usepackage{longtable}
\usepackage{booktabs}
\usepackage{color,soul} % for hl
\usepackage{fullpage}
\usepackage{xfrac} % for sfrac

\providecommand{\tightlist}{\setlength{\itemsep}{0pt}\setlength{\parskip}{0pt}}

\newcommand{\prob}[1]{\mathbb{P}\left[{#1}\right]}
\newcommand{\expect}[1]{\mathbb{E}\!\left[{#1}\right]}
\newcommand{\var}[1]{\mathbb{V}\left[{#1}\right]}
\newcommand{\cov}[1]{\mathrm{Cov}\!\left[{#1}\right]}
\newcommand{\defeq}{\stackrel{\text{def}}{=}}
\newcommand{\rveq}{\stackrel{\text{rv}}{=}}
\newcommand{\dd}{\, \mathrm{d}}

\title{A short introduction to advanced statistics for scientists}
\author{Scott W. Olesen}

\begin{document}
\maketitle

\tableofcontents

\frontmatter

%!TEX root=main.tex

\chapter{Preface}

As a graduate student and postdoc in the life sciences, I saw that many of my
colleauges, had substantial quantitative training and experience. They were
very able to hack together fairly sensible quantitative and statistical methods
to answer their scientific questions. I found, however, that there was often a
steep drop-off when it came time to apply statistical rigor to these hacky
methods.

I think this gap between the ability to hack something together and the
understanding to make something rigorous is partly born out of a gap in
educational materials. There are plenty of introductory statistics
textbooks that explain what a mean is, and there are plenty of statistical
test cookbooks that well you what assumptions go into a $t$-test, and
there are plenty of books and articles on statistics meants for people
with a graduate-level education in statistics or math. There are few
resources for people who are mature and shrewd quantitative thinkers but
who don't have a half dozen statistics courses under their belt.

I think the situation is analogous to learning a foreign language as an
adult. I find that it is easy to find books for children, which are at my
level in terms of grammar and vocabulary but thematically boring, and
there are books for adults, which are thematically interesting but way
over my head. I wanted to write a book about statistical tests that was
thematically \emph{and} ``gramatically'' appropriate.

This is not a book for people who want to push the boundaries of what
statisticians think are interesting mathematical problems. Nor is it
a a cookbook that tells you what statistical test to run on your data.
I want to give you the ability to reason about why to do a test and how to
formulate one. Rather than telling you which steps to generate a $p$-value
in a specific case, I show how $p$-values come about in general and how to
derive them for specific cases.

I hope you find it useful!


\mainmatter

%!TEX root=main.tex

\chapter{Introduction to statistical testing}

\section{Statistical testing is young}

Statistical hypothesis testing, or significance testing, is one of the most
widespread methods in science. If you pick up an article of \textit{Science} or
\textit{Nature} and read any scientific article, there will be a $p$-value in
it. I put statistical tests after the incredibly fundamental concepts of
observation, quantitation, and experimentation as the building blocks of
contemporary science.

Statistical testing is the youngest of these fundamental concepts. In the
Western scientific tradition, observation as a method goes back to Aristotle
(b. 384 BC). The ancient Greeks were also using quantitative measurements to
determine things like the diameter of the Earth. Experimentation may as we now
think of it was championed by Francis Bacon (b. 1561 AD) but probably has much
older roots. Our earliest examples of statistical hypothesis testing date to
the 1700s ---one of the early examples will be explored in a later chapter---
but testing as we now understand it was formalized by Ronald Fisher (b. 1890
AD), Jerzy Neyman (b. 1894 AD), and Egon Pearson (b. 1895 AD).

The fact that a young method, whose formalization is less than 100 years old,
has permeated nearly all of science speaks to its intellectual appeal. I think
it also clarifies why statistical testing and interpretation of $p$-values is
such a controversial issue in contemporary science: we as a scientific
community simply have not had enough time to fully digest the idea. Not only is
it not fully refined, we have also not found the best ways to explain it to one
another.

\section{Statistical inference is philosophically confusing}

Aside from being a relatively young idea, statistical testing and
statistical inference run into some of the critical philosophical
foundations of probability theory. My experience is that my colleagues
stumble on the philosophical problems of statistical inference as much as
on the mathematical aspects! Rather than sweep those under the rug, I want
to lay them out, to avoid confusion later.

\emph{Probability} is, mathematically speaking, a well-defined concept: it
is a function (more technically, a ``measure'') that links elements in
a set of possible outcomes of an experiment with numbers between $0$ and
$1$. For example, if I show the outcome ``coin flip comes up heads'' to
the probability function, it will return to me the number $\tfrac{1}{2}$.

\subsection{Frequentist and Bayesian probability}

Philosophically, probability has two main interpretations:
\emph{frequentist} and \emph{Bayesian}. When doing math and science,
frequentist probability statistics is the default: when people say
``statistics'', they almost always mean frequentist statistics, unless
they specifically say ``Bayesian statistics''. The principal mistake made
in interpreting statistical tests is that, although the test was
frequentist in construction, it is interpreted in a Bayesian way.

It is therefore important to understand the two different definitions of
probability and their implications for inference:
\begin{itemize}

\item \emph{Frequentist probability is a proportion.} ``The probability of
outcome $X$ from this experiment is $p$'' means that, as the number of 

\end{itemize}


\part{Probability}

%!TEX root = main.tex

\chapter{Functions}
\label{chapter:functions}

It may seem weird to start a book about statistics with something as abstract
as ``functions'', but I think that probability theory becomes very confusing if you
are not entirely comfortable with the formalism of mathematical functions.

As an example, as they are typically meant in statistical writing, $x=2$ is a
true or false statement, while $X=2$ is an \emph{event}, which we will define
later. This distinction often causes enormous confusion. After reading this
chapter, you should comfortable with the idea that the ``equals'' sign in the
statements $x=2$ and $X=2$ are actually different functions.

\section{Functions map things to other things}

In mathematical terms, a \emph{function} is relationship between sets, linking
each element of one set to an element of another set. For example, say I have a
function $f$ that links each number to the that number plus one:
\begin{gather*}
f(1) = 2 \\
f(2) = 3 \\
\text{etc.}
\end{gather*}
This function is relatively simple in that it has only one input and it links
elements of a set, the integers, onto other members of the same set. As we will
see, functions can be much more complicated than this.

For many people, especially those used to computer programming, it may be more
natural to think of a function as a factory: $f$ ``takes'' an input number $x$,
adds $1$ to $x$, and then ``returns'' the output $x+1$. This analogy can be
confusing because, in computer science, a ``function'' can actually refer to
different things. The correct analogy to the mathematical concept of
``function'' is called a ``pure'' function in computer science, where ``pure''
means that the function is guaranteed to produce the same output whenever given
the same input.\footnote{A pure function also has no ``side effects'', which
means that calling that function cannot change anything about the statement of
the program.} For example, a random number generator, while it might be called
a ``function'' in a program, is not a pure function. To the user's eye, a
random number generator is called multiple times, with no input, and returns
different numbers every time.

Some functions, like my example above, map one number to another, single
number. Other functions take two numbers and return one number. For example, say
the function $g$ adds two numbers:
\begin{equation*}
g(1, 2) = 3
\end{equation*}
In fact, this is just the ``plus'' function that we usually write with \emph{infix
notation}, meaning that we put the operator between the arguments, like $1+2$,
rather than at the front, like $\mathord{+}(1, 2)$.

A function can take something that is not a number and return a number. For
example, say the determinant of a matrix $A$ is $5$:
\begin{equation*}
\mathrm{det}(A) = 5
\end{equation*}
The reverse is of course possible. For example, a function could take as input
a positive integer and return the identity matrix of that size:
\begin{equation*}
    \mathrm{Identity}(3) = \begin{bmatrix} 1 & 0 & 0 \\ 0 & 1 & 0 \\ 0 & 0 & 1 \\ \end{bmatrix}
\end{equation*}

Or a function can take numbers and return something that is not a number. For
exmaple, the statement ``$5$ is less than or equal to $2$'' is false: the
operator ``less than or equal two'' takes two numbers ($5$ and $2$) and returns
a Boolean value (``false'').

A function can also take a function as input and return a number. For example,
the minimum of the function $f(x) = x^2$ is zero. A function (``minimum'') took
a function ($f$) as input and returned a number (zero).

A function can also take a function and return a different function. For
example, the mirror of $f(x)$ around the $x=0$ point is a new function $g(x) =
f(-x)$. The function ``mirror around $x=0$'' took a function ($f$) as input and
returned another function ($g$). In computer science, functions that return
functions are sometimes called ``function factories''.

I hope it's clear that the inputs or outputs could be anything. I could even
say that I have a function $h$ that maps from the space of quadrilaterals to
the space of vegetables:
\begin{equation*}
h(\lozenge) = \text{Brussels sprouts}.
\end{equation*}
There is a special notation to make it clear what kinds of spaces a function
maps from and to:
\begin{equation*}
h : \text{quadrilaterals} \to \text{vegetables}.
\end{equation*}
Perhaps more familiar, the function $f$ above maps from real numbers to other
real numbers:
\begin{equation*}
f : \mathbb{R} \to \mathbb{R}
\end{equation*}
Functions with multiple inputs, like $g$ above, get a Cartesian product
``cross'':
\begin{equation*}
g : \mathbb{R} \times \mathbb{R} \to \mathbb{R}
\end{equation*}



\section{Functions are written in different ways}

Confusingly, we write functions in many different ways. The most obvious is the
``prefix'' or ``Eulerian'' style, where single-letter, italic text functions or
3-ish-letter, normal text functions precede their inputs:
\begin{gather*}
f(1) = 2 \\
\sin \pi = 1 \\
\mathrm{add}(1, 2) = 3
\end{gather*}



Sometimes the ``dot'' notation, like $f(\cdot)$, is used to distinguish the
function $f$ from the number $f(x)$ that the function outputs for the input
$x$. The dot avoids the confusion about whether $x$ is a stand-in or whether it
is an actual number by using the dot, which is clearly not a number.

But sometimes we write certain functions of two inputs with an ``infix'' style
in between the two inputs: we write $1 + 2 = 3$ rather than $\mathord{+}(1, 2)
= 3$. In a few cases, we write a function as ``around-fix''. For example, the
determinant of the matrix $A$ is often written $|A|$. We even use a ``postfix''
for things the factorial function: $4! = 24$.

Don't be alarmed. These are all just functions, mappings from one set of things
to another set of things. If you see some notation you do not understand, ask,
``What is this a function of? What kinds of things is it mapping from and to?''

\section{Different functions are written in the same way}

Computer scientists are likely comfortable with \emph{polymorphism}, the
notion that the same ``function'', applied to different inputs, takes different
actions. For example, in many programming languages, \texttt{+} is both
numeric addition and string concatenation: \texttt{1 + 2} returns \texttt{3},
but \texttt{"Stats " + "rules"} gives \texttt{"Stats rules"}. Somehow the
``function'' \texttt{+} knew to add in one case and to concatenate in the
other.

We do the same thing with fairly common symbols, like ``equals'' ($=$). For
example, say $f(x) = x^2$. Then $f(2) = 4$ is true but $f(2) = 5$ is false.
The ``equals'' in $f(2) = 5$ is a function that takes two numbers, $f(2)$ and
$5$, and returns the Boolean ``false''. Extra confusingly, the ``equals'' in
$f(x) = x^2$ is not a function at all; it is a signal to you, the reader, that
a function is being defined! I try to avoid this confusion in this book by
writing ``def'' on top of any equal sign that is a mathematically important
definition.

Polymorphism arises in a few critical and very confusing moments in probability
theory. For example, if I say ``the probability that $X$ equals two'', you will
respond with a number between zero and one. Say the value is one-third:
\begin{equation*}
\prob{X = 2} = \tfrac{1}{3}
\end{equation*}
Apparently, ``probability'' is a function that maps statements like
``$X$ equals two'' to a number:
\begin{equation*}
\mathbb{P} : \text{statements like ``$X=2$''} \to [0, 1]
\end{equation*}
Now if I say, ``the probability that one equals two'', you will say ``zero'':
\begin{equation*}
\prob{1 = 2} = 0
\end{equation*}
This makes it seem like ``$X=2$'' and ``$1=2$'' are the same
kinds of statements.

However, in another sense, ``one equals two'' is clearly ``false'', but
``the probability of false is zero'' doesn't make a lot of sense. The probability
function does not take Booleans; it takes something called \emph{events} that
will be defined in the next chapter. The point is that the ``equals'' in ``$X=2$''
means a different thing from the ``equals'' in ``$1=2$'', since ``$X=2$'' need
not resolve to simply ``true'' or ``false''.

%!TEX root = main.tex

\chapter{Probability theory}

Probability is an entire branch of mathematics. To understand statistics, it's
critical to have a grasp of basic concepts in probability. Indeed, inferential
statistics was originally called ``inverse probability''. ``Probability''
meant the study of going from rules to outcomes. For example, given the rules of
poker, how likely am I to draw a royal flush? ``Inverse probability''
meant going from outcomes to rules. Given the large number of times
you've drawn royal flushes, can I conclude that you are a cheater?

Data follow the rules of probability, so to analyze data, we need to
understand probability theory. Probability theory is all about a thing called
\emph{probability}. Before going any further, let's examine this concept.

\section{The definition of probability}

We talk about and reason about probability every day, mostly for matters of
prediction. Given the weather report, it it worth it for me to carry an
umbrella? Given the results
of my experiment, how likely is it that my hypothesis was correct?

We intuitively think about probability in a mostly
\emph{Bayesian} way. Statistics, when not prefaced by the word
``Bayesian'', refers to a different philosophical branch call
\emph{frequentist} statistics. These philosophical distinctions have
important practical implications.

In Bayesian statistics, \emph{probability} refers to a difficult-to-articulate
sense of confidence about a future event. If I say the
probability of an event is 1\%, it means, in a practical sense, that I'm willing to bet a lot of money it
won't happen. Philosophically, it's hard to say what ``1\%
confident'' means. Even if it's philosophically challenging, I think this is
how we all think about probability, so I won't go on about it.\footnote{\textbf{swo: sunrise problem?}}

A more mathematically and philosophically simple way to think about
probability is the \emph{frequentist} approach. In this framework, probabilities are all
proportions, or ``frequencies''.
A 50\% chance of
flipping a coin and seeing heads means that, as you flip the coin more and
more times, the proportion of flips that come up heads will approach 50\%.

The problem with the frequentist approach is that probabilities can only be
assigned to events that can be repeated. It therefore doesn't make sense to
ask about the probability that your hypothesis is correct, since your
hypothesis is either correct or incorrect. It's like asking the probability
that it rained yesterday. It either did or it didn't.
We live in just one universe, so you can't
ask, in a frequentist scheme, about the probability of a state of nature, like
whether your hypothesis is correct.

This is deeply dissatifying: the whole point of statistical inference is to
figure out what's going on in the world. But the Bayesian approach, which does
allow you to ask about the probability of states of nature, is mathematically
and technically more challenging. Also, some prominent statisticians thought
it was nonsense and mostly directed the field of statistics away from the Bayesian
approach for many years.

So what's the definition of probability? It's either:
\begin{enumerate}
\item in the frequentist sense, the proportion of identically-repeated trials that have some outcome, or
\item in the Bayesian sense, some number between 0 and 1 that encodes our sense of the---for lack of a better word---probability of that outcome.
\end{enumerate}

\section{The mathematical definition of probability}

In the last section I used the words ``trial'', ``outcome'', and
``probability''.  There is a sophisticated mathematical theory about all these
concepts called \emph{measure theory}. Although fascinating, it's pretty heavy
lifting, so I'll aim to use definitions that will give us a lot of practical
benefit, albeit at the cost of creating a mathematically incomplete theory.

To start, we define \emph{outcomes} or \emph{sample space} as the set of all
things that could happen when we run some trial. For example, if a flip a
coin, it will land heads or tails. The collection of all possible combinations
of outcomes are the possible \emph{events}.\footnote{This is where the
mathematical incompleteness comes in. If the outcomes are finite (i.e.,
discrete, like in our cards example), it's straightforward to enumerate the
events as the finite set of all possible subsets of the outcomes. If the
outcomes are continuous, things get tricky. You can't just ``count'' an
infinite number of events, so you need to be careful about how you define
``all possible combinations''. The right way to do things is with a
\emph{$\sigma$-algebra}.} For flipping a coin, the events include the
individual outcomes (heads or tails), the combinations of outcomes (either
heads or tails), and no outcome (didn't flip anything).

\begin{table}
\centering
\begin{tabular}{p{3cm}p{4cm}p{6cm}}
\toprule
Situation & Outcomes & Events \\
\midrule
Flipping a coin & heads, tails & $\varnothing$, $H$, $T$, $\Omega$ \\
Rolling a die & $1, 2, \ldots, 6$ & $\varnothing$; $1, 2, \ldots, 6$; $\binom{6}{2}$ 2-event outcomes (e.g., 1 or 2); $\binom{6}{3}$ 3-outcome events (e.g., 1, 2, or 3); $\ldots$; $\binom{6}{5}$ 5-outcome events (e.g., any except 1); $\Omega$ \\
Drawing a card & 2 of Clubs, $\ldots$, Ace of Hearts & $\varnothing$; 52 individual events (e.g., 2 of Clubs); $\binom{52}{2}$ 2-outcome events (e.g., black Ace); $\ldots$; $\binom{52}{51}$ 51-outcome events (e.g., any except 2 of Clubs); $\Omega$ \\
Pick a random number between 0 and 1 & All real numbers between 0 and 1 & $\varnothing$; intervals like $[0, \tfrac{1}{2}]$; ``all subsets'' of $[0, 1]$; $\Omega$ \\
Your experiment & Each possible configuration of atoms at measurement & ``Cancer cell died'', ``Electron had energy between 1 and 2 eV'', etc. \\
\bottomrule
\end{tabular}
\caption{Distinction between outcomes and events for various example situations. $\varnothing$ is the nothing-happened event. $\Omega$ is the something-happened event. The ``all subsets'' is a $\sigma$-algebra, which we won't dive into.}
\label{tab:outcome-event}
\end{table}

The difference between \emph{outcomes} and \emph{events} might seem trivial,
but it will come up a few times, so I'll repeat that \emph{outcomes} are
individual, real things that might happen, while \emph{events} are abstract
groupings of outcomes. Importantly, nothing-happened is an event but clearly
not an outcome: if I run a trial, it can't be that nothing happened, but
for mathematical reasons, nothing-happened is an event.

Each event is associated with a probability via a function that has no real
name except ``probability''. It is a basic axiom of probability theory that
probabilities always within zero and one. The probability function, written
here as $\mathbb{P}$, links each event $\omega$ in the space $\Omega$ of
events with a probability $p$:
\begin{gather*}
\mathbb{P} : \Omega \to [0, 1] \\
\prob{\omega} = p
\end{gather*}
For example, if the event $H$ is flipping heads, then we say
$\prob{H} = \tfrac{1}{2}$. I use the square brackets to emphasize
that $\mathbb{P}$ is a function of something that is not a number: it takes
events $\omega$ and not numbers like 5.

Probability theory axioms specify that the anything-happened event $\Omega$ has
probability one and the nothing-happened event $\varnothing$ has probability zero:
\begin{align*}
\prob{\Omega} &= 1 \\
\prob{\varnothing} &= 0
\end{align*}
Something must happen, and nothing cannot happen.

\section{Manipulating probabilities}

We're often interested in the relationships between events. What is that
probability that this \emph{or} that happened? What is the probability that
this \emph{and} that happened?

\subsection{``And'' adds event probabilities}

If $A$ and $B$ are two events that don't have any constituent outcomes in
common, we call then \emph{disjoint}, and their probabilities add. For
example, the probability of flipping a heads or a tails is
the probability of heads plus the probability of tails.
Mathematically we write this as
$$
\text{if } A \cap B = \varnothing \text{, then } \prob{A \cup B} = \prob{A} + \prob{B},
$$
where the ``cap'' $\cap$ means \emph{intersection} (``and'') and the ``cup'' $\cup$ means
\emph{union} (``or''), so this reads ``if no outcomes are in both $A$ and $B$,
then the probability of $A$ or $B$ is the sum of their individual probabilities.''

If $A$ and $B$ do have some overlap, you need to subtract out the probability
of the overlap. For example, consider drawing a card
from a standard 52-card deck. What is the probability of drawing a Jack or a
Diamond? If you add up the probability of drawing a Jack and the probability
of drawing a Diamond, you end up double-counting the Jack of Diamonds event,
so the solution is to subtract out the double-counted event:
$$
\prob{A \cup B} = \prob{A} + \prob{B} - \prob{A \cap B}.
$$
We mostly deal with disjoint events, so I won't belabor this point.

\subsection{``Or'' multiplies events}

We just said that ``or'' for disjoint events adds probabilities. How do we
find the probability of $A$ and $B$? Say I flip two coins. What's the
probability that I flip heads on the first coin and heads on the second?

If $A$ and $B$ are \emph{independent} events, then their probabilities
multiply. The probability of flipping two heads is $\tfrac{1}{2} \times
\tfrac{1}{2} = \tfrac{1}{4}$.

\subsection{Independence and conditional probability}

How do you know if two events are independent? Mathematically, $A$ and $B$
are indepedent if and only if their probabilities multiply.

This might feel circular. Intuitively, independence means that the events
can't depend on each other. I flip two separate coins, so the flip of one
can't affect the flip of the other.

``Depends on'' also has a mathematical notation. While the probability function
took a single event and returned a single number, \emph{conditional
probability} is a function that takes two events and returns a probability:
$$
\mathbb{P}[A | B] = p.
$$
This equation is read as ``the probability of $A$ given $B$ is $p$''.

In a frequesting mindset, where probabilities are proportions, $\prob{A | B}$
is, on a denominator of the trials in which $B$ happened, the proportion of
trials  in which $A$ also happened.\footnote{This is another place in which my
simple definition hides controversy and subtlety. I'm using the ``Kolmogorov
definition'' of conditional probability, but there's more than one way to
approach it. Probability is a philosophically and mathematically complex notion.}
Say $n_B$ is the number of trials in which $B$ happened, $n_{AB}$ is the
number of trials in which $A$ and $B$ happened, and $n$ is the total number of
trials. Then the proportion we're talking about is $n_{AB} / n_B$.

\begin{figure}
\caption{Conditional probability as shrinking of universe}
\label{fig:conditional-probability}
\end{figure}

To get to the more general definition of conditional probability, divide top
and bottom by $n$ to get $(n_{AB}/n) / (n_B/n)$. The numerator is the
proportion of trials where $A$ and $B$ happened. The denominator is the
proportion of trials were $B$ happened. Thinking of proportions of trials as
probabilities, it follows that: $$ \prob{A | B} \defeq \frac{\prob{A \cap
B}}{\prob{B}}. $$ A pictoral way of thinking about conditional probabilities
is to say you're interested in the probability of $A$, not in the original
universe of events $\Omega$, but in the smaller universe of events $B$ (Figure
\ref{fig:conditional-probability}).

This notation clarifies our definition of independence. If $A$ and $B$ are
independent, then $\prob{A \cap B} = \prob{A} \times \prob{B}$, so, by the
definition, $\prob{A | B} = \prob{A}$. In other words, independence means that
the probability of one event doesn't depend on the other.
%!TEX root=main.tex

\chapter{Random variables}

In the previous chapter, we examined outcomes, events, and their
probabilities. Outcomes are specific configurations of atoms, which is not at
all a practical thing for a scientist think about. Events are abstract
groupings of those configurations, which is also impractical. Instead, in
science, we measure the results of experiments with numbers. We link whatever
we see with a number. In microbiology, we could cells. In physics, we measure
the energy of an electron.

In probability theory, \emph{random variables} link outcomes with numbers.
They are the mathematical analogue of scientific measurement. Random variables
will lead us to familiar concepts like means, standard deviations, and
probability distributions.

\section{Definition of random variables}

Confusingly, a ``random variable'' is a function, not a number. We might say
that a random variable ``takes on'' a value, but the random variable itself,
in mathematical terms is always a function.

A \emph{discrete} random variable $X$ is a function that associates outcomes
$\omega \in \Omega$ with a finite set of numbers $x_i$:
\begin{gather*}
X : \Omega \to \{x_1, \ldots x_k\} \\
X[\omega] = x_i
\end{gather*}
for some finite $k$. A \emph{continuous} random variables associates outcomes with real
numbers $x \in \mathbb{R}$:
\begin{gather*}
X : \Omega \to \mathbb{R} \\
X[\omega] = x
\end{gather*}
This is a simplification: measure theory says that a random variable can map
the outcome space to any \emph{measurable space}. We won't dive into that.

For example, when flipping a die, you could imagine a random variable $X$
encoding ``number of heads flipped'':
\begin{align*}
X[H] &= 1 \\
X[T] &= 0
\end{align*}
Roughly speaking, the values of the random variable group the outcomes into
events, which each have probabilities. This means it makes sense to ask about
the event in which a random variable takes on a value, e.g., $\prob{X = 1} = \tfrac{1}{2}$.
Note that ``$X = 1$'' does not mean that $X$ is the number one; instead, the
entire phrase ``$X = 1$'' refers to the event where $X$ took on the value one.

Random variables therefore allow us to abstract over the outcomes and look
only at the values taken on by the random variable. From now on, we will have
no reason to think about the specific configuration of atoms in our
experiment. We only think about the numbers delivered by our apparatus. Soon
we'll be able to say things like ``$X$ is normally distributed'' without having
to specify what are the outcomes and events.

The previous chapter was not a waste, however, because the
rules for manipulating events and probabilities work just as well for
events like ``$X = 1$'' as they did for ``flipped heads''. The philosophical and
technical definition of probability still holds. For example, $\prob{(X = 1)
\cup (X = 0)}$, the probability that $X$ takes on the values zero or one, is the
sum of the two constituent probabilities $\prob{X=1}$ and
$\prob{X=0}$.

\section{Distribution of random variables}

How is $X$ ``distributed''? If you're like me, you think in terms of
\emph{probability distribution functions} (pdf's) and consider
\emph{cumulative distribution functions} (cdf's) as a derived quantity, but in
fact it's mathematically easier to go the reverse route. If you aren't
familiar with either, never fear.

\subsection{Cumulative distribution functions}

The first way of defining the ``distribution'' of a random variable $X$ is its
\emph{cumulative distribution function}, also called the cumulative
\emph{density} function and abbreviated ``cdf''. The cdf of a discrete random
variable $X$ is the probability that the random variable takes on a value
below some threshold $x_i$:
\begin{gather*}
F_X : \{x_1, \ldots, x_k\} \to [0, 1] \\
F_X(x_i) \defeq \prob{X \leq x_i} = \sum_{j \,:\, x_j \leq x_i} \prob{X = x_j}
\end{gather*}
A continuous random variable has a cdf that takes real numbers:
\begin{gather*}
F_X : \mathbb{R} \to [0, 1] \\
F_X(x) \defeq \prob{X \leq x},
\end{gather*}
Note that the cdf is a function of numbers, which I emphasize by
using regular parentheses around $x$ in $F_X(x)$.

Take careful note that ``$X \leq x$'' is an event, not a Boolean expression
like ``$1 \leq 2$''. This may seem pedantic, but I think having a really good
grasp of the notation will help you articulate correct thoughts more clearly.
Things will quickly get confusing if you think that $\prob{X \leq x}$ means the same
thing as $\prob{x \leq x}$.

% swo: need a figure here with the laddering

A discrete random variable takes on a maximum value $x_\mathrm{max}$ so that
\begin{gather}
\text{if } x < x_\mathrm{min} \text{, then } F_X(x) = 0 \\
F_X(x_\mathrm{max}) = 1.
\end{gather}
If a continuous random variable can take on any real number, then  the cdf of
a continuous random variable approaches zero and one as $x$ goes out toward
infinity:
\begin{gather}
\lim_{x \to -\infty} F_X(x) = 0 \\
\lim_{x \to \infty} F_X(x) = 1
\end{gather}

% need a figure here with, say, normal distribution

\subsection{Probability distribution function}

For a discrete random variable, it's straightforward to define its
\emph{probability mass function} or ``pmf'', which is just the probability
that it takes on each discrete value $x_i$:
\begin{gather*}
f_X : \{x_1, \ldots, x_k\} \to [0, 1] \\
f_X(x_i) = \prob{X = x_i}
\end{gather*}

The analogue for continuous random variables is the probability
\emph{distribution} function, also called probability \emph{density} function
or ``pdf'':
\begin{gather*}
f_X(x) : \mathbb{R} \to [0, 1] \\
f_X(x) \defeq \frac{d}{dx} F_X(x)
\end{gather*}

Note that the pdf of a continuous random variable is not $\prob{X = x}$. This
may seem like a pedantic diversion, but I actually think it's important to
avoid confusion. For a continuous random variable, $\prob{X = x}$ is zero. For
example, say $X$ takes on values between 0 and 1 symmetrically, e.g., the
probability that $X$ comes out between 0 and 0.5 is the same as between 0.5
and 1. Say you're asking about the probability that $X$ comes out as 0.5. I'll
grant the following:
\begin{align*}
\prob{0 \leq X \leq 1} &= 1 \\
\prob{0.495 \leq X \leq 0.505} &= 0.1 \\
\prob{0.4995 \leq X \leq 0.5005} &= 0.01 \\
&\vdots
\end{align*}
and so on. We normally don't say that $\prob{X = 0.500\ldots} = 0$, since that
makes it sound like it's impossible for $X$ to take on the value $0.5$.
Nevertheless, it should be clear that, from any practical point of view, it
doesn't make sense to write things like $\prob{X = 0.5}$. It does make sense
to write $f_X(0.5)$; it does not make sense to write $\prob{X =
0.5}$.\footnote{Those dissatisfied with this explanation will need to learn
some measure theory.}

The word ``density'' in probability density function emphasizes that the pdf,
when integrated, gives probabilities:
\begin{equation}\label{eq:integrated_pdf}
\int_{x_0}^{x_1} f_X(x') \,dx' = F_X(x_1) - F_X(x_0) = \prob{x_0 < X \leq x_1}
\end{equation}
If $X$ can take on any value, then it follows from the definition of the cdf and
some fundamental calculus that
\begin{equation}
\int_{-\infty}^\infty f_X(x') \,dx' = 1
\end{equation}
that is, that all the probability must be somewhere, and
\begin{equation}
\lim_{x \to \pm \infty} = 0.
\end{equation}

These definitions I've given are simple ones, and they need to be refined to
deal with more complicated functions. For example, if the cdf has
discontinuities, you need careful with the limits on the integrals.

\subsection{Known distributions}

If a random variable $X$ has the same cdf as another random variable $Y$ (and
therefore also the same pdf), we say that $X$ is ``distributed like'' $Y$:
$$
X \sim Y.
$$
This notation is used to refer to well-documented distributions. For example,
you might say that $X$ is distributed like a normal random variable with mean
$\mu$ and variance $\sigma^2$:
$$
X \sim \mathcal{N}(x; \mu, \sigma^2),
$$
which is a shorthand for saying
\begin{equation*}
f_X(x) = f_\mathcal{N}(x; \mu, \sigma^2) \defeq
  \frac{1}{\sqrt{2\pi\sigma^2}} \exp \left\{-\frac{1}{2} \frac{(x-\mu)^2}{\sigma^2} \right\}.
\end{equation*}
Similarly $X \sim \mathrm{Binom}(n, p)$ means that $X$ follows a binomial distribution:
\begin{equation*}
f_X(x) = f_\mathrm{binom}(x; n, p) \defeq \binom{n}{x} x^p (n-x)^{1-p}.
\end{equation*}
We'll come back to these distributions and their pdf's later; I just want to
you understand that, when we say that ``$X$ is normally distributed'', we
making a mathematically precise statement about $X$'s pdf and cdf.

\subsection{Independent, identically-distributed random variables}

A common construct is to consider \emph{independent, identically-distributed}
(or iid) random variables. This means you have a collection of random variables $X_i$
that are all independent of one another but follow the same distribution. This
might be written as
\begin{equation*}
X \stackrel{\text{iid}}{\sim} \mathcal{N}(0, 1)
\end{equation*}
to mean that $f_{X_i}(x)$ is the standard normal pdf for every random variable
$X_i$ and that all the $X_i$ and $X_j$ are independent for all $i \neq j$.
For example, in my card drawing example, if I draw one card, shuffle the deck,
and draw another card, any random variable associated with the cards in the
two draws will be independent and identically distributed.\footnote{I hope
it's clear that ``independent'' and ``identically-distributed'' are not
contradictory: the only knowledge you could potentially get from the outcomes
$X_1, X_2, \ldots X_n$ that would tell you something about the as-of-yet
hidden value of $X_{n+1}$ is what distribution the $X_i$ are drawn from. This,
however, is not a secrete when you say that ``the $X_i$ are iid standard
normal variables''. Later on we will definitely get into how you use the
values taken on by $X_i$ to try to guess what distribution they come. That's
the point of inferential statistics!}

If some random variables are independent and indentically distributed, then
their histogram approximates the probability distribution function. To see why
this is, remember that, at least in the frequentist scheme, the proportion of
``trials'', which we can now more concretely associate with the individual
random variables in this group of iid random variables, that fall into some
range $[a, b]$ is the probability of \emph{each} of the identically-distributed
random variables taking on values in that range, which is $\prob{a
< X \leq b} = \int_a^b f_X(x') \,dx'$. When we talk about ``drawing from a
normal distribution'', what we're talking about in a rigorous mathematical
sense is examining a collection of iid variables with that distribution.

\section{Sums and products of random variables}

If you know the cdf and pdf for a random variable $X$, and you also know these
values for another random variable $Y$, you might be interested in the
behavior of these variables together.\footnote{An important example is the sum
of many independent, identically-distributed random variables, which we will
see are intimately linked to the normal distribution.} Many specific
distributions have nice shortcuts for figuring out the behavior of sums of
random variables with those distributions.\footnote{For example, if $X \sim
\mathcal{N}(\mu_1, \sigma_1^2)$ and $Y \sim \mathcal{N}(\mu_2, \sigma_2^2)$,
then $X + Y \sim \mathcal{N}(\mu_1 + \mu_2, \sigma_1^2 + \sigma_2^2)$.} Where
do those nice rules come from?

First, let's be clear about the notation. If $X$ and $Y$ are random variables,
then what does $Z = X + Y$ mean? Remember that $X$ and $Y$ are both functions,
so the plus sign in $X + Y$ doesn't mean the same thing as the plus sign in $1
+ 2$. Instead, it's an intuitive shorthand:
\begin{equation}
Z = X + Y \implies Z[\omega] = X[\omega] + Y[\omega],
\end{equation}
that is, for every outcome $\omega$, $Z$ takes on the sum of the values that
$X$ and $Y$ take on. More pedantically, $Z$ is a function that, for every input
$\omega$, outputs the sum of the outputs of $X$ and $Y$ when they in turns are
fed the input $\omega$.

This means that, for some event, if $X$ takes on the value $x$ and $Y$ takes
on the value $y$, then $Z$ takes on $x+y$. Reversing that, we can say that, if
$Z$ takes on the value $z$ and $X$ takes on $x$, then it must be that $Y$
takes on $z - x$. This is how we write the pmf for a discrete $Z = X + Y$ when
$X$ and $Y$ are independent:
\begin{align*}
Z = X + Y \implies f_Z(z) &= \sum_i \prob{(X=x_i) \cap (Y=z - x_i)} \\
  &= \sum_i \prob{X=x_i} \prob{Y=z-x_i} \text{ (independence)} \\
  &= \sum_i f_X(x_i) f_Y(z - x_i)
\end{align*}
Note that $f_Y(z-x_i)$ could be zero for many cases. For example, if $X$ and
$Y$ are two die rolls, then $f_Z(5)$ if $X$ was 6 then $Y$ would have to be $5
- 6 = -1$, which is clearly impossible.

A similar definition holds for when $Z = X + Y$ is continuous. Rather than
summing over all the outcomes in which $X$ takes on certain values, we integrate:
\begin{equation}
f_Z(z) = \int f_X(x') f_Y(z - x') \,dx'
\end{equation}
The right way to derive this is to start with the cdf for $Z$, but it's pretty
boring algebraic manipulation, so you can trust that it works out.

The process is similar for a product of random variables $Z = XY$, expect
that, if $Z = z$ and $X = x$, then it must be that $Y = z/x$. You can also
take a ratio of random variables $Z = X/Y$, which will have terms where $Y =
zx$. The math behind all these can be a real pain, and cases in which there is
a nice syllogism like ``if $X$ is whatever distributed and $Y$ is whatever
distributed, then $Z = X + Y$ is whatever distributed'' are glorious
exceptions.

I want you to see that, even if the math showing some fact about the sum,
product, or ratio of random variables is hard---for example, the ratio of two
normally distributed variables is \emph{not} normally distributed---there's
nothing mysterious about where the fact comes from, or how you could derive
your own facts like that yourself, if you had the time and algebraic ability.

\section{Properties of random variables}

Two properties of random variables, their \emph{expected values} and
\emph{variances}, will come up repeatedly.

\subsection{Expected value (but don't always expect it)}

A random variable's \emph{expected value} is a sort of ``average'' value of
the random variable. For a discrete variable, you find the weighted average of
the values the random variable takes on, where the weights are the probability
that that value comes up. For a continuous random variable, you need to
integrate:
\begin{equation}
\expect{X} = \begin{dcases}
  \sum_i x_i f_X(x_i) & \text{ for discrete random variables} \\
  \int x' f_X(x') \,dx' & \text{ for continuous random variables}
\end{dcases}
\end{equation}
where the $i$ refer to the different values taken by $X$ and the limits of the
integral are the limits of the values taken on by $X$. You might also see
people use notation like $E[X]$ or $\mathbb{E}X$ or, for reasons that will
become confusing later, $\overline{X}$.

The name ``expected value'' is a little confusing. We previously noted that,
for continuous random variables, the probability of getting any particular
number is essentially zero, so there's no particular number you should
``expect''. For discrete random variables, where there is finite probability
of getting each particular number, the expected value need not be any of the
values actually taken on by $X$. In the example of rolling a die, each of the
numbers 1 through 6 arises with equal probability, so:
\begin{equation}
\expect{X} = \sum_{k=1}^6 k f_X(k) = \sum_{k=1}^6 k \times \frac{1}{6} = 3.5.
\end{equation}
You certainly don't ever expect to roll a 3.5!

What the expected value \emph{is} is a handy mathematical tool and a measure
of ``central tendency'', a rough of measure of what kind of values $X$ is
``centered'' around, although even this is slippery, since, as you might
recall from learning in high school about the difference between mean, mode, and median, what ``the
center'' means is not necessarily obvious. It's worth noting, though, that if
the values $x_i$ taken on by a random variable $X$ are all equiprobable, then
the expected value of $X$ is just the plain-old arithmetic mean of the $x_i$:
\begin{equation}
f_X(x_i) = f_X(x_j) \text{ for all } i, j \implies \expect{X} = \frac{1}{n} \sum_i x_i.
\end{equation}

\hl{proof of linearity of expectation}

Expected values have some nice properties that make them really easy to deal
with. For example, the expected value of a shifted random variable like $X +
1$ is just $\expect{X} + 1$: you're still ``averaging'' $X$ overall the
outomes, but you're just adding 1 to everything you averaged, which is clearly
just the ``average'' $X$ plus 1. Similarly, you should be able to see that
$\expect{2X} = 2 \,\expect{X}$.

You could think of ``1'' in the previous example as being a really boring
random variable: it gives 1 for every outcome. This might lead you to suspect
that $\expect{X + Y} = \expect{X} + \expect{Y}$, which is true. The fact that
you can easily transform the expected values of random variables multiplied by
numbers and the sum of random variables is called the \emph{linearity of expectation}:
\begin{equation}
\expect{aX + bY} = a \,\expect{X} + b \,\expect{Y}.
\end{equation}

Perhaps surprisingly, this property holds whether $X$ and $Y$ are independent
or not. Here's the proof: We previously showed that if $Z = X + Y$, then
\begin{equation}
f_Z(z) = \sum_i f_X(x_i) f_Y(z - x_i).
\end{equation}
With some re-arranging, you can show that this is equivalent to
\begin{equation}
f_Z(z) = \sum_i \sum_j (x_i + y_j) f_X(x_i) f_Y(y_j).
\end{equation}
Split this up into two sums:
\begin{align*}
f_Z(z) &= \sum_i \sum_j x_i f_X(x_i) f_Y(y_j) + \sum_i \sum_j y_j f_X(x_i) f_Y(y_j) \\
  &= \left(\sum_i x_i f_X(x_i)\right) \left(\sum_j f_Y(y_j)\right) +
    \left(\sum_j y_j f_Y(y_j)\right) \left(\sum_i f_X(x_i)\right).
\end{align*}
You'll recall that summing up the pmf gives 1, that is, $\sum_i x_i f_X(x_i) =
1$, so the second and fourth bits are just 1. Then notice that the first bit is
just $\expect{X}$ and the third is just $\expect{Y}$.

\hl{Another way to think about this is to say that, even if $X$ and $Y$ depend on
each other,}

\footnote{Here's a problem that sounds really hard, but becomes
really easy with expectation: there are $n$ chickens, standing in a circle.
Suddenly, every chicken pecks the chicken to its left or to its right with
equal probability. How many chickens got pecked?---If you interpret ``how many''
as expected value, then notice that the total number of pecked chickens is the
sum of iid random variables, one for each chicken, with 0 indicating not pecked
and 1 indicating pecked. To be not pecked, the chicken to the left must peck
left, and the chicken to the right must peck right, each with probability
$1/2$, so the expected value for each chicken is $1/4$. The
expected number of pecked chickens is $n/4$. Computing the
\emph{variance} in the number of pecked chickens is definitely harder! \hl{cite}}

Interestingly, the \emph{product} of two random variables is nice only if the
variables are independent. In that case, $\expect{XY} = \expect{X}\,\expect{Y}$.
More generally, there is some residual term call the \emph{covariance}:
\begin{equation}
\expect{XY} = \expect{X} \, \expect{Y} + \cov{X, Y}.
\end{equation}
Before getting too much into covariance, we should start with regular variance!

\subsection{Variance}

If the expected value is some measure of position, then \emph{variance} is the
corresponding measurement of spread. For a random variable $X$, the variance
is the expected value of a random variable $(X - \expect{X})^2$.\footnote{We
learned about how to take the product of random variables, so this is totally
kosher: we start with $X$, then make a new random variable $X - \expect{X}$,
and then look at a third random variable $(X - \expect{X})^2$.} In words, this
is the expectation value (i.e., some kind of sense of position or magnitude)
or the square of the deviations of $X$ from \emph{its} expected value. I'll
write variance with a weird V:
\begin{equation}
\var{X} \defeq \expect{(X - \expect{X})^2}.
\end{equation}
You might also see notation like $\mathrm{Var}[X]$ or $V(X)$, or, for reasons
that will become confusing later on, $\sigma^2_X$. Again, I use the square brackets
to emphasize that $\mathbb{V}$ is not a function over numbers: it takes random
variables and gives back a number.

Note that $\var{X} \geq 0$, since it's an expected value over squares, which
are nonnegative.\footnote{The uninteresting case $\var{X} = 0$ means that $X$
always takes on the same value---that is, that it's a constant.}

Unlike the expected value, variance is not linear. Consider the random variable $aX$,
where $a$ is just a number:
\begin{equation}
\var{aX} = \expect{(aX - \expect{aX})^2} = \expect{a^2(X - \expect{X})} = a^2 \expect{X}.
\end{equation}
For a sum, the situation is even more confusing. First I'll note that
\begin{equation}
\var{X} = \expect{X^2 - 2X \expect{X} - \expect{X}^2}
  = \expect{X^2} - \expect{X}^2.
\end{equation}
In words, the variance is the expectation of the square minus the square of
the expectation. I did this so I could more easily expand out:
\begin{align*}
\var{X + Y} &= \expect{(X+Y)^2} - \expect{X+Y}^2 \\
  &= \expect{X^2} + \expect{Y^2} + 2\, \expect{XY} - \left(\expect{X}+\expect{Y}\right)^2 \\
  &= \left(\expect{X^2} - \expect{X}^2\right) +
    \left(\expect{Y^2} - \expect{Y}^2\right) +
    2 \left(\expect{XY} - \expect{X} \, \expect{Y}\right) \\
  &= \var{X} + \var{Y} + 2 \cov{X, Y}
\end{align*}
where $\cov{X, Y}$, introduced in the end of the last section, is the
\emph{covariance} of $X$ and $Y$. I think it's way easier to write it out as
\begin{equation}
\cov{X, Y} \defeq \expect{(X - \expect{X})(Y - \expect{Y})}.
\end{equation}
In other words, the covariance is the expected value of the product of the
deviations of $X$ and $Y$ from their expected values.

\subsection{Correlation}

Rather than covariance, you're probably more
familiar with the term \emph{correlation}, especially the familiar Pearson's
correlation coefficient, usually written $\rho$, which is
\begin{equation}
\rho[X, Y] \defeq \frac{\cov{X, Y}}{\sqrt{\var{X} \var{Y}}}.
\end{equation}
I want to emphasize that we're still talking about random variables, which are
functions, and not numbers, the ``realizations'' of random variables. So, as I
wrote it, $\rho$ is a \emph{function} that takes two random variables, which
are themselves functions, and returns a number.\footnote{This is different
from the correlation coefficient as you're probably used to it, which is
computed as a number from a set of numbers. If you find this curious, don't
worry, that's what the next chapter is about!}

To see how covariance and correlation are related, think about a correlated
$X$ and $Y$, like a simple bivariate normal distribution. \hl{figure} In
general, the points for which $X$ is greater than its mean value are the same
points for which $Y$ is greater than its mean value. Similarly, when one
deviation is negative, the other tends to be negative. This way, the overall
\emph{product} of the two deviations tends to be positive. In constrast, if
two variables are \emph{anticorrelated}, then when one is higher than average
the other tends to be lower than average, so the product of deviations tends
to be negative.

The reason for dividing by the two variances was so that $\rho$ is between
$-1$ and 1. To see that that's true, define the ``standardized'' random
variable \begin{equation} X' = \frac{X - \expect{X}}{\sqrt{\var{X}}}
\end{equation} which has has mean 0 and variance 1. Similarly define $Y'$, and
then check that $\rho[X, Y] = \rho[X', Y'] = \expect{X'Y'}$. Now make another
random variable $Z = X' - \rho Y'$.\footnote{If you're familiar with linear
regression, notice that $Z$ is like the residuals that come from predicting
$X'$ using independent variable $Y'$ and slope $\rho$: $\rho Y'$ are the
predicted values. The fact that $\var{Z} = 1 - \rho^2$ is why sometimes people
call $\rho^2$, or $R^2$, the ``fraction of variance explained'' by the
regression. If $\rho$ is 1, then $Z$, the residuals, are a constant zero:
you've ``explained'' all the variation.} Its variance is
\begin{align*}
\var{Z} &= \expect{(X' - \rho Y')^2} \\
  &= \expect{X'^2 - 2 \rho X' Y' + \rho^2 Y'^2} \\
  &= \expect{X'^2} - 2\rho\,\expect{X'Y'} + \rho^2 \expect{Y'^2} \\
  &= 1 - 2\rho^2 + \rho^2 \\
  &= 1 - \rho^2
\end{align*}
Variances are always nonnegative, so $1 - \rho^2 \geq 0$, from which you can see
that $-1 \leq \rho \leq 1$.\footnote{This proof is a specific case of the
Cauchy-Schwartz inequality.}


\part{Descriptive statistics}

\chapter{What is statistics?}

``Statistics'' is confusing because it means two things. It's a plural noun that
refers to multiple things, each of which is called a ``statistic''. It's also a
singular noun that refers to the study of these mathematical objects.

Statistics is a tool.\footnote{I really enjoyed reading this anecdote from a
paper by Gene Glass, the inventor of meta-analysis (doi:10.1002/jrsm.1133).
Glass. He ends up sitting on the plane next to the famous statistician
Tukey. Tukey asks, "Along which dimension do you see the greatest
variability in effects?" By investigator. "Then jackknife on investigator."
The point is that there's a wide gap between being able to understand what
the jackknife is in a mathematical sense (i.e., how to compute it, whether
it's an unbiased way to determine the variance in the median) and how to apply
it. It also goes to show that "real" statisticians can be pretty hacky!}

A \emph{statistic} is some function of the sample data. Although the field of
statistics is often described as consisting of "descriptive statistics" and
"inferential statistics"---that is, the study of mathematical objects used to
describe data and to make inferences about the populations the data were taken
from---in this book I want to emphasize that, for both purposes, we are
interested in the properties of statistics.

The most interesting statistics are the ones that are used to estimate some
property of the population. These kinds of statistics are sensibly called
\emph{estimators}. Most of the study of statistics comes down to figuring out things
about estimators. One of the most critical is understanding the variance of
estimators, which is essential for both descriptive statistics---so that you
can put error bars on your measurements---and inferential statistics---so you
can make a guess about how probable it is that your data arose under some null
hypothesis. All of this is about estimators and their variance.
%!TEX root=main

\chapter{Parameters and estimators}

Parameter is a function of the distribution and gives a number. Estimator is a
function of data and gives a number and is intended to estimate the parameter.

So if there are more and more $X_i$ with iid $F_X$, then the idea is that the estimator of the $\hat{\theta}(\vec{x})$ should approach $\theta(F_X)$.

There's a ``sample'' probability distribution $\hat{f}_X$. It's hard to say whether this is a maximum likelihood estimator or something. What does ``sample'' mean? Like ``sample mean''?
%!TEX root=main

\chapter{Confidence intervals}

\section{Definition}

So far we have worked with \emph{point estimates}, that is, guesses for
population parameters that are just a single number. This presents a problem,
because statistics and probability are all about uncertainty, and we know that
the point estimate we make for, say, $B$ in the German tank problem will likely
never be exactly correct, even if the estimator is consistent, unbiased, and
efficient.

In the frequentist framework, the approach for expressing uncertainty is using
\emph{confidence intervals}. A confidence interval for a parameter $\theta$ with estimator $\hat{\theta}$
at \emph{level} $1-\alpha$ is a pair of estimators $\hat{\theta}_-$ and $\hat{\theta}_+$
defined such that, for any parameter value $\theta$, it holds that:
\begin{equation}
  \prob{\hat{\theta}_- \leq \theta \leq \hat{\theta}_+} \geq 1 - \alpha.
\end{equation}
A typical value for $\alpha$ is 5\%, or $0.05$, which yields 95\% confidence intervals.
A higher $\alpha$ corresponds to a lower probability of error, and thus wider confidence
intervals, for which we are more sure that the interval actually contains the true value.
Thus a 99\% confidence interval is wide and a 90\% confidence interval is narrow.

This equation is typically read as saying ``the probability that the value of
the unknown parameter $\theta$ is between $\hat{\theta}_-$ and $\hat{\theta}_+$ is
95\%''. The typical conceptualization is then to imagine that $\theta$ is varying,
bopping around inside the confidence interval set by $\hat{\theta}_-$ and
$\hat{\theta}_+$. In fact, however, it is the intervals that are the random variables,
and the parameter is the constant fixed point around which those random variables
are bopping.

Note that, in the frequentist approach, we develop two statistics ---functions
of the observed data--- such that their corresponding estimators will, with some
probability, enclose the true value, for every possible true value. Strictly speaking,
you must select a method for constructing confidence intervals such that, for any
parameter $\theta$ I choose, the proportion of infinitely many repeated trials will
produce realizations of the confidence interval that enclose $\theta$.

\section{Meaning and interpretation}

If this introduction has been a bit befuddling, then you are in good company.
Confidence intervals are a very new concept. They were first introduced in the statistical literature in 1937 and become commonplace in scientific work only in the later 20th century.
It is perhaps no surprise, given its youth, that the concept of the confidence interval
is very confusing.

The most common misconception about confidence intervals is that they represent
\emph{confidence}. (One might argue that ``confidence'' was a poor word to use to name this
thing!) In this misconception, we can have 95\% confidence that the true parameter $\theta$
lies inside the confidence interval constructed after the experimental data has been gathered.
Strictly speaking, this is incorrect. In a frequentist framework, the true
value is either inside the realized confidence interval, or it is not; our ignorance of
the true value has nothing to do with probability. In fact, the interval that encloses
the true value with some ``confidence'' is a Bayesian concept, the \emph{credible interval}.
``Confidence'' is part of the Bayesian definition of probability; it is foreign to the
frequentist construction.

To a practicing scientist, this distinction might appear entirely semantic. But if anyone
tells you there is a 95\% probability that the true value of a parameter falls within
some fairly specific range, you should be very skeptical. Would you, for example, bet
twenty-to-one odds that they were correct? I would not, not because they had miscomputed
their statistics, but because their experimental approach is very likely imperfect.

\section{Constructing intervals}

Regardless of what confidence intervals really mean, they are useful for
producing some quantification of uncertainty.

To show how confidence intervals are constructed, consider the German
tank problem again. Our unbiased point estimate is $\hat{B} = \tfrac{n+1}{n} \max_i X_i$.
We will design a \emph{symmetric} confidence interval, which means that:
\begin{equation*}
    \prob{\hat{B}_- \leq B} = \prob{\hat{B}_+ \geq B} = 1 - \frac{\alpha}{2}
\end{equation*}
This confidence interval is symmetric because the probability that the confidence interval
will not include $B$ because it is too low is equal to the probability that it will not
include $B$ because it is too high. For a 95\% confidence interval, the probability of the
confidence interval being wrong on either side is $2.5\%$, for a 5\% total probability of
error.

To construct $\hat{B}_-$ and $\hat{B}_+$, we will find the range of values in which the point
estimator $\hat{B}$ is to expected fall, given $B$. We found that the cdf for the biased
estimator $\max_i X_i$ was $(x/B)^n$, so the cdf for the
unbiased point estimator is:
\begin{equation*}
    \prob{\hat{B} \leq x} = \frac{n+1}{n} \left(\frac{x}{B}\right)^n
\end{equation*}
First, we use this cdf to find the value of $x$ such that $\hat{B}$ will fall below it
with probability $1-\frac{\alpha}{2}$:
\begin{equation*}
    \prob{\hat{B} \leq x} = 1 - \frac{\alpha}{2} \implies x = \left( \frac{\alpha}{2} \frac{n}{n+1}\right)^{1/n} B
\end{equation*}
Next, we ``invert'' this relationship:
\begin{equation*}
    1 - \frac{\alpha}{2}
    = \prob{\hat{B} \leq \left( \frac{\alpha}{2} \frac{n}{n+1}\right)^{1/n} B}
    = \prob{\left( \frac{\alpha}{2} \frac{n}{n+1}\right)^{-1/n} \hat{B} \leq B}
\end{equation*}
And thus, we have found our lower confidence interval:
\begin{equation*}
    \hat{B}_- = \left( \frac{\alpha}{2} \frac{n}{n+1}\right)^{-1/n} \hat{B}
\end{equation*}
A similar exercise shows that:
\begin{equation*}
    \hat{B}_+ = \left[ \left(1- \frac{\alpha}{2}\right) \frac{n}{n+1}\right]^{-1/n} \hat{B}
\end{equation*}

As an example, for $n=5$, $\max x_i = 1$, and $\alpha = 5\%$, we have $\hat{B} = 1.2$, $\hat{B}_- = 1.04$ and $\hat{B}_+ = 2.17$. For $n=100$ and the same observed maximum $1$,
we have $\hat{B} = 1.01$,
$\hat{B}_- = 1.0004$, and $\hat{B}_+ = 1.04$. In both these cases, we do not say what $B$
actually is, so we cannot say whether or not the confidence intervals contain the true value
or not. We only know that, regardless of what $B$ is, there is a 95\% probability that the
resulting data will produce, according to our definitions, values of $\hat{B}_-$ and
$\hat{B}_+$ that contain $B$.

\section{Properties of confidence intervals}

We could have been much lazier with our confidence intervals. For example, I could have
defined $\hat{B}_- = 0$. Because we know \textit{a priori} that $B>0$, this lower end
of the confidence interval will always be correct. In fact, it means that the confidence
intervals will be contain $B$ with a probability greater than $1-\alpha$. This is an
undesirable feature. Confidence intervals with the desirable property that they have a
error probability of exactly $\alpha$ and no more are termed \emph{valid} confidence intervals.

You might imagine that we could have constructed different confidence intervals. For example,
if we relaxed the symmetry requirement, then we might be able to produce confidence intervals
that are overall narrower. For certain situations, there is a well-acknowledged single best
method for constructing a confidence interval. In other situations, there are multiple very
reasonable ways to construct confidence intervals. For example, for the binomial distribution,
there are pros and cons to the Wilson method and the Clopper-Pearson methods for computing
confidence intervals.

\section{Resamplign methods}

One might object that the approach this so far has been theoretically pleasant
but practically nearly useless. As a scientist collecting data, one is often
unsure what distribution the data will follow. More often, one might be
confident that the data do not follow any well-studied statistical distribution
and so cannot count on there being an already-derived methodology for computing
confidence intervals. In these situations, we turn to resampling methods like the
\emph{bootstrap} or the \emph{jackknife}.

\subsection{Bootstrap}



\subsection{Jackknife}

\section{Profile method}


\part{Inferential statistics}

%!TEX root=main.tex

\chapter{Inferential statistics}

inference vs. decision-making under uncertainty
wilk's theorem
%!TEX root=main.tex

\chapter{Frequentist inferential statistics: statistical tests and confidence intervals}

The previous chapter laid out the conceptual points of inference. In this chapter, I lay out the nuts and bolts of what most people think statistics is: tests.

The mathematics of statistical tests follows directly from the previous mathematical chapters: we define some \emph{estimators}, functions of the data that predict some property of the data, and determine their variance. We just give things different names and interpret them differently.

But rather than begin with the mathematics of tests, I'll start with the concepts of how they work, and show how to implement them algorithmically and computationally. Once you understand what the test is doing, we can get back to unpacking some of the math you will hear bandied about around them.

\section{The ingredients in a statistical test}

To run a statistical test, you need:

\begin{enumerate}
\item Observed data.
\item A listing of the universe of observations that could have been made, of which the actual observation is one instance.
\item A \emph{null hypothesis} that associates a probability with each possible observation.
\item A \emph{test statistic} that summarizes how interesting the observed data are in a single number.
\end{enumerate}

Every statistical test that you know of can be defined by how it selects its
universe of possible observations, its null hypothesis, and its test statistic.

The test will give you two outputs:

\begin{enumerate}
\item A $p$-value, which is the probability of observing data as extreme (or ``interesting'') as what you did observe, given the null hypothesis.
\item Confidence intervals on the test statistic.
\end{enumerate}

More concretely, the $p$-value is the proportion of the observations in the universe of observations, weighted by their probabilities, that have test statistics that are more extreme (i.e., either larger or smaller) than the test statistic for the observed data. The confidence interval says that, using the observed data as our best estimates of the underlying truth, what range of test statistics would we expect?

\section{Interpreting $p$-values and confidence intervals}

The $p$-value was developed by Fisher, probably the one person who had the most influence on modern statistics. He used the $p$-value, a rigorously-derived number, to reason informally about data. If the $p$-value too big, by some standard to be determined \textit{ad hoc}, then the data didn't show anything interesting. In different books and papers, for different data sets, Fisher used different standards for ``too big'', but $0.05$ was a recurring choice.

Later, Neyman and Pearson (the younger) reformulated Fisher's method as a kind
of decision-making under uncertainty. If you were a decision-maker, and you had
to make a decision, what would you do when faced with certain data? They
suggested you pick some specific \emph{confidence level}, written $\alpha$, and
make your decision based on whether $p$ was greater than or less than $\alpha$.

Although Fisher furiously disagreed with Neyman and Pearson about how to interpret $p$-values, the canonical $\alpha = 0.05$ somehow made it into the scientific literature.

As laid out in the previous chapter, ``what is the probability that this data is true, given the data?'' is a very tricky question. The critical point is that the $p$-value is, in the language of that chapter, $\prob{X | \theta}$, but we want to know about $\prob{\theta | X}$. The link between these is the mysterious prior probability $\prob{\theta}$.

I think my only take-away is that you should trust your gut more than a $p$-value. First, Fisher developed statistics as a way to rigorously show that data \emph{weren't} interesting, a mathematical counterbalance to humans' ability to find interesting patterns in data, especially their own. He meant $p$-values as a way to deflate puffed-up conclusions, not to puff them up. Second, Fisher, who invented $p$-values, couldn't really agree with other extremely intelligent people about exactly what they meant. So don't make yourself crazy trying to figure it out too precisely either! Just know that big $p$-values, where ``big'' depends on the intellectual context, mean the data aren't interesting, where ``interesting'' is defined by the null hypothesis.

Confidence intervals were also controversial when first introduced, and they are also philosophically confusing. To say that the 95\% confidence interval on $X$ is from here to there means that, if the test statistic observed in this experiment were the true one (which it almost certainly is not), then if you repeated the experiment many times, in what interval would the measured test statistics tend to fall? This is \emph{not} the same as saying that there is a 95\% chance that the true value is in that interval, although exactly what it \emph{is} saying is confusing.

\section{The simplest test: checking for a fair coin}

To understand how to go from a test's ingredients to its outputs, I will work
through many examples of increasing complexity. The first and simplest is this: I flip a coin $n$ times, seeing some sequence of heads and tails, and I want to infer whether the coin is fair.

\begin{itemize}
\item The \emph{observed data} is the sequence of heads and tails I flipped.
\item The universe of possible observations is all the sequences of heads and tails I could have flipped. So if $n=2$, the universe is the four cases $HH, HT, TH, TT$.
\item The \emph{null hypothesis} is that heads and tails are equally likely.
\item The \emph{test statistic} is the number of heads I flipped. This statistic, like all the other statistics discussed in chapter XX, is a function of the data: given some sequence of heads and tails, it just counts up the number of heads.
\end{itemize}

\subsection{One- and two-sided $p$-values}

The $p$-value is the probability of data as extreme as what we observed, or
more extreme, given the null hypothesis that $p = 0.5$.\footnote{I apologize
that $p$ means the expected proportion of heads and $p$-value means something
totally different.} The tricky part of this definition is understand what ``as
extreme or more'' means.

``Extremeness'' is quantified by the test statistic, which summarizes each of
the possible observations ---each length-$n$ sequence of heads and tails---
into a single number, the number of heads in that sequence. ``As extreme or
more'' can mean 3 things:
\begin{itemize}
    \item as small as what was observed, or smaller,
    \item as large as what was observed, or larger, or
    \item as far away from the ``center'' as was observed, or further away.
\end{itemize}

The first two cases call for \emph{one-sided} $p$-values and can be used if you
have an \textit{a priori} hypothesis that the number of heads that you will
observe will be much higher or lower than what you would expect if the coin
were fair. So if I want to seek support for my \textit{a priori} hypothesis
that the coin is more likely to land tails than heads, then I would use that
one-sided $p$-value. If I don't have an \textit{a priori} expectation about
whether the flips will be enriched for heads or tails ---if I just think that
the coin is not fair, but I don't know which way--- then I need a more
conservative, \emph{two-sided} $p$-value.

In this example, the null hypothesis of a fair coin means that all the possible
sequences of heads and tails are equally likely, so to compute a one-sided
$p$-value, I just need to count up the number of sequences that are more
extreme than the observed one:
\begin{equation}
    \text{one-sided $p$-value} = \frac{\text{number of possible sequences more extreme than observed}}{\text{total number of sequences}}
\end{equation}
You may know that the number of length-$n$ heads-tails sequences is $2^n$ and
that the number of those sequences that has $x$ heads is $\binom{n}{x} = n! /
\left[ x! (n-x)! \right]$. Thus, for this example, if we have a ``low-sided''
expectation about the number of heads:
\begin{equation}
    \text{``low-sided'' one-sided $p$-value} = \frac{\sum_{y=0}^x \binom{n}{x}}{2^n}
\end{equation}
A ``high-sided'' $p$-value would sum from $x$ up to $n$.

For the two-sided $p$-value, we need to sum up the ``low-side'' and ``high-side'' $p$-values. Because the observed $x$ falls to one side of the central expectation $\expect{x} = n/2$, we need to symmetrize to find the other value. For example if $x$ is low, we make another, mirrored ``high'' $x_\mathrm{hi} = n-x$:
\begin{equation}
    \text{two-sided $p$-value} = \frac{\sum_{y=0}^x \binom{n}{x} + \sum_{y=n-x}^n \binom{n}{x}}{2^n}
\end{equation}
If $x > n/2$, then the sums would go from $0$ to $n-x$ and from $x$ to $n$.

\section{Checking for a specified unfairness of a coin}

In the first example, we used a null hypothesis that heads and tails were equal, which simplifed thinking about $p$-values because we could simply count up the equiprobable sequences of heads and tails.

Now, we generalize slightly, allowing a null hypothesis that the probability $p$ of flipping heads is not exactly $\tfrac{1}{2}$:
\begin{itemize}
\item The \emph{observed data} is, again, the sequence of heads and tails I flipped.
\item The universe of possible observations is, again, all the sequences of heads and tails I could have flipped.
\item The \emph{null hypothesis} is that the probability of flipping heads is $p$.
\item The \emph{test statistic} is the number of heads I flipped.
\end{itemize}

The mathematically savvy will know that the probability of $y$ successful trials among $n$ total trials, with each trial having a probability $p$ of success, is the \emph{binomial distribution}:
\begin{equation}
    \mathrm{Binom}(y; n, p) = \binom{n}{y} p^y (1-p)^{n-y}
\end{equation}
This is just like the above, where we counted the number of sequences, only now
we need to weight them by the number of more (or less) probably heads (and
tails):
\begin{equation}
    \text{``low-sided'' one-sided $p$-value} = \sum_{y=0}^x \mathrm{Binom}(y; n, p)
\end{equation}

\subsection{Computing confidence intervals}

Recall that a \emph{confidence interval} is a method: it is a pair of
statistics, functions of the data, that will, in a certain proportion $\alpha$
of replicate experiments, will contain the true value, regardless of what the
true value is.

In this example, the ``true value'' in question is the unknown probability
$p_\mathrm{true}$ that the coin will flip heads, which would be anything
between 0 and 1.  So we need a method that, whether $p_\mathrm{true} = 10^{-6}$
or $p_\mathrm{true} = 0.999$, will produce an interval that contains the true
$p_\mathrm{true}$ in a proportion of $\alpha$ of the data that would arise in
replicates of that experiment.

As mentioned earlier, there are multiple methods that can achieve this result.
The one that doesn't rely on any kind of approximation, which we will address
later, is the \emph{Clopper-Pearson} interval. This says to find the largest
value $p_+$ for which $\sum_{y=0}^x \mathrm{Binom}(y; n, p_+) > \alpha/2$ and
the smallest value $p_-$ for which $\sum_{y=x}^n \mathrm{Binom}(y; n, p_-) >
\alpha/2$. In other words, if $\alpha=95\%$, pick an uppper limit on
$p_\mathrm{true}$ so that, if that were the true value, in exactly $2.5\%$ of
replicates of the experiment, you would get numbers of heads larger than what
you observed. Similarly, pick a lower limit so that, if that were the true
probability of flipping heads, in $2.5\%$ of replicates, you would get numbers
of heads larger than what you observed.

If this seems like a big, confusing headache, I hope you take it as a sign that
confidence intervals are confusing, not that you're not understanding!

\section{Approximating the binomial proportion test}

The elegance of the previous example is that it works by simply enumerating all
the possible sets of observations. The inelegant part is that there can be
many: $2^n$ grows quickly with $n$, making the sums in the $p$-value
calculations onerous. Furthermore, the math of the Clopper-Pearson interval is
a bit of a mess and hard to articulate.

In fact, rather than enumerating all $2^n$ possibilities, the typical approach
is to approximate the binomial distribution with the normal distribution. It turns out that, when $n$ is not too small, that:
\begin{equation}
    \mathrm{Binom}(x; n, p) \approx \mathcal{N}\left(x; \; \mu = np, \; \sigma^2 = \frac{p (1-p)}{n}\right),
\end{equation}
in other words, you can use a normal distribution with mean a mean and variance
computed from $n$ and $p$ to approximate the binomial distribution.

\textbf{FIGURE}

The binomial counting approach was intellectually elegant but mathematically
inelegant. The normal approximation approach is mathematically elegant but
intellectually confusing, as it makes the ingredients in the test very
artificial:
\begin{itemize}
    \item The observed data is not the sequence of heads and tails but rather simply $x$, which was previously the observed test statistics, that is, the number of heads flipped.
    \item The universe of possible observations is all values between $0$ and $n$. Of course, this isn't really true, as you can only observe integer $x$. But for the math to work out, you need to imagine that you could have observed $x = 1.02$, for example.
    \item The null hypothesis, as above, is some assertion about the probability $p$, only now $p$ is not the probability of any particular event, it's just a parameter in the normal distribution's mean and variance.
    \item The test statistic is just $x$, the observed number.
\end{itemize}

The other intellectually weird thing is that now, because the universe of possible observations is infinite, you cannot count up the states but must integrate over them:
\begin{equation}
    \text{``low-sided'' one-sided $p$-value} = \int_{y=0}^x \mathcal{N}\left(y; np, \frac{p(1-p)}{n} \right) \,\mathrm{d}y
\end{equation}
This is where the mathematical elegance comes in: the integral of the normal distribution, which is its cumulative distribution function, is oft-calculated and so easy to look up. So now rather than summing over something like $2^n$ states, we just look up a single number.

The confidence intervals become similarly simple: I look for a lower bound $p_-$ such that the probability that I would observe a value smaller than what I did observe is $\alpha/2$:
\begin{equation}
    \alpha/2 = \int_{y=0}^x \mathcal{N} \left(y; np_-, \frac{p_-(1-p_-)}{n} \right) \,\mathrm{d}y
\end{equation}

There are a few other approximations to the binomial distribution that are more
accurate, but they all come down to the same logic: they exchange the
intellectual simplicity of counting up over a finite universe of states for the
mathematical elegance of integrating over a well-known function.

\textbf{point out how this works with a z-test?}

\section{The data with different null hypotheses and statistics: paired tests}

So far we've had null hypotheses and test statistics that are obvious matches for the data that we're looking at. In the real world, when you're making up your own statistical tests, you'll find that you actually have multiple choices for null hypotheses and test statistics. To show how this can work, I'll analyze the same kind of data in a few ways.

Charles Darwin wanted to know if cross-fertilized corn (\textit{Zea mays}) grew better than self-fertilized corn. He took 15 pairs of plants, one self- and one cross-fertilized, each having germinated on the same day, and planted them together in one of 4 pots. He measured the heights of the plants, to one-eight of an inch, after some time (Table \ref{table:darwin}).

\begin{table}
\centering
\renewcommand{\arraystretch}{1.2}
\begin{tabular}{ccll}
\toprule
& & \multicolumn{2}{c}{Height (inches)} \\
\cmidrule{3-4}
Pair & Pot & Crossed & Self-fert. \\
\midrule
1 & 1 & $23 \tfrac{1}{2}$ & $17 \tfrac{3}{8}$ \\
2 & 1 & $12$ & $20 \tfrac{3}{8}$ \\
3 & 1 & $21$ & $20$ \\[1.0ex]
4 & 2 & $22$ & $20$ \\
5 & 2 & $19 \tfrac{1}{8}$ & $18 \tfrac{3}{8}$ \\
6 & 2 & $21 \tfrac{1}{2}$ & $18 \tfrac{5}{8}$ \\[1.0ex]
7 & 3 & $22 \tfrac{1}{8}$ & $18 \tfrac{5}{8}$ \\
8 & 3 & $20 \tfrac{3}{8}$ & $15 \tfrac{1}{4}$ \\
9 & 3 & $18 \tfrac{1}{4}$ & $16 \tfrac{1}{2}$ \\
10 & 3 & $21 \tfrac{5}{8}$ & $18$ \\
11 & 3 & $23 \tfrac{1}{4}$ & $16 \tfrac{1}{4}$ \\[1.0ex]
12 & 4 & $21$ & $18$ \\
13 & 4 & $22 \tfrac{1}{8}$ & $12 \tfrac{3}{4}$ \\
14 & 4 & $23$ & $15 \tfrac{1}{2}$ \\
15 & 4 & $12$ & $18$ \\
\bottomrule
\end{tabular}
\caption{Darwin's \textit{Zea mays} data}
\label{table:darwin}
\end{table}

The obvious question is: did the hybridized plants grow faster than the self-fertilized plants?

\subsection{A sign test}

This just reduces down to the binomial test above. Null hypothesis is that the median is zero.

\subsection{Wilcoxon's signed rank test}

Null hypothesis is zero median and symmetric distribution. Much more complex test statistic.

\subsection{Fisher's sum test}

Null is zero median and symmetric distribution. Simple test statistic.

\subsection{Paired $t$-test}

\subsection{Scott's}

Something about permuting within the pots? Bootstrapping?

\section{Parametric inference}

- z-test
- t-test
- Wilk's theorem? and likelihood-ratio tests
- F-tests in general, variance. Then go to ANOVA

\section{Non-parametric inference}

- Kolmogorov-Smirnov

\section{$t$-test}

\subsection{Equal variance}\label{equal-variance}

The (old school) \emph{t}-test is two sample, assuming equal variances.
We're interested in the difference in the means between the two
populations.

The null hypothesis is that we're drawing \(n_1 + n_2\) samples from a
population that has this equal variance, and that the labels on the two
``populations'' are just fictitious.

Our estimator \(s_p^2\) for the pooled variance is just the average of
the variances of the two ``populations'', weighted by \(n_i - 1\) (which
is a better estimator than weighting by just \(n_i\)): \[
s_p^2 = \frac{(n_1 - 1) s_1^2 + (n_2 - 1) s_2^2}{n_1 + n_2 - 2}.
\]

To see why this is, say that we're going to develop some pooled estimator
$\hat{\mathbb{V}}_{X,p}$, which is a constant times the sum of squares:
\begin{align}
\hat{\mathbb{V}}_{X,p} &= C \left[ \sum_i (X_{1i} - \overline{X}_1)^2 + \ldots \right] \\
  &= C \left[ \sigma^2 \sum_i (Z_{1i} - \overline{Z}_1^2 ) + \ldots \right] \\
  &= C \sigma^2 \left[ \sum_i Z_{1i}^2 - n \overline{Z}_1^2 + \ldots \right] \\
\expect{\hat{\mathbb{V}}_{X,p}} &= C \sigma^2 \left[ \sum_i \expect{Z_{1i}^2} - n \expect{\overline{Z}_1^2} + \ldots \right] \\
  &= C \sigma^2 \left( n_1 - 1 + n_2 - 1 \right) \\
\end{align}
which implies that
\begin{equation}
C = \frac{1}{n_1 + n_2 - 2}.
\end{equation}

To see the last bit, use this identity:
\begin{align}
\sum_i (Z_i - \overline{Z})^2 &= \sum_i (Z_i^2 - 2 Z_i \overline{Z} + \overline{Z}^2) \\
  &= \sum_i Z_i^2 - 2 n \overline{Z} \overline{Z} + n \overline{Z}^2 \\
  &= \sum_i Z_i^2 - n \overline{Z}^2.
\end{align}

The thing we're observing is the difference between the mean of \(n_1\)
samples from a (potentially ficitious) variable \(X_1\) and \(n_2\) from
\(X_2\): \[
\overline{X}_1 - \overline{X}_2 = \frac{1}{n_1} \sum_{i=1}^{n_1} X_{1i} - \frac{1}{n_2} \sum_{i=1}^{n_2} X_{2i}.
\] It would be nice if our statistic was distributed like
\(\mathcal{N}(0, 1)\), so we compute the variance of this observation:
\[
\begin{aligned}
\mathrm{Var}\left[ \overline{X}_1 - \overline{X}_2 \right]
  &= \frac{1}{n_1^2} \sum_i \mathrm{Var}[X_1] + \frac{1}{n_2^2} \sum_i \mathrm{Var}[X_2] \\
  &= \frac{1}{n_1^2} n_1 s_p^2 + \frac{1}{n_2^2} n_2 s_p^2 \\
  &= \left( \frac{1}{n_1} + \frac{1}{n_2} \right) s_p^2.
\end{aligned}
\]

So the statistic for this test is just the observation over its
variance: \[
t = \frac{\overline{X}_1 - \overline{X}_2}{s_p \sqrt{\frac{1}{n_1} + \frac{1}{n_2}}}.
\]

The confusing thing is that \(\overline{X}_1\), \(\overline{X}_2\), and
\(s_p\) are all random variables. We know how to take the sum (or
difference) of two random variables (i.e., how to figure out the
distribution of the numerator), but it's not immediately obvious how to
find the distribution of the whole thing, which has a different random
variable in its denominator.

\subsubsection{Computational approach}\label{computational-approach}

\begin{itemize}
\tightlist
\item
  Compute the observed \(t\) statistic
\item
  Compute the observed sizes, means, and standard deviations for the two
  sample populations
\item
  Many times, generate two sets of random variates. One set of variates
  is drawn from a normal distribution with the first sample mean and
  variance.
\item
  For each iteration, compute the simulated \(t\) statistic.
\item
  The empirical \(p\)-value is the fraction (not true! need \(r+1/n+1\))
  of simulated statistics that are greater than the observed statistic.
\end{itemize}

\subsection{Unequal variance (Welch's)}\label{unequal-variance-welchs}

This is the Behrens-Fisher problem. It stumped Fisher! He came up with a
weird statistic with a weird distribution (Behrens-Fisher), but it
didn't really stick, since he couldn't calculate confidence intervals
(?).

Instead, people went for the Welch-Satterthwaite equation, which
approximates the interesting distribution using a more handy one by
matching the first and second moments. (Maybe worth discussing those? Or
just say mean and variance?)

\section{anova}\label{anova}

Say you have some (equally-sized) groups. Each group was drawn from a
normal distribution (all with the same variance). Are the data
consistent with the model in which all those groups have the same mean?

The statistic is \(F \equiv \frac{\mathrm{MS}_B}{\mathrm{MS}_W}\), where
\(MS_B\) is the mean of the squares of the residuals(?) between the
group means and the grand mean (``between'') and \(MS_W\) is the mean of
the squares of the residuals between the data points and the group means
(``within'').

Again, focus on what's the population we're sampling from. It's easy to
think about a finite population, where you can just do all the possible
combinations and compare their \(F\) statistics. Then move on to say
that, if you believe that the particular variances and means you
measured are the exact, true distribution that you're sampling from, ask
what happens when you sample from that infinite population.

\subsection{z-test example}\label{z-test-example}

What's the nonparametric equivalent of this? It's just saying what the
empirical cdf is! Then say, if you really truly believe your mean and
standard deviation, then you can do that.

In other words, you say you are absolutely sure what population you are
drawing from. The same is actually true of the \emph{t}-test, except
that the \emph{z}-test is asking about a single value, distributed like
\(N(\mu, \sigma^2)\), while the \emph{t}-test is about the mean of the
\(n\) points, which is distributed like \(N(\mu, \sigma^2/n)\).

\section{Paired differences}\label{paired-differences}

\subsection{Historical example and
motivation}\label{historical-example-and-motivation}

Darwin's thing with the pots, as described in Fisher's \emph{Design of
Experiments}

We'll make a tower of the kinds of assumptions made to test Darwin's
hypothesis.

\subsection{Sign test}\label{sign-test}

Assume that it's meaningful if a hybrid plant is taller than a
self-fertilized plant, but don't assign any meaning beyond that. Then
the data are effectively dichotomized: you get some number of cases in
which one is taller and some number of cases in which the other is
taller.

Better to say that we're sampling from any distribution that has zero
median. You could even say you're sampling from \emph{all}
distributions. That's confusing, mathematically, because there are
infinitely many distributions, and it's not obvious how you should
sample from that functional space, but it works out, because all those
distributions will have the same distribution of pluses and minuses.

This is now just a binomial test.

\subsection{Rank test (Mann-Whitney $U$)}

Assume that the \emph{ranks} of the differences are meaningful.

Now you're sampling from any distribution that is symmetric about zero.
That means it has zero median also.

\subsection{Fisher's weird sum test}\label{fishers-weird-sum-test}

Not sure if there's any name for this. Assume that the actual values of
the differences are meaningful.

\subsection{Welch's $t$-test}

Assume that the two populations are normally distributed and, and
therefore that that the variances of the populations are meaningful.
Then you can infer where this set of differences would stand in an
infinite set of such differences.

For early statisticians, this was really appealing, mostly from a
computational point of view: you could actually compute the mean and
standard deviation with pen and paper in a reasonable amount of time,
but you definitely couldn't do all \(2^n\) different ways of taking
sums. Fisher does it for one example in his book, and I'm sure it was
pretty crazy. He makes it clear that he went to great lengths to do it,
and his conclusion is that the results are basically the same, so you
should probably be doing the easier thing and not worry about it.
Nowadays it's gotten pretty easy to do the other thing!, so it's

\section{Wilcox test and Mann-Whitney
test}\label{wilcox-test-and-mann-whitney-test}

\textbf{Walsh averages and confidence intervals}, from
\href{http://www.stat.umn.edu/geyer/old03/5102/notes/rank.pdf}{here}

There a few different names for these things:

\begin{itemize}
\tightlist
\item
  One-sample test: is this distribution symmetric about zero (or
  whatever)?
\item
  Two-sample unpaired (independent; Mann-Whitney): does one of these
  distributions ``stochastically dominate'' the other (i.e., is it that
  a random value drawn from population \(A\) is more than 50\% probable
  to be greater than a random value from \(B\))?
\item
  Two-sample paired (dependent): are the differences between paired data
  points symmetric about zero?
\end{itemize}

\subsection{Wilcoxon}\label{wilcoxon}

\begin{enumerate}
\def\labelenumi{\arabic{enumi}.}
\tightlist
\item
  For each pair \(i\), compute the magnitude and sign \(s_i\) of the
  difference. Exclude tied pairs.
\item
  Order the pairs by the magnitude of their difference: \(i=1\) is the
  pair with the smallest magnitude. Now \(i\) is the rank.
\item
  Compute the \(W = \sum_i i s_i\).
\end{enumerate}

Thus, the bigger differences get more weight.

(There might be a way to do a visualization of this: as you walk along
the data points, you get a good bump for every rank that is in the
``high'' data set, and you get a bad hit for every rank that is not.
Then it settles out pretty quickly, and you want to know the meaning of
the final intercept.)

For small \(W\), the distribution has to be computed for each integer
\(W\). For larger values (\(\geq 50\)), a normal approximation works.

Compare the sign test, which does not use ranks, and which assumes the
median is zero, but not that the distributions are symmetric. That's
just a binomial test of the number of pluses or minuses you get. It's
like setting the weights, which in \(W\) are the ranks, all equal.

\subsection{Mann-Whitney}\label{mann-whitney}

\begin{enumerate}
\def\labelenumi{\arabic{enumi}.}
\tightlist
\item
  Assign ranks to every observation.
\item
  Compute \(R_1\), the sum ranks that belong to points for sample 1.
  Note that \(R_1 + R_2 = \sum_{i=1}^N = N(N+1)/2\).
\item
  Compute \(U_1 = R_1 - n_1(n_1+1)/2\) and \(U_2\). Use the smaller of
  \(U_1\) or \(U_2\) when looking at a table.
\end{enumerate}

At minimum \(U_1 = 0\), which means that sample 1 had ranks
\(1,2,\ldots,n_1\). Note that \(U_1 + U_2 = n_1 n_2\).

For large \(U\), there is a normal distribution approximation.

\subsubsection{Generation}\label{generation}

Say you have \(N\) total points and \(n_1\) in sample 1. Find all the
ways to draw \(n_1\) numbers from the sequence \(1, 2, \ldots N\).
Compute \(U\) for each of those. Voila.

Note that, if you fix \(n_1\), then you don't have to subtract the
\(n_1(n_1+1)/2\) to get the right \(p\)-value.

\section{Statistical power: Cochrane-Armitage
test}\label{statistical-power-cochrane-armitage-test}

We never want to run just any test: we want to use the test that is most
capable of distinguishing between the scenarios we're interested in.
Usually this is a matter of choosing the test that has the right
assumptions: the one-sample \emph{t}-test is more powerful than the
Wilcoxon test if the data come from a truly normally-distributed
population.

In other cases, you might have more flexibility. There's a somewhat
obscure test that is, I think, a great illustration of this.

Imagine that you have some data with a dichotomous outcome for some
categorical predictor value. One classic example is drug dosing: you
think that, as the dosage of the drug goes up, you have more good
outcomes than bad outcomes. Did a greater proportion of people on board
the Titanic survive as you go up from crew to Third Class to Second to
First? Did the proportion of some kind of event increase over years?
Technically, this means you have a \(2 \times k\) table of counts, with
two outcomes and \(k\) predictor categories.

\begin{longtable}[]{@{}llll@{}}
\toprule
Outcome & Dose 1 & Dose 2 & Dose 3\tabularnewline
\midrule
\endhead
Good & 1 & 5 & 9\tabularnewline
Bad & 9 & 4 & 1\tabularnewline
\bottomrule
\end{longtable}

You could use a \(\chi^2\) test with equal expected frequencies across
the columns. In other words, there might be more ``good'' than ``bad''
outcomes, but you don't expect that proportion to differ meaningfully
across categories. You would pool the data across categories, use the
observed proportion of good outcomes as you best guess of the true
proportion \(f\), and compare the actual data with you expectation that
a fraction \(f\) of the counts in each column are ``good''.

In our examples, we think the data have some \emph{particular} kind of
pattern. The \(\chi^2\) test doesn't look for any particular pattern; it
just looks for any deviation from the null. The test statistic for the
\(\chi^2\) distribution is based around the sum of the square deviations
from the expected values, usually written \(\sum_i (O_i - E_i)^2\), with
some stuff in the denominator to make the distribution of the statistic
easier to work with. If the sum of the squared deviations is too large,
then we have evidence that the observed values are not ``sticking to''
the expected frequencies.

The trick I'm going to show you is to keep the same null
hypothesis---that outcome doesn't depend on dose---but adjust the test
so that it's more sensitive to particular kinds of dependencies.

This is a fair approach because we're still just trying to say, ``OK,
say you (the nameless antagonist) were right, and there really was no
pattern in the data. Then I'm free to make up any test statistic, so
long as, if you're right, we can show that the observed data were likely
to have arisen by chance.''

To start constructing the test, think about each flip as a weighted
binomial trial. We'll use these weights to adjust the test statistic to
be more sensitive to what we suspect the true pattern in the data is,
but we'll need to derive the distribution of the test statistic so that
we can satisfy the nameless antagonist.

Say each flip \(y_i\), which is in some category \(x_i\), gets some
associated weight \(w_i\). A really simple statistic would be
\(\sum_i w_i y_i\), the sum of the weights of the ``successful'' trials.
It would be nice to have this be zero-centered: \[
\sum_i \left\{ w_i y_i - \mathbb{E}\left[ w_i y_i \right] \right\} = \sum_i w_i (y_i - \overline{p}),
\] where \(\overline{p} = (1/N)\sum_i y_i\).

It would also be nice for this to have variance 1, so we can divide by
the square root of \[
\mathrm{Var}\left[ \sum_i w_i (y_i - \overline{p}) \right] = \overline{p} (1-\overline{p}) \sum_i w_i^2
\] to produce the statistic \[
T = \frac{\sum_i w_i (y_i - \overline{p})}{\sqrt{\overline{p} (1-\overline{p}) \sum_i w_i^2}}.
\]

You could also conceive of this being a table with two rows and some
number of columns. We bin the trials by their weights: all trials with
the same weight are in the same column. Successes go in the top row;
failures in the bottom. Now write \(t_c\) as the weight of the trials in
the \(c\)-th column, \(n_{1c}\) is the number of successful trials with
weight \(t_c\) (i.e., in column \(c\)), and \(n_{2c}\) is the number of
failures. Then some math shows that you can rewrite \(T\) as \[
T = \frac{\sum_c w_c (n_{1c} n_{2\bullet} - n_{2c} n_{1\bullet})}{\sqrt{(n_{1\bullet} n_{2\bullet} / n_{\bullet\bullet}^2) \sum_c n_{\bullet c} w_c^2}}
\] where \(n_{r\bullet}\) are the row margins, \(n_{\bullet c}\) are the
column margins, and \(n_{\bullet\bullet}\) is the total number of
trials.

\emph{N.B.}: The wiki page gives a different answer, but I don't trust
it, since the variance formula doesn't assume independence of the
\(y_i\). A fact sheet about the PASS software that shows the formula in
terms of the \(y_i\) seems to make a mistake by using \(i\) as an index
both for individual trials and for the weight categories.

The confusing thing here is how to pick the weights. This test is mostly
used to look for linear trends: imagine that each \(y_i\) is associated
with some \(x_i\), so that the weights would be \(x_i\) or
\(x_i - \overline{x}\). Why you pick these exact weights has to do with
the \emph{sensitivity} of the test. There could, of course, be a
nonlinear trend, like a U-shape, that would lead to a zero expectation
for this statistic. The \(\chi^2\) test can find that, but
Cochrane-Armitage with these weights cannot.

To see why you use those weights for a linear test, imagine that
\(p_i \propto x_i\), and zero-center the \(x_i\) such that
\(p_i = m x_i + \overline{p}\).

Then the question is what \(w_i\) maximize \(\mathbb{E}[T]\)? You can
quickly see that this is equivalent to maximizing
\(\sum_i w_i x_i / \sqrt{\sum_i w_i^2}\), and taking a derivative with
respect to \(w_j\) shows that, at the extremum,
\(x_j \sum_i w_i^2 = w_j \sum_i w_i x_i\), which \(w_i = x_i\) for all
\(i\) satisfies. So those weights are the best way to get a large
statistic if you think that there actually is a linear test.


\chapter{copied material}

\section{Things to include}\label{things-to-include}

\begin{itemize}
\item
  Gauss: a minimum-variance, mean-unbiased estimator minimizes the
  squared-error loss function. Laplace: among median-unbiased
  estimators, a minimum-average-absolute-deviation estimator minimizes
  the absolute loss function. Maybe it's better to allow some bias so
  you can get less variance. That's the domain of statistical theory. \hl{Move to MLE section?}
\item
  Fisher's crazy sum test is the same thing as is used in \emph{TRANSIT}
  (DeJesus \emph{et al}.): they treat TA sites in the same gene as
  independent; the statistic is the difference in the sum of the
  (normalized) number of insertions in two treatments; the null
  distribution is generated by shuffling the values across the two
  datasets. OK, it's not \emph{exactly} like Fisher's test, since it's
  not paired, but it's pretty close. Fisher probably wouldn't have
  wanted to to the \(\binom{n}{2}\) options, compared to the \(2^n\)
  that he did. \hl{Example of how people come up with tests?}
\end{itemize}

\subsection{Estimators about
estimators}\label{estimators-about-estimators}

\hl{Move this up, into the descriptive section}

\subsubsection{Jackknife}\label{jackknife}

You have \(n\) data points and compute an estimator \(\hat{\theta}\) for
some population parameter \(\theta\). If you don't know how the
population is structured, then it's not clear what you expect the
variance of \(\hat{\theta}\) to be. How sure can you be of this value?
In terms of inference, can you make any inference with it?

Compute the \emph{jackknife replicates}\footnote{The ``jackknife''
  method is so called because Tukey compared the method, which is
  ``rough-and-ready'', to another rough-and-ready tool, the pocket
  knife, also known as a jackknife. Although this name has the
  disadvantage of giving you no clue what it is about, it had the
  advantage of having more brevity and vivacity than ``delete-1
  resampling'', which is probably the more accurate name.}
\(\hat{\theta}_j\), which are the estimators computed using all the data
points except the \(j\)-th one.

That seems like a weird thing to have done, but you can use them to
compute two handy things:

\begin{enumerate}
\def\labelenumi{\arabic{enumi}.}
\tightlist
\item
  An estimate of the variance of the estimator. This can help you for
  description---by giving a confidence interval(?)---and for
  inference---by giving you a sense of the ``random'' ranges you would
  expect from two samples.
\item
  An estimate of the bias in the estimator. This is helpful if you don't
  want want your estimator to be biased but you don't know how to fix
  it.
\end{enumerate}

\paragraph{Jackknife variance
estimator}\label{jackknife-variance-estimator}

The variance estimator is \[
\widehat{\mathrm{Var}}_\mathrm{jk}[\hat{\theta}] := \frac{n-1}{n}  \sum_j \left( \hat{\theta}_j - \hat{\theta}_{(\cdot)} \right)^2,
\] where \(\hat{\theta}_{(\cdot)}\) is the average of the jackknife
replicates: \[
\hat{\theta}_{(\cdot)} := \frac{1}{n} \sum_j \hat{\theta}_j.
\] In other words, it's the variance of the jackknife replicates with
some rescaling: \[
\mathrm{Var}[\hat{\theta}_j] = \frac{1}{n-1} \sum_j \left( \hat{\theta}_j - \hat{\theta}_{(\cdot)} \right)^2 \implies
  \widehat{\mathrm{Var}}_\mathrm{jk}[\hat{\theta}] = \frac{(n-1)^2}{n} \mathrm{Var}[\hat{\theta}_j].
\]

The reason for that scaling factor is beyond the scope of this book
(Efron \& Stein 1981?), but the exercise gives you a sense of why it has
to be true for a specific case.

Some other work, also beyond the scope of this book, shows that the
jackknife estimate of variance is biased: it tends to overestimate the
true variance. This makes the jackknife a conservative tool.

\textbf{Exercise}. Let \(\theta\) be the mean. Show that the scaling
factor is what we think. Hints:

\begin{itemize}
\tightlist
\item
  Show that \(\hat{\theta}_{(\cdot)}\) is the sample mean.
\item
  Show that
  \(\hat{\theta}_j - \hat{\theta}_{(\cdot)} = (n \overline{x} - x_j) / (n - 1)\).
\item
  Show that that value is equal to \((\overline{x} - x_j) / (n - 1)\).
\end{itemize}

That exercise is from McIntosh's bioRxiv about jackknife resampling.

\paragraph{Jackknife bias estimator}\label{jackknife-bias-estimator}

The jackknife estimate of bias is
\((n-1) \left( \hat{\theta}_{(\cdot)} - \theta \right)\). This is the
sum of the deviations of the jackknife replicates from the observed
value \(\hat{\theta}\). Again, the reason that you would take the
average deviation and scale it up to the sum is beyond the scope.

However, if you have an expectation about the bias in an estimator, you
can make an unbiased estimator by subtracting out that bias: \[
\hat{\theta}_\mathrm{jk} := \hat{\theta} - \widehat{\mathrm{Bias}}_\mathrm{jk}[\theta].
\]

\textbf{Exercise}. Show that the jackknife estimate of bias for the
variance gives you the familiar unbiased variance estimator.

\textbf{Exercise}. Something about the maximum estimator?

\paragraph{Pros and cons of the
jackknife}\label{pros-and-cons-of-the-jackknife}

It's a piece of cake to implement. There are only \(n\) replicates to
do, so it's tractable. Those replicates are deterministic, so you only
run it once.

The cons are that it doesn't always work. For example, a jackknife
estimate of the variance of a median (\textbf{swo check Knight}) is not
consistent. It's also overly conservative: it's biased toward higher
variances. You can rescue some properties if you move to a delete-\(d\)
resampling and pick \(d\) from the correct range.

\subsubsection{Bootstrap}\label{bootstrap}

\section{Example from Efron, ``Thinking the unthinkable''}

There's some true distribution $f_X(x)$, and you're approximating it with
$\hat{f}_X(x)$, which is a pmf. If you took $N$ data points, then bootstrapping
means that you're picking a vector $\vec{c}$, where $c_i$ is the number of
times that the $i$-th data point makes it into the bootstrap sample. This
begs the question, how is $\vec{c}$ behaved? It's just a multinomial, with
probability $1/N$ for each of the $N$ cells.

Normally you compute a statistic $T(\vec{x})$ of the data. Instead, formulate
this in terms of a function $g(\vec{c})$ If you can write $T(\vec{x}) = \sum_i t(x_i)$,
then $g(\vec{c}) = \sum_i (c_i/N) t(x_i)$. In a Taylor expansion around
$\vec{c}_\mathrm{ML} = (1, 1, \ldots, 1)$:
$$
g(\vec{c}) = g(\vec{c}_\mathrm{ML}) + \sum_i \frac{dg}{dc_i} (c_i - 1) + \mathcal{O}(c_i^2)
$$
So the variance of the values $g(\vec{c})$ that you will get from
bootstrapping is approximately
\begin{align}
\expect{\left[g(\vec{c}) - g(\vec{c}_\mathrm{ML})\right]^2}
  &= \expect{\left(\sum_i \frac{dg}{dc_i} (c_i-1)\right)^2} + \mathcal{O}(c_i^2) \\
  &= \expect{\sum_i \left( \frac{dg}{dc_i} (c_i-1)\right)^2} + \mathcal{O}(c_i^2) \\
  &= \sum_i \left(\frac{dg}{dc_i}\right)^2 \expect{(c_i-1)^2} + \mathcal{O}(c_i^2) \\
\end{align}

And then an $n^2$ comes out? The point is that the jackknife is basically
doing a finite estimation of the gradient, by leaving out a single point at a
time.

\section{\texorpdfstring{What does it mean to
``sample''?}{What does it mean to sample?}}\label{what-does-it-mean-to-sample}

\hl{Earlier in text, get clearer about RVs and their ``realizations''. and ``samples''}

Does it make sense to compute a confidence interval when you're sampled
all the 50 United States?

\textbf{Finite correction factor} to point out that there's a difference
between simple random sampling and something else. Then need to explain
what simple random sampling is!


\subsection{\texorpdfstring{\emph{t}-distribution}{t-distribution}}\label{t-distribution}

\hl{Maybe just mention in passing how difficult the math gets when you want to estimate many quantities simultaneously? Contrast t- and z-tests}

Let's think about how to construct that method. Say you knew the true
variance \(\sigma^2\). Then we know that the sample means are drawn from
\(\mathcal{N}(0, \sigma^2/n)\). So it's pretty easy to see that
\((\overline{x} - \mu) / (\sigma^2) \sim \mathcal{N}(0, 1)\), from which
the familiar \(1.96\), etc. come.

What if you \emph{don't} know the true variance? The means are still
drawn from \(\mathcal{N}(0, \sigma^2/n)\), but now the sample variance
is also a random variable.

We know the confidence interval is some function of the sample mean and
variance, and let's guess that it's symmetric about the sample mean and
is some linear function of sample variance: \[
\mathrm{CI}_\pm(\overline{x}, s) = \overline{x} \pm A s.
\] We want to find \(A\) such that \[
\mathbb{P}\left[ \mathrm{CI}_- < \mu < \mathrm{CI}_+ \right] = 95\%,
\] or, if we're willing to trust in symmetry, \[
2.5\% = \mathbb{P}\left[ \mathrm{CI}_- > \mu \right] = \mathbb{P}\left[ \frac{\overline{x} - \mu}{A} - s > 0 \right].
\] We know the distribution of the first thing: \[
(\overline{x}-\mu)/A \sim \mathcal{N}\left(0, \frac{\sigma^2}{n A^2}\right).
\] Some math shows that \[
\frac{(n-1) s^2}{\sigma^2} \sim \chi^2(n-1).
\]

Call the first thing \(K\) and the second \(L\). We're interested in the
distribution of \(M \equiv K - L\): \[
f_M(m) = \int_0^\infty f_K(m + l) f_L(l) \,\mathrm{d}l,
\]

where the limits come from the fact that variance is positive. You're
probably not excited to do this integral, which was considered a major
achievement (well, it was the thought leading up to the integral, which
we've just outlined, but whatever). This major achievement was made by
William Sealy Gosset, who made it while he was a researcher for Guinness
ensuring the quality of their beer. Guinness had a policy of not
allowing its employee to publish their results, so Gosset signed his
paper ``a student'', so the result of that integral is now called
Student's \emph{t}-distribution: \[
f_t(x; \nu) = \frac{\Gamma(\frac{\nu+1}{2})}{\sqrt{\nu\pi} \Gamma\left(\frac{\nu}{2}\right)}
  \left(1+ \frac{x^2}{\nu}\right)^{-\frac{\nu+1}{2}},
\] where the (badly named) ``degrees of freedom'' \(\nu\) is \(n-1\) for
our purposes. I write this out fully because it is one of the things we
will \emph{not} derive in this book.

\section{Contingency tables}\label{contingency-tables}

\hl{Move Fisher's exact, Barnard, chi-square up, into a section on p-values? Or maybe statistical power?}

These are nice examples for how to do statistical thinking.

\subsection{Barnard's test}\label{barnards-test}

The classic example is whether a certain treatment causes more of the
outcome of interest than just doing nothing. In medicine, that means
splitting your participants into a placebo group and a treatment group
and asking what fraction of each gets well. In a biology experiment, you
might split your mice into a treatment group and a control group and ask
what proportion of the mice in each group get cancer.

In statistics jargon, this is called a \(2 \times 2\) contingency table:

\begin{longtable}[]{@{}llll@{}}
\toprule
Group & Outcome \(p\) & Outcome not-\(p\) & Row sums\tabularnewline
\midrule
\endhead
A & \(a\) & \(c\) & \(m\)\tabularnewline
B & \(b\) & \(d\) & \(n\)\tabularnewline
Column sums & \(r\) & \(s\) & \(N\)\tabularnewline
\bottomrule
\end{longtable}

Because we picked \(m\) and \(n\), the sizes of the two groups, those
are fixed parameters. The question is whether the way that \(m\) gets
distributed into \(a\) and \(c\) (and that way that the \(n\) get put
into the \(b\) and \(d\)) is consistent with there being a common
probability \(p\) of the outcome of interest.

So we might say that \(a\) is distributed like a binomial distribution
with \(m\) draws and probability \(p_a\) of success, and \(b\) is
distributed like a binomial with \(n\) draws and a probability of
\(p_b\) of success. The null hypothesis is that \(p_a = p_b\). What's
the likelihood of the data given the null?

If we didn't assume the null, and gave the two binomials their own
probabilities, the likelihood of the data would be: \[
P(a, b | p_a, p_b) = \mathrm{Bin}(a; m, p_a) \times \mathrm{Bin}(b; n, p_b).
\] But, given that the probabilities are the same, we can collapse it:
\[
\begin{aligned}
\mathcal{P}[a, b | p_a = p_b = p] &= \mathrm{Bin}(a; m, p) \times \mathrm{Bin}(b; n, p) \\
  &= \binom{m}{a} p^a (1-p)^{m-a} \times \binom{n}{b} p^b (1-p)^{n-b} \\
  &= \binom{m}{a} \binom{n}{b} p^{a+b} (1-p)^{m+n-(a+b)} \\
  &= \frac{m! \, n!}{a! \, b! \, c! \, d!} p^r (1-p)^s.
\end{aligned}
\]

This result is a little confusing\footnote{I trotted out this test
  because these two confusions are actually great learning
  opportunities.}, for two reasons:

\begin{enumerate}
\def\labelenumi{\arabic{enumi}.}
\tightlist
\item
  The probability \(p\) of the outcome of interest might be interesting
  to design a later experiment, but it's \emph{not} interesting for
  designing a test. We certainly don't want to deliver a result like,
  ``Well, if the null hypothesis is true, \emph{and} \(p\) happens to be
  exactly such-and-such, then your \(p\)-value is so-and-so.'' The value
  \(p\) is called a \emph{nuisance parameter} since we don't actually
  care about its value.
\item
  We're usually not interested in the likelihood of exactly this data,
  but rather in the likelihood of data \emph{at least this extreme}. We
  usually measure ``extremeness'' using a statistic---a single
  number---so it's clear that ``more extreme'' means ``bigger'' (or
  ``smaller'' or ``bigger or smaller'', depending on if it's a one-sided
  or two-sided test). Here, we have two numbers, \(a\) and \(b\), so
  there aren't two ``sides'' to the distribution: there are four!
\end{enumerate}

To resolve the first point, we say that the null hypothesis
\(p_a = p_b = p\) doesn't restrict us to a particular value of \(p\). In
other words, the null hypothesis, which functions as a sort of Annoying
Skeptic, is free to pick \(p\) to make our results as uninteresting as
possible. Mathematically, this means that, when computing the
\(p\)-value, we should optimize over all values of \(p\), choosing the
one that makes our results as uninteresting as possible (i.e., which
maximizes the \(p\)-value).

We can't really ``resolve'' the second point, since it demonstrates that
our previous way of thinking about extremeness was not sufficient for
all cases. As Barnard notes in his original paper\footnote{Barnard
  conceived of the \((a, b)\) as points ``in a plane lattice diagram of
  points with integer co-ordinates'', that is, that \(a\) is like the
  \(x\)-axis and \(b\) is like the \(y\)-axis. Then the possible
  outcomes of the experiment are the points in the rectangle bounded by
  the horizontal lines \(a = 0\) and \(a = m\) and the vertical lines
  \(b = 0\) and \(b = n\). He then said that you should pick the
  non-extremal points (i.e., the values of \((a, b)\) for which you
  would not reject the null) such that they ``consist of as many points
  as possible, and should like away from that diagonal of the rectangle
  which passes through the origin. Formulated mathematically, these
  latter requirements mean that the {[}points for which you would reject
  the null{]} must in a certain sense be convex, symmetrical and
  minimal.''}, there are actually many ways to choose the pairs
\((a, b)\) that produce a \(p\)-value more than our threshold. This gets
into some fancy footwork to articulate exactly how you should pick this
area, but the basic results are pretty intuitive: when \(a/m\) and
\(b/n\) are similar, you tend to be under the rejection threshold; when
they are far apart, you tend to be over.

The interesting point here is that, whatever fancy footwork you pick to
choose that region, and no matter how ``reasonable'' your footwork is,
it's still footwork that doesn't obviously follow from the simple
definition of a hypothesis test. We'll encounter this problem again in
Bayesian statistics, when we find that the Bayesian analog of a
confidence interval is not unique: there are many ranges of values that
are compatible with our ignorance.

\subsection{Fisher's test to the
rescue(?)}\label{fishers-test-to-the-rescue}

If you've worked with contingency tables, you're probably saying, ``I've
never heard of this crazy Bernard's test, with its weird multi-sided
rejection space and its requirement to maximize over \(p\). We have
Fisher's exact test, which is the exactly right test to use here!''

Looking at the same contingency table, Fisher's test asks, given the row
marginals \(m\) and \(n\), the first column marginal \(r\), and the
grand total \(N\), what is the probability of a table at least this
extreme?

This is just a combinatoric problem: if you're as likely to assign items
in \(m\) to \(a\) as to \(c\) (and, analogously, to assign items from
\(n\) to \(b\) or \(d\)), then ``what's the probability of this table''
is equivalent to asking ``given the marginals, how many ways are there
to choose this table?''. More specifically, how many ways are there to
choose \(a\) items from a bank of \(m\) items and \(b\) items from a
bank of \(n\), given that we chose \(r = a + b\) items from the total
\(N\)? Mathematically: \[
\mathbb{P}[a | m, n, r, s] = \frac{\binom{m}{a} \binom{n}{b}}{\binom{N}{r}} = \frac{m! \, n! \, r! \, s!}{N! \, a! \, b! \, c! \, d!}.
\]

Computing the \(p\)-value is easier here than with Barnard's test
because we need to keep the row \emph{and column} marginals the same. In
Barnard's test, we just kept the row marginals constant, because we
considered those as fixed parameters, corresponding to things like the
number of patients we assigned to each of the placebo and treatment
groups. It doesn't make sense to allow the Annoying Skeptic to fiddle
with those values.

In Banard's test, we \emph{did} allow the Annoying Skeptic to fiddle
with the column marginals, since it wasn't clear, before the experiment
began, that \(r\) would have the outcome of interest. In other words, we
didn't know that \(r\) people in both the placebo and treatment groups
would get well.

Fisher's test, however, \emph{does} keep the column marginal constant.
This makes it a lot easier to compute the \(p\)-value. First, the
nuisance parameter \(p\) doesn't appear in the likelihood, so we don't
need to do the weird maximization. Second, we only need to vary one
value, \(a\) (or, equivalently, \(b\)), since, if you know the
marginals, there is only one axis along which to change the values in
the table. In other words, if you know \[
\begin{aligned}
a + c &= m \\
b + d &= n \\
a + b &= r,
\end{aligned}
\] then that's three equations with four unknowns (\(a\), \(b\), \(c\),
\(d\)), so specifying any one of \(a\), \(b\), \(c\), or \(d\) specifies
all the others. (You might be looking for a fourth equation
\(c + d = s\), but you can get that by adding the first two equations
and subtracting the third.)

Here's an example:

\begin{longtable}[]{@{}llll@{}}
\toprule
Group & Success & Failure & Row sums\tabularnewline
\midrule
\endhead
A & 1 & 9 & 10\tabularnewline
B & 11 & 3 & 14\tabularnewline
Column sums & 12 & 12 & 14\tabularnewline
\bottomrule
\end{longtable}

There's only one way to make this table more ``extreme'' without
changing the marginals: you can take the one group A success and make it
a group A failure and simultaneously make a group B failure into a group
B success. Similarly, there's only one way to make this table less
extreme: turn a group A failure into success, and turn a group B success
into failure.

So keeping the column sums constant made it way easier to compute the
\(p\)-value: count this table and all the tables with a more extreme
upper-left or bottom-right and see if your summed probability hits the
rejection threshold.

However, this simplicity came at a cost, which you may have noticed:
does it make sense to keep the columns constant? Experimentally, this
means that you're restricting the Annoying Skeptic to only consider
cases in which, say, the number of patients who got well \emph{in both
groups} is equal to the experimentally observed value. This is a little
weird. It suggest that your experimental design was like this:

\begin{enumerate}
\def\labelenumi{\arabic{enumi}.}
\tightlist
\item
  Pick \(m\), \(n\), and \(r\).
\item
  Assign \(m\) patients to placebo and \(n\) to treatment.
\item
  Wait until \(r\) patients \emph{across both groups} have gotten well.
\item
  Stop the experiment.
\end{enumerate}

This is almost certainly not reflective of how typical experiments are
run\footnote{It is, however, the way the famous ``lady tea tasting''
  experiment was designed. The myth is that Fisher didn't believe it
  when a high-class lady told him that she could detect whether tea was
  added to a cup with milk in it or whether the milk was added to the
  tea. He designed an experiment with \(m\) cups prepared one way, \(n\)
  prepared the other, and told her to detect the \(r = m\) cups that
  were prepared the first way. A Barnard-style experiment, in which the
  same \(m\) and \(n\) cups}.

\textbf{Fisherian small data}

\textbf{What happens if I use the ``wrong'' test? Chi-square as an
example of wrongness}

\section{Regression}\label{regression}

\hl{Make this a chapter? In descriptive statistics? Or just say this regression is an ML problem? Link and error distributions.}

\section{$\chi^2$ test}

\hl{Merge with section above on chi-square}

Say you have $k$ iid standard normal random variables:
\begin{equation}
X_i \stackrel{\text{iid}}{\sim} \mathcal{N}(0, 1).
\end{equation}
Then $Y = \sum_{i=1}^k x^2$ (??) is $\chi^2$-distributed with $k$ degrees of freedom.

Let's start with a simple case where you have a table with two cells with
expected probabilities $p_1$ or $p_2 = 1-p_1$. We got $n$ total observations,
with $O_1$ in the first cell and $O_2 = n - O_1$ in the second. You probably
remember how to compute the test statistic from Stats 101:
\begin{equation}
\chi^2 = \sum_{k=1}^2 \frac{(O_i - E_i)^2}{E_i} = \frac{(O_1 - np_1)^2}{np_1} + \frac{(O_2 - np_2)^2}{np_2},
\end{equation}
where $E_i$ is the ``expected'' number of counts in each cell.

Consider the numerator of the second term:
\begin{equation*}
(O_2 - np_2)^2 = \left[(n - O_1) - n(1 - p_1)\right]^2 = (-O_1 + np_1)^2 = (O_1 - np_1)^2.
\end{equation*}
Handy, that's the same as numerator of the first term! That means we can re-write things:
\begin{equation*}
\chi^2 = \frac{(O_1 - np_1)^2}{n}\left( \frac{1}{p_1} + \frac{1}{p_2}\right).
\end{equation*}
A little algebra shows that $1/p_1 + 1/p_2 = 1/p_1(1-p_1)$, so that
\begin{equation}
\chi^2 = \frac{(O_1 - np_1)^2}{np_1(1-p_1)} = \left( \frac{O_1-np_1}{\sqrt{np_1(1-p_1)}} \right)^2.
\end{equation}
That might look terrible, but it's actually pretty cool. Here's why: $O_1$ is the observed value, $np_1$ is the expected mean, and $\sqrt{np_1(1-p_1)}$ is the standard deviation of the binomial distribution. I'll re-write that last equation with more suggestive notation:
\begin{equation}
\chi^2 = \left( \frac{x_1 - \mu_1}{\sigma_1} \right)^2
\end{equation}

This certainly \emph{looks} like a $\mathcal{N}(0, 1)$ variable, although we
said previously that the counts in the two cells follow a binomial
distribution. This is where the central limit theorem comes in: the sum of any
large set of (well-behaved) iid random variables approaches a normal
distribution. The binomial distribution approaches the normal distribution
particularly quickly such that (if the distribution is not highly skewed) you
only need about 5 counts for the normal approximation to be pretty
good.\footnote{The normal approximation to the binomial was proved long before
the central limit theorem. This special case, called the \emph{de Moive-Laplace
theorem}, was first published by de Moivre in 1738. Laplace published
the reverse result, that the binomial approximates the normal, 75 years later,
in 1812. The general central limit theorem was proven, more than 150 years
after de Moivre's original result only, in 1901 by Lyapunov. \hl{Put CLT in with regression? Or z-test?}}

So, so long as each cell has (ish) 5 or more counts, then we can approximate
the binomial variables with normal variables, which means that the test
statistic $\chi^2$ that I wrote is actually just the square of a single,
standard normal variable, which happens to be $\chi^2$-square distributed with
1 degree of freedom. Two cells in the table ($k=2$) meant $k-1=1$ degrees of
for the $\chi^2$ distribution.

The same result holds, that the sum of the $(O_i - E_i)^2/E_i$ values follows
a $\chi^2$ distribution with $k-1$ degrees of freedom, for $k>2$. The math is
a lot more involved because the $k$ cells in the table are distributed
according to a multinomial distribution. In other words, conditioned on the
total number $n$ of counts, the values in the different cells are not
independent: if cell 1 has a lot of counts, cells 2, 3, etc. can't have that
many cells. Like we've seen before, covariance makes the calculations hard!
Nevertheless, the same restrictions apply: you can only count on the normal
approximation working if you have enough counts in every cell.

\section{Coda}

\hl{New material}

- Bayes
- Bayes sampling?
- MCMC for copmlex models?
- Nature Biotech Bayes example
- Optimization for MLE
- Regression and mixed models
- Random variable neq variable with a random value
- When discussing RVs, note that cdf defines it. Then don't ever talk about events, just look at joint cdfs, etc. Make this a whole section unto itself.
- Do joint pdf's so it's easier to talk about independence

Tony's ideas:

- Neyman Pearson lemma
- Likelihood ratio tests
- Why is frequentist so good? CLT, convergence, etc.
- Information theory, model simplicity, AIC?

\end{document}
