%!TEX root=main.tex

\chapter{Preface}

\section*{Who this book is for}

As a graduate student and postdoc, I found that many graduate
students and postdocs, especially in the life sciences, have a fair amount of quantitative
training, often including statistics, and can hack together
fairly sensible quantitative methods that answer their scientific questions. I
found, though, that there was often a steep drop-off when it came time to apply
statistical rigor to these hacky methods. This is my
own story too.

This book is for people who are experienced quantitative and scientific
thinkers, have decent algebra, and who want to learn
about how to use statistical thinking to improve their scientific thinking.
A rusty memory of college-level statistics might make it more interesting
but is not required.

\section*{What this book is for}

As I tried to fill gaps in my statistics knowedge, I found that books about
statistics tended to come in two varities: utterly simplistic and
overwhelmingly complex. It felt like trying to find books to read as an
adult second-language leanrer.
Grown-up books have interesting topics but overwhelming grammar and vocabulary,
while children's books have accessible language but childish topics.

I wanted this book to be interesting for scientific ``adults'' without having
an overwhelming grammar or vocabulary. I introduce a lot of math in the first
few sections, but it assumes very little mathematical training and focuses on
the building blocks of the practical discussions later on. There are very few
proofs in this book.

\section*{What this book is not}

This \emph{is} a book for people who want
to use rigorous statistics to do science, and maybe even to develop some
new statistical methods. It is \emph{not} a book for people who want to
push the boundaries of what statisticians think are interesting mathematical
problems.

This is not a cookbook that tells you what statistical test to run on your
data. Instead, I try to give you tools to reason about why to do a test and
how to formulate one. Rather than telling you which steps to generate a
$p$-value in a specific case, I show how $p$-values come about in general and
how to derive them for specific cases.

I did not fill this book with scientific examples, because I find that it's
very easy to think that your particular scientific question is very
interesting, and that the statistics that motivate are interesting, but that
someone \emph{else's} scientific questions or statistical problems are entirely
uninteresting. I try to keep it general so you can imagine you are the hero of the story.

\section*{What you should be able to do after reading this book}

I want you to be able to articulate you own statistical tests, modify
and critique existing methodologies, and develop a healthy skepticism of the
idea of statistical inference in general.

\section*{What else should I read}

My principal sources of motivation for this book were Allen Downey's
\textit{Think Stats} and \textit{Think Bayes}. Stephen Stigler's
\textit{History of Statistics} helped me understand many of the unspoken
assumptions present in contemporary statistical thinking. Miran Lipva\v{c}a's
\textit{Learn You a Haskell for Great Good} helped me clarify the language
around the mathematical objects that come up. The first four chapters of
\textit{Probability and Random Processes} by Grimmett and Stirzaker are an
outstanding introduction to probability.

\section*{Notes on notation}

\begin{table}
\centering
\begin{tabular}{ll}
\toprule
Type of object & Notation \\
\midrule
function that maps $A$ to $B$ & $f : A \to B$ \\
function of a number $x$ with parameters $\theta$ & $f(x; \theta)$ \\
function of non-number input $X$ & $f[X]$ \\
probability function & $\mathbb{P} : \Omega \to [0, 1]$ \\
probability & $p \in [0, 1]$ \\
discrete cumulative distribution function for random variable $X$ & $F_X : \mathbb{Z} \to [0, 1]$ \\
continuous cumulative distribution function for random variable $X$ & $F_X : \mathbb{R} \to [0, 1]$ \\
probability mass function for random variable $X$ & $f_X : \mathbb{Z} \to [0, 1]$ \\
probability distribution function for random variable $X$ & $f_X : \mathbb{R} \to [0, 1]$ \\
random variable $X$ is distributed like $Y$ & $X \sim Y$ \\
estimator of statistic $X$ & $\hat{X}$ \\
variance of a random variable $X$ & $\mathbb{V}[X]$, $\sigma_{X}^2$ \\
expected value of a random variable $X$ & $\mathbb{E}[X]$, $\mu_X$ \\
estimator for mean of iid random variables $X_i$ & $\hat{\mathbb{E}}[X]$, $\hat{\mu}_X$, $\overline{X}$ \\
estimator for variance of iid random variables $X_i$ & $\hat{\mathbb{V}}[X]$, $\hat{\sigma}_X^2$ \\
observed mean of data & $\overline{x}$ \\
observed variance of data & $s^2$ \\
\bottomrule
\end{tabular}
\caption{Notation used in this book.}
\label{tab:notation}
\end{table}

Notation is important, especially in probability and statistics, where related
but crucially different things can be represented by similar pieces of
notation. I present each piece of notation as it arises, but I put them here
also for reference (Table~\ref{tab:notation}).
