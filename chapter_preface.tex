%!TEX root=main.tex

\chapter{Preface}

As a graduate student and postdoc in the life sciences, I saw that many of my
colleauges, had substantial quantitative training and experience. They were
very able to hack together fairly sensible quantitative and statistical methods
to answer their scientific questions. I found, however, that there was often a
steep drop-off when it came time to apply statistical rigor to these hacky
methods.

I think this gap between the ability to hack something together and the
understanding to make something rigorous is partly born out of a gap in
educational materials. There are plenty of introductory statistics
textbooks that explain what a mean is, and there are plenty of statistical
test cookbooks that well you what assumptions go into a $t$-test, and
there are plenty of books and articles on statistics meants for people
with a graduate-level education in statistics or math. There are few
resources for people who are mature and shrewd quantitative thinkers but
who don't have a half dozen statistics courses under their belt.

I think the situation is analogous to learning a foreign language as an
adult. I find that it is easy to find books for children, which are at my
level in terms of grammar and vocabulary but thematically boring, and
there are books for adults, which are thematically interesting but way
over my head. I wanted to write a book about statistical tests that was
thematically \emph{and} ``gramatically'' appropriate.

This is not a book for people who want to push the boundaries of what
statisticians think are interesting mathematical problems. Nor is it
a a cookbook that tells you what statistical test to run on your data.
I want to give you the ability to reason about why to do a test and how to
formulate one. Rather than telling you which steps to generate a $p$-value
in a specific case, I show how $p$-values come about in general and how to
derive them for specific cases.

I hope you find it useful!
