%!TEX root=main.tex

\chapter{Preface}

As a graduate student and postdoc in the life sciences, I saw that many of my
colleauges had substantial training and experience in quantitative methods. They were
very able to hack together sensible ways to approach their statistical problems.
I found, however, that there was often a
steep drop-off in their ability to apply statistical rigor to these \textit{ad hoc}
methods.

I think this gap between the ability to hack something together and the
understanding to make something rigorous is partly born out of a gap in
educational materials. There are plenty of introductory statistics
textbooks that explain what a mean is. There are also plenty of statistical
test cookbooks that well you what assumptions are made when using a $t$-test.
And finally, there are plenty of books and articles on statistics meants for people
with a graduate-level education in statistics or math. However, there are few
resources for people who are mature and shrewd quantitative thinkers but
who do not have a half dozen statistics courses under their belt.

To me, this situation is analogous to when I tried to learn a foreign language as an
adult. It was easy for me to find books written for children. These books are at my
reading level in terms of grammar and vocabulary, but they are thematically boring.
On the other hand, books for adults are thematically interesting but completely
intractable in terms of vocabulary and grammar.
I wanted to write a book about statistics that is
thematically \emph{and} ``gramatically'' appropriate.

I am not a statistician, and this is not a book for people who want to push the
boundaries of what
statisticians think are interesting problems. It is also not
a cookbook that tells you what statistical test to run on your data. There
are plenty of those already.
Instead, my goal is to give you the ability to reason about why to do a statistical
test and how to
formulate your own. Rather than telling you which steps to generate a $p$-value
in a specific case, I show how $p$-values come about in general and how to
derive them for specific cases.

I hope you find it useful!
