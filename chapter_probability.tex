%!TEX root = main.tex

\chapter{Probability theory}
\label{chapter:probability}

To understand statistics, it's critical to have a grasp of basic concepts in
probability, which is an entire branch of mathematics. The central concept in
probability theory is, unsurprisingly, \emph{probability}.

\section{The definition of probability}

We talk about and reason about probability every day, mostly for matters of
prediction. Given the weather report, is it worth it for me to carry an
umbrella? Given the results of my experiment, how likely is it that my
hypothesis was correct?

We intuitively think about probability in a mostly \emph{Bayesian} way.
Statistics, when not prefaced by the word ``Bayesian'', refers to a different
philosophical branch call \emph{frequentist} statistics. The philosophical
distinction between Bayesian probability and frequentist probability have
important practical implications, so it's important to understand the
difference.

In Bayesian statistics, \emph{probability} refers to a difficult-to-articulate
sense of confidence about an unknown thing, often a future event. If I say the
probability of an event is 1\%, it means, in a practical sense, that I'm
willing to bet a fair amount of money it won't happen. Philosophically, it's
hard to say what ``1\% confident'' means, but the concept should be very
intuitive.

A more mathematically and philosophically simple way to think about
probability is the \emph{frequentist} approach. In this framework,
probabilities are all proportions, or ``frequencies''. A 50\% chance of
flipping a coin and seeing heads means that, as you flip the coin more and
more times, the proportion of flips that come up heads will approach 50\%.

The problem with the frequentist approach is that probabilities can only be
assigned to events that can be repeated. It therefore doesn't make sense to
ask about the probability that your hypothesis is correct, since your
hypothesis is either correct or incorrect. It's like asking the probability
that it rained yesterday. It either did or it didn't. We live in just one
universe, so you can't ask, in a frequentist scheme, about the probability of
a state of nature, like whether your hypothesis is correct.

This is deeply dissatifying: the whole point of statistical inference is to
figure out what's going on in the world. But the Bayesian approach, which does
allow you to ask about the probability of states of nature, is mathematically
and technically more challenging. Also, some prominent statisticians thought
it was nonsense and mostly directed the field of statistics away from the Bayesian
approach for many years.

So what's the definition of probability? It's either:
\begin{enumerate}
\item in the frequentist sense, the proportion of identically-repeated trials that have some outcome, or
\item in the Bayesian sense, some number between 0 and 1 that encodes our sense of the---for lack of a better word---probability of that outcome.
\end{enumerate}

\section{The mathematical definition of probability}

In the last section I used the words ``trial'', ``outcome'', and
``probability''.  There is a sophisticated mathematical theory about all these
concepts called \emph{measure theory}. Although fascinating, it's pretty heavy
lifting, so I'll aim to use definitions that will give us a lot of practical
benefit, albeit at the cost of creating a mathematically incomplete theory.

To start, we define \emph{outcomes} or \emph{sample space} as the set of all
things that could happen when we run some trial. How you define ``set of all
things'' is a complicated question. In the discussion that follows, I'll use
simple, abstract events. For example, if you flip a coin, we'll say it lands
heads or tails. The collection of all possible combinations of outcomes are
the possible \emph{events}.\footnote{This is where the mathematical
incompleteness comes in. If the outcomes are finite (i.e., discrete, like in
our cards example), it's straightforward to enumerate the events as the finite
set of all possible subsets of the outcomes. If the outcomes are continuous,
things get tricky. You can't just ``count'' an infinite number of events, so
you need to be careful about how you define ``all possible combinations''. The
right way to do things is with a \emph{$\sigma$-algebra}.} For flipping a
coin, the events include the individual outcomes (heads or tails), the
combinations of outcomes (either heads or tails), and no outcome (didn't flip
anything).

\begin{table}
\centering
\begin{tabular}{p{3cm}p{4cm}p{6cm}}
\toprule
Situation & Outcomes & Events \\
\midrule
Flipping a coin & heads, tails & $\varnothing$, $H$, $T$, $\Omega$ \\
Rolling a die & $1, 2, \ldots, 6$ & $\varnothing$; $1, 2, \ldots, 6$; $\binom{6}{2}$ 2-event outcomes (e.g., 1 or 2); $\binom{6}{3}$ 3-outcome events (e.g., 1, 2, or 3); $\ldots$; $\binom{6}{5}$ 5-outcome events (e.g., any except 1); $\Omega$ \\
Drawing a card & 2 of Clubs, $\ldots$, Ace of Hearts & $\varnothing$; 52 individual events (e.g., 2 of Clubs); $\binom{52}{2}$ 2-outcome events (e.g., black Ace); $\ldots$; $\binom{52}{51}$ 51-outcome events (e.g., any except 2 of Clubs); $\Omega$ \\
Pick a random number between 0 and 1 & All real numbers between 0 and 1 & $\varnothing$; intervals like $[0, \tfrac{1}{2}]$; ``all subsets'' of $[0, 1]$; $\Omega$ \\
Your experiment & Each possible configuration of atoms at measurement & ``Cancer cell died'', ``Electron had energy between 1 and 2 eV'', etc. \\
\bottomrule
\end{tabular}
\caption{Distinction between outcomes and events for various example situations. $\varnothing$ is the nothing-happened event. $\Omega$ is the something-happened event. The ``all subsets'' is a $\sigma$-algebra, which we won't dive into.}
\label{tab:outcome-event}
\end{table}

The difference between \emph{outcomes} and \emph{events} might seem trivial,
but it will come up a few times, so I'll repeat that \emph{outcomes} are
individual, real things that might happen, while \emph{events} are abstract
groupings of outcomes. Importantly, nothing-happened is an event but clearly
not an outcome: if I run a trial, it can't be that nothing happened, but
for mathematical reasons, nothing-happened is an event.

Each event is associated with a probability via a function that has no real
name except ``probability''. It is a basic axiom of probability theory that
probabilities are always between zero and one. The probability function, written
here as $\mathbb{P}$, links each event $\omega$ in the space $\Omega$ of
events with a probability $p$:
\begin{gather*}
\mathbb{P} : \Omega \to [0, 1] \\
\prob{\omega} = p
\end{gather*}
For example, if the event $H$ is flipping heads, then we say
$\prob{H} = \tfrac{1}{2}$. I use the square brackets to emphasize
that $\mathbb{P}$ is a function of something that is not a number: it takes
events $\omega$ and not numbers like 5.

Probability theory axioms specify that the anything-happened event $\Omega$ has
probability one and the nothing-happened event $\varnothing$ has probability zero:
\begin{align*}
\prob{\Omega} &= 1 \\
\prob{\varnothing} &= 0
\end{align*}
Something must happen, and nothing cannot happen.

\section{Manipulating probabilities}

We're often interested in the relationships between events. What is that
probability that this \emph{or} that happened? What is the probability that
this \emph{and} that happened?

\subsection{``And'' adds event probabilities}

If $A$ and $B$ are two events that don't have any constituent outcomes in
common, we call then \emph{disjoint}, and their probabilities add. For
example, the probability of flipping a heads or a tails is
the probability of heads plus the probability of tails.
Mathematically we write this as
$$
\text{if } A \cap B = \varnothing \text{, then } \prob{A \cup B} = \prob{A} + \prob{B},
$$
where the ``cap'' $\cap$ means \emph{intersection} (``and'') and the ``cup'' $\cup$ means
\emph{union} (``or''), so this reads ``if no outcomes are in both $A$ and $B$,
then the probability of $A$ or $B$ is the sum of their individual probabilities.''

If $A$ and $B$ do have some overlap, you need to subtract out the probability
of the overlap. For example, consider drawing a card
from a standard 52-card deck. What is the probability of drawing a Jack or a
Diamond? If you add up the probability of drawing a Jack and the probability
of drawing a Diamond, you end up double-counting the Jack of Diamonds event,
so the solution is to subtract out the double-counted event:
$$
\prob{A \cup B} = \prob{A} + \prob{B} - \prob{A \cap B}.
$$
We mostly deal with disjoint events, so I won't belabor this point.

\subsection{``Or'' multiplies events}

We just said that ``or'' for disjoint events adds probabilities. How do we
find the probability of $A$ and $B$? Say I flip two coins. What's the
probability that I flip heads on the first coin and heads on the second?

If $A$ and $B$ are \emph{independent} events, then their probabilities
multiply. The probability of flipping two heads is $\tfrac{1}{2} \times
\tfrac{1}{2} = \tfrac{1}{4}$.

\subsection{Independence and conditional probability}

How do you know if two events are independent? Mathematically, $A$ and $B$
are indepedent if and only if their probabilities multiply.

This might feel circular. Intuitively, independence means that the events
can't depend on each other. I flip two separate coins, so the flip of one
can't affect the flip of the other.

``Depends on'' also has a mathematical notation. While the probability function
took a single event and returned a single number, \emph{conditional
probability} is a function that takes two events and returns a probability:
$$
\mathbb{P}[A | B] = p.
$$
This equation is read as ``the probability of $A$ given $B$ is $p$''.

In a frequesting mindset, where probabilities are proportions, $\prob{A | B}$
is, on a denominator of the trials in which $B$ happened, the proportion of
trials  in which $A$ also happened.\footnote{This is another place in which my
simple definition hides controversy and subtlety. I'm using the ``Kolmogorov
definition'' of conditional probability, but there's more than one way to
approach it. Probability is a philosophically and mathematically complex notion.}
Say $n_B$ is the number of trials in which $B$ happened, $n_{AB}$ is the
number of trials in which $A$ and $B$ happened, and $n$ is the total number of
trials. Then the proportion we're talking about is $n_{AB} / n_B$.

\begin{figure}
\caption{Conditional probability as shrinking of universe}
\label{fig:conditional-probability}
\end{figure}

To get to the more general definition of conditional probability, divide top
and bottom by $n$ to get $(n_{AB}/n) / (n_B/n)$. The numerator is the
proportion of trials where $A$ and $B$ happened. The denominator is the
proportion of trials were $B$ happened. Thinking of proportions of trials as
probabilities, it follows that: $$ \prob{A | B} \defeq \frac{\prob{A \cap
B}}{\prob{B}}. $$ A pictoral way of thinking about conditional probabilities
is to say you're interested in the probability of $A$, not in the original
universe of events $\Omega$, but in the smaller universe of events $B$ (Figure
\ref{fig:conditional-probability}).

This notation clarifies our definition of independence. If $A$ and $B$ are
independent, then $\prob{A \cap B} = \prob{A} \times \prob{B}$, so, by the
definition, $\prob{A | B} = \prob{A}$. In other words, independence means that
the probability of one event doesn't depend on the other.
