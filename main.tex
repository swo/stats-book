\documentclass{book}

\usepackage{amsmath,amsfonts,amssymb}
\usepackage{bm}
\usepackage{mathtools} % for dcases
\usepackage{hyperref}
\usepackage{longtable}
\usepackage{booktabs}
\usepackage{color,soul} % for hl
\usepackage{fullpage}
\usepackage{xfrac} % for sfrac

\providecommand{\tightlist}{\setlength{\itemsep}{0pt}\setlength{\parskip}{0pt}}

\newcommand{\prob}[1]{\mathbb{P}\left[{#1}\right]}
\newcommand{\expect}[1]{\mathbb{E}\!\left[{#1}\right]}
\newcommand{\var}[1]{\mathbb{V}\left[{#1}\right]}
\newcommand{\cov}[1]{\mathrm{Cov}\!\left[{#1}\right]}
\newcommand{\defeq}{\stackrel{\text{def}}{=}}
\newcommand{\rveq}{\stackrel{\text{rv}}{=}}
\newcommand{\dd}{\, \mathrm{d}}

\title{A short introduction to advanced statistics for scientists}
\author{Scott W. Olesen}

\begin{document}
\maketitle

\tableofcontents

\frontmatter

%!TEX root=main.tex

\chapter{Preface}

As a graduate student and postdoc in the life sciences, I saw that many of my
colleauges had substantial training and experience in quantitative methods. They were
very able to hack together sensible ways to approach their statistical problems.
I found, however, that there was often a
steep drop-off in their ability to apply statistical rigor to these \textit{ad hoc}
methods.

I think this gap between the ability to hack something together and the
understanding to make something rigorous is partly born out of a gap in
educational materials. There are plenty of introductory statistics
textbooks that explain what a mean is. There are also plenty of statistical
test cookbooks that well you what assumptions are made when using a $t$-test.
And finally, there are plenty of books and articles on statistics meants for people
with a graduate-level education in statistics or math. However, there are few
resources for people who are mature and shrewd quantitative thinkers but
who do not have a half dozen statistics courses under their belt.

To me, this situation is analogous to when I tried to learn a foreign language as an
adult. It was easy for me to find books written for children. These books are at my
reading level in terms of grammar and vocabulary, but they are thematically boring.
On the other hand, books for adults are thematically interesting but completely
intractable in terms of vocabulary and grammar.
I wanted to write a book about statistics that is
thematically \emph{and} ``gramatically'' appropriate.

I am not a statistician, and this is not a book for people who want to push the
boundaries of what
statisticians think are interesting problems. It is also not
a cookbook that tells you what statistical test to run on your data. There
are plenty of those already.
Instead, my goal is to give you the ability to reason about why to do a statistical
test and how to
formulate your own. Rather than telling you which steps to generate a $p$-value
in a specific case, I show how $p$-values come about in general and how to
derive them for specific cases.

I hope you find it useful!


\mainmatter

%!TEX root=main.tex

\chapter{Introduction to statistical testing}

\section{Statistical testing is young}

Statistical hypothesis testing, or ``significance testing'', is one of the most
widespread methods in science. If you pick up an article of \textit{Science} or
\textit{Nature} and read any scientific article, there will almost certainly
be a $p$-value in
it. I put statistical tests after the incredibly fundamental concepts of
observation, quantitation, and experimentation as the building blocks of
contemporary science.

Of these fundamental scientific concepts, statistical testing is by far the youngest.
In the
Western scientific tradition, observation as a method goes back to Aristotle
(b. 384 BC). Other ancient Greeks were using quantitative measurements and
geometry to
determine things like the diameter of the Earth. Experimentation may as we now
think of it was championed by Francis Bacon (b. 1561 AD) but probably has much
older roots. Our earliest examples of statistical hypothesis testing date to
the 1700s ---one of the early examples will be explored in a later chapter---
but testing as we now understand it was formalized by Ronald Fisher (b. 1890),
Jerzy Neyman (b. 1894), and Egon Pearson (b. 1895).

The fact that a young method, whose formalization is less than 100 years old,
has permeated nearly all of science speaks to its intellectual appeal. I think
it also clarifies why statistical testing and interpretation of $p$-values is
such a controversial issue in contemporary science: we as a scientific
community simply have not had enough time to fully digest the idea. Not only is
the conceptual basis of statistical testing not fully refined, we have also
not found the best ways to explain this concept to one another.

\section{Statistical inference is philosophically confusing}

Not only is statistical testing a relatively young idea, but statistical
inference as a conceptual approach is deeply entwined with some of the
fundamental philosophical questions that underly probability theory.  In my
experience, the students and practitioners that are my target audience for this
book are challenged by these philosophical problems of statistical inference
just as much as they stuggle with the mathematical aspects of statistical
testing!

Rather than sweep those philosophical questions under the rug, I will lay them
out, to avoid confusion later.


\part{Probability}

%!TEX root = main.tex

\chapter{Functions}
\label{chapter:functions}

It may seem weird to start a book about statistics with something as abstract
as ``functions'', but I think that probability theory becomes very confusing if you
are not entirely comfortable with the formalism of mathematical functions.

As an example, as they are typically meant in statistical writing, $x=2$ is a
true or false statement, while $X=2$ is an \emph{event}, which we will define
later. This distinction often causes enormous confusion. After reading this
chapter, you should comfortable with the idea that the ``equals'' sign in the
statements $x=2$ and $X=2$ are actually different functions.

\section{Functions map things to other things}

In mathematical terms, a \emph{function} is relationship between sets, linking
each element of one set to an element of another set. For example, say I have a
function $f$ that links each number to the that number plus one:
\begin{gather*}
f(1) = 2 \\
f(2) = 3 \\
\text{etc.}
\end{gather*}
This function is relatively simple in that it has only one input and it links
elements of a set, the integers, onto other members of the same set. As we will
see, functions can be much more complicated than this.

For many people, especially those used to computer programming, it may be more
natural to think of a function as a factory: $f$ ``takes'' an input number $x$,
adds $1$ to $x$, and then ``returns'' the output $x+1$. This analogy can be
confusing because, in computer science, a ``function'' can actually refer to
different things. The correct analogy to the mathematical concept of
``function'' is called a ``pure'' function in computer science, where ``pure''
means that the function is guaranteed to produce the same output whenever given
the same input.\footnote{A pure function also has no ``side effects'', which
means that calling that function cannot change anything about the statement of
the program.} For example, a random number generator, while it might be called
a ``function'' in a program, is not a pure function. To the user's eye, a
random number generator is called multiple times, with no input, and returns
different numbers every time.

Some functions, like my example above, map one number to another, single
number. Other functions take two numbers and return one number. For example, say
the function $g$ adds two numbers:
\begin{equation*}
g(1, 2) = 3
\end{equation*}
In fact, this is just the ``plus'' function that we usually write with \emph{infix
notation}, meaning that we put the operator between the arguments, like $1+2$,
rather than at the front, like $\mathord{+}(1, 2)$.

A function can take something that is not a number and return a number. For
example, say the determinant of a matrix $A$ is $5$:
\begin{equation*}
\mathrm{det}(A) = 5
\end{equation*}
The reverse is of course possible. For example, a function could take as input
a positive integer and return the identity matrix of that size:
\begin{equation*}
    \mathrm{Identity}(3) = \begin{bmatrix} 1 & 0 & 0 \\ 0 & 1 & 0 \\ 0 & 0 & 1 \\ \end{bmatrix}
\end{equation*}

Or a function can take numbers and return something that is not a number. For
exmaple, the statement ``$5$ is less than or equal to $2$'' is false: the
operator ``less than or equal two'' takes two numbers ($5$ and $2$) and returns
a Boolean value (``false'').

A function can also take a function as input and return a number. For example,
the minimum of the function $f(x) = x^2$ is zero. A function (``minimum'') took
a function ($f$) as input and returned a number (zero).

A function can also take a function and return a different function. For
example, the mirror of $f(x)$ around the $x=0$ point is a new function $g(x) =
f(-x)$. The function ``mirror around $x=0$'' took a function ($f$) as input and
returned another function ($g$). In computer science, functions that return
functions are sometimes called ``function factories''.

I hope it's clear that the inputs or outputs could be anything. I could even
say that I have a function $h$ that maps from the space of quadrilaterals to
the space of vegetables:
\begin{equation*}
h(\lozenge) = \text{Brussels sprouts}.
\end{equation*}
There is a special notation to make it clear what kinds of spaces a function
maps from and to:
\begin{equation*}
h : \text{quadrilaterals} \to \text{vegetables}.
\end{equation*}
Perhaps more familiar, the function $f$ above maps from real numbers to other
real numbers:
\begin{equation*}
f : \mathbb{R} \to \mathbb{R}
\end{equation*}
Functions with multiple inputs, like $g$ above, get a Cartesian product
``cross'':
\begin{equation*}
g : \mathbb{R} \times \mathbb{R} \to \mathbb{R}
\end{equation*}



\section{Functions are written in different ways}

Confusingly, we write functions in many different ways. The most obvious is the
``prefix'' or ``Eulerian'' style, where single-letter, italic text functions or
3-ish-letter, normal text functions precede their inputs:
\begin{gather*}
f(1) = 2 \\
\sin \pi = 1 \\
\mathrm{add}(1, 2) = 3
\end{gather*}



Sometimes the ``dot'' notation, like $f(\cdot)$, is used to distinguish the
function $f$ from the number $f(x)$ that the function outputs for the input
$x$. The dot avoids the confusion about whether $x$ is a stand-in or whether it
is an actual number by using the dot, which is clearly not a number.

But sometimes we write certain functions of two inputs with an ``infix'' style
in between the two inputs: we write $1 + 2 = 3$ rather than $\mathord{+}(1, 2)
= 3$. In a few cases, we write a function as ``around-fix''. For example, the
determinant of the matrix $A$ is often written $|A|$. We even use a ``postfix''
for things the factorial function: $4! = 24$.

Don't be alarmed. These are all just functions, mappings from one set of things
to another set of things. If you see some notation you do not understand, ask,
``What is this a function of? What kinds of things is it mapping from and to?''

\section{Different functions are written in the same way}

Computer scientists are likely comfortable with \emph{polymorphism}, the
notion that the same ``function'', applied to different inputs, takes different
actions. For example, in many programming languages, \texttt{+} is both
numeric addition and string concatenation: \texttt{1 + 2} returns \texttt{3},
but \texttt{"Stats " + "rules"} gives \texttt{"Stats rules"}. Somehow the
``function'' \texttt{+} knew to add in one case and to concatenate in the
other.

We do the same thing with fairly common symbols, like ``equals'' ($=$). For
example, say $f(x) = x^2$. Then $f(2) = 4$ is true but $f(2) = 5$ is false.
The ``equals'' in $f(2) = 5$ is a function that takes two numbers, $f(2)$ and
$5$, and returns the Boolean ``false''. Extra confusingly, the ``equals'' in
$f(x) = x^2$ is not a function at all; it is a signal to you, the reader, that
a function is being defined! I try to avoid this confusion in this book by
writing ``def'' on top of any equal sign that is a mathematically important
definition.

Polymorphism arises in a few critical and very confusing moments in probability
theory. For example, if I say ``the probability that $X$ equals two'', you will
respond with a number between zero and one. Say the value is one-third:
\begin{equation*}
\prob{X = 2} = \tfrac{1}{3}
\end{equation*}
Apparently, ``probability'' is a function that maps statements like
``$X$ equals two'' to a number:
\begin{equation*}
\mathbb{P} : \text{statements like ``$X=2$''} \to [0, 1]
\end{equation*}
Now if I say, ``the probability that one equals two'', you will say ``zero'':
\begin{equation*}
\prob{1 = 2} = 0
\end{equation*}
This makes it seem like ``$X=2$'' and ``$1=2$'' are the same
kinds of statements.

However, in another sense, ``one equals two'' is clearly ``false'', but
``the probability of false is zero'' doesn't make a lot of sense. The probability
function does not take Booleans; it takes something called \emph{events} that
will be defined in the next chapter. The point is that the ``equals'' in ``$X=2$''
means a different thing from the ``equals'' in ``$1=2$'', since ``$X=2$'' need
not resolve to simply ``true'' or ``false''.

%!TEX root = main.tex

\chapter{Probability}
\label{chapter:probability}

To understand statistics, it's critical to have a grasp of basic concepts in
probability, which is an entire branch of mathematics. Here I will give you
the essentials so you can have a grasp of probability theory.

\section{Probability has two definitions}

We talk about and reason about probability every day, mostly for matters of
prediction. Given the weather report, is it worth it for me to carry an
umbrella? Given the results of my experiment, how likely is it that my
hypothesis was correct?

We intuitively think about probability in a mostly \emph{Bayesian} way.
Statistics, when not prefaced by the word ``Bayesian'', refers to a different
philosophical branch call \emph{frequentist} statistics. The philosophical
distinction between Bayesian probability and frequentist probability has
practical implications, so it's important to understand the difference.

In Bayesian statistics, \emph{probability} refers to a sense of confidence
about an unknown thing, often a future event. If I say the probability of an
event is 1\%, it means, in a practical sense, that I'm willing to bet a fair
amount of money it won't happen. Philosophically, it's hard to say what ``1\%
confident'' really means, but the concept should be very intuitive.

The mathematically and philosophically simpler definition of probability is as
a proportion or ``frequency''. A 50\% probability of flipping a coin and seeing
heads means that, as you flip the coin more and more times, the proportion of
flips that come up heads will approach 50\%. One might equivalently say that
the ``frequency'' of heads approaches 50\%. This is called the
\emph{frequentist} definition of probability.


\section{The frequentist system is more coherent but dissatisfying}

The problem with the frequentist definition is that probabilities can only
be assigned to experiments or situations that can be repeated infinitely many
times. It does not make sense to ask about the probability that it will rain
tomorrow any more than it makes sense to ask about the probability that it
rained yesterday: it either rained or it didn't, either it will rain or it
won't.

The fact that we live in just one universe means that, in a frequentist scheme,
you cannot ask about the probability of a state of nature. You cannot ask about
the probability your hypothesis is correct. If you get anything other than zero
or one, you did the math wrong.

As a scientist, this is deeply dissatisying.  The whole point of statistical
inference is to figure out what's going on in the world. I don't want to feed
my hard-won experimental data into a statistical algorithm that says, ``If your
hypothesis is true, then it is; and if it's not, it's not.''

The Bayesian approach is more appealing in that it does in fact allow you to
ask about the probability of states of nature. There is such a thing as a
Bayesian probability that it will rain tomorrow or that your scientific
hypothesis is correct.

Unfortunately, Bayesian statistics is more mathematically and technically
challenging. It also requires an assertion of a \emph{prior probability}: you
must state the probability that your hypothesis was true before you started the
experiment. This idea is bizarre and repugnant to many scientists today. More
importantly, some early, prominent statisticians thought it was nonsense and
mostly directed the field of statistics away from the Bayesian approach for
many years.

Perhaps not surprisingly, frequentist statistics are more common.
``Statistics'' means frequentist statistics unless someone explicitly says
``Bayesian''. I follow this convention and will discuss only frequentist
statistics. Sorry.


\section{The mathematical definition of probability}


Above we said that the frequentist definition of probability has to do with the
proportions of infinitely-repeatable trials that have some \emph{outcome}. An
outcome is any thing that can happen as a result of an infinitely-repeatable
trial. For example, if you flip a coin, you will get heads, or you will get
tails.

The \emph{sample space}, written $\Omega$, consists of all possible subsets of
\emph{outcomes}, each written $\omega$. Members of the sample space are called
\emph{events}.  Clearly, each outcome is itself an event: ``flipped heads'' and
``flipped tails'' are both events. But the empty set $\varnothing$ is also an
event (``nothing happened''), and the set of all outcomes $\Omega$ is also an
event (``something happened''). Some other examples of outcomes, events, and
sample spaces are in Table \ref{tab:outcome-event}. I emphasize that
\emph{outcomes} are individual, real things that might happen, while
\emph{events} are abstract groupings of outcomes.

\begin{table}
\centering
\begin{tabular}{p{3cm}p{4cm}p{6cm}}
\toprule
Situation & Outcomes & Events \\
\midrule
Flipping a coin & heads, tails & $\varnothing$, $H$, $T$, $\Omega$ \\
Rolling a die & $1, 2, \ldots, 6$ & $\varnothing$; $1, 2, \ldots, 6$; $\binom{6}{2}$ 2-event outcomes (e.g., 1 or 2); $\binom{6}{3}$ 3-outcome events (e.g., 1, 2, or 3); $\ldots$; $\binom{6}{5}$ 5-outcome events (e.g., any except 1); $\Omega$ \\
Drawing a card & 2 of Clubs, $\ldots$, Ace of Hearts & $\varnothing$; 52 individual events (e.g., 2 of Clubs); $\binom{52}{2}$ 2-outcome events (e.g., black Ace); $\ldots$; $\binom{52}{51}$ 51-outcome events (e.g., any except 2 of Clubs); $\Omega$ \\
Pick a random number between 0 and 1 & All real numbers between 0 and 1 & $\varnothing$; intervals like $[0, \tfrac{1}{2}]$; ``all subsets'' of $[0, 1]$; $\Omega$ \\
Your experiment & Each possible configuration of atoms at measurement & ``Cancer cell died'', ``Electron had energy between 1 and 2 eV'', etc. \\
\bottomrule
\end{tabular}
\caption{Distinction between outcomes and events for various example
situations.}
\label{tab:outcome-event}
\end{table}

As an aside, I note that defining ``all possible subsets of outcomes'' is
straightforward for a discrete case like flipping a coin or drawing a card, but
it is complicated for a continuous case like drawing a random real number
between $0$ and $1$. In continuous cases, a rigorous definition for the event
space $\Omega$ requires concepts from \emph{measure theory}, including a
$\sigma$-\emph{algebra}. I avoid these complexities in this chapter at the cost
of presenting a mathematically incomplete theory of probability.

\emph{Probability} is a function that maps events to numbers between $0$ and
$1$:
\begin{equation*}
\mathbb{P} : \Omega \to [0, 1]
\end{equation*}
For example, if $H$ is the event of flipping heads, then $\prob{H} =
\tfrac{1}{2}$. I use the square brackets to emphasize that $\mathbb{P}$ is a
function of something other than numbers: $H$ is not a number like $5$, it is
an \emph{event}, a distinct mathematical object.

Confusingly, ``probability'' refer both to the function $\mathbb{P}$ as well as
the number $p = \prob{\omega}$ that comes from applying this function to an
event $\omega$.


\section{Manipulating probabilities}

\subsection{Probability functions follow certain axiomatic rules}

Comfortingly, probability functions are axiomatically required to be defined so
that that the probability that ``nothing happened'' is zero and the probability
that ``something happened'' is one:
\begin{align*}
\prob{\Omega} &= 1 \\
\prob{\varnothing} &= 0
\end{align*}
Something must happen, and nothing cannot happen.

The axioms about probability functions also mean they follow specific rules
around manipulating probabilities. For example, we're often interested in the
relationships between events. What is that probability that this \emph{or} that
happened? What is the probability that this \emph{and} that happened?

\subsection{``Or'' adds event probabilities}

If $A$ and $B$ are two events that don't have any constituent outcomes in
common, we call then \emph{disjoint}, and their probabilities add. For example,
the probability of flipping a heads \emph{or} flipping a tails is the
probability of heads plus the probability of tails.  Mathematically we write
this as
$$
\text{if } A \cap B = \varnothing \text{, then } \prob{A \cup B} = \prob{A} + \prob{B},
$$
where the ``cap'' $\cap$ means \emph{intersection} (``and'') and the ``cup'' $\cup$ means
\emph{union} (``or''), so this reads ``if no outcomes are in both $A$ and $B$,
then the probability of $A$ or $B$ is the sum of their individual probabilities.''

If $A$ and $B$ do have some overlap, you need to subtract out the probability
of the overlap. For example, consider drawing a card
from a standard 52-card deck. What is the probability of drawing a Jack \emph{or} a
Diamond? If you add up the probability of drawing a Jack and the probability
of drawing a Diamond, you end up double-counting the Jack of Diamonds event,
so the solution is to subtract out the double-counted event:
$$
\prob{A \cup B} = \prob{A} + \prob{B} - \prob{A \cap B}.
$$
We mostly deal with disjoint events, so I won't belabor this point.

\subsection{``And'' multiplies events}

We just said that ``or'' for disjoint events adds probabilities. How do we
find the probability of $A$ and $B$? Say I flip two coins. What's the
probability that I flip heads on the first coin \emph{and} heads on the second?

If $A$ and $B$ are \emph{independent} events, then their probabilities
multiply. The probability of flipping two heads in a row is $\tfrac{1}{2} \times
\tfrac{1}{2} = \tfrac{1}{4}$.

\subsection{Independence and conditional probability}

Mathematically, $A$ and $B$ are indepedent if and only if their probabilities
multiply. This might feel circular, so I will show you the logic.

Intuitively, two events are independent is they do not depend on each other. If
I flip two separate coins, the flip of one coin can't affect the flip of the
other.

Perhaps surprisingly, ``depends on'' hits on a deep philosophical question.
Just as ``probability'' had frequentist and Bayesian definitions, so there are
multiple definitions of \emph{conditional probability}. The easiest
(``Kolomogorov'') definition is:
\begin{equation*}
\prob{A | B} = \frac{\prob{A \cap B}}{\prob{B}},
\end{equation*}
where $\prob{A | B}$ is pronounced ``the probability of $A$ given $B$''.

Although the definition of conditional probability is a mathematical axiom that
cannot be proven or disproven, it is easy to get an intuitive picture of why it
is chosen as an axiom. In a frequentist interpretation of probability, $\prob{A
| B}$ is, on a denominator of the trials in which $B$ happened, the proportion
of trials in which $A$ also happened. Say $n_B$ is the number of trials in
which $B$ happened, $n_{AB}$ is the number of trials in which $A$ and $B$
happened, and $n$ is the total number of trials. Then the proportion we're
talking about is $n_{AB} / n_B$, which corresponds to $\prob{A \cap B} /
\prob{B}$. I like to think of conditional probability as ``zooming in'' to a
smaller sample space: rather than thinking about all of $\Omega$, we are
thinking only about the subset of outcomes $\omega$ where $B$ happened (Figure
\ref{fig:conditional-probability}).

\begin{figure}
\caption{Conditional probability as shrinking of universe}
\label{fig:conditional-probability}
\end{figure}

If $A$ and $B$ are independent, then $A$ does not ``depend on'' $B$: the
probability of $A$ given that $B$ happened is equal to the probability of $A$
without knowing anything about $B$: if $A$ and $B$ are independent, then
$\prob{A | B} = \prob{A}$, and thus $\prob{A \cap B} = \prob{A} \times
\prob{B}$.

%!TEX root=main.tex

\chapter{Random variables}

In the previous chapter, we examined outcomes, events, and their
probabilities. Outcomes are specific configurations of atoms, which is not at
all a practical thing for a scientist think about. Events are abstract
groupings of those configurations, which is also impractical. Instead, in
science, we measure the results of experiments with numbers. We link whatever
we see with a number. In microbiology, we could cells. In physics, we measure
the energy of an electron.

In probability theory, \emph{random variables} link outcomes with numbers.
They are the mathematical analogue of scientific measurement. Random variables
will lead us to familiar concepts like means, standard deviations, and
probability distributions.

\section{Definition of random variables}

Confusingly, a ``random variable'' is a function, not a number. We might say
that a random variable ``takes on'' a value, but the random variable itself,
in mathematical terms is always a function.

A \emph{discrete} random variable $X$ is a function that associates outcomes
$\omega \in \Omega$ with a finite set of numbers $x_i$:
\begin{gather*}
X : \Omega \to \{x_1, \ldots x_k\} \\
X[\omega] = x_i
\end{gather*}
for some finite $k$. A \emph{continuous} random variables associates outcomes with real
numbers $x \in \mathbb{R}$:
\begin{gather*}
X : \Omega \to \mathbb{R} \\
X[\omega] = x
\end{gather*}
This is a simplification: measure theory says that a random variable can map
the outcome space to any \emph{measurable space}. We won't dive into that.

For example, when flipping a die, you could imagine a random variable $X$
encoding ``number of heads flipped'':
\begin{align*}
X[H] &= 1 \\
X[T] &= 0
\end{align*}
Roughly speaking, the values of the random variable group the outcomes into
events, which each have probabilities. This means it makes sense to ask about
the event in which a random variable takes on a value, e.g., $\prob{X = 1} = \tfrac{1}{2}$.
Note that ``$X = 1$'' does not mean that $X$ is the number one; instead, the
entire phrase ``$X = 1$'' refers to the event where $X$ took on the value one.

Random variables therefore allow us to abstract over the outcomes and look
only at the values taken on by the random variable. From now on, we will have
no reason to think about the specific configuration of atoms in our
experiment. We only think about the numbers delivered by our apparatus. Soon
we'll be able to say things like ``$X$ is normally distributed'' without having
to specify what are the outcomes and events.

The previous chapter was not a waste, however, because the
rules for manipulating events and probabilities work just as well for
events like ``$X = 1$'' as they did for ``flipped heads''. The philosophical and
technical definition of probability still holds. For example, $\prob{(X = 1)
\cup (X = 0)}$, the probability that $X$ takes on the values zero or one, is the
sum of the two constituent probabilities $\prob{X=1}$ and
$\prob{X=0}$.

\section{Distribution of random variables}

How is $X$ ``distributed''? If you're like me, you think in terms of
\emph{probability distribution functions} (pdf's) and consider
\emph{cumulative distribution functions} (cdf's) as a derived quantity, but in
fact it's mathematically easier to go the reverse route. If you aren't
familiar with either, never fear.

\subsection{Cumulative distribution functions}

The first way of defining the ``distribution'' of a random variable $X$ is its
\emph{cumulative distribution function}, also called the cumulative
\emph{density} function and abbreviated ``cdf''. The cdf of a discrete random
variable $X$ is the probability that the random variable takes on a value
below some threshold $x_i$:
\begin{gather*}
F_X : \{x_1, \ldots, x_k\} \to [0, 1] \\
F_X(x_i) \defeq \prob{X \leq x_i} = \sum_{j \,:\, x_j \leq x_i} \prob{X = x_j}
\end{gather*}
A continuous random variable has a cdf that takes real numbers:
\begin{gather*}
F_X : \mathbb{R} \to [0, 1] \\
F_X(x) \defeq \prob{X \leq x},
\end{gather*}
Note that the cdf is a function of numbers, which I emphasize by
using regular parentheses around $x$ in $F_X(x)$.

Take careful note that ``$X \leq x$'' is an event, not a Boolean expression
like ``$1 \leq 2$''. This may seem pedantic, but I think having a really good
grasp of the notation will help you articulate correct thoughts more clearly.
Things will quickly get confusing if you think that $\prob{X \leq x}$ means the same
thing as $\prob{x \leq x}$.

% swo: need a figure here with the laddering

A discrete random variable takes on a maximum value $x_\mathrm{max}$ so that
\begin{gather}
\text{if } x < x_\mathrm{min} \text{, then } F_X(x) = 0 \\
F_X(x_\mathrm{max}) = 1.
\end{gather}
If a continuous random variable can take on any real number, then  the cdf of
a continuous random variable approaches zero and one as $x$ goes out toward
infinity:
\begin{gather}
\lim_{x \to -\infty} F_X(x) = 0 \\
\lim_{x \to \infty} F_X(x) = 1
\end{gather}

% need a figure here with, say, normal distribution

\subsection{Probability distribution function}

For a discrete random variable, it's straightforward to define its
\emph{probability mass function} or ``pmf'', which is just the probability
that it takes on each discrete value $x_i$:
\begin{gather*}
f_X : \{x_1, \ldots, x_k\} \to [0, 1] \\
f_X(x_i) = \prob{X = x_i}
\end{gather*}

The analogue for continuous random variables is the probability
\emph{distribution} function, also called probability \emph{density} function
or ``pdf'':
\begin{gather*}
f_X(x) : \mathbb{R} \to [0, 1] \\
f_X(x) \defeq \frac{d}{dx} F_X(x)
\end{gather*}

Note that the pdf of a continuous random variable is not $\prob{X = x}$. This
may seem like a pedantic diversion, but I actually think it's important to
avoid confusion. For a continuous random variable, $\prob{X = x}$ is zero. For
example, say $X$ takes on values between 0 and 1 symmetrically, e.g., the
probability that $X$ comes out between 0 and 0.5 is the same as between 0.5
and 1. Say you're asking about the probability that $X$ comes out as 0.5. I'll
grant the following:
\begin{align*}
\prob{0 \leq X \leq 1} &= 1 \\
\prob{0.495 \leq X \leq 0.505} &= 0.1 \\
\prob{0.4995 \leq X \leq 0.5005} &= 0.01 \\
&\vdots
\end{align*}
and so on. We normally don't say that $\prob{X = 0.500\ldots} = 0$, since that
makes it sound like it's impossible for $X$ to take on the value $0.5$.
Nevertheless, it should be clear that, from any practical point of view, it
doesn't make sense to write things like $\prob{X = 0.5}$. It does make sense
to write $f_X(0.5)$; it does not make sense to write $\prob{X =
0.5}$.\footnote{Those dissatisfied with this explanation will need to learn
some measure theory.}

The word ``density'' in probability density function emphasizes that the pdf,
when integrated, gives probabilities:
\begin{equation}\label{eq:integrated_pdf}
\int_{x_0}^{x_1} f_X(x') \,dx' = F_X(x_1) - F_X(x_0) = \prob{x_0 < X \leq x_1}
\end{equation}
If $X$ can take on any value, then it follows from the definition of the cdf and
some fundamental calculus that
\begin{equation}
\int_{-\infty}^\infty f_X(x') \,dx' = 1
\end{equation}
that is, that all the probability must be somewhere, and
\begin{equation}
\lim_{x \to \pm \infty} = 0.
\end{equation}

These definitions I've given are simple ones, and they need to be refined to
deal with more complicated functions. For example, if the cdf has
discontinuities, you need careful with the limits on the integrals.

\subsection{Known distributions}

If a random variable $X$ has the same cdf as another random variable $Y$ (and
therefore also the same pdf), we say that $X$ is ``distributed like'' $Y$:
$$
X \sim Y.
$$
This notation is used to refer to well-documented distributions. For example,
you might say that $X$ is distributed like a normal random variable with mean
$\mu$ and variance $\sigma^2$:
$$
X \sim \mathcal{N}(x; \mu, \sigma^2),
$$
which is a shorthand for saying
\begin{equation*}
f_X(x) = f_\mathcal{N}(x; \mu, \sigma^2) \defeq
  \frac{1}{\sqrt{2\pi\sigma^2}} \exp \left\{-\frac{1}{2} \frac{(x-\mu)^2}{\sigma^2} \right\}.
\end{equation*}
Similarly $X \sim \mathrm{Binom}(n, p)$ means that $X$ follows a binomial distribution:
\begin{equation*}
f_X(x) = f_\mathrm{binom}(x; n, p) \defeq \binom{n}{x} x^p (n-x)^{1-p}.
\end{equation*}
We'll come back to these distributions and their pdf's later; I just want to
you understand that, when we say that ``$X$ is normally distributed'', we
making a mathematically precise statement about $X$'s pdf and cdf.

\subsection{Independent, identically-distributed random variables}

A common construct is to consider \emph{independent, identically-distributed}
(or ``iid'') random variables. This means you have a collection of random variables $X_i$
that all have the same cdf and are all independent of one another. For example,
\begin{equation*}
X \stackrel{\text{iid}}{\sim} \mathcal{N}(\mu, \sigma^2)
\end{equation*}
means that $f_{X_i}(x) = f_\mathcal{N}(x; \mu, \sigma^2)$ for every random variable
$X_i$ and that all the $X_i$ and $X_j$ are independent for all $i \neq j$.

Independent, identically-distributed random variables are important for
frequentist statistics because they represent the ``many independent trials''
in the frequentist definition of probability as a proportion of many
independent trials in which an outcome emerges. As the number of iid variables
increases, the proportion of discrete iid random variables that take on each
discrete value will approach the probability of that value. Similarly, the
proportion of continuous iid random variable that take on values in a range $a
\leq X < b$ will approach $F_X(b) - F_X(a)$.

In day-to-day speech, we might say, ``I draw $n$ values from this random
variable $X$''. Mathematically, this means, ``Consider $n$ iid random
variables distributed like $X$''.

% swo: do events or outcomes get probabilities?

\section{Sums and products of random variables}

If you know the cdf and pdf for a random variable $X$, and you also know these
values for another random variable $Y$, you might be interested in the
behavior of these variables together. An important example is the sum of many
independent, identically-distributed random variables, which we will see are
intimately linked to the normal distribution. Many specific distributions have
nice shortcuts for figuring out the behavior of sums of random variables with
those distributions. For example, if $X \sim \mathcal{N}(\mu_1, \sigma_1^2)$
and $Y \sim \mathcal{N}(\mu_2, \sigma_2^2)$, then $X + Y \sim
\mathcal{N}(\mu_1 + \mu_2, \sigma_1^2 + \sigma_2^2)$. Where do those nice
rules come from?

First, let's be clear about the notation. If $X$ and $Y$ are random variables,
then what does $Z = X + Y$ mean? Remember that $X$ and $Y$ are both functions,
so the plus sign in $X + Y$ doesn't mean ``add numbers''. Instead, it's an intuitive shorthand:
\begin{equation}
Z = X + Y \text{ means } Z[\omega] = X[\omega] + Y[\omega].
\end{equation}
That is, for every outcome $\omega$, $Z$ takes on the sum of the values that
$X$ and $Y$ take on. In other words, $Z$ is a function that, for every input
$\omega$, outputs the sum of the outputs of $X$ and $Y$ when they in turns are
fed the input $\omega$.

This means that, for some event, if $X$ takes on the value $x$ and $Y$ takes
on the value $y$, then $Z$ takes on $x+y$. It follows that, if
$Z$ takes on the value $z$ and $X$ takes on $x$, then it must be that $Y$
took on $z - x$. Therefore the probability that $Z$ takes on value $z$ is
sum of the probabilities that $X$ took on $x_i$ and $Y$ takes on $z - x_i$.
If $X$ and $Y$ are independent,
$\prob{(X = x_i) \cap (Y = z - x_i)} = \prob{X = x_i} \times \prob{Y = z - x_i}$
so that
\begin{align*}
f_Z(z) &= \sum_i \prob{(X=x_i) \cap (Y=z - x_i)} \\
  &= \sum_i \prob{X=x_i} \prob{Y=z-x_i} \\
  &= \sum_i f_X(x_i) f_Y(z - x_i).
\end{align*}

A similar definition holds for when $Z = X + Y$ is continuous. Rather than
summing over all the outcomes in which $X$ takes on certain values, we integrate:
\begin{equation}
f_Z(z) = \int f_X(x') f_Y(z - x') \,dx'
\end{equation}

For a product of random variables $Z = XY$, we say that,
if $Z = z$ and $X = x$, then it must be that $Y = z/x$.
For a ratio of random variables $Z = X/Y$, it must be that $Y = zx$.

For any particular $X$ and $Y$, the derived variables $X + Y$, $XY$, and $X/Y$
are usually distributed in some really messy way. For example, although the
sum of two normally-distributed random variables is normally distributed,
neither the product nor the ratio of normally-distributed random variables are
normally distributed.

\section{Properties of random variables}

Two properties of random variables, their \emph{expected values} and
\emph{variances}, will come up repeatedly.

\subsection{Don't expect the expected value}

A random variable's \emph{expected value} is the probability-weighted average
of the values it takes on. It is a function that links random variables with
numbers:
\begin{equation*}
\mathbb{E} : \text{random variables} \to \mathbb{R}
\end{equation*}
For a discrete random variable, you simply sum. For a continuous random
variable, you need to integrate:
\begin{equation}
\expect{X} = \begin{dcases}
  \sum_i x_i f_X(x_i) & \text{ for discrete random variables} \\
  \int x' f_X(x') \,dx' & \text{ for continuous random variables}
\end{dcases}
\end{equation}
You might also see people use notation like $E[X]$, $\mathbb{E}X$, or $\mu_X$.

% swo: expected vs. expectation. go with "expected"

The name ``expectation value'' is a little confusing. We previously noted that,
for continuous random variables, the probability of getting any particular
number is essentially zero, so there's no particular number you should
``expect''. For discrete random variables, where there is finite probability
of getting each particular number, the expected value need not be any of the
values actually taken on by $X$. In our coin flip example, the random variable
$X$ measuring the number of heads flipped has expectation value:
\begin{equation}
\expect{X} = 0 \times \prob{T} + 1 \times \prob{H} = \tfrac{1}{2}.
\end{equation}
You certainly don't ever expect to flip $0.5$ heads.

Even if you don't expect the expectation value, it is a handy mathematical
tool, and it is a rough of measure of what kind of values $X$ is
``centered'' around. Even this is slippery, since, as you might
recall from learning in high school about the difference between mean, mode, and median, what ``the
center'' means is not necessarily obvious.

\subsection{Linearity of expectation}

Expected values have some nice properties that make them really easy to deal
with. For example, the expectation value of a shifted random variable like $X +
1$ is just $\expect{X} + 1$: you're still ``averaging'' $X$ overall the
outomes, but you're just adding 1 to everything you averaged, which is clearly
just the ``average'' $X$ plus 1. Similarly, you should be able to see that
$\expect{2X} = 2 \,\expect{X}$.

You could think of ``1'' in the previous example as being a really boring
random variable: it gives 1 for every outcome. This might lead you to suspect
that $\expect{X + Y} = \expect{X} + \expect{Y}$, which is true. The fact that
you can easily transform the expectation values of random variables multiplied by
numbers and the sum of random variables is called the \emph{linearity of expectation}:
\begin{equation}
\expect{aX + bY} = a \,\expect{X} + b \,\expect{Y}.
\end{equation}
The proof of this fact follows from the logic used to find the pmf of $Z = X +
Y$. It's not hard, but it's not particularly enlightening either.

Interestingly, the product $XY$ of two random variables is nice only if the
variables $X$ and $Y$ are independent. In that case, $\expect{XY} = \expect{X}\,\expect{Y}$.
More generally, there is some residual term call the \emph{covariance}:
\begin{equation}
\expect{XY} = \expect{X} \, \expect{Y} + \cov{X, Y}.
\end{equation}
Before getting too much into covariance, we should start with regular variance.

\subsection{Variance}

If the expected value is some measure of position, then \emph{variance} is the
corresponding measurement of spread. It is a function of random variables that
returns a nonnegative number:
\begin{equation*}
\mathbb{V} : \text{random variables} \to [0, \infty).
\end{equation*}
The variance of $X$ is also often written $\mathrm{Var}[X]$, $V(X)$, or $\sigma^2_X$.

For a random variable $X$, the variance is the expected value of the square of
the deviation of the random variable from its own expected value:
\begin{equation*}
\var{X} \defeq \expect{(X - \expect{X})^2}
\end{equation*}
Because all those nested brackets are confusing, people sometimes write
$\expect{X}$, which is just a number, as $\mu$ so that $\var{X} = \expect{(X -
\mu)^2}$.

Variance is always nonnegative because it's a weighted average over squares,
which are always nonnegative. The variance of a random variable is zero only
in the very boring case where that random variable always takes on the same
value.

Note that the square in $(X - \expect{X})^2$ is a shorthand for taking the
product of two iid random variables: $X^2$ refers to the product $X_1 X_2$,
where $X_1$ and $X_2$ are distributed like $X$.

Unlike the expected value, variance is not linear. Consider the random variable $aX$,
where $a$ is just a number:
\begin{equation}
\var{aX} = \expect{(aX - \expect{aX})^2} = \expect{a^2(X - \expect{X})} = a^2 \, \expect{X}.
\end{equation}
For a sum, the situation is even more confusing. Some algebra shows that
\begin{equation*}
\var{X + Y} = \var{X} + \var{Y} + 2 \,\cov{X, Y},
\end{equation*}
where $\cov{X, Y}$, introduced in the end of the last section, is the
\emph{covariance} of $X$ and $Y$:
\begin{equation}
\cov{X, Y} \defeq \expect{(X - \expect{X})(Y - \expect{Y})}.
\end{equation}

In words, the covariance is the expected value of the product of the
deviations of $X$ and $Y$ from their expected values. If, when $X$ takes on
values greater than its expected value, $Y$ also tends to take on values
greater than its expected value, then $\cov{X, Y} > 0$. Conversely, if, when
$X$ has positive deviations, $Y$ has negative deviations, then $\cov{X, Y} <
0$. If $X$ and $Y$ are independent, then $\cov{X, Y} = 0$.

\subsection{Correlation}

Rather than covariance, you're probably more familiar with the term
\emph{correlation}, especially the familiar Pearson's correlation coefficient,
usually written $\rho$ or $r$. The correlation coefficient is a function of
two random variables that gives a number between $-1$ and $1$:
\begin{gather*}
\rho : (\text{random variables}) \times (\text{random variables}) \to [-1, 1] \\
\rho[X, Y] \defeq \frac{\cov{X, Y}}{\sqrt{\var{X} \var{Y}}},
\end{gather*}
In other words, the correlation between $X$ and $Y$ is their covariance
``normalized'' by their variances. The reason for dividing by the two
variances was so that $\rho$ is between $-1$ and $1$. The math isn't hard, but
it's not particularly useful for us.

% swo: figure, or integrate this paragraph

To see how covariance and correlation are related, think about a correlated
$X$ and $Y$, like a simple bivariate normal distribution. \hl{figure} In
general, the points for which $X$ is greater than its mean value are the same
points for which $Y$ is greater than its mean value. Similarly, when one
deviation is negative, the other tends to be negative. This way, the overall
\emph{product} of the two deviations tends to be positive. In constrast, if
two variables are \emph{anticorrelated}, then when one is higher than average
the other tends to be lower than average, so the product of deviations tends
to be negative.


\part{Descriptive statistics}

\chapter{What is statistics?}

``Statistics'' is confusing because it means two things. It's a plural noun that
refers to multiple things, each of which is called a ``statistic''. It's also a
singular noun that refers to the study of these mathematical objects.

Statistics is a tool.\footnote{I really enjoyed reading this anecdote from a
paper by Gene Glass, the inventor of meta-analysis (doi:10.1002/jrsm.1133).
Glass. He ends up sitting on the plane next to the famous statistician
Tukey. Tukey asks, "Along which dimension do you see the greatest
variability in effects?" By investigator. "Then jackknife on investigator."
The point is that there's a wide gap between being able to understand what
the jackknife is in a mathematical sense (i.e., how to compute it, whether
it's an unbiased way to determine the variance in the median) and how to apply
it. It also goes to show that "real" statisticians can be pretty hacky!}

A \emph{statistic} is some function of the sample data. Although the field of
statistics is often described as consisting of "descriptive statistics" and
"inferential statistics"---that is, the study of mathematical objects used to
describe data and to make inferences about the populations the data were taken
from---in this book I want to emphasize that, for both purposes, we are
interested in the properties of statistics.

The most interesting statistics are the ones that are used to estimate some
property of the population. These kinds of statistics are sensibly called
\emph{estimators}. Most of the study of statistics comes down to figuring out things
about estimators. One of the most critical is understanding the variance of
estimators, which is essential for both descriptive statistics---so that you
can put error bars on your measurements---and inferential statistics---so you
can make a guess about how probable it is that your data arose under some null
hypothesis. All of this is about estimators and their variance.
%!TEX root=main

\chapter{Estimators}

\section{A statistic is a function of data}

In common speech, we use the terms like expected value, variance, covariance, and
correlation in reference to data, that is, to sets of fixed numbers, rather than
to mathematical objects like random variables. In this chapter, we will bridge the
gap between the probability theory we have considered so far and the practice of
statistics.

The term ``statistics'' has two meanings. As a singular noun, a \emph{statistic},
is a function that maps data, that is, an ordered list of numbers, to a real
number. We use $t$ as a stand-in to mean ``any old statistic'':
\begin{equation*}
    t : \mathbb{R}^n \to \mathbb{R}
\end{equation*}
To study the properties of a statistic, we will define a random variable that is
analogous to that statistic. A random variable that is analogous to a statistic is
called an \emph{estimator}, often written $T$.

For example, the arithmetic mean maps a list of numbers $x_i$ to a single number, and
so is a statistic, which I will write as $t$, such that $t \defeq (1/n) \sum_i x_i$.
There is an analogous estimator $T \defeq t(X) = (1/n) \sum_i X_i$. As I emphasized in the
previous chapter, although the notation in these two equations is almost identical,
just replacing some lowercase letters with uppercase ones, it is critical to remember
that adding random variables is not the same thing as adding numbers. The function
$t$ maps numbers to a number, so $t(X)$ is, strictly speaking, nonsense. But in the
same way that we use the plus sign when adding numbers, like in $x + y$, and for
``adding'' random variables, like in $X + Y$, we use $t$ to refer to an operation on
numbers as well as to the analogous operation on random variables.

The word ``statistics'' ending in \textit{s} could either be the plural of statistic,
that is, many functions of the data, or it can refer to the field of study of
estimators. Thus it is syntactically and semantically correct to say ``statistics
\textit{is} fun''.

Up to this point, we have discussed random variables in a deductive way:
given the distribution of the random variable, you ask
what its expected value or variance is. Estimators are the starting point of
statistics (the field of study), which goes the opposite way: given some data
points, what can we say about the distribution of the random variables that gave
rise to that data?

\section{Parameters, populations, and samples}

Some people refer to statistics (the functions) as \emph{sample statistics} to
emphasize that they are functions of a sample of data that was drawn from some
larger \emph{population}, which has some fixed an unknowable \emph{parameters}
that describe it. For example, if you draw many data points $x_i$ from a distribution
and compute the mean of the drawn data points, you do not expect that the
\emph{sample} mean that you compute will be exactly equal to the true
\emph{population} mean.

In mathematical terms, we say that a random variable $X$ has some expected value
$\mathbb{E}[X]$ that is fixed. A random variable is a function, and the expected
value is another function that links the random variable with a single number.
There is nothing ``random'' about this so far. The analogy to the random sample
are iid random variables $X_i$, so the sample mean is analogous to the estimator
$t(X)$.

Ideally, the estimator $t(X)$ will actually reflect the parameter $\mathbb{E}[X]$
that we wanted to learn about. The rest of this section will discuss what makes
a ``good'' estimator.

\section{A useful example: the ``German tank'' problem}

To examine the properties of estimators and what makes an estimator a ``good'' one,
I will use an example that is mathematically tractable but just unfamiliar enough
to make you think.

During World War II, it was important to the Allies to know how many tanks
Germany was producing. The traditional approach was to use spies and aerial
reconnaissance. The new approach was to use statistics on serial numbers,
which turned out to be much more accurate.

A serial number is a unique number written on a manufactured part. Serial numbers
are usually assigned in sequential order, so that older parts have lower
serial numbers and newer parts have higher numbers. German tanks had serial
numbers. When
the Allies captured German tanks, they took note of those numbers, which gave
them a clue about how may thanks there were. For example, if you captured three
tanks and found serial numbers 1, 3, and 5, you know there are at least 5 tanks
total, and there probably aren't more than 10 or so. If you find serial numbers
100, 300, and 500, then you know there are at least 500 tanks, and there are
probably more like 1,000.

The Allies had a fairly complex problem, because they wanted to estimate the
rate of production, and they had many serial numbers on many different parts
of the tank, some of which were not
exactly sequential. Let's instead consider an abstracted, simplified version of
this problem. Say you drew $n$ numbers from \emph{uniform distribution} ranging
from $A$ to some unknown upper limit $B$. You want to estimate $B$, which is
analogous to the number of German tanks out there.

Mathematically, we are interested in iid random variables $X$ that follow the
uniform distribution:
\begin{gather*}
    f_X(x) = \frac{1}{B-A} \text{ for } A \leq x \leq B \\
    F_X(x) = \frac{x-A}{B-A} \text{ for } A \leq x \leq B
\end{gather*}
For simplicity, let's say that $A=0$ so that $f_X(x) = 1/B$ and $f_X(x) = x/B$
for $0 \leq x \leq B$.

Our challenge is to create a statistic, a function of the data, that estimates
the parameter $B$ and then examine the mathematical properties of the corresponding estimator. It's typical to denote estimators with a
``hat'', so we will write our estimators like $\hat{B}$. It's also typical to denote
statistics with lower-case letters reminiscent of the true value, so we'll
use names like $b$. To be clear, $b$ is a function of numbers, although we also
write $\hat{B} = b(X)$, where now ``$b$'' means some manipulation of a set of iid
random variables to create a new random variable.

\section{Consistent estimators}

Let me start with what will seem like a very inane choice of statistic.
Maybe my favorite number is 3, so I will say that, regardless of what data I
collect, I will just always guess that $B$ is 3:
\begin{equation*}
    b(x_1, \ldots, x_n) \defeq 3.
\end{equation*}
The corresponding estimator $\hat{B}$ is very simple. It takes on the
value 3 with 100\% probability:
\begin{equation*}
    f_{\hat{B}}(x) = \begin{cases}
        1 &\text{if $x = 3$} \\
        0 &\text{otherwise}
    \end{cases}
\end{equation*}

This is clearly a bad estimator. No matter how much data I accumulate, my
estimate doesn't improve. I will only be ``right'' if $B$, by some miracle,
happens to be exactly $3$. By contrast, my expectation would be that, in
the limit of collecting a lot of data, the statistic $b$ would be almost
guaranteed to be very close to $B$. This requirement is called \emph{consistency}.

Mathematically, I want the probability that $\hat{B}$ takes on a value far from $B$ to get close to zero as I collect more and more data. This requires defining what a ``limit'' is for a series of random variables. For a sequence of numbers, a limit means that, for any \textit{a priori}, fixed threshold $\varepsilon > 0$, there is some integer $n$ such that every value in the sequence after the $n$-th one is within $\varepsilon$ of the true value:
\begin{equation*}
    \lim_{n\to\infty} a_n = L \text{ means that for any $\varepsilon > 0$ there exists some $n$ such that } |a_i - L| < \varepsilon \text{ for all $i \geq n$}.
\end{equation*}
For example, given the geometric series $1, \tfrac{1}{2}, \tfrac{1}{4}, \ldots$,
if you pick the threshold $\varepsilon = \tfrac{1}{100}$, I can pick $n=7$, since the terms in the series are equal to $\tfrac{1}{2^n}$, and $\tfrac{1}{2^7} = \tfrac{1}{128}$. All subsequent terms in the series will be even closer to zero than that term.

In probability, if you give me $\varepsilon$, in most cases I cannot pick an
$n$ such that the estimator maps all values outside that threshold to zero probability.
Casually speaking, even after a trillion data points, there is no guarantee that
a series of random data points won't stray from their limit.
Instead, in probability, we say that a sequence of random variables $X_n$
\emph{converges} toward a number $a$ if
\begin{equation}
\lim_{n \to \infty} \prob{|X_n - a| > \varepsilon} = 0,
\end{equation}
that is, if the probability that $X_n$
takes on a value more than $\varepsilon$ away from $a$ approaches zero as $n$
increases. An estimator is \emph{consistent} if the series of random variables
corresponding to collecting more and more data converge toward the true value.

In our example, $\hat{B}_1$ corresponds to the estimator when we only collect
1 data point, $\hat{B}_2$ when we collect 2, and so on. In the dumb example
where $\hat{B}$ only takes on the value 3, the $\hat{B}_n$ converge to 3, but not
to $B$, except in the unlikely case that $B$ just happens to be 3.

For our second guess, let's use a more reasonable statistic. Say that $b$
is the maximum of whatever data points we collected:
\begin{equation*}
    b(x_1, \ldots, x_n) = \max_i x_i.
\end{equation*}
I know I can write out an analogous equation to define the estimator:
\begin{equation*}
    \hat{B} = \max_i X_i.
\end{equation*}
But what does it mean to take the maximum of two random variables? Recall that,
we computed the distribution of a sum of two random variables $Z = X + Y$ as
$f_Z(z) = \int f_X(x) f_Y(z-x) \,dx$. For a maximum, it is easier to express the
new variable's distribution using its cdf. To say $Z = \max[X, Y]$ means that,
if $X$ and $Y$ are independent,
\begin{align*}
    F_Z(z) &= \prob{Z \leq z} \\
    &= \prob{(X \leq x) \cap (Y \leq y)} \\
    &= \prob{X \leq x} \, \prob{Y \leq y} \\
    &= F_X(x) F_Y(y)
\end{align*}
So if $F_X(x) = x/B$, then $F_{\hat{B}_n} = (x/B)^n$. Note that $\hat{B}_n$ can
only take on values between $0$ and $B$, so if it diverges from $B$, it will do
so by falling short:
\begin{align*}
    \prob{|\hat{B}_n - B| > \varepsilon}
    &= \prob{\hat{B}_n \leq B - \varepsilon} \\
    &= F_{\hat{B}_n}(B - \varepsilon) \\
    &= \left(\frac{B - \varepsilon}{B}\right)^n \\
    &= \left(1 - \frac{\varepsilon}{B}\right)^n.
\end{align*}
For $\varepsilon > 0$, this limit of this number, as $n \to \infty$, is zero,
and thus $\hat{B}_n$ converges to $B$, so $\hat{B}$ is a consistent estimator.

\section{Unbiased estimators}

It is a nice thing that our estimator $\hat{B}$ is consistent: as we get more
and more data, the statistic $b$ will end up closer and closer to the true value
$B$. It is disappointing, though, that $b$ is always an underestimate. If the
data points $x_i$ are randomly distributed between $0$ and $B$, and we take
$b = \max_i x_i$, it is necessarily the case that $b < B$. Because $b$ is not
``centered around'' $B$, we say that $\hat{B}$ is a \emph{biased} estimator.

Consider the extreme case where $n=1$: we draw only one data point $x$, which is
somewhere between $0$ and $B$. Intuitively, I might expect $x$ to be about
halfway between $0$ and $B$, that is, that $x \approx \tfrac{1}{2}B$, so that
$b$ underestimates $B$ by a factor of $\tfrac{1}{2}$. And if
$n=2$, then I might expect those points, roughly speaking, to land somewhere
like $\tfrac{1}{3}B$ and $\tfrac{2}{3}B$, so that $b$ underestimates $B$ by
a factor of about $\tfrac{1}{3}$. These factors appear to shrink as $n$ increases,
which makes sense, since $\hat{B}$ is a consistent estimator. But can we do
something to figure out, mathematically, what that factor should be?

We do this by examining the expected value of the estimator:
\begin{equation*}
    \expect{\hat{B}} = \int_0^B x \, f_{\hat{B}}(x) \, dx.
\end{equation*}
We said above that $F_{\hat{B}} = (x/B)^n$, from which it follows that
\begin{equation*}
    f_{\hat{B}}(x) = \frac{d}{dx} F_{\hat{B}} = \frac{n}{B^n} x^{n-1}.
\end{equation*}
Plugging this definition of $f_{\hat{B}}$ into the integral gives
\begin{equation*}
    \expect{\hat{B}} = \int_0^B \frac{n}{B^n} x^n \, dx = \frac{n}{n+1} B.
\end{equation*}
This result accords exactly with our intuition above: for $n=1$, the expected
value of $\hat{B}$ is $\tfrac{1}{2}B$. For $n=2$, it is $\tfrac{2}{3}B$, and
so forth. Based on this finding, we can define a new estimator:
\begin{equation}
\hat{B}_\mathrm{unbiased} \defeq \frac{n+1}{n} \max_i X_i.
\end{equation}
This ensures that our estimates for $B$ are ``centered'' around $B$, because
$\expect{\hat{B}_\mathrm{unbiased}} = B$. More generally, we say that an estimator
is \emph{unbiased} if its expected value is equal to the parameter it is estimating.

Note the subtlety in what we did: without actually knowing what $B$ is, we found
a way to adjust the way that we estimate $B$ to ensure that we come up with a more
accurate estimate. I was mystified in introductory statistics that, while we computed
variance as $\tfrac{1}{n} \sum_i (x_i - \mu)^2$, we computed \emph{sample variance}
as $\tfrac{1}{n-1} \sum_i (x_i - \mu)^2$. The standard explanation has something to
do with ``degrees of freedom'', which I never understood. There is, I think, a
more straightforward explanation, which you can now understand, which is that the
$n-1$ is there in the sample variance, instead of $n$, because it corrects for a
bias.

\section{Efficient estimators}

Consistency ensures that, in the limit of infinite data, our estimate approaches
the true value. How can we be sure that we are getting the best estimate for our
data, given the fact that we can't collect infinite data? This question relates
to the concept of \emph{efficiency}. One unbiased estimator $\hat{X}_1$ is 
more \emph{efficient} than another
unbiased estimator $\hat{X}_2$ if it has lower variance:
\begin{equation*}
    \var{\hat{X}_1} < \var{\hat{X}_2}.
\end{equation*}
As we will see, confidence intervals are related to the variance of an estimator,
so a more efficient estimator means that we can get more narrow confidence intervals
for the same amount of data, which is clearly a desirable thing. I don't want to
appear more ignorant than I have to be, just because I picked a poor estimator!

It turns out that there is a theoretical lower limit to the variance of estimators
called the \emph{Cram\'{e}r-Rao bound}, and many estimators actually hit this limit.
Therefore, rather than saying that one estimator is \emph{more} efficient than
another, we usually just say that an estimator is ``efficient'', meaning that it
hits the lower bound on variance and is maximally efficient.

The math behind efficiency is somewhat complex, so I will simply tell you that
a whole class of estimators, \emph{maximum likelihood} estimators, are efficient.

\section{Maximum likelihood estimators}

The German tank problem has a useful toy example, but it's hard to imagine deriving
estimators by hand for the kind of complex data analysis that most scientists do.
Fortunately, modern computing means that we can address all sorts of complex data
using a \emph{maximum likelihood} (ML) approach. In a maximum likelihood approach, you
specify a theoretical model for how your data are generated, and then you use optimization to estimate values for the parameters that best ``fit'' the data.

The word \emph{likelihood} is confusing because, in common speech, ``probability''
and ``likelihood'' are synonymous. In statistics jargon, the likelihood function
is actually related to the probability density function, only it emphasizes the
parameters over the data. Say we have some data $x$ that was drawn from a probability 
distribution with density function $f_X$. We often think about families of probability
distributions and name them by their parameters. For example, we say that a normal
distribution is defined by two parameters, its mean and its variance. So say that
the density function $f_X$ changes depending on some parameter (or parameters) $\theta$.
We write the density function as $f_X(x; \theta)$ to emphasize that, while the
parameters $\theta$ are relevant, we think of them as fixed. By contrast, we write
the likelihood function $L$ as $L(\theta; x) \defeq f_X(x; \theta)$. For likelihood,
we consider the data fixed and vary the parameters.

To see how this plays out, let's return to the German tank problem. Given some data
points $x_i$, what is the probability density of drawing that data, for some varying
parameter $B$? We look for the value of $B$ that maximizes the probability density
function for all the measured data points $x_i$. This kind of optimization is called \emph{argmax}. While \emph{max} asks for the maximum value of some function $f(x)$ when varying $x$, argmax asks for the value of $x$ that leads to $f(x)$ being maximized. Thus:

\begin{align*}
    \hat{B}_\mathrm{ML}
    &= \underset{B}{\operatorname{argmax}} \prod_i f_X(x_i; B) \\
    &= \underset{B}{\operatorname{argmax}} \prod_i \begin{cases}
      1/B &\text{ if } 0 \leq x_i \leq B \\
      0 &\text{ if } x_i > B
    \end{cases} \\
    &= \underset{B}{\operatorname{argmax}} \begin{cases}
      (1/B)^n \&text{ if } x_i > B \text{ for any $i$} \\
      0 &\text{ otherwise}
     \end{cases} \\
    &= \max_i x_i
\end{align*}
Note that, in the maximum likelihood approach, when computing $f_X(x_i; B)$, we assume that we know $B$, because it is the parameter we are optimizing over. This makes the maximum likelihood computation relatively simple: we cannot choose $\hat{B}_\mathrm{ML}$ to be smaller than the maximum of the observed values $x_i$, because that would make the observed data impossible, and we set $\hat{B}_\mathrm{ML}$ at exactly $\max_i x_i$ because choosing any greater value would make the observed data seem unusual. If $B$ were actually much, much higher than $\max_i x_i$, the observed data would be rare, being clustered very close to zero, relative to that very large $B$.

Note also that the maximum likelihood approach for the German tank problem gave us a consistent estimator, but not an unbiased one. Given certain mathematical necessities that are probably true for your applications, maximum likelihood estimators are consistent, but they are in general not biased.

Again, a key practical advantage of maximum likelihood estimation is that it requires only a computer and an articulation of how the data are generated.


\section{The arithmetic mean as an estimator}

The German tank problem has simple math, but it is not a familiar application. Let us consider instead a very familiar estimator, the arithmetic mean, that is, the simple average of numbers. Just as the mean of some set of numbers $x_i$ is $(1/n) \sum_i x_i$, we analogously define the estimator $\hat{\mathbb{E}}_X \defeq (1/n) \sum_i X_i$ of the mean of i.i.d. random variables $X_i$.

It is immediately clear that $\hat{\mathbb{E}}_X$ is an unbiased estimator of $\expect{X}$:
\begin{equation*}
    \expect{\hat{\mathbb{E}}_X} = \expect{\frac{1}{n} \sum_i X_i} = \frac{1}{n} \expect{X_i} = \expect{X}.
\end{equation*}
But is $\hat{\mathbb{E}}_X$ as consistent estimator of $\expect{X}$? The mathematician Jacob Bernoulli published the first proof of this fact for a special case of random variables (i.e., those following the Bernoulli distribution) in 1713. It took him 20 years to develop a rigorous proof for what he proudly called the ``Golden Theorem''. Now the proof is considerably simpler and broader, and we call it the \emph{law of large numbers}.

The modern proof of the law of large numbers follows from an observation about the variance of the estimator $\hat{\mathbb{E}}_X$ that was surprising to many early statisticians:
\begin{equation*}
    \var{\hat{\mathbb{E}}_X} = \var{\frac{1}{n} \sum_i X_i} = \frac{1}{n^2} \sum_i \var{X_i} = \frac{1}{n} \var{X}
\end{equation*}
In words, the variance of the estimator is $1/n$ of the variance of the underlying random variable it is estimating. Intuitively, this means that, as you take more data, you are ever more likely to get an estimate near to the true mean of the underlying population you are sampling. This result makes it clear that $\hat{\mathbb{E}}_X$ is a consistent estimator, because for arbitrarily high $n$, it has arbitrarily low variance.

Because the \emph{standard deviation}, the square root of the variance, is more intuitive, one can say that the standard deviation of the estimator decreases with the square root of $n$. To distinguish the standard deviation of the estimator from the standard deviation of the underlying population, $\var{\hat{\mathbb{E}}_X}$ is also called the \emph{standard error of the mean}. With more data points $n$, you can get an ever more precise estimate of the mean, so the standard error of the mean declines, but the standard error of the underlying population $\var{X}$ is fixed.

This result was surprising because it was expected that the standard error would scale like $1/n$ rather than $1/\sqrt{n}$. In other words, it was expected that, if you collected double the data, you would get double the precision, when in fact, you get only a $\sqrt{2}$ improvement, that is, a 41\% improvement. More starkly, if you take one hundred times the data, you get only a ten-fold increase in precision.

%!TEX root=main

\chapter{Confidence intervals}

\section{Definition}

So far we have worked with \emph{point estimates}, that is, guesses for
population parameters that are just a single number. This presents a problem,
because statistics and probability are all about uncertainty, and we know that
the point estimate we make for, say, $B$ in the German tank problem will likely
never be exactly correct, even if the estimator is consistent, unbiased, and
efficient.

In the frequentist framework, the approach for expressing uncertainty is using
\emph{confidence intervals}. A confidence interval for a parameter $\theta$ with estimator $\hat{\theta}$
at \emph{level} $1-\alpha$ is a pair of estimators $\hat{\theta}_-$ and $\hat{\theta}_+$
defined such that, for any parameter value $\theta$, it holds that:
\begin{equation}
  \prob{\hat{\theta}_- \leq \theta \leq \hat{\theta}_+} \geq 1 - \alpha.
\end{equation}
A typical value for $\alpha$ is 5\%, or $0.05$, which yields 95\% confidence intervals.

In other words, in the frequentist approach, we develop two statistics ---functions
of the observed data--- such that their corresponding estimators will, with some
probability, enclose the true value, for every possible true value. Strictly speaking,
you must select a method for constructing confidence intervals such that, for any
parameter $\theta$ I choose, the proportion of infinitely many repeated trials will
produce realizations of the confidence interval that enclose $\theta$.

Note that $\alpha$ is the probability of an error, that is, of the confidence intervals not
enclosing the true value. Thus, a higher confidence percentage, which translates to a lower 
$\alpha$, means a lower probability of the
confidence intervals being incorrect, and thus wider confidence intervals.

\section{Meaning and interpretation}

Confidence intervals are a very new concept. They were first introduced in the statistical literature in 1937 and become commonplace in scientific work only in the later 20th century.
It is perhaps no surprise, given its youth, that the concept of the confidence interval
is very confusing.

The most common misconception about confidence intervals is that they represent
\emph{confidence}. (One might argue that ``confidence'' was a poor word to use to name this
thing!) In this misconception, we can have 95\% confidence that the true parameter $\theta$
lies inside the confidence interval constructed after the experimental data has been gathered.
Strictly speaking, this is incorrect. In a frequentist framework, the true
value is either inside the realized confidence interval, or it is not; our ignorance of
the true value has nothing to do with probability. In fact, the interval that encloses
the true value with some ``confidence'' is a Bayesian concept, the \emph{credible interval}.
``Confidence'' is part of the Bayesian definition of probability; it is foreign to the
frequentist construction.

To a practicing scientist, this distinction might appear entirely semantic. Regardless of
what they specifically mean, it is useful to have some quantification of uncertainty.

\section{Constructing intervals}

To show how confidence intervals are constructed, consider the German
tank problem again. Our unbiased point estimate is $\hat{B} = \tfrac{n+1}{n} \max_i X_i$.
We will design a \emph{symmetric} confidence interval, which means that:
\begin{equation*}
    \prob{\hat{B}_- \leq B} = \prob{\hat{B}_+ \geq B} = 1 - \frac{\alpha}{2}
\end{equation*}
This confidence interval is symmetric because the probability that the confidence interval
will not include $B$ because it is too low is equal to the probability that it will not
include $B$ because it is too high. For a 95\% confidence interval, the probability of the
confidence interval being wrong on either side is $2.5\%$, for a 5\% total probability of
error.

To construct $\hat{B}_-$ and $\hat{B}_+$, we will find the range of values in which the point
estimator $\hat{B}$ is expected fall, given $B$. We found that the cdf for the biased
estimator $\max_i X_i$ was $(x/B)^n$, so the cdf for the
unbiased point estimator:
\begin{equation*}
    \prob{\hat{B} \leq x} = \frac{n+1}{n} \left(\frac{x}{B}\right)^n
\end{equation*}
First, we use this cdf to find the value of $x$ such that $\hat{B}$ will fall below it
with probability $1-\frac{\alpha}{2}$:
\begin{equation*}
    \prob{\hat{B} \leq x} = 1 - \frac{\alpha}{2} \implies x = \left( \frac{\alpha}{2} \frac{n}{n+1}\right)^{1/n} B
\end{equation*}
Next, we ``invert'' this relationship:
\begin{equation*}
    1 - \frac{\alpha}{2}
    = \prob{\hat{B} \leq \left( \frac{\alpha}{2} \frac{n}{n+1}\right)^{1/n} B}
    = \prob{\left( \frac{\alpha}{2} \frac{n}{n+1}\right)^{-1/n} \hat{B} \leq B}
\end{equation*}
And thus, we have found our lower confidence interval:
\begin{equation*}
    \hat{B}_- = \left( \frac{\alpha}{2} \frac{n}{n+1}\right)^{-1/n} \hat{B}
\end{equation*}
A similar exercise shows that:
\begin{equation*}
    \hat{B}_+ = \left[ \left(1- \frac{\alpha}{2}\right) \frac{n}{n+1}\right]^{-1/n} \hat{B}
\end{equation*}

As an example, for $n=5$, $\max x_i = 1$, and $\alpha = 5\%$, we have $\hat{B} = 1.2$, $\hat{B}_- = 1.04$ and $\hat{B}_+ = 2.17$. For $n=100$ and the same observed maximum $1$,
we have $\hat{B} = 1.01$,
$\hat{B}_- = 1.0004$, and $\hat{B}_+ = 1.04$. In both these cases, we do not say what $B$
actually is, so we cannot say whether or not the confidence intervals contain the true value
or not. We only know that, regardless of what $B$ is, there is a 95\% probability that the
resulting data will produce, according to our definitions, a value of $\hat{B}_-$ and
$\hat{B}_+$, that contain $B$.

We could have been much lazier with our confidence intervals. For example, I could have
defined $\hat{B}_- = 0$. Because we know \textit{a priori} that $B>0$, this lower end
of the confidence interval will always be correct. In fact, it means that the confidence
intervals will be contain $B$ with a probability greater than $1-\alpha$. This is an
undesirable feature. Confidence intervals with the desirable property that they have a
error probability of exactly $\alpha$ and no more are termed \emph{valid} confidence intervals.

\section{Bootstrapping}

\section{Jackknife}

\section{Profile method}

\part{Inferential statistics}

%!TEX root=main.tex

\chapter{Inferential statistics}

\section{The philosophy of statistical inference}

- inference vs. decision-making under uncertainty
- the p-value debate
- note that the math is still the same as before, we still thinking about properties of estimators
- inference as one of the most widely-spread notions in science

\section{Parametric inference}

- z-test
- t-test
- Wilk's theorem? and likelihood-ratio tests
- F-tests in general, variance. Then go to ANOVA

\section{Non-parametric inference}

- Kolmogorov-Smirnov

\section{$t$-test}

\subsection{Equal variance}\label{equal-variance}

The (old school) \emph{t}-test is two sample, assuming equal variances.
We're interested in the difference in the means between the two
populations.

The null hypothesis is that we're drawing \(n_1 + n_2\) samples from a
population that has this equal variance, and that the labels on the two
``populations'' are just fictitious.

Our estimator \(s_p^2\) for the pooled variance is just the average of
the variances of the two ``populations'', weighted by \(n_i - 1\) (which
is a better estimator than weighting by just \(n_i\)): \[
s_p^2 = \frac{(n_1 - 1) s_1^2 + (n_2 - 1) s_2^2}{n_1 + n_2 - 2}.
\]

To see why this is, say that we're going to develop some pooled estimator
$\hat{\mathbb{V}}_{X,p}$, which is a constant times the sum of squares:
\begin{align}
\hat{\mathbb{V}}_{X,p} &= C \left[ \sum_i (X_{1i} - \overline{X}_1)^2 + \ldots \right] \\
  &= C \left[ \sigma^2 \sum_i (Z_{1i} - \overline{Z}_1^2 ) + \ldots \right] \\
  &= C \sigma^2 \left[ \sum_i Z_{1i}^2 - n \overline{Z}_1^2 + \ldots \right] \\
\expect{\hat{\mathbb{V}}_{X,p}} &= C \sigma^2 \left[ \sum_i \expect{Z_{1i}^2} - n \expect{\overline{Z}_1^2} + \ldots \right] \\
  &= C \sigma^2 \left( n_1 - 1 + n_2 - 1 \right) \\
\end{align}
which implies that
\begin{equation}
C = \frac{1}{n_1 + n_2 - 2}.
\end{equation}

To see the last bit, use this identity:
\begin{align}
\sum_i (Z_i - \overline{Z})^2 &= \sum_i (Z_i^2 - 2 Z_i \overline{Z} + \overline{Z}^2) \\
  &= \sum_i Z_i^2 - 2 n \overline{Z} \overline{Z} + n \overline{Z}^2 \\
  &= \sum_i Z_i^2 - n \overline{Z}^2.
\end{align}

The thing we're observing is the difference between the mean of \(n_1\)
samples from a (potentially ficitious) variable \(X_1\) and \(n_2\) from
\(X_2\): \[
\overline{X}_1 - \overline{X}_2 = \frac{1}{n_1} \sum_{i=1}^{n_1} X_{1i} - \frac{1}{n_2} \sum_{i=1}^{n_2} X_{2i}.
\] It would be nice if our statistic was distributed like
\(\mathcal{N}(0, 1)\), so we compute the variance of this observation:
\[
\begin{aligned}
\mathrm{Var}\left[ \overline{X}_1 - \overline{X}_2 \right]
  &= \frac{1}{n_1^2} \sum_i \mathrm{Var}[X_1] + \frac{1}{n_2^2} \sum_i \mathrm{Var}[X_2] \\
  &= \frac{1}{n_1^2} n_1 s_p^2 + \frac{1}{n_2^2} n_2 s_p^2 \\
  &= \left( \frac{1}{n_1} + \frac{1}{n_2} \right) s_p^2.
\end{aligned}
\]

So the statistic for this test is just the observation over its
variance: \[
t = \frac{\overline{X}_1 - \overline{X}_2}{s_p \sqrt{\frac{1}{n_1} + \frac{1}{n_2}}}.
\]

The confusing thing is that \(\overline{X}_1\), \(\overline{X}_2\), and
\(s_p\) are all random variables. We know how to take the sum (or
difference) of two random variables (i.e., how to figure out the
distribution of the numerator), but it's not immediately obvious how to
find the distribution of the whole thing, which has a different random
variable in its denominator.

\subsubsection{Computational approach}\label{computational-approach}

\begin{itemize}
\tightlist
\item
  Compute the observed \(t\) statistic
\item
  Compute the observed sizes, means, and standard deviations for the two
  sample populations
\item
  Many times, generate two sets of random variates. One set of variates
  is drawn from a normal distribution with the first sample mean and
  variance.
\item
  For each iteration, compute the simulated \(t\) statistic.
\item
  The empirical \(p\)-value is the fraction (not true! need \(r+1/n+1\))
  of simulated statistics that are greater than the observed statistic.
\end{itemize}

\subsection{Unequal variance (Welch's)}\label{unequal-variance-welchs}

This is the Behrens-Fisher problem. It stumped Fisher! He came up with a
weird statistic with a weird distribution (Behrens-Fisher), but it
didn't really stick, since he couldn't calculate confidence intervals
(?).

Instead, people went for the Welch-Satterthwaite equation, which
approximates the interesting distribution using a more handy one by
matching the first and second moments. (Maybe worth discussing those? Or
just say mean and variance?)

\section{anova}\label{anova}

Say you have some (equally-sized) groups. Each group was drawn from a
normal distribution (all with the same variance). Are the data
consistent with the model in which all those groups have the same mean?

The statistic is \(F \equiv \frac{\mathrm{MS}_B}{\mathrm{MS}_W}\), where
\(MS_B\) is the mean of the squares of the residuals(?) between the
group means and the grand mean (``between'') and \(MS_W\) is the mean of
the squares of the residuals between the data points and the group means
(``within'').

Again, focus on what's the population we're sampling from. It's easy to
think about a finite population, where you can just do all the possible
combinations and compare their \(F\) statistics. Then move on to say
that, if you believe that the particular variances and means you
measured are the exact, true distribution that you're sampling from, ask
what happens when you sample from that infinite population.

\subsection{z-test example}\label{z-test-example}

What's the nonparametric equivalent of this? It's just saying what the
empirical cdf is! Then say, if you really truly believe your mean and
standard deviation, then you can do that.

In other words, you say you are absolutely sure what population you are
drawing from. The same is actually true of the \emph{t}-test, except
that the \emph{z}-test is asking about a single value, distributed like
\(N(\mu, \sigma^2)\), while the \emph{t}-test is about the mean of the
\(n\) points, which is distributed like \(N(\mu, \sigma^2/n)\).

\section{Paired differences}\label{paired-differences}

\subsection{Historical example and
motivation}\label{historical-example-and-motivation}

Darwin's thing with the pots, as described in Fisher's \emph{Design of
Experiments}

We'll make a tower of the kinds of assumptions made to test Darwin's
hypothesis.

\subsection{Sign test}\label{sign-test}

Assume that it's meaningful if a hybrid plant is taller than a
self-fertilized plant, but don't assign any meaning beyond that. Then
the data are effectively dichotomized: you get some number of cases in
which one is taller and some number of cases in which the other is
taller.

Better to say that we're sampling from any distribution that has zero
median. You could even say you're sampling from \emph{all}
distributions. That's confusing, mathematically, because there are
infinitely many distributions, and it's not obvious how you should
sample from that functional space, but it works out, because all those
distributions will have the same distribution of pluses and minuses.

This is now just a binomial test.

\subsection{Rank test (Mann-Whitney $U$)}

Assume that the \emph{ranks} of the differences are meaningful.

Now you're sampling from any distribution that is symmetric about zero.
That means it has zero median also.

\subsection{Fisher's weird sum test}\label{fishers-weird-sum-test}

Not sure if there's any name for this. Assume that the actual values of
the differences are meaningful.

\subsection{Welch's $t$-test}

Assume that the two populations are normally distributed and, and
therefore that that the variances of the populations are meaningful.
Then you can infer where this set of differences would stand in an
infinite set of such differences.

For early statisticians, this was really appealing, mostly from a
computational point of view: you could actually compute the mean and
standard deviation with pen and paper in a reasonable amount of time,
but you definitely couldn't do all \(2^n\) different ways of taking
sums. Fisher does it for one example in his book, and I'm sure it was
pretty crazy. He makes it clear that he went to great lengths to do it,
and his conclusion is that the results are basically the same, so you
should probably be doing the easier thing and not worry about it.
Nowadays it's gotten pretty easy to do the other thing!, so it's

\section{Wilcox test and Mann-Whitney
test}\label{wilcox-test-and-mann-whitney-test}

\textbf{Walsh averages and confidence intervals}, from
\href{http://www.stat.umn.edu/geyer/old03/5102/notes/rank.pdf}{here}

There a few different names for these things:

\begin{itemize}
\tightlist
\item
  One-sample test: is this distribution symmetric about zero (or
  whatever)?
\item
  Two-sample unpaired (independent; Mann-Whitney): does one of these
  distributions ``stochastically dominate'' the other (i.e., is it that
  a random value drawn from population \(A\) is more than 50\% probable
  to be greater than a random value from \(B\))?
\item
  Two-sample paired (dependent): are the differences between paired data
  points symmetric about zero?
\end{itemize}

\subsection{Wilcoxon}\label{wilcoxon}

\begin{enumerate}
\def\labelenumi{\arabic{enumi}.}
\tightlist
\item
  For each pair \(i\), compute the magnitude and sign \(s_i\) of the
  difference. Exclude tied pairs.
\item
  Order the pairs by the magnitude of their difference: \(i=1\) is the
  pair with the smallest magnitude. Now \(i\) is the rank.
\item
  Compute the \(W = \sum_i i s_i\).
\end{enumerate}

Thus, the bigger differences get more weight.

(There might be a way to do a visualization of this: as you walk along
the data points, you get a good bump for every rank that is in the
``high'' data set, and you get a bad hit for every rank that is not.
Then it settles out pretty quickly, and you want to know the meaning of
the final intercept.)

For small \(W\), the distribution has to be computed for each integer
\(W\). For larger values (\(\geq 50\)), a normal approximation works.

Compare the sign test, which does not use ranks, and which assumes the
median is zero, but not that the distributions are symmetric. That's
just a binomial test of the number of pluses or minuses you get. It's
like setting the weights, which in \(W\) are the ranks, all equal.

\subsection{Mann-Whitney}\label{mann-whitney}

\begin{enumerate}
\def\labelenumi{\arabic{enumi}.}
\tightlist
\item
  Assign ranks to every observation.
\item
  Compute \(R_1\), the sum ranks that belong to points for sample 1.
  Note that \(R_1 + R_2 = \sum_{i=1}^N = N(N+1)/2\).
\item
  Compute \(U_1 = R_1 - n_1(n_1+1)/2\) and \(U_2\). Use the smaller of
  \(U_1\) or \(U_2\) when looking at a table.
\end{enumerate}

At minimum \(U_1 = 0\), which means that sample 1 had ranks
\(1,2,\ldots,n_1\). Note that \(U_1 + U_2 = n_1 n_2\).

For large \(U\), there is a normal distribution approximation.

\subsubsection{Generation}\label{generation}

Say you have \(N\) total points and \(n_1\) in sample 1. Find all the
ways to draw \(n_1\) numbers from the sequence \(1, 2, \ldots N\).
Compute \(U\) for each of those. Voila.

Note that, if you fix \(n_1\), then you don't have to subtract the
\(n_1(n_1+1)/2\) to get the right \(p\)-value.

\section{Statistical power: Cochrane-Armitage
test}\label{statistical-power-cochrane-armitage-test}

We never want to run just any test: we want to use the test that is most
capable of distinguishing between the scenarios we're interested in.
Usually this is a matter of choosing the test that has the right
assumptions: the one-sample \emph{t}-test is more powerful than the
Wilcoxon test if the data come from a truly normally-distributed
population.

In other cases, you might have more flexibility. There's a somewhat
obscure test that is, I think, a great illustration of this.

Imagine that you have some data with a dichotomous outcome for some
categorical predictor value. One classic example is drug dosing: you
think that, as the dosage of the drug goes up, you have more good
outcomes than bad outcomes. Did a greater proportion of people on board
the Titanic survive as you go up from crew to Third Class to Second to
First? Did the proportion of some kind of event increase over years?
Technically, this means you have a \(2 \times k\) table of counts, with
two outcomes and \(k\) predictor categories.

\begin{longtable}[]{@{}llll@{}}
\toprule
Outcome & Dose 1 & Dose 2 & Dose 3\tabularnewline
\midrule
\endhead
Good & 1 & 5 & 9\tabularnewline
Bad & 9 & 4 & 1\tabularnewline
\bottomrule
\end{longtable}

You could use a \(\chi^2\) test with equal expected frequencies across
the columns. In other words, there might be more ``good'' than ``bad''
outcomes, but you don't expect that proportion to differ meaningfully
across categories. You would pool the data across categories, use the
observed proportion of good outcomes as you best guess of the true
proportion \(f\), and compare the actual data with you expectation that
a fraction \(f\) of the counts in each column are ``good''.

In our examples, we think the data have some \emph{particular} kind of
pattern. The \(\chi^2\) test doesn't look for any particular pattern; it
just looks for any deviation from the null. The test statistic for the
\(\chi^2\) distribution is based around the sum of the square deviations
from the expected values, usually written \(\sum_i (O_i - E_i)^2\), with
some stuff in the denominator to make the distribution of the statistic
easier to work with. If the sum of the squared deviations is too large,
then we have evidence that the observed values are not ``sticking to''
the expected frequencies.

The trick I'm going to show you is to keep the same null
hypothesis---that outcome doesn't depend on dose---but adjust the test
so that it's more sensitive to particular kinds of dependencies.

This is a fair approach because we're still just trying to say, ``OK,
say you (the nameless antagonist) were right, and there really was no
pattern in the data. Then I'm free to make up any test statistic, so
long as, if you're right, we can show that the observed data were likely
to have arisen by chance.''

To start constructing the test, think about each flip as a weighted
binomial trial. We'll use these weights to adjust the test statistic to
be more sensitive to what we suspect the true pattern in the data is,
but we'll need to derive the distribution of the test statistic so that
we can satisfy the nameless antagonist.

Say each flip \(y_i\), which is in some category \(x_i\), gets some
associated weight \(w_i\). A really simple statistic would be
\(\sum_i w_i y_i\), the sum of the weights of the ``successful'' trials.
It would be nice to have this be zero-centered: \[
\sum_i \left\{ w_i y_i - \mathbb{E}\left[ w_i y_i \right] \right\} = \sum_i w_i (y_i - \overline{p}),
\] where \(\overline{p} = (1/N)\sum_i y_i\).

It would also be nice for this to have variance 1, so we can divide by
the square root of \[
\mathrm{Var}\left[ \sum_i w_i (y_i - \overline{p}) \right] = \overline{p} (1-\overline{p}) \sum_i w_i^2
\] to produce the statistic \[
T = \frac{\sum_i w_i (y_i - \overline{p})}{\sqrt{\overline{p} (1-\overline{p}) \sum_i w_i^2}}.
\]

You could also conceive of this being a table with two rows and some
number of columns. We bin the trials by their weights: all trials with
the same weight are in the same column. Successes go in the top row;
failures in the bottom. Now write \(t_c\) as the weight of the trials in
the \(c\)-th column, \(n_{1c}\) is the number of successful trials with
weight \(t_c\) (i.e., in column \(c\)), and \(n_{2c}\) is the number of
failures. Then some math shows that you can rewrite \(T\) as \[
T = \frac{\sum_c w_c (n_{1c} n_{2\bullet} - n_{2c} n_{1\bullet})}{\sqrt{(n_{1\bullet} n_{2\bullet} / n_{\bullet\bullet}^2) \sum_c n_{\bullet c} w_c^2}}
\] where \(n_{r\bullet}\) are the row margins, \(n_{\bullet c}\) are the
column margins, and \(n_{\bullet\bullet}\) is the total number of
trials.

\emph{N.B.}: The wiki page gives a different answer, but I don't trust
it, since the variance formula doesn't assume independence of the
\(y_i\). A fact sheet about the PASS software that shows the formula in
terms of the \(y_i\) seems to make a mistake by using \(i\) as an index
both for individual trials and for the weight categories.

The confusing thing here is how to pick the weights. This test is mostly
used to look for linear trends: imagine that each \(y_i\) is associated
with some \(x_i\), so that the weights would be \(x_i\) or
\(x_i - \overline{x}\). Why you pick these exact weights has to do with
the \emph{sensitivity} of the test. There could, of course, be a
nonlinear trend, like a U-shape, that would lead to a zero expectation
for this statistic. The \(\chi^2\) test can find that, but
Cochrane-Armitage with these weights cannot.

To see why you use those weights for a linear test, imagine that
\(p_i \propto x_i\), and zero-center the \(x_i\) such that
\(p_i = m x_i + \overline{p}\).

Then the question is what \(w_i\) maximize \(\mathbb{E}[T]\)? You can
quickly see that this is equivalent to maximizing
\(\sum_i w_i x_i / \sqrt{\sum_i w_i^2}\), and taking a derivative with
respect to \(w_j\) shows that, at the extremum,
\(x_j \sum_i w_i^2 = w_j \sum_i w_i x_i\), which \(w_i = x_i\) for all
\(i\) satisfies. So those weights are the best way to get a large
statistic if you think that there actually is a linear test.

%!TEX root=main.tex

Things to do:

- If you cannot construct the entire universe of possible observations (events?), then you are actually sampling from the $p$-value distribution
- You observe a specific statistic, and then you sample another *n*.
- Under the null hypothesis, the observed value and the other *n* came from the same distribution.
- The *p*-value is the probability of getting a value as small as the observed one, given that the null is true.
- If you know the "shape" of the distribution, you can compute this exactly (e.g., as we did in our sign test).
- If you *don't* know the shape, then you need to do this empirically, with a parametric test.
- You say, I have *n+1* items in a list. What's the chance that this particular item falls at rank *m* or below (or above)? That's $(r+1)/(n+1)$.
- It's a mistake to do $r/n$, but not a terrible mistake, except that it gives you $p=0$!
- Overestimates small $p$ (because you go up to 1/n+1) and underestimates large, but that's normally not a problem. The important thing is avoiding p=0 bias.
- Also, this provides an *estimate* of the $p$-value, since you're basically doing a binomial test: of $n+1$ total coin flips, $r+1$ come out heads. What's the probability of heads? This has a confidence interval around it.

\chapter{Frequentist inferential statistics: statistical tests and confidence intervals}

\hl{1. Define p-value (1 and 2-sided). 2. Zea mays example data. 3. Binomial sign test. 4. Fisher's weird sum. 5. Paired test (wilcox)?. 6. Paired t-test.}

\hl{Add a chapter for statistical power and sample size. Define Type I and Type II errors. Chapters on parametric vs. nonparametric. Permutation tests.}

$T=t(X)$ is statistic. $t^\star = t(x)$ is obs. value. $p=\prob{T>t^\star}$ or $\prob{T<t^\star}$. 2-sided is $\prob{T<t^\star \text{ or } T > t_0 + (t^\star-t_0)}$, where $t_0$ is some central value?

The previous chapter laid out the conceptual points of inference. In this chapter, I lay out the nuts and bolts of what most people think statistics is: tests.

The mathematics of statistical tests follows directly from the previous mathematical chapters: we define some \emph{estimators}, functions of the data that predict some property of the data, and determine their variance. We just give things different names and interpret them differently.

But rather than begin with the mathematics of tests, I'll start with the concepts of how they work, and show how to implement them algorithmically and computationally. Once you understand what the test is doing, we can get back to unpacking some of the math you will hear bandied about around them.

\section{The ingredients in a statistical test}

To run a statistical test, you need:

\begin{enumerate}
\item Observed data.
\item A listing of the universe of observations that could have been made, of which the actual observation is one instance.
\item A \emph{null hypothesis} that associates a probability with each possible observation.
\item A \emph{test statistic} that summarizes how interesting the observed data are in a single number.
\end{enumerate}

Every statistical test that you know of can be defined by how it selects its
universe of possible observations, its null hypothesis, and its test statistic.

The test will give you two outputs:

\begin{enumerate}
\item A $p$-value, which is the probability of observing data as extreme (or ``interesting'') as what you did observe, given the null hypothesis.
\item Confidence intervals on the test statistic.
\end{enumerate}

More concretely, the $p$-value is the proportion of the observations in the universe of observations, weighted by their probabilities, that have test statistics that are more extreme (i.e., either larger or smaller) than the test statistic for the observed data. The confidence interval says that, using the observed data as our best estimates of the underlying truth, what range of test statistics would we expect?

\section{Interpreting $p$-values and confidence intervals}

The $p$-value was developed by Fisher, probably the one person who had the most influence on modern statistics. He used the $p$-value, a rigorously-derived number, to reason informally about data. If the $p$-value too big, by some standard to be determined \textit{ad hoc}, then the data didn't show anything interesting. In different books and papers, for different data sets, Fisher used different standards for ``too big'', but $0.05$ was a recurring choice.

Later, Neyman and Pearson (the younger) reformulated Fisher's method as a kind
of decision-making under uncertainty. If you were a decision-maker, and you had
to make a decision, what would you do when faced with certain data? They
suggested you pick some specific \emph{confidence level}, written $\alpha$, and
make your decision based on whether $p$ was greater than or less than $\alpha$.

Although Fisher furiously disagreed with Neyman and Pearson about how to interpret $p$-values, the canonical $\alpha = 0.05$ somehow made it into the scientific literature.

As laid out in the previous chapter, ``what is the probability that this data is true, given the data?'' is a very tricky question. The critical point is that the $p$-value is, in the language of that chapter, $\prob{X | \theta}$, but we want to know about $\prob{\theta | X}$. The link between these is the mysterious prior probability $\prob{\theta}$.

I think my only take-away is that you should trust your gut more than a $p$-value. First, Fisher developed statistics as a way to rigorously show that data \emph{weren't} interesting, a mathematical counterbalance to humans' ability to find interesting patterns in data, especially their own. He meant $p$-values as a way to deflate puffed-up conclusions, not to puff them up. Second, Fisher, who invented $p$-values, couldn't really agree with other extremely intelligent people about exactly what they meant. So don't make yourself crazy trying to figure it out too precisely either! Just know that big $p$-values, where ``big'' depends on the intellectual context, mean the data aren't interesting, where ``interesting'' is defined by the null hypothesis.

\hl{Move all of the above to previous chapter where p-values are defined}

\section{A test with a simple explanation but difficult math: the binomial proportion test}

To understand how to go from a test's ingredients to its outputs, I will work
through many examples of increasing complexity. The first and simplest is this:
I flip a coin $n$ times, seeing some sequence of heads and tails, and I want to
infer whether the coin is fair.

\begin{itemize}
\item The \emph{observed data} is the sequence of heads and tails I flipped.
\item The universe of possible observations is all the sequences of heads and tails I could have flipped. So if $n=2$, the universe is the four cases $HH, HT, TH, TT$.
\item The \emph{null hypothesis} is that heads and tails are equally likely.
\item The \emph{test statistic} is the number of heads I flipped. This statistic, like all the other statistics discussed in chapter XX, is a function of the data: given some sequence of heads and tails, it just counts up the number of heads.
\end{itemize}

\subsection{One- and two-sided $p$-values}

The $p$-value is the probability of data as extreme as what we observed, or
more extreme, given the null hypothesis that $p = 0.5$.\footnote{I apologize
that $p$ means the expected proportion of heads and $p$-value means something
totally different.} The tricky part of this definition is understand what ``as
extreme or more'' means.

``Extremeness'' is quantified by the test statistic, which summarizes each of
the possible observations ---each length-$n$ sequence of heads and tails---
into a single number, the number of heads in that sequence. ``As extreme or
more'' can mean 3 things:
\begin{itemize}
    \item as small as what was observed, or smaller,
    \item as large as what was observed, or larger, or
    \item as far away from the ``center'' as was observed, or further away.
\end{itemize}

\hl{Exact tests: you can count every possibility for a binomial}

The first two cases call for \emph{one-sided} $p$-values and can be used if you
have an \textit{a priori} hypothesis that the number of heads that you will
observe will be much higher or lower than what you would expect if the coin
were fair. So if I want to seek support for my \textit{a priori} hypothesis
that the coin is more likely to land tails than heads, then I would use that
one-sided $p$-value. If I don't have an \textit{a priori} expectation about
whether the flips will be enriched for heads or tails ---if I just think that
the coin is not fair, but I don't know which way--- then I need a more
conservative, \emph{two-sided} $p$-value.

In this example, the null hypothesis of a fair coin means that all the possible
sequences of heads and tails are equally likely, so to compute a one-sided
$p$-value, I just need to count up the number of sequences that are more
extreme than the observed one:
\begin{equation}
    \text{one-sided $p$-value} = \frac{\text{number of possible sequences more extreme than observed}}{\text{total number of sequences}}
\end{equation}
You may know that the number of length-$n$ heads-tails sequences is $2^n$ and
that the number of those sequences that has $x$ heads is $\binom{n}{x} = n! /
\left[ x! (n-x)! \right]$. Thus, for this example, if we have a ``low-sided''
expectation about the number of heads:
\begin{equation}
    \text{``low-sided'' one-sided $p$-value} = \frac{\sum_{y=0}^x \binom{n}{x}}{2^n}
\end{equation}
A ``high-sided'' $p$-value would sum from $x$ up to $n$.

For the two-sided $p$-value, we need to sum up the ``low-side'' and ``high-side'' $p$-values. Because the observed $x$ falls to one side of the central expectation $\expect{x} = n/2$, we need to symmetrize to find the other value. For example if $x$ is low, we make another, mirrored ``high'' $x_\mathrm{hi} = n-x$:
\begin{equation}
    \text{two-sided $p$-value} = \frac{\sum_{y=0}^x \binom{n}{x} + \sum_{y=n-x}^n \binom{n}{x}}{2^n}
\end{equation}
If $x > n/2$, then the sums would go from $0$ to $n-x$ and from $x$ to $n$.

\section{Checking for a specified unfairness of a coin}

In the first example, we used a null hypothesis that heads and tails were equal, which simplifed thinking about $p$-values because we could simply count up the equiprobable sequences of heads and tails.

Now, we generalize slightly, allowing a null hypothesis that the probability $p$ of flipping heads is not exactly $\tfrac{1}{2}$:
\begin{itemize}
\item The \emph{observed data} is, again, the sequence of heads and tails I flipped.
\item The universe of possible observations is, again, all the sequences of heads and tails I could have flipped.
\item The \emph{null hypothesis} is that the probability of flipping heads is $p$.
\item The \emph{test statistic} is the number of heads I flipped.
\end{itemize}

The mathematically savvy will know that the probability of $y$ successful trials among $n$ total trials, with each trial having a probability $p$ of success, is the \emph{binomial distribution}:
\begin{equation}
    \mathrm{Binom}(y; n, p) = \binom{n}{y} p^y (1-p)^{n-y}
\end{equation}
This is just like the above, where we counted the number of sequences, only now
we need to weight them by the number of more (or less) probably heads (and
tails):
\begin{equation}
    \text{``low-sided'' one-sided $p$-value} = \sum_{y=0}^x \mathrm{Binom}(y; n, p)
\end{equation}

\subsection{Computing confidence intervals}

Recall that a \emph{confidence interval} is a method: it is a pair of
statistics, functions of the data, that will, in a certain proportion $\alpha$
of replicate experiments, will contain the true value, regardless of what the
true value is.

In this example, the ``true value'' in question is the unknown probability
$p_\mathrm{true}$ that the coin will flip heads, which would be anything
between 0 and 1.  So we need a method that, whether $p_\mathrm{true} = 10^{-6}$
or $p_\mathrm{true} = 0.999$, will produce an interval that contains the true
$p_\mathrm{true}$ in a proportion of $\alpha$ of the data that would arise in
replicates of that experiment.

As mentioned earlier, there are multiple methods that can achieve this result.
The one that doesn't rely on any kind of approximation, which we will address
later, is the \emph{Clopper-Pearson} interval. This says to find the largest
value $p_+$ for which $\sum_{y=0}^x \mathrm{Binom}(y; n, p_+) > \alpha/2$ and
the smallest value $p_-$ for which $\sum_{y=x}^n \mathrm{Binom}(y; n, p_-) >
\alpha/2$. In other words, if $\alpha=95\%$, pick an uppper limit on
$p_\mathrm{true}$ so that, if that were the true value, in exactly $2.5\%$ of
replicates of the experiment, you would get numbers of heads larger than what
you observed. Similarly, pick a lower limit so that, if that were the true
probability of flipping heads, in $2.5\%$ of replicates, you would get numbers
of heads larger than what you observed.

If this seems like a big, confusing headache, I hope you take it as a sign that
confidence intervals are confusing, not that you're not understanding!

\hl{Now with computers, Clopper-Pearson is easy}

\section{A test with simple math but a difficult explanation: the $z$-test}

The elegance of the previous example is that it works by simply enumerating all
the possible sets of observations. The inelegant part is that there can be
many: $2^n$ grows quickly with $n$, making the sums in the $p$-value
calculations onerous. Furthermore, the math of the Clopper-Pearson interval is
a bit of a mess and hard to articulate.

As a counterexample, consider the humble $z$-test, in which your experiment consists of drawing a single number from a pre-specified population, specifically, a normal distribution with known mean $\mu$ and variance $\sigma^2$:
\begin{itemize}
\item The observed data is the single number $x$ you drew.
\item The universe of possible observations is all numbers, from $-\infty$ to $+\infty$.
\item The null hypothesis is that the number $x$ was drawn from a normal population with mean $\mu$ and variance $\sigma^2$.
\item The test statistic is just the number $x$.
\end{itemize}

In this example, rather than summing up over the \emph{discrete} universe of possibilities, you need to integrate to find the $p$-value:
\begin{equation}
  \text{``low-sided'' $p$-value} = \int_{y=-\infty}^x f_\mathcal{N}(y; \mu, \sigma^2) \,\mathrm{d}y = F_\mathcal{N}(y; \mu, \sigma^2).
\end{equation}

\subsection{cis}

\begin{align}
1 - \alpha &= \prob{X_- \leq \mu \leq X_+} \\
\alpha / 2 &= \prob{X_- \geq \mu}
\end{align}
Making the inspired guess that
\begin{equation}
  \frac{X_- - \mu}{\sigma} = \frac{X - \mu}{\sigma} - b
\end{equation}
that is, that $X_- = X - \sigma b$, we find that
\begin{align}
  \alpha/2 &= \prob{X - \sigma b \geq \mu} \\
  &= \prob{\frac{X - \mu}{\sigma} > b},
\end{align}
which you can find that, for $\alpha = 0.05$, you get $b = 1.96$.


\section{approx}

In fact, rather than enumerating all $2^n$ possibilities, the typical approach
is to approximate the binomial distribution with the normal distribution. It turns out that, when $n$ is not too small, that:
\begin{equation}
    \mathrm{Binom}(x; n, p) \approx \mathcal{N}\left(x; \; \mu = np, \; \sigma^2 = \frac{p (1-p)}{n}\right),
\end{equation}
in other words, you can use a normal distribution with mean a mean and variance
computed from $n$ and $p$ to approximate the binomial distribution.

\textbf{FIGURE}

The binomial counting approach was intellectually elegant but mathematically
inelegant. The normal approximation approach is mathematically elegant but
intellectually confusing, as it makes the ingredients in the test very
artificial:
\begin{itemize}
    \item The observed data is not the sequence of heads and tails but rather simply $x$, which was previously the observed test statistics, that is, the number of heads flipped.
    \item The universe of possible observations is all values between $0$ and $n$. Of course, this isn't really true, as you can only observe integer $x$. But for the math to work out, you need to imagine that you could have observed $x = 1.02$, for example.
    \item The null hypothesis, as above, is some assertion about the probability $p$, only now $p$ is not the probability of any particular event, it's just a parameter in the normal distribution's mean and variance.
    \item The test statistic is just $x$, the observed number.
\end{itemize}

The other intellectually weird thing is that now, because the universe of possible observations is infinite, you cannot count up the states but must integrate over them:
\begin{equation}
    \text{``low-sided'' one-sided $p$-value} = \int_{y=0}^x \mathcal{N}\left(y; np, \frac{p(1-p)}{n} \right) \,\mathrm{d}y
\end{equation}
This is where the mathematical elegance comes in: the integral of the normal distribution, which is its cumulative distribution function, is oft-calculated and so easy to look up. So now rather than summing over something like $2^n$ states, we just look up a single number.

The confidence intervals become similarly simple: I look for a lower bound $p_-$ such that the probability that I would observe a value smaller than what I did observe is $\alpha/2$:
\begin{equation}
    \alpha/2 = \int_{y=0}^x \mathcal{N} \left(y; np_-, \frac{p_-(1-p_-)}{n} \right) \,\mathrm{d}y
\end{equation}

There are a few other approximations to the binomial distribution that are more
accurate, but they all come down to the same logic: they exchange the
intellectual simplicity of counting up over a finite universe of states for the
mathematical elegance of integrating over a well-known function.

\textbf{point out how this works with a z-test?}

\section{The data with different null hypotheses and statistics: paired tests}

So far we've had null hypotheses and test statistics that are obvious matches for the data that we're looking at. In the real world, when you're making up your own statistical tests, you'll find that you actually have multiple choices for null hypotheses and test statistics. To show how this can work, I'll analyze the same kind of data in a few ways.

Charles Darwin wanted to know if cross-fertilized corn (\textit{Zea mays}) grew better than self-fertilized corn. He took 15 pairs of plants, one self- and one cross-fertilized, each having germinated on the same day, and planted them together in one of 4 pots. He measured the heights of the plants, to one-eight of an inch, after some time (Table \ref{table:darwin}).

\begin{table}
\centering
\renewcommand{\arraystretch}{1.2}
\begin{tabular}{ccll}
\toprule
& & \multicolumn{2}{c}{Height (inches)} \\
\cmidrule{3-4}
Pair & Pot & Crossed & Self-fert. \\
\midrule
1 & 1 & $23 \tfrac{1}{2}$ & $17 \tfrac{3}{8}$ \\
2 & 1 & $12$ & $20 \tfrac{3}{8}$ \\
3 & 1 & $21$ & $20$ \\[1.0ex]
4 & 2 & $22$ & $20$ \\
5 & 2 & $19 \tfrac{1}{8}$ & $18 \tfrac{3}{8}$ \\
6 & 2 & $21 \tfrac{1}{2}$ & $18 \tfrac{5}{8}$ \\[1.0ex]
7 & 3 & $22 \tfrac{1}{8}$ & $18 \tfrac{5}{8}$ \\
8 & 3 & $20 \tfrac{3}{8}$ & $15 \tfrac{1}{4}$ \\
9 & 3 & $18 \tfrac{1}{4}$ & $16 \tfrac{1}{2}$ \\
10 & 3 & $21 \tfrac{5}{8}$ & $18$ \\
11 & 3 & $23 \tfrac{1}{4}$ & $16 \tfrac{1}{4}$ \\[1.0ex]
12 & 4 & $21$ & $18$ \\
13 & 4 & $22 \tfrac{1}{8}$ & $12 \tfrac{3}{4}$ \\
14 & 4 & $23$ & $15 \tfrac{1}{2}$ \\
15 & 4 & $12$ & $18$ \\
\bottomrule
\end{tabular}
\caption{Darwin's \textit{Zea mays} data}
\label{table:darwin}
\end{table}

The obvious question is: did the hybridized plants grow faster than the self-fertilized plants?

\subsection{A sign test}

This just reduces down to the binomial test above. Null hypothesis is that the median is zero.

\subsection{Wilcoxon's signed rank test}

Null hypothesis is zero median and symmetric distribution. Much more complex test statistic.

\subsection{Fisher's sum test}

Null is zero median and symmetric distribution. Simple test statistic.

\subsection{Paired $t$-test}

\subsection{Scott's}

Something about permuting within the pots? Bootstrapping?

\section{Parametric inference}

- z-test
- t-test
- Wilk's theorem? and likelihood-ratio tests
- F-tests in general, variance. Then go to ANOVA

\section{Non-parametric inference}

- Kolmogorov-Smirnov

\section{$t$-test}

\subsection{Equal variance}\label{equal-variance}

The (old school) \emph{t}-test is two sample, assuming equal variances.
We're interested in the difference in the means between the two
populations.

The null hypothesis is that we're drawing \(n_1 + n_2\) samples from a
population that has this equal variance, and that the labels on the two
``populations'' are just fictitious.

Our estimator \(s_p^2\) for the pooled variance is just the average of
the variances of the two ``populations'', weighted by \(n_i - 1\) (which
is a better estimator than weighting by just \(n_i\)): \[
s_p^2 = \frac{(n_1 - 1) s_1^2 + (n_2 - 1) s_2^2}{n_1 + n_2 - 2}.
\]

To see why this is, say that we're going to develop some pooled estimator
$\hat{\mathbb{V}}_{X,p}$, which is a constant times the sum of squares:
\begin{align}
\hat{\mathbb{V}}_{X,p} &= C \left[ \sum_i (X_{1i} - \overline{X}_1)^2 + \ldots \right] \\
  &= C \left[ \sigma^2 \sum_i (Z_{1i} - \overline{Z}_1^2 ) + \ldots \right] \\
  &= C \sigma^2 \left[ \sum_i Z_{1i}^2 - n \overline{Z}_1^2 + \ldots \right] \\
\expect{\hat{\mathbb{V}}_{X,p}} &= C \sigma^2 \left[ \sum_i \expect{Z_{1i}^2} - n \expect{\overline{Z}_1^2} + \ldots \right] \\
  &= C \sigma^2 \left( n_1 - 1 + n_2 - 1 \right) \\
\end{align}
which implies that
\begin{equation}
C = \frac{1}{n_1 + n_2 - 2}.
\end{equation}

To see the last bit, use this identity:
\begin{align}
\sum_i (Z_i - \overline{Z})^2 &= \sum_i (Z_i^2 - 2 Z_i \overline{Z} + \overline{Z}^2) \\
  &= \sum_i Z_i^2 - 2 n \overline{Z} \overline{Z} + n \overline{Z}^2 \\
  &= \sum_i Z_i^2 - n \overline{Z}^2.
\end{align}

The thing we're observing is the difference between the mean of \(n_1\)
samples from a (potentially ficitious) variable \(X_1\) and \(n_2\) from
\(X_2\): \[
\overline{X}_1 - \overline{X}_2 = \frac{1}{n_1} \sum_{i=1}^{n_1} X_{1i} - \frac{1}{n_2} \sum_{i=1}^{n_2} X_{2i}.
\] It would be nice if our statistic was distributed like
\(\mathcal{N}(0, 1)\), so we compute the variance of this observation:
\[
\begin{aligned}
\mathrm{Var}\left[ \overline{X}_1 - \overline{X}_2 \right]
  &= \frac{1}{n_1^2} \sum_i \mathrm{Var}[X_1] + \frac{1}{n_2^2} \sum_i \mathrm{Var}[X_2] \\
  &= \frac{1}{n_1^2} n_1 s_p^2 + \frac{1}{n_2^2} n_2 s_p^2 \\
  &= \left( \frac{1}{n_1} + \frac{1}{n_2} \right) s_p^2.
\end{aligned}
\]

So the statistic for this test is just the observation over its
variance: \[
t = \frac{\overline{X}_1 - \overline{X}_2}{s_p \sqrt{\frac{1}{n_1} + \frac{1}{n_2}}}.
\]

The confusing thing is that \(\overline{X}_1\), \(\overline{X}_2\), and
\(s_p\) are all random variables. We know how to take the sum (or
difference) of two random variables (i.e., how to figure out the
distribution of the numerator), but it's not immediately obvious how to
find the distribution of the whole thing, which has a different random
variable in its denominator.

\subsubsection{Computational approach}\label{computational-approach}

\begin{itemize}
\tightlist
\item
  Compute the observed \(t\) statistic
\item
  Compute the observed sizes, means, and standard deviations for the two
  sample populations
\item
  Many times, generate two sets of random variates. One set of variates
  is drawn from a normal distribution with the first sample mean and
  variance.
\item
  For each iteration, compute the simulated \(t\) statistic.
\item
  The empirical \(p\)-value is the fraction (not true! need \(r+1/n+1\))
  of simulated statistics that are greater than the observed statistic.
\end{itemize}

\subsection{Unequal variance (Welch's)}\label{unequal-variance-welchs}

This is the Behrens-Fisher problem. It stumped Fisher! He came up with a
weird statistic with a weird distribution (Behrens-Fisher), but it
didn't really stick, since he couldn't calculate confidence intervals
(?).

Instead, people went for the Welch-Satterthwaite equation, which
approximates the interesting distribution using a more handy one by
matching the first and second moments. (Maybe worth discussing those? Or
just say mean and variance?)

\section{anova}\label{anova}

Say you have some (equally-sized) groups. Each group was drawn from a
normal distribution (all with the same variance). Are the data
consistent with the model in which all those groups have the same mean?

The statistic is \(F \equiv \frac{\mathrm{MS}_B}{\mathrm{MS}_W}\), where
\(MS_B\) is the mean of the squares of the residuals(?) between the
group means and the grand mean (``between'') and \(MS_W\) is the mean of
the squares of the residuals between the data points and the group means
(``within'').

Again, focus on what's the population we're sampling from. It's easy to
think about a finite population, where you can just do all the possible
combinations and compare their \(F\) statistics. Then move on to say
that, if you believe that the particular variances and means you
measured are the exact, true distribution that you're sampling from, ask
what happens when you sample from that infinite population.

\subsection{z-test example}\label{z-test-example}

What's the nonparametric equivalent of this? It's just saying what the
empirical cdf is! Then say, if you really truly believe your mean and
standard deviation, then you can do that.

In other words, you say you are absolutely sure what population you are
drawing from. The same is actually true of the \emph{t}-test, except
that the \emph{z}-test is asking about a single value, distributed like
\(N(\mu, \sigma^2)\), while the \emph{t}-test is about the mean of the
\(n\) points, which is distributed like \(N(\mu, \sigma^2/n)\).

\section{Paired differences}\label{paired-differences}

\subsection{Historical example and
motivation}\label{historical-example-and-motivation}

Darwin's thing with the pots, as described in Fisher's \emph{Design of
Experiments}

We'll make a tower of the kinds of assumptions made to test Darwin's
hypothesis.

\subsection{Sign test}\label{sign-test}

Assume that it's meaningful if a hybrid plant is taller than a
self-fertilized plant, but don't assign any meaning beyond that. Then
the data are effectively dichotomized: you get some number of cases in
which one is taller and some number of cases in which the other is
taller.

Better to say that we're sampling from any distribution that has zero
median. You could even say you're sampling from \emph{all}
distributions. That's confusing, mathematically, because there are
infinitely many distributions, and it's not obvious how you should
sample from that functional space, but it works out, because all those
distributions will have the same distribution of pluses and minuses.

This is now just a binomial test.

\subsection{Rank test (Mann-Whitney $U$)}

Assume that the \emph{ranks} of the differences are meaningful.

Now you're sampling from any distribution that is symmetric about zero.
That means it has zero median also.

\subsection{Fisher's weird sum test}\label{fishers-weird-sum-test}

Not sure if there's any name for this. Assume that the actual values of
the differences are meaningful.

\subsection{Welch's $t$-test}

Assume that the two populations are normally distributed and, and
therefore that that the variances of the populations are meaningful.
Then you can infer where this set of differences would stand in an
infinite set of such differences.

For early statisticians, this was really appealing, mostly from a
computational point of view: you could actually compute the mean and
standard deviation with pen and paper in a reasonable amount of time,
but you definitely couldn't do all \(2^n\) different ways of taking
sums. Fisher does it for one example in his book, and I'm sure it was
pretty crazy. He makes it clear that he went to great lengths to do it,
and his conclusion is that the results are basically the same, so you
should probably be doing the easier thing and not worry about it.
Nowadays it's gotten pretty easy to do the other thing!, so it's

\section{Wilcox test and Mann-Whitney
test}\label{wilcox-test-and-mann-whitney-test}

\textbf{Walsh averages and confidence intervals}, from
\href{http://www.stat.umn.edu/geyer/old03/5102/notes/rank.pdf}{here}

There a few different names for these things:

\begin{itemize}
\tightlist
\item
  One-sample test: is this distribution symmetric about zero (or
  whatever)?
\item
  Two-sample unpaired (independent; Mann-Whitney): does one of these
  distributions ``stochastically dominate'' the other (i.e., is it that
  a random value drawn from population \(A\) is more than 50\% probable
  to be greater than a random value from \(B\))?
\item
  Two-sample paired (dependent): are the differences between paired data
  points symmetric about zero?
\end{itemize}

\subsection{Wilcoxon}\label{wilcoxon}

\begin{enumerate}
\def\labelenumi{\arabic{enumi}.}
\tightlist
\item
  For each pair \(i\), compute the magnitude and sign \(s_i\) of the
  difference. Exclude tied pairs.
\item
  Order the pairs by the magnitude of their difference: \(i=1\) is the
  pair with the smallest magnitude. Now \(i\) is the rank.
\item
  Compute the \(W = \sum_i i s_i\).
\end{enumerate}

Thus, the bigger differences get more weight.

(There might be a way to do a visualization of this: as you walk along
the data points, you get a good bump for every rank that is in the
``high'' data set, and you get a bad hit for every rank that is not.
Then it settles out pretty quickly, and you want to know the meaning of
the final intercept.)

For small \(W\), the distribution has to be computed for each integer
\(W\). For larger values (\(\geq 50\)), a normal approximation works.

Compare the sign test, which does not use ranks, and which assumes the
median is zero, but not that the distributions are symmetric. That's
just a binomial test of the number of pluses or minuses you get. It's
like setting the weights, which in \(W\) are the ranks, all equal.

\subsection{Mann-Whitney}\label{mann-whitney}

\begin{enumerate}
\def\labelenumi{\arabic{enumi}.}
\tightlist
\item
  Assign ranks to every observation.
\item
  Compute \(R_1\), the sum ranks that belong to points for sample 1.
  Note that \(R_1 + R_2 = \sum_{i=1}^N = N(N+1)/2\).
\item
  Compute \(U_1 = R_1 - n_1(n_1+1)/2\) and \(U_2\). Use the smaller of
  \(U_1\) or \(U_2\) when looking at a table.
\end{enumerate}

At minimum \(U_1 = 0\), which means that sample 1 had ranks
\(1,2,\ldots,n_1\). Note that \(U_1 + U_2 = n_1 n_2\).

For large \(U\), there is a normal distribution approximation.

\subsubsection{Generation}\label{generation}

Say you have \(N\) total points and \(n_1\) in sample 1. Find all the
ways to draw \(n_1\) numbers from the sequence \(1, 2, \ldots N\).
Compute \(U\) for each of those. Voila.

Note that, if you fix \(n_1\), then you don't have to subtract the
\(n_1(n_1+1)/2\) to get the right \(p\)-value.

\section{Statistical power: Cochrane-Armitage
test}\label{statistical-power-cochrane-armitage-test}

\hl{Do I need this example, if I'm showing the Zea mays choices instead?}

We never want to run just any test: we want to use the test that is most
capable of distinguishing between the scenarios we're interested in.
Usually this is a matter of choosing the test that has the right
assumptions: the one-sample \emph{t}-test is more powerful than the
Wilcoxon test if the data come from a truly normally-distributed
population.

In other cases, you might have more flexibility. There's a somewhat
obscure test that is, I think, a great illustration of this.

Imagine that you have some data with a dichotomous outcome for some
categorical predictor value. One classic example is drug dosing: you
think that, as the dosage of the drug goes up, you have more good
outcomes than bad outcomes. Did a greater proportion of people on board
the Titanic survive as you go up from crew to Third Class to Second to
First? Did the proportion of some kind of event increase over years?
Technically, this means you have a \(2 \times k\) table of counts, with
two outcomes and \(k\) predictor categories.

\begin{longtable}[]{@{}llll@{}}
\toprule
Outcome & Dose 1 & Dose 2 & Dose 3\tabularnewline
\midrule
\endhead
Good & 1 & 5 & 9\tabularnewline
Bad & 9 & 4 & 1\tabularnewline
\bottomrule
\end{longtable}

You could use a \(\chi^2\) test with equal expected frequencies across
the columns. In other words, there might be more ``good'' than ``bad''
outcomes, but you don't expect that proportion to differ meaningfully
across categories. You would pool the data across categories, use the
observed proportion of good outcomes as you best guess of the true
proportion \(f\), and compare the actual data with you expectation that
a fraction \(f\) of the counts in each column are ``good''.

In our examples, we think the data have some \emph{particular} kind of
pattern. The \(\chi^2\) test doesn't look for any particular pattern; it
just looks for any deviation from the null. The test statistic for the
\(\chi^2\) distribution is based around the sum of the square deviations
from the expected values, usually written \(\sum_i (O_i - E_i)^2\), with
some stuff in the denominator to make the distribution of the statistic
easier to work with. If the sum of the squared deviations is too large,
then we have evidence that the observed values are not ``sticking to''
the expected frequencies.

The trick I'm going to show you is to keep the same null
hypothesis---that outcome doesn't depend on dose---but adjust the test
so that it's more sensitive to particular kinds of dependencies.

This is a fair approach because we're still just trying to say, ``OK,
say you (the nameless antagonist) were right, and there really was no
pattern in the data. Then I'm free to make up any test statistic, so
long as, if you're right, we can show that the observed data were likely
to have arisen by chance.''

To start constructing the test, think about each flip as a weighted
binomial trial. We'll use these weights to adjust the test statistic to
be more sensitive to what we suspect the true pattern in the data is,
but we'll need to derive the distribution of the test statistic so that
we can satisfy the nameless antagonist.

Say each flip \(y_i\), which is in some category \(x_i\), gets some
associated weight \(w_i\). A really simple statistic would be
\(\sum_i w_i y_i\), the sum of the weights of the ``successful'' trials.
It would be nice to have this be zero-centered: \[
\sum_i \left\{ w_i y_i - \mathbb{E}\left[ w_i y_i \right] \right\} = \sum_i w_i (y_i - \overline{p}),
\] where \(\overline{p} = (1/N)\sum_i y_i\).

It would also be nice for this to have variance 1, so we can divide by
the square root of \[
\mathrm{Var}\left[ \sum_i w_i (y_i - \overline{p}) \right] = \overline{p} (1-\overline{p}) \sum_i w_i^2
\] to produce the statistic \[
T = \frac{\sum_i w_i (y_i - \overline{p})}{\sqrt{\overline{p} (1-\overline{p}) \sum_i w_i^2}}.
\]

You could also conceive of this being a table with two rows and some
number of columns. We bin the trials by their weights: all trials with
the same weight are in the same column. Successes go in the top row;
failures in the bottom. Now write \(t_c\) as the weight of the trials in
the \(c\)-th column, \(n_{1c}\) is the number of successful trials with
weight \(t_c\) (i.e., in column \(c\)), and \(n_{2c}\) is the number of
failures. Then some math shows that you can rewrite \(T\) as \[
T = \frac{\sum_c w_c (n_{1c} n_{2\bullet} - n_{2c} n_{1\bullet})}{\sqrt{(n_{1\bullet} n_{2\bullet} / n_{\bullet\bullet}^2) \sum_c n_{\bullet c} w_c^2}}
\] where \(n_{r\bullet}\) are the row margins, \(n_{\bullet c}\) are the
column margins, and \(n_{\bullet\bullet}\) is the total number of
trials.

\emph{N.B.}: The wiki page gives a different answer, but I don't trust
it, since the variance formula doesn't assume independence of the
\(y_i\). A fact sheet about the PASS software that shows the formula in
terms of the \(y_i\) seems to make a mistake by using \(i\) as an index
both for individual trials and for the weight categories.

The confusing thing here is how to pick the weights. This test is mostly
used to look for linear trends: imagine that each \(y_i\) is associated
with some \(x_i\), so that the weights would be \(x_i\) or
\(x_i - \overline{x}\). Why you pick these exact weights has to do with
the \emph{sensitivity} of the test. There could, of course, be a
nonlinear trend, like a U-shape, that would lead to a zero expectation
for this statistic. The \(\chi^2\) test can find that, but
Cochrane-Armitage with these weights cannot.

To see why you use those weights for a linear test, imagine that
\(p_i \propto x_i\), and zero-center the \(x_i\) such that
\(p_i = m x_i + \overline{p}\).

Then the question is what \(w_i\) maximize \(\mathbb{E}[T]\)? You can
quickly see that this is equivalent to maximizing
\(\sum_i w_i x_i / \sqrt{\sum_i w_i^2}\), and taking a derivative with
respect to \(w_j\) shows that, at the extremum,
\(x_j \sum_i w_i^2 = w_j \sum_i w_i x_i\), which \(w_i = x_i\) for all
\(i\) satisfies. So those weights are the best way to get a large
statistic if you think that there actually is a linear test.


\chapter{copied material}

\section{Things to include}\label{things-to-include}

\begin{itemize}
\item
  Gauss: a minimum-variance, mean-unbiased estimator minimizes the
  squared-error loss function. Laplace: among median-unbiased
  estimators, a minimum-average-absolute-deviation estimator minimizes
  the absolute loss function. Maybe it's better to allow some bias so
  you can get less variance. That's the domain of statistical theory. \hl{Move to MLE section?}
\item
  Fisher's crazy sum test is the same thing as is used in \emph{TRANSIT}
  (DeJesus \emph{et al}.): they treat TA sites in the same gene as
  independent; the statistic is the difference in the sum of the
  (normalized) number of insertions in two treatments; the null
  distribution is generated by shuffling the values across the two
  datasets. OK, it's not \emph{exactly} like Fisher's test, since it's
  not paired, but it's pretty close. Fisher probably wouldn't have
  wanted to to the \(\binom{n}{2}\) options, compared to the \(2^n\)
  that he did. \hl{Example of how people come up with tests?}
\end{itemize}

\subsection{Estimators about
estimators}\label{estimators-about-estimators}

\hl{Move this up, into the descriptive section}

\subsubsection{Jackknife}\label{jackknife}

You have \(n\) data points and compute an estimator \(\hat{\theta}\) for
some population parameter \(\theta\). If you don't know how the
population is structured, then it's not clear what you expect the
variance of \(\hat{\theta}\) to be. How sure can you be of this value?
In terms of inference, can you make any inference with it?

Compute the \emph{jackknife replicates}\footnote{The ``jackknife''
  method is so called because Tukey compared the method, which is
  ``rough-and-ready'', to another rough-and-ready tool, the pocket
  knife, also known as a jackknife. Although this name has the
  disadvantage of giving you no clue what it is about, it had the
  advantage of having more brevity and vivacity than ``delete-1
  resampling'', which is probably the more accurate name.}
\(\hat{\theta}_j\), which are the estimators computed using all the data
points except the \(j\)-th one.

That seems like a weird thing to have done, but you can use them to
compute two handy things:

\begin{enumerate}
\def\labelenumi{\arabic{enumi}.}
\tightlist
\item
  An estimate of the variance of the estimator. This can help you for
  description---by giving a confidence interval(?)---and for
  inference---by giving you a sense of the ``random'' ranges you would
  expect from two samples.
\item
  An estimate of the bias in the estimator. This is helpful if you don't
  want want your estimator to be biased but you don't know how to fix
  it.
\end{enumerate}

\paragraph{Jackknife variance
estimator}\label{jackknife-variance-estimator}

The variance estimator is \[
\widehat{\mathrm{Var}}_\mathrm{jk}[\hat{\theta}] := \frac{n-1}{n}  \sum_j \left( \hat{\theta}_j - \hat{\theta}_{(\cdot)} \right)^2,
\] where \(\hat{\theta}_{(\cdot)}\) is the average of the jackknife
replicates: \[
\hat{\theta}_{(\cdot)} := \frac{1}{n} \sum_j \hat{\theta}_j.
\] In other words, it's the variance of the jackknife replicates with
some rescaling: \[
\mathrm{Var}[\hat{\theta}_j] = \frac{1}{n-1} \sum_j \left( \hat{\theta}_j - \hat{\theta}_{(\cdot)} \right)^2 \implies
  \widehat{\mathrm{Var}}_\mathrm{jk}[\hat{\theta}] = \frac{(n-1)^2}{n} \mathrm{Var}[\hat{\theta}_j].
\]

The reason for that scaling factor is beyond the scope of this book
(Efron \& Stein 1981?), but the exercise gives you a sense of why it has
to be true for a specific case.

Some other work, also beyond the scope of this book, shows that the
jackknife estimate of variance is biased: it tends to overestimate the
true variance. This makes the jackknife a conservative tool.

\textbf{Exercise}. Let \(\theta\) be the mean. Show that the scaling
factor is what we think. Hints:

\begin{itemize}
\tightlist
\item
  Show that \(\hat{\theta}_{(\cdot)}\) is the sample mean.
\item
  Show that
  \(\hat{\theta}_j - \hat{\theta}_{(\cdot)} = (n \overline{x} - x_j) / (n - 1)\).
\item
  Show that that value is equal to \((\overline{x} - x_j) / (n - 1)\).
\end{itemize}

That exercise is from McIntosh's bioRxiv about jackknife resampling.

\paragraph{Jackknife bias estimator}\label{jackknife-bias-estimator}

The jackknife estimate of bias is
\((n-1) \left( \hat{\theta}_{(\cdot)} - \theta \right)\). This is the
sum of the deviations of the jackknife replicates from the observed
value \(\hat{\theta}\). Again, the reason that you would take the
average deviation and scale it up to the sum is beyond the scope.

However, if you have an expectation about the bias in an estimator, you
can make an unbiased estimator by subtracting out that bias: \[
\hat{\theta}_\mathrm{jk} := \hat{\theta} - \widehat{\mathrm{Bias}}_\mathrm{jk}[\theta].
\]

\textbf{Exercise}. Show that the jackknife estimate of bias for the
variance gives you the familiar unbiased variance estimator.

\textbf{Exercise}. Something about the maximum estimator?

\paragraph{Pros and cons of the
jackknife}\label{pros-and-cons-of-the-jackknife}

It's a piece of cake to implement. There are only \(n\) replicates to
do, so it's tractable. Those replicates are deterministic, so you only
run it once.

The cons are that it doesn't always work. For example, a jackknife
estimate of the variance of a median (\textbf{swo check Knight}) is not
consistent. It's also overly conservative: it's biased toward higher
variances. You can rescue some properties if you move to a delete-\(d\)
resampling and pick \(d\) from the correct range.

\subsubsection{Bootstrap}\label{bootstrap}

\section{Example from Efron, ``Thinking the unthinkable''}

There's some true distribution $f_X(x)$, and you're approximating it with
$\hat{f}_X(x)$, which is a pmf. If you took $N$ data points, then bootstrapping
means that you're picking a vector $\vec{c}$, where $c_i$ is the number of
times that the $i$-th data point makes it into the bootstrap sample. This
begs the question, how is $\vec{c}$ behaved? It's just a multinomial, with
probability $1/N$ for each of the $N$ cells.

Normally you compute a statistic $T(\vec{x})$ of the data. Instead, formulate
this in terms of a function $g(\vec{c})$ If you can write $T(\vec{x}) = \sum_i t(x_i)$,
then $g(\vec{c}) = \sum_i (c_i/N) t(x_i)$. In a Taylor expansion around
$\vec{c}_\mathrm{ML} = (1, 1, \ldots, 1)$:
$$
g(\vec{c}) = g(\vec{c}_\mathrm{ML}) + \sum_i \frac{dg}{dc_i} (c_i - 1) + \mathcal{O}(c_i^2)
$$
So the variance of the values $g(\vec{c})$ that you will get from
bootstrapping is approximately
\begin{align}
\expect{\left[g(\vec{c}) - g(\vec{c}_\mathrm{ML})\right]^2}
  &= \expect{\left(\sum_i \frac{dg}{dc_i} (c_i-1)\right)^2} + \mathcal{O}(c_i^2) \\
  &= \expect{\sum_i \left( \frac{dg}{dc_i} (c_i-1)\right)^2} + \mathcal{O}(c_i^2) \\
  &= \sum_i \left(\frac{dg}{dc_i}\right)^2 \expect{(c_i-1)^2} + \mathcal{O}(c_i^2) \\
\end{align}

And then an $n^2$ comes out? The point is that the jackknife is basically
doing a finite estimation of the gradient, by leaving out a single point at a
time.

\section{\texorpdfstring{What does it mean to
``sample''?}{What does it mean to sample?}}\label{what-does-it-mean-to-sample}

\hl{Earlier in text, get clearer about RVs and their ``realizations''. and ``samples''}

Does it make sense to compute a confidence interval when you're sampled
all the 50 United States?

\textbf{Finite correction factor} to point out that there's a difference
between simple random sampling and something else. Then need to explain
what simple random sampling is!


\subsection{\texorpdfstring{\emph{t}-distribution}{t-distribution}}\label{t-distribution}

\hl{Maybe just mention in passing how difficult the math gets when you want to estimate many quantities simultaneously? Contrast t- and z-tests}

Let's think about how to construct that method. Say you knew the true
variance \(\sigma^2\). Then we know that the sample means are drawn from
\(\mathcal{N}(0, \sigma^2/n)\). So it's pretty easy to see that
\((\overline{x} - \mu) / (\sigma^2) \sim \mathcal{N}(0, 1)\), from which
the familiar \(1.96\), etc. come.

What if you \emph{don't} know the true variance? The means are still
drawn from \(\mathcal{N}(0, \sigma^2/n)\), but now the sample variance
is also a random variable.

We know the confidence interval is some function of the sample mean and
variance, and let's guess that it's symmetric about the sample mean and
is some linear function of sample variance: \[
\mathrm{CI}_\pm(\overline{x}, s) = \overline{x} \pm A s.
\] We want to find \(A\) such that \[
\mathbb{P}\left[ \mathrm{CI}_- < \mu < \mathrm{CI}_+ \right] = 95\%,
\] or, if we're willing to trust in symmetry, \[
2.5\% = \mathbb{P}\left[ \mathrm{CI}_- > \mu \right] = \mathbb{P}\left[ \frac{\overline{x} - \mu}{A} - s > 0 \right].
\] We know the distribution of the first thing: \[
(\overline{x}-\mu)/A \sim \mathcal{N}\left(0, \frac{\sigma^2}{n A^2}\right).
\] Some math shows that \[
\frac{(n-1) s^2}{\sigma^2} \sim \chi^2(n-1).
\]

Call the first thing \(K\) and the second \(L\). We're interested in the
distribution of \(M \equiv K - L\): \[
f_M(m) = \int_0^\infty f_K(m + l) f_L(l) \,\mathrm{d}l,
\]

where the limits come from the fact that variance is positive. You're
probably not excited to do this integral, which was considered a major
achievement (well, it was the thought leading up to the integral, which
we've just outlined, but whatever). This major achievement was made by
William Sealy Gosset, who made it while he was a researcher for Guinness
ensuring the quality of their beer. Guinness had a policy of not
allowing its employee to publish their results, so Gosset signed his
paper ``a student'', so the result of that integral is now called
Student's \emph{t}-distribution: \[
f_t(x; \nu) = \frac{\Gamma(\frac{\nu+1}{2})}{\sqrt{\nu\pi} \Gamma\left(\frac{\nu}{2}\right)}
  \left(1+ \frac{x^2}{\nu}\right)^{-\frac{\nu+1}{2}},
\] where the (badly named) ``degrees of freedom'' \(\nu\) is \(n-1\) for
our purposes. I write this out fully because it is one of the things we
will \emph{not} derive in this book.

\section{Contingency tables}\label{contingency-tables}

\hl{Move Fisher's exact, Barnard, chi-square up, into a section on p-values? Or maybe statistical power?}

These are nice examples for how to do statistical thinking.

\subsection{Barnard's test}\label{barnards-test}

The classic example is whether a certain treatment causes more of the
outcome of interest than just doing nothing. In medicine, that means
splitting your participants into a placebo group and a treatment group
and asking what fraction of each gets well. In a biology experiment, you
might split your mice into a treatment group and a control group and ask
what proportion of the mice in each group get cancer.

In statistics jargon, this is called a \(2 \times 2\) contingency table:

\begin{longtable}[]{@{}llll@{}}
\toprule
Group & Outcome \(p\) & Outcome not-\(p\) & Row sums\tabularnewline
\midrule
\endhead
A & \(a\) & \(c\) & \(m\)\tabularnewline
B & \(b\) & \(d\) & \(n\)\tabularnewline
Column sums & \(r\) & \(s\) & \(N\)\tabularnewline
\bottomrule
\end{longtable}

Because we picked \(m\) and \(n\), the sizes of the two groups, those
are fixed parameters. The question is whether the way that \(m\) gets
distributed into \(a\) and \(c\) (and that way that the \(n\) get put
into the \(b\) and \(d\)) is consistent with there being a common
probability \(p\) of the outcome of interest.

So we might say that \(a\) is distributed like a binomial distribution
with \(m\) draws and probability \(p_a\) of success, and \(b\) is
distributed like a binomial with \(n\) draws and a probability of
\(p_b\) of success. The null hypothesis is that \(p_a = p_b\). What's
the likelihood of the data given the null?

If we didn't assume the null, and gave the two binomials their own
probabilities, the likelihood of the data would be: \[
P(a, b | p_a, p_b) = \mathrm{Bin}(a; m, p_a) \times \mathrm{Bin}(b; n, p_b).
\] But, given that the probabilities are the same, we can collapse it:
\[
\begin{aligned}
\mathcal{P}[a, b | p_a = p_b = p] &= \mathrm{Bin}(a; m, p) \times \mathrm{Bin}(b; n, p) \\
  &= \binom{m}{a} p^a (1-p)^{m-a} \times \binom{n}{b} p^b (1-p)^{n-b} \\
  &= \binom{m}{a} \binom{n}{b} p^{a+b} (1-p)^{m+n-(a+b)} \\
  &= \frac{m! \, n!}{a! \, b! \, c! \, d!} p^r (1-p)^s.
\end{aligned}
\]

This result is a little confusing\footnote{I trotted out this test
  because these two confusions are actually great learning
  opportunities.}, for two reasons:

\begin{enumerate}
\def\labelenumi{\arabic{enumi}.}
\tightlist
\item
  The probability \(p\) of the outcome of interest might be interesting
  to design a later experiment, but it's \emph{not} interesting for
  designing a test. We certainly don't want to deliver a result like,
  ``Well, if the null hypothesis is true, \emph{and} \(p\) happens to be
  exactly such-and-such, then your \(p\)-value is so-and-so.'' The value
  \(p\) is called a \emph{nuisance parameter} since we don't actually
  care about its value.
\item
  We're usually not interested in the likelihood of exactly this data,
  but rather in the likelihood of data \emph{at least this extreme}. We
  usually measure ``extremeness'' using a statistic---a single
  number---so it's clear that ``more extreme'' means ``bigger'' (or
  ``smaller'' or ``bigger or smaller'', depending on if it's a one-sided
  or two-sided test). Here, we have two numbers, \(a\) and \(b\), so
  there aren't two ``sides'' to the distribution: there are four!
\end{enumerate}

To resolve the first point, we say that the null hypothesis
\(p_a = p_b = p\) doesn't restrict us to a particular value of \(p\). In
other words, the null hypothesis, which functions as a sort of Annoying
Skeptic, is free to pick \(p\) to make our results as uninteresting as
possible. Mathematically, this means that, when computing the
\(p\)-value, we should optimize over all values of \(p\), choosing the
one that makes our results as uninteresting as possible (i.e., which
maximizes the \(p\)-value).

We can't really ``resolve'' the second point, since it demonstrates that
our previous way of thinking about extremeness was not sufficient for
all cases. As Barnard notes in his original paper\footnote{Barnard
  conceived of the \((a, b)\) as points ``in a plane lattice diagram of
  points with integer co-ordinates'', that is, that \(a\) is like the
  \(x\)-axis and \(b\) is like the \(y\)-axis. Then the possible
  outcomes of the experiment are the points in the rectangle bounded by
  the horizontal lines \(a = 0\) and \(a = m\) and the vertical lines
  \(b = 0\) and \(b = n\). He then said that you should pick the
  non-extremal points (i.e., the values of \((a, b)\) for which you
  would not reject the null) such that they ``consist of as many points
  as possible, and should like away from that diagonal of the rectangle
  which passes through the origin. Formulated mathematically, these
  latter requirements mean that the {[}points for which you would reject
  the null{]} must in a certain sense be convex, symmetrical and
  minimal.''}, there are actually many ways to choose the pairs
\((a, b)\) that produce a \(p\)-value more than our threshold. This gets
into some fancy footwork to articulate exactly how you should pick this
area, but the basic results are pretty intuitive: when \(a/m\) and
\(b/n\) are similar, you tend to be under the rejection threshold; when
they are far apart, you tend to be over.

The interesting point here is that, whatever fancy footwork you pick to
choose that region, and no matter how ``reasonable'' your footwork is,
it's still footwork that doesn't obviously follow from the simple
definition of a hypothesis test. We'll encounter this problem again in
Bayesian statistics, when we find that the Bayesian analog of a
confidence interval is not unique: there are many ranges of values that
are compatible with our ignorance.

\subsection{Fisher's test to the
rescue(?)}\label{fishers-test-to-the-rescue}

If you've worked with contingency tables, you're probably saying, ``I've
never heard of this crazy Bernard's test, with its weird multi-sided
rejection space and its requirement to maximize over \(p\). We have
Fisher's exact test, which is the exactly right test to use here!''

Looking at the same contingency table, Fisher's test asks, given the row
marginals \(m\) and \(n\), the first column marginal \(r\), and the
grand total \(N\), what is the probability of a table at least this
extreme?

This is just a combinatoric problem: if you're as likely to assign items
in \(m\) to \(a\) as to \(c\) (and, analogously, to assign items from
\(n\) to \(b\) or \(d\)), then ``what's the probability of this table''
is equivalent to asking ``given the marginals, how many ways are there
to choose this table?''. More specifically, how many ways are there to
choose \(a\) items from a bank of \(m\) items and \(b\) items from a
bank of \(n\), given that we chose \(r = a + b\) items from the total
\(N\)? Mathematically: \[
\mathbb{P}[a | m, n, r, s] = \frac{\binom{m}{a} \binom{n}{b}}{\binom{N}{r}} = \frac{m! \, n! \, r! \, s!}{N! \, a! \, b! \, c! \, d!}.
\]

Computing the \(p\)-value is easier here than with Barnard's test
because we need to keep the row \emph{and column} marginals the same. In
Barnard's test, we just kept the row marginals constant, because we
considered those as fixed parameters, corresponding to things like the
number of patients we assigned to each of the placebo and treatment
groups. It doesn't make sense to allow the Annoying Skeptic to fiddle
with those values.

In Banard's test, we \emph{did} allow the Annoying Skeptic to fiddle
with the column marginals, since it wasn't clear, before the experiment
began, that \(r\) would have the outcome of interest. In other words, we
didn't know that \(r\) people in both the placebo and treatment groups
would get well.

Fisher's test, however, \emph{does} keep the column marginal constant.
This makes it a lot easier to compute the \(p\)-value. First, the
nuisance parameter \(p\) doesn't appear in the likelihood, so we don't
need to do the weird maximization. Second, we only need to vary one
value, \(a\) (or, equivalently, \(b\)), since, if you know the
marginals, there is only one axis along which to change the values in
the table. In other words, if you know \[
\begin{aligned}
a + c &= m \\
b + d &= n \\
a + b &= r,
\end{aligned}
\] then that's three equations with four unknowns (\(a\), \(b\), \(c\),
\(d\)), so specifying any one of \(a\), \(b\), \(c\), or \(d\) specifies
all the others. (You might be looking for a fourth equation
\(c + d = s\), but you can get that by adding the first two equations
and subtracting the third.)

Here's an example:

\begin{longtable}[]{@{}llll@{}}
\toprule
Group & Success & Failure & Row sums\tabularnewline
\midrule
\endhead
A & 1 & 9 & 10\tabularnewline
B & 11 & 3 & 14\tabularnewline
Column sums & 12 & 12 & 14\tabularnewline
\bottomrule
\end{longtable}

There's only one way to make this table more ``extreme'' without
changing the marginals: you can take the one group A success and make it
a group A failure and simultaneously make a group B failure into a group
B success. Similarly, there's only one way to make this table less
extreme: turn a group A failure into success, and turn a group B success
into failure.

So keeping the column sums constant made it way easier to compute the
\(p\)-value: count this table and all the tables with a more extreme
upper-left or bottom-right and see if your summed probability hits the
rejection threshold.

However, this simplicity came at a cost, which you may have noticed:
does it make sense to keep the columns constant? Experimentally, this
means that you're restricting the Annoying Skeptic to only consider
cases in which, say, the number of patients who got well \emph{in both
groups} is equal to the experimentally observed value. This is a little
weird. It suggest that your experimental design was like this:

\begin{enumerate}
\def\labelenumi{\arabic{enumi}.}
\tightlist
\item
  Pick \(m\), \(n\), and \(r\).
\item
  Assign \(m\) patients to placebo and \(n\) to treatment.
\item
  Wait until \(r\) patients \emph{across both groups} have gotten well.
\item
  Stop the experiment.
\end{enumerate}

This is almost certainly not reflective of how typical experiments are
run\footnote{It is, however, the way the famous ``lady tea tasting''
  experiment was designed. The myth is that Fisher didn't believe it
  when a high-class lady told him that she could detect whether tea was
  added to a cup with milk in it or whether the milk was added to the
  tea. He designed an experiment with \(m\) cups prepared one way, \(n\)
  prepared the other, and told her to detect the \(r = m\) cups that
  were prepared the first way. A Barnard-style experiment, in which the
  same \(m\) and \(n\) cups}.

\textbf{Fisherian small data}

\textbf{What happens if I use the ``wrong'' test? Chi-square as an
example of wrongness}

\section{Regression}\label{regression}

\hl{Make this a chapter? In descriptive statistics? Or just say this regression is an ML problem? Link and error distributions.}

\section{$\chi^2$ test}

\hl{Merge with section above on chi-square}

Say you have $k$ iid standard normal random variables:
\begin{equation}
X_i \stackrel{\text{iid}}{\sim} \mathcal{N}(0, 1).
\end{equation}
Then $Y = \sum_{i=1}^k x^2$ (??) is $\chi^2$-distributed with $k$ degrees of freedom.

Let's start with a simple case where you have a table with two cells with
expected probabilities $p_1$ or $p_2 = 1-p_1$. We got $n$ total observations,
with $O_1$ in the first cell and $O_2 = n - O_1$ in the second. You probably
remember how to compute the test statistic from Stats 101:
\begin{equation}
\chi^2 = \sum_{k=1}^2 \frac{(O_i - E_i)^2}{E_i} = \frac{(O_1 - np_1)^2}{np_1} + \frac{(O_2 - np_2)^2}{np_2},
\end{equation}
where $E_i$ is the ``expected'' number of counts in each cell.

Consider the numerator of the second term:
\begin{equation*}
(O_2 - np_2)^2 = \left[(n - O_1) - n(1 - p_1)\right]^2 = (-O_1 + np_1)^2 = (O_1 - np_1)^2.
\end{equation*}
Handy, that's the same as numerator of the first term! That means we can re-write things:
\begin{equation*}
\chi^2 = \frac{(O_1 - np_1)^2}{n}\left( \frac{1}{p_1} + \frac{1}{p_2}\right).
\end{equation*}
A little algebra shows that $1/p_1 + 1/p_2 = 1/p_1(1-p_1)$, so that
\begin{equation}
\chi^2 = \frac{(O_1 - np_1)^2}{np_1(1-p_1)} = \left( \frac{O_1-np_1}{\sqrt{np_1(1-p_1)}} \right)^2.
\end{equation}
That might look terrible, but it's actually pretty cool. Here's why: $O_1$ is the observed value, $np_1$ is the expected mean, and $\sqrt{np_1(1-p_1)}$ is the standard deviation of the binomial distribution. I'll re-write that last equation with more suggestive notation:
\begin{equation}
\chi^2 = \left( \frac{x_1 - \mu_1}{\sigma_1} \right)^2
\end{equation}

This certainly \emph{looks} like a $\mathcal{N}(0, 1)$ variable, although we
said previously that the counts in the two cells follow a binomial
distribution. This is where the central limit theorem comes in: the sum of any
large set of (well-behaved) iid random variables approaches a normal
distribution. The binomial distribution approaches the normal distribution
particularly quickly such that (if the distribution is not highly skewed) you
only need about 5 counts for the normal approximation to be pretty
good.\footnote{The normal approximation to the binomial was proved long before
the central limit theorem. This special case, called the \emph{de Moive-Laplace
theorem}, was first published by de Moivre in 1738. Laplace published
the reverse result, that the binomial approximates the normal, 75 years later,
in 1812. The general central limit theorem was proven, more than 150 years
after de Moivre's original result only, in 1901 by Lyapunov. \hl{Put CLT in with regression? Or z-test?}}

So, so long as each cell has (ish) 5 or more counts, then we can approximate
the binomial variables with normal variables, which means that the test
statistic $\chi^2$ that I wrote is actually just the square of a single,
standard normal variable, which happens to be $\chi^2$-square distributed with
1 degree of freedom. Two cells in the table ($k=2$) meant $k-1=1$ degrees of
for the $\chi^2$ distribution.

The same result holds, that the sum of the $(O_i - E_i)^2/E_i$ values follows
a $\chi^2$ distribution with $k-1$ degrees of freedom, for $k>2$. The math is
a lot more involved because the $k$ cells in the table are distributed
according to a multinomial distribution. In other words, conditioned on the
total number $n$ of counts, the values in the different cells are not
independent: if cell 1 has a lot of counts, cells 2, 3, etc. can't have that
many cells. Like we've seen before, covariance makes the calculations hard!
Nevertheless, the same restrictions apply: you can only count on the normal
approximation working if you have enough counts in every cell.

\section{Coda}

\hl{New material}

- Bayes
- Bayes sampling?
- MCMC for copmlex models?
- Nature Biotech Bayes example
- Optimization for MLE
- Regression and mixed models
- Random variable neq variable with a random value
- When discussing RVs, note that cdf defines it. Then don't ever talk about events, just look at joint cdfs, etc. Make this a whole section unto itself.
- Do joint pdf's so it's easier to talk about independence

Tony's ideas:

- Neyman Pearson lemma
- Likelihood ratio tests
- Why is frequentist so good? CLT, convergence, etc.
- Information theory, model simplicity, AIC?

\end{document}
