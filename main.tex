\documentclass{book}

\usepackage{amsmath,amsfonts,amssymb}
\usepackage{bm}
\usepackage{mathtools} % for dcases
\usepackage{hyperref}
\usepackage{longtable}
\usepackage{booktabs}
\usepackage{color,soul} % for hl

\providecommand{\tightlist}{%
  \setlength{\itemsep}{0pt}\setlength{\parskip}{0pt}}

 \newcommand{\prob}[1]{\mathbb{P}\left[{#1}\right]}
 \newcommand{\expect}[1]{\mathbb{E}\!\left[{#1}\right]}
 \newcommand{\var}[1]{\mathbb{V}\left[{#1}\right]}
 \newcommand{\cov}[1]{\mathrm{Cov}\!\left[{#1}\right]}
 \newcommand{\defeq}{\stackrel{\text{def}}{=}}

\title{Statistic for people who do numbers and science}
\author{Scott W. Olesen}

\begin{document}
\maketitle

\frontmatter

%!TEX root=main.tex

\chapter{Preface}

\section*{Who this book is for}

I finished my PhD at MIT in Biological Engineering in 2016. While I was a
graduate student, and afterward when I was a postdoc, I found that graduate
students, especially in the life sciences, have a fair amount of quantitative
training, often including statistics, and are capable of hacking together
fairly sensible quantitative methods that answer their scientific questions. I
found, though, that there was often a steep drop-off when it came time to apply
statistical rigor to these hacky methods. I say this, in part, because it is my
own story too.

This book is for people who are experienced quantitative and scientific
thinkers, have decent algebra and maybesome rusty stats, and who want to learn
about how to use statistical thinking to improve their scientific thinking.

I also tried to have fun when I wrote this book, so there are lots of things
in it that I think are fun, like history\footnote{I think history is actually
a useful part of the study of statistics, because it gets you away from
thinking about statistics as some monolithic thing and gets you toward
thinking about it as a work-in-progress whose shape has been determined by the
abilities and prejudices of its makers. Ronald Fisher's personal tastes will
come up a lot; I have a pet theory that a lot of the flavor of contemporary
statistics is due to his tastes.} and footnotes.

\section*{What this book is for}

As I tried to fill gaps in my statistics knowedge, I found that books about
statistics tended to come in two varities: utterly simplistic and
overwhelmingly complex. It was a similar feeling as trying to learn a new
language as an adult: I have the mind of an adult, and am interested in
grown-up things, but I can only read my target language at the level of an
elementary schooler. Books about grown-up things have overwhelming vocabulary,
while books are childish things are way below my level.

I intend this book to be conceptually advanced but technically manageable. Very
often there is sophisticated mathematical machinery behind something that can
be summarized, from a scientific view, very quickly. On the other hand, there
are some times that a little algebra goes a long way to illustrating a
statistical concept.

For example, "maximum likelihood" is a key concept in statistics. Rather than
show you how you can use linear algebra to compute maximum likelihood values
for linear regression, I will just tell you that there is such software as an
optimizer that finds maxima. On the other hand, I won't shy away from
explaining to you why you divide by $n$ when you compute the mean but you
divide by $n-1$ when you compute the standard deviation, which is pretty easy
to derive algebraically.

\section*{What this book is not}

Maybe it's just as easy to say that this \emph{is} a book for people who want
to \emph{use} rigorous statistics to do science, and maybe even to develop some
new statistical methods, but it is \emph{not} a book for people who want to
push the boundaries of what statisticians think are interesting mathematical
problems.

This is not a cookbook that tells you what steps to do to get a $p$-value. This
is more like Harold McGee's \textit{On Food and Cooking}, that tries to help
give you the principles by which you can understand what the tests are doing.

I also do not intend to fill this book with examples, because I find that it's
very easy to think that your particular scientific question is very
itneresting, and that the statistics that motivate are interesting, but that
it's hard to get motivated by hearing about someone \emph{else's} statistical
woes. I try to keep it general so you can imagine you are the hero of the story.

\section*{What you should be able to do after reading this book}

I basically want you to be able to articulate you own statistical tests, modify
and critique existing methodologies, and develop a healthy skepticism of the
idea of statistical inference in general.

\section*{What else should I read}

My principal sources of motivation for this book were Allen Downey's
\textit{Think Stats} and \textit{Think Bayes}, Miran Lipva\v{c}a's
\textit{Learn You a Haskell for Great Good}, and Stephen Stigler's
\textit{History of Statistics}. That order is probably the most profitable.

\section*{Notes on notation}

\paragraph{Functions} When introducing a function $f$ that maps things in a domain
$D$ to things in a range $R$, I write that as
$$
f : D \to R.
$$
When writing a function $f$ that takes numbers, I use normal parentheses:
$f(x)$. When writing a function that takes things that aren't numbers, I use
square brackets: $\prob{A}$. My idea is to emphasize that which functions
are number-taking and which are something-else-taking.

When a function takes multiple inputs, I use a semicolon to distinguish
between the input(s) that, in that context, we think of as changing and the
input(s) that we think of as fixed. I put the ``variables'' before the
semicolon and the ``parameters'' after. So I would write the probability of
getting $x$ successes out of $n$ trials each with probability $p$ of success
as $f(x; n, p)$. Other people might write this as $f(x | n, p)$, especially
for likelihood, but I like to reserve the bar specifically for conditional
probabilities $\prob{A | B}$.

\paragraph{Probability} When probability is a function, I use the
``blackboard'' letter $\mathbb{P}$. When probability is a number---that is, an
output of the function $\mathbb{P}$---I write $p$. I try to avoid capital
$P$, because I think it's not clear whether that's a function or a number.

\paragraph{Expected value} When expected value is a function, I use the
``blackboard'' letter $\mathbb{E}$. I use an ``overline'' to write the arithmetic
mean of a group of numbers: $\overline{x} = (1/n)\sum_i x_i$. I try to avoid
using the overline with random variables and use, say, $\mu$ as the true mean
and, as per the note below, $\hat{\mu}$ as an estimator of the mean (which,
incidentally, $\bar{x}$ is).

\paragraph{Variance} When variance is a function, I use the ``blackboard''
letter $\mathbb{V}$. As per the note below, I write the estimated variance
as $\hat{\mathbb{V}}$ but the sample variance as $\sigma^2$. I do this in
part to clarify why we use $n$ in the denominator of $\sigma^2$ and $n-1$
in the denominator of $\hat{\mathbb{V}}$.

\paragraph{Estimators} I use ``hats'' for estimators, so $\hat{X}$ is an
estimator for $X$. When it's important to distinguish between estimators,
I use subscripts, so maybe $\hat{X}_\mathrm{ML}$ is the maximum likelihood
estimator for $X$.


\mainmatter

\chapter{What is statistics?}

``Statistics'' is confusing because it means two things. It's a plural noun that
refers to multiple things, each of which is called a ``statistic''. It's also a
singular noun that refers to the study of these mathematical objects.

Statistics is a tool.\footnote{I really enjoyed reading this anecdote from a
paper by Gene Glass, the inventor of meta-analysis (doi:10.1002/jrsm.1133).
Glass. He ends up sitting on the plane next to the famous statistician
Tukey. Tukey asks, "Along which dimension do you see the greatest
variability in effects?" By investigator. "Then jackknife on investigator."
The point is that there's a wide gap between being able to understand what
the jackknife is in a mathematical sense (i.e., how to compute it, whether
it's an unbiased way to determine the variance in the median) and how to apply
it. It also goes to show that "real" statisticians can be pretty hacky!}

A \emph{statistic} is some function of the sample data. Although the field of
statistics is often described as consisting of "descriptive statistics" and
"inferential statistics"---that is, the study of mathematical objects used to
describe data and to make inferences about the populations the data were taken
from---in this book I want to emphasize that, for both purposes, we are
interested in the properties of statistics.

The most interesting statistics are the ones that are used to estimate some
property of the population. These kinds of statistics are sensibly called
\emph{estimators}. Most of the study of statistics comes down to figuring out things
about estimators. One of the most critical is understanding the variance of
estimators, which is essential for both descriptive statistics---so that you
can put error bars on your measurements---and inferential statistics---so you
can make a guess about how probable it is that your data arose under some null
hypothesis. All of this is about estimators and their variance.
%!TEX root = main.tex

\chapter{Probability}
\label{chapter:probability}

To understand statistics, it's critical to have a grasp of basic concepts in
probability, which is an entire branch of mathematics. Here I will give you
the essentials so you can have a grasp of probability theory.

\section{Probability has two definitions}

We talk about and reason about probability every day, mostly for matters of
prediction. Given the weather report, is it worth it for me to carry an
umbrella? Given the results of my experiment, how likely is it that my
hypothesis was correct?

We intuitively think about probability in a mostly \emph{Bayesian} way.
Statistics, when not prefaced by the word ``Bayesian'', refers to a different
philosophical branch call \emph{frequentist} statistics. The philosophical
distinction between Bayesian probability and frequentist probability has
practical implications, so it's important to understand the difference.

In Bayesian statistics, \emph{probability} refers to a sense of confidence
about an unknown thing, often a future event. If I say the probability of an
event is 1\%, it means, in a practical sense, that I'm willing to bet a fair
amount of money it won't happen. Philosophically, it's hard to say what ``1\%
confident'' really means, but the concept should be very intuitive.

The mathematically and philosophically simpler definition of probability is as
a proportion or ``frequency''. A 50\% probability of flipping a coin and seeing
heads means that, as you flip the coin more and more times, the proportion of
flips that come up heads will approach 50\%. One might equivalently say that
the ``frequency'' of heads approaches 50\%. This is called the
\emph{frequentist} definition of probability.


\section{The frequentist system is more coherent but dissatisfying}

The problem with the frequentist definition is that probabilities can only
be assigned to experiments or situations that can be repeated infinitely many
times. It does not make sense to ask about the probability that it will rain
tomorrow any more than it makes sense to ask about the probability that it
rained yesterday: it either rained or it didn't, either it will rain or it
won't.

The fact that we live in just one universe means that, in a frequentist scheme,
you cannot ask about the probability of a state of nature. You cannot ask about
the probability your hypothesis is correct. If you get anything other than zero
or one, you did the math wrong.

As a scientist, this is deeply dissatisying.  The whole point of statistical
inference is to figure out what's going on in the world. I don't want to feed
my hard-won experimental data into a statistical algorithm that says, ``If your
hypothesis is true, then it is; and if it's not, it's not.''

The Bayesian approach is more appealing in that it does in fact allow you to
ask about the probability of states of nature. There is such a thing as a
Bayesian probability that it will rain tomorrow or that your scientific
hypothesis is correct.

Unfortunately, Bayesian statistics is more mathematically and technically
challenging. It also requires an assertion of a \emph{prior probability}: you
must state the probability that your hypothesis was true before you started the
experiment. This idea is bizarre and repugnant to many scientists today. More
importantly, some early, prominent statisticians thought it was nonsense and
mostly directed the field of statistics away from the Bayesian approach for
many years.

Perhaps not surprisingly, frequentist statistics are more common.
``Statistics'' means frequentist statistics unless someone explicitly says
``Bayesian''. I follow this convention and will discuss only frequentist
statistics. Sorry.


\section{The mathematical definition of probability}


Above we said that the frequentist definition of probability has to do with the
proportions of infinitely-repeatable trials that have some \emph{outcome}. An
outcome is any thing that can happen as a result of an infinitely-repeatable
trial. For example, if you flip a coin, you will get heads, or you will get
tails.

The \emph{sample space}, written $\Omega$, consists of all possible subsets of
\emph{outcomes}, each written $\omega$. Members of the sample space are called
\emph{events}.  Clearly, each outcome is itself an event: ``flipped heads'' and
``flipped tails'' are both events. But the empty set $\varnothing$ is also an
event (``nothing happened''), and the set of all outcomes $\Omega$ is also an
event (``something happened''). Some other examples of outcomes, events, and
sample spaces are in Table \ref{tab:outcome-event}. I emphasize that
\emph{outcomes} are individual, real things that might happen, while
\emph{events} are abstract groupings of outcomes.

\begin{table}
\centering
\begin{tabular}{p{3cm}p{4cm}p{6cm}}
\toprule
Situation & Outcomes & Events \\
\midrule
Flipping a coin & heads, tails & $\varnothing$, $H$, $T$, $\Omega$ \\
Rolling a die & $1, 2, \ldots, 6$ & $\varnothing$; $1, 2, \ldots, 6$; $\binom{6}{2}$ 2-event outcomes (e.g., 1 or 2); $\binom{6}{3}$ 3-outcome events (e.g., 1, 2, or 3); $\ldots$; $\binom{6}{5}$ 5-outcome events (e.g., any except 1); $\Omega$ \\
Drawing a card & 2 of Clubs, $\ldots$, Ace of Hearts & $\varnothing$; 52 individual events (e.g., 2 of Clubs); $\binom{52}{2}$ 2-outcome events (e.g., black Ace); $\ldots$; $\binom{52}{51}$ 51-outcome events (e.g., any except 2 of Clubs); $\Omega$ \\
Pick a random number between 0 and 1 & All real numbers between 0 and 1 & $\varnothing$; intervals like $[0, \tfrac{1}{2}]$; ``all subsets'' of $[0, 1]$; $\Omega$ \\
Your experiment & Each possible configuration of atoms at measurement & ``Cancer cell died'', ``Electron had energy between 1 and 2 eV'', etc. \\
\bottomrule
\end{tabular}
\caption{Distinction between outcomes and events for various example
situations.}
\label{tab:outcome-event}
\end{table}

As an aside, I note that defining ``all possible subsets of outcomes'' is
straightforward for a discrete case like flipping a coin or drawing a card, but
it is complicated for a continuous case like drawing a random real number
between $0$ and $1$. In continuous cases, a rigorous definition for the event
space $\Omega$ requires concepts from \emph{measure theory}, including a
$\sigma$-\emph{algebra}. I avoid these complexities in this chapter at the cost
of presenting a mathematically incomplete theory of probability.

\emph{Probability} is a function that maps events to numbers between $0$ and
$1$:
\begin{equation*}
\mathbb{P} : \Omega \to [0, 1]
\end{equation*}
For example, if $H$ is the event of flipping heads, then $\prob{H} =
\tfrac{1}{2}$. I use the square brackets to emphasize that $\mathbb{P}$ is a
function of something other than numbers: $H$ is not a number like $5$, it is
an \emph{event}, a distinct mathematical object.

Confusingly, ``probability'' refer both to the function $\mathbb{P}$ as well as
the number $p = \prob{\omega}$ that comes from applying this function to an
event $\omega$.


\section{Manipulating probabilities}

\subsection{Probability functions follow certain axiomatic rules}

Comfortingly, probability functions are axiomatically required to be defined so
that that the probability that ``nothing happened'' is zero and the probability
that ``something happened'' is one:
\begin{align*}
\prob{\Omega} &= 1 \\
\prob{\varnothing} &= 0
\end{align*}
Something must happen, and nothing cannot happen.

The axioms about probability functions also mean they follow specific rules
around manipulating probabilities. For example, we're often interested in the
relationships between events. What is that probability that this \emph{or} that
happened? What is the probability that this \emph{and} that happened?

\subsection{``Or'' adds event probabilities}

If $A$ and $B$ are two events that don't have any constituent outcomes in
common, we call then \emph{disjoint}, and their probabilities add. For example,
the probability of flipping a heads \emph{or} flipping a tails is the
probability of heads plus the probability of tails.  Mathematically we write
this as
$$
\text{if } A \cap B = \varnothing \text{, then } \prob{A \cup B} = \prob{A} + \prob{B},
$$
where the ``cap'' $\cap$ means \emph{intersection} (``and'') and the ``cup'' $\cup$ means
\emph{union} (``or''), so this reads ``if no outcomes are in both $A$ and $B$,
then the probability of $A$ or $B$ is the sum of their individual probabilities.''

If $A$ and $B$ do have some overlap, you need to subtract out the probability
of the overlap. For example, consider drawing a card
from a standard 52-card deck. What is the probability of drawing a Jack \emph{or} a
Diamond? If you add up the probability of drawing a Jack and the probability
of drawing a Diamond, you end up double-counting the Jack of Diamonds event,
so the solution is to subtract out the double-counted event:
$$
\prob{A \cup B} = \prob{A} + \prob{B} - \prob{A \cap B}.
$$
We mostly deal with disjoint events, so I won't belabor this point.

\subsection{``And'' multiplies events}

We just said that ``or'' for disjoint events adds probabilities. How do we
find the probability of $A$ and $B$? Say I flip two coins. What's the
probability that I flip heads on the first coin \emph{and} heads on the second?

If $A$ and $B$ are \emph{independent} events, then their probabilities
multiply. The probability of flipping two heads in a row is $\tfrac{1}{2} \times
\tfrac{1}{2} = \tfrac{1}{4}$.

\subsection{Independence and conditional probability}

Mathematically, $A$ and $B$ are indepedent if and only if their probabilities
multiply. This might feel circular, so I will show you the logic.

Intuitively, two events are independent is they do not depend on each other. If
I flip two separate coins, the flip of one coin can't affect the flip of the
other.

Perhaps surprisingly, ``depends on'' hits on a deep philosophical question.
Just as ``probability'' had frequentist and Bayesian definitions, so there are
multiple definitions of \emph{conditional probability}. The easiest
(``Kolomogorov'') definition is:
\begin{equation*}
\prob{A | B} = \frac{\prob{A \cap B}}{\prob{B}},
\end{equation*}
where $\prob{A | B}$ is pronounced ``the probability of $A$ given $B$''.

Although the definition of conditional probability is a mathematical axiom that
cannot be proven or disproven, it is easy to get an intuitive picture of why it
is chosen as an axiom. In a frequentist interpretation of probability, $\prob{A
| B}$ is, on a denominator of the trials in which $B$ happened, the proportion
of trials in which $A$ also happened. Say $n_B$ is the number of trials in
which $B$ happened, $n_{AB}$ is the number of trials in which $A$ and $B$
happened, and $n$ is the total number of trials. Then the proportion we're
talking about is $n_{AB} / n_B$, which corresponds to $\prob{A \cap B} /
\prob{B}$. I like to think of conditional probability as ``zooming in'' to a
smaller sample space: rather than thinking about all of $\Omega$, we are
thinking only about the subset of outcomes $\omega$ where $B$ happened (Figure
\ref{fig:conditional-probability}).

\begin{figure}
\caption{Conditional probability as shrinking of universe}
\label{fig:conditional-probability}
\end{figure}

If $A$ and $B$ are independent, then $A$ does not ``depend on'' $B$: the
probability of $A$ given that $B$ happened is equal to the probability of $A$
without knowing anything about $B$: if $A$ and $B$ are independent, then
$\prob{A | B} = \prob{A}$, and thus $\prob{A \cap B} = \prob{A} \times
\prob{B}$.

%!TEX root=main.tex

\chapter{Random variables}

In the previous chapter, we examined outcomes, events, and their
probabilities. Outcomes are specific configurations of atoms, which is not at
all a practical thing for a scientist think about. Events are abstract
groupings of those configurations, which is also impractical. Instead, in
science, we measure the results of experiments with numbers. We link whatever
we see with a number. In microbiology, we could cells. In physics, we measure
the energy of an electron.

In probability theory, \emph{random variables} link outcomes with numbers.
They are the mathematical analogue of scientific measurement. Random variables
will lead us to familiar concepts like means, standard deviations, and
probability distributions.

\section{Definition of random variables}

Confusingly, a ``random variable'' is a function, not a number. We might say
that a random variable ``takes on'' a value, but the random variable itself,
in mathematical terms is always a function.

A \emph{discrete} random variable $X$ is a function that associates outcomes
$\omega \in \Omega$ with a finite set of numbers $x_i$:
\begin{gather*}
X : \Omega \to \{x_1, \ldots x_k\} \\
X[\omega] = x_i
\end{gather*}
for some finite $k$. A \emph{continuous} random variables associates outcomes with real
numbers $x \in \mathbb{R}$:
\begin{gather*}
X : \Omega \to \mathbb{R} \\
X[\omega] = x
\end{gather*}
This is a simplification: measure theory says that a random variable can map
the outcome space to any \emph{measurable space}. We won't dive into that.

For example, when flipping a die, you could imagine a random variable $X$
encoding ``number of heads flipped'':
\begin{align*}
X[H] &= 1 \\
X[T] &= 0
\end{align*}
Roughly speaking, the values of the random variable group the outcomes into
events, which each have probabilities. This means it makes sense to ask about
the event in which a random variable takes on a value, e.g., $\prob{X = 1} = \tfrac{1}{2}$.
Note that ``$X = 1$'' does not mean that $X$ is the number one; instead, the
entire phrase ``$X = 1$'' refers to the event where $X$ took on the value one.

Random variables therefore allow us to abstract over the outcomes and look
only at the values taken on by the random variable. From now on, we will have
no reason to think about the specific configuration of atoms in our
experiment. We only think about the numbers delivered by our apparatus. Soon
we'll be able to say things like ``$X$ is normally distributed'' without having
to specify what are the outcomes and events.

The previous chapter was not a waste, however, because the
rules for manipulating events and probabilities work just as well for
events like ``$X = 1$'' as they did for ``flipped heads''. The philosophical and
technical definition of probability still holds. For example, $\prob{(X = 1)
\cup (X = 0)}$, the probability that $X$ takes on the values zero or one, is the
sum of the two constituent probabilities $\prob{X=1}$ and
$\prob{X=0}$.

\section{Distribution of random variables}

How is $X$ ``distributed''? If you're like me, you think in terms of
\emph{probability distribution functions} (pdf's) and consider
\emph{cumulative distribution functions} (cdf's) as a derived quantity, but in
fact it's mathematically easier to go the reverse route. If you aren't
familiar with either, never fear.

\subsection{Cumulative distribution functions}

The first way of defining the ``distribution'' of a random variable $X$ is its
\emph{cumulative distribution function}, also called the cumulative
\emph{density} function and abbreviated ``cdf''. The cdf of a discrete random
variable $X$ is the probability that the random variable takes on a value
below some threshold $x_i$:
\begin{gather*}
F_X : \{x_1, \ldots, x_k\} \to [0, 1] \\
F_X(x_i) \defeq \prob{X \leq x_i} = \sum_{j \,:\, x_j \leq x_i} \prob{X = x_j}
\end{gather*}
A continuous random variable has a cdf that takes real numbers:
\begin{gather*}
F_X : \mathbb{R} \to [0, 1] \\
F_X(x) \defeq \prob{X \leq x},
\end{gather*}
Note that the cdf is a function of numbers, which I emphasize by
using regular parentheses around $x$ in $F_X(x)$.

Take careful note that ``$X \leq x$'' is an event, not a Boolean expression
like ``$1 \leq 2$''. This may seem pedantic, but I think having a really good
grasp of the notation will help you articulate correct thoughts more clearly.
Things will quickly get confusing if you think that $\prob{X \leq x}$ means the same
thing as $\prob{x \leq x}$.

% swo: need a figure here with the laddering

A discrete random variable takes on a maximum value $x_\mathrm{max}$ so that
\begin{gather}
\text{if } x < x_\mathrm{min} \text{, then } F_X(x) = 0 \\
F_X(x_\mathrm{max}) = 1.
\end{gather}
If a continuous random variable can take on any real number, then  the cdf of
a continuous random variable approaches zero and one as $x$ goes out toward
infinity:
\begin{gather}
\lim_{x \to -\infty} F_X(x) = 0 \\
\lim_{x \to \infty} F_X(x) = 1
\end{gather}

% need a figure here with, say, normal distribution

\subsection{Probability distribution function}

For a discrete random variable, it's straightforward to define its
\emph{probability mass function} or ``pmf'', which is just the probability
that it takes on each discrete value $x_i$:
\begin{gather*}
f_X : \{x_1, \ldots, x_k\} \to [0, 1] \\
f_X(x_i) = \prob{X = x_i}
\end{gather*}

The analogue for continuous random variables is the probability
\emph{distribution} function, also called probability \emph{density} function
or ``pdf'':
\begin{gather*}
f_X(x) : \mathbb{R} \to [0, 1] \\
f_X(x) \defeq \frac{d}{dx} F_X(x)
\end{gather*}

Note that the pdf of a continuous random variable is not $\prob{X = x}$. This
may seem like a pedantic diversion, but I actually think it's important to
avoid confusion. For a continuous random variable, $\prob{X = x}$ is zero. For
example, say $X$ takes on values between 0 and 1 symmetrically, e.g., the
probability that $X$ comes out between 0 and 0.5 is the same as between 0.5
and 1. Say you're asking about the probability that $X$ comes out as 0.5. I'll
grant the following:
\begin{align*}
\prob{0 \leq X \leq 1} &= 1 \\
\prob{0.495 \leq X \leq 0.505} &= 0.1 \\
\prob{0.4995 \leq X \leq 0.5005} &= 0.01 \\
&\vdots
\end{align*}
and so on. We normally don't say that $\prob{X = 0.500\ldots} = 0$, since that
makes it sound like it's impossible for $X$ to take on the value $0.5$.
Nevertheless, it should be clear that, from any practical point of view, it
doesn't make sense to write things like $\prob{X = 0.5}$. It does make sense
to write $f_X(0.5)$; it does not make sense to write $\prob{X =
0.5}$.\footnote{Those dissatisfied with this explanation will need to learn
some measure theory.}

The word ``density'' in probability density function emphasizes that the pdf,
when integrated, gives probabilities:
\begin{equation}\label{eq:integrated_pdf}
\int_{x_0}^{x_1} f_X(x') \,dx' = F_X(x_1) - F_X(x_0) = \prob{x_0 < X \leq x_1}
\end{equation}
If $X$ can take on any value, then it follows from the definition of the cdf and
some fundamental calculus that
\begin{equation}
\int_{-\infty}^\infty f_X(x') \,dx' = 1
\end{equation}
that is, that all the probability must be somewhere, and
\begin{equation}
\lim_{x \to \pm \infty} = 0.
\end{equation}

These definitions I've given are simple ones, and they need to be refined to
deal with more complicated functions. For example, if the cdf has
discontinuities, you need careful with the limits on the integrals.

\subsection{Known distributions}

If a random variable $X$ has the same cdf as another random variable $Y$ (and
therefore also the same pdf), we say that $X$ is ``distributed like'' $Y$:
$$
X \sim Y.
$$
This notation is used to refer to well-documented distributions. For example,
you might say that $X$ is distributed like a normal random variable with mean
$\mu$ and variance $\sigma^2$:
$$
X \sim \mathcal{N}(x; \mu, \sigma^2),
$$
which is a shorthand for saying
\begin{equation*}
f_X(x) = f_\mathcal{N}(x; \mu, \sigma^2) \defeq
  \frac{1}{\sqrt{2\pi\sigma^2}} \exp \left\{-\frac{1}{2} \frac{(x-\mu)^2}{\sigma^2} \right\}.
\end{equation*}
Similarly $X \sim \mathrm{Binom}(n, p)$ means that $X$ follows a binomial distribution:
\begin{equation*}
f_X(x) = f_\mathrm{binom}(x; n, p) \defeq \binom{n}{x} x^p (n-x)^{1-p}.
\end{equation*}
We'll come back to these distributions and their pdf's later; I just want to
you understand that, when we say that ``$X$ is normally distributed'', we
making a mathematically precise statement about $X$'s pdf and cdf.

\subsection{Independent, identically-distributed random variables}

A common construct is to consider \emph{independent, identically-distributed}
(or ``iid'') random variables. This means you have a collection of random variables $X_i$
that all have the same cdf and are all independent of one another. For example,
\begin{equation*}
X \stackrel{\text{iid}}{\sim} \mathcal{N}(\mu, \sigma^2)
\end{equation*}
means that $f_{X_i}(x) = f_\mathcal{N}(x; \mu, \sigma^2)$ for every random variable
$X_i$ and that all the $X_i$ and $X_j$ are independent for all $i \neq j$.

Independent, identically-distributed random variables are important for
frequentist statistics because they represent the ``many independent trials''
in the frequentist definition of probability as a proportion of many
independent trials in which an outcome emerges. As the number of iid variables
increases, the proportion of discrete iid random variables that take on each
discrete value will approach the probability of that value. Similarly, the
proportion of continuous iid random variable that take on values in a range $a
\leq X < b$ will approach $F_X(b) - F_X(a)$.

In day-to-day speech, we might say, ``I draw $n$ values from this random
variable $X$''. Mathematically, this means, ``Consider $n$ iid random
variables distributed like $X$''.

% swo: do events or outcomes get probabilities?

\section{Sums and products of random variables}

If you know the cdf and pdf for a random variable $X$, and you also know these
values for another random variable $Y$, you might be interested in the
behavior of these variables together. An important example is the sum of many
independent, identically-distributed random variables, which we will see are
intimately linked to the normal distribution. Many specific distributions have
nice shortcuts for figuring out the behavior of sums of random variables with
those distributions. For example, if $X \sim \mathcal{N}(\mu_1, \sigma_1^2)$
and $Y \sim \mathcal{N}(\mu_2, \sigma_2^2)$, then $X + Y \sim
\mathcal{N}(\mu_1 + \mu_2, \sigma_1^2 + \sigma_2^2)$. Where do those nice
rules come from?

First, let's be clear about the notation. If $X$ and $Y$ are random variables,
then what does $Z = X + Y$ mean? Remember that $X$ and $Y$ are both functions,
so the plus sign in $X + Y$ doesn't mean ``add numbers''. Instead, it's an intuitive shorthand:
\begin{equation}
Z = X + Y \text{ means } Z[\omega] = X[\omega] + Y[\omega].
\end{equation}
That is, for every outcome $\omega$, $Z$ takes on the sum of the values that
$X$ and $Y$ take on. In other words, $Z$ is a function that, for every input
$\omega$, outputs the sum of the outputs of $X$ and $Y$ when they in turns are
fed the input $\omega$.

This means that, for some event, if $X$ takes on the value $x$ and $Y$ takes
on the value $y$, then $Z$ takes on $x+y$. It follows that, if
$Z$ takes on the value $z$ and $X$ takes on $x$, then it must be that $Y$
took on $z - x$. Therefore the probability that $Z$ takes on value $z$ is
sum of the probabilities that $X$ took on $x_i$ and $Y$ takes on $z - x_i$.
If $X$ and $Y$ are independent,
$\prob{(X = x_i) \cap (Y = z - x_i)} = \prob{X = x_i} \times \prob{Y = z - x_i}$
so that
\begin{align*}
f_Z(z) &= \sum_i \prob{(X=x_i) \cap (Y=z - x_i)} \\
  &= \sum_i \prob{X=x_i} \prob{Y=z-x_i} \\
  &= \sum_i f_X(x_i) f_Y(z - x_i).
\end{align*}

A similar definition holds for when $Z = X + Y$ is continuous. Rather than
summing over all the outcomes in which $X$ takes on certain values, we integrate:
\begin{equation}
f_Z(z) = \int f_X(x') f_Y(z - x') \,dx'
\end{equation}

For a product of random variables $Z = XY$, we say that,
if $Z = z$ and $X = x$, then it must be that $Y = z/x$.
For a ratio of random variables $Z = X/Y$, it must be that $Y = zx$.

For any particular $X$ and $Y$, the derived variables $X + Y$, $XY$, and $X/Y$
are usually distributed in some really messy way. For example, although the
sum of two normally-distributed random variables is normally distributed,
neither the product nor the ratio of normally-distributed random variables are
normally distributed.

\section{Properties of random variables}

Two properties of random variables, their \emph{expected values} and
\emph{variances}, will come up repeatedly.

\subsection{Don't expect the expected value}

A random variable's \emph{expected value} is the probability-weighted average
of the values it takes on. It is a function that links random variables with
numbers:
\begin{equation*}
\mathbb{E} : \text{random variables} \to \mathbb{R}
\end{equation*}
For a discrete random variable, you simply sum. For a continuous random
variable, you need to integrate:
\begin{equation}
\expect{X} = \begin{dcases}
  \sum_i x_i f_X(x_i) & \text{ for discrete random variables} \\
  \int x' f_X(x') \,dx' & \text{ for continuous random variables}
\end{dcases}
\end{equation}
You might also see people use notation like $E[X]$, $\mathbb{E}X$, or $\mu_X$.

% swo: expected vs. expectation. go with "expected"

The name ``expectation value'' is a little confusing. We previously noted that,
for continuous random variables, the probability of getting any particular
number is essentially zero, so there's no particular number you should
``expect''. For discrete random variables, where there is finite probability
of getting each particular number, the expected value need not be any of the
values actually taken on by $X$. In our coin flip example, the random variable
$X$ measuring the number of heads flipped has expectation value:
\begin{equation}
\expect{X} = 0 \times \prob{T} + 1 \times \prob{H} = \tfrac{1}{2}.
\end{equation}
You certainly don't ever expect to flip $0.5$ heads.

Even if you don't expect the expectation value, it is a handy mathematical
tool, and it is a rough of measure of what kind of values $X$ is
``centered'' around. Even this is slippery, since, as you might
recall from learning in high school about the difference between mean, mode, and median, what ``the
center'' means is not necessarily obvious.

\subsection{Linearity of expectation}

Expected values have some nice properties that make them really easy to deal
with. For example, the expectation value of a shifted random variable like $X +
1$ is just $\expect{X} + 1$: you're still ``averaging'' $X$ overall the
outomes, but you're just adding 1 to everything you averaged, which is clearly
just the ``average'' $X$ plus 1. Similarly, you should be able to see that
$\expect{2X} = 2 \,\expect{X}$.

You could think of ``1'' in the previous example as being a really boring
random variable: it gives 1 for every outcome. This might lead you to suspect
that $\expect{X + Y} = \expect{X} + \expect{Y}$, which is true. The fact that
you can easily transform the expectation values of random variables multiplied by
numbers and the sum of random variables is called the \emph{linearity of expectation}:
\begin{equation}
\expect{aX + bY} = a \,\expect{X} + b \,\expect{Y}.
\end{equation}
The proof of this fact follows from the logic used to find the pmf of $Z = X +
Y$. It's not hard, but it's not particularly enlightening either.

Interestingly, the product $XY$ of two random variables is nice only if the
variables $X$ and $Y$ are independent. In that case, $\expect{XY} = \expect{X}\,\expect{Y}$.
More generally, there is some residual term call the \emph{covariance}:
\begin{equation}
\expect{XY} = \expect{X} \, \expect{Y} + \cov{X, Y}.
\end{equation}
Before getting too much into covariance, we should start with regular variance.

\subsection{Variance}

If the expected value is some measure of position, then \emph{variance} is the
corresponding measurement of spread. It is a function of random variables that
returns a nonnegative number:
\begin{equation*}
\mathbb{V} : \text{random variables} \to [0, \infty).
\end{equation*}
The variance of $X$ is also often written $\mathrm{Var}[X]$, $V(X)$, or $\sigma^2_X$.

For a random variable $X$, the variance is the expected value of the square of
the deviation of the random variable from its own expected value:
\begin{equation*}
\var{X} \defeq \expect{(X - \expect{X})^2}
\end{equation*}
Because all those nested brackets are confusing, people sometimes write
$\expect{X}$, which is just a number, as $\mu$ so that $\var{X} = \expect{(X -
\mu)^2}$.

Variance is always nonnegative because it's a weighted average over squares,
which are always nonnegative. The variance of a random variable is zero only
in the very boring case where that random variable always takes on the same
value.

Note that the square in $(X - \expect{X})^2$ is a shorthand for taking the
product of two iid random variables: $X^2$ refers to the product $X_1 X_2$,
where $X_1$ and $X_2$ are distributed like $X$.

Unlike the expected value, variance is not linear. Consider the random variable $aX$,
where $a$ is just a number:
\begin{equation}
\var{aX} = \expect{(aX - \expect{aX})^2} = \expect{a^2(X - \expect{X})} = a^2 \, \expect{X}.
\end{equation}
For a sum, the situation is even more confusing. Some algebra shows that
\begin{equation*}
\var{X + Y} = \var{X} + \var{Y} + 2 \,\cov{X, Y},
\end{equation*}
where $\cov{X, Y}$, introduced in the end of the last section, is the
\emph{covariance} of $X$ and $Y$:
\begin{equation}
\cov{X, Y} \defeq \expect{(X - \expect{X})(Y - \expect{Y})}.
\end{equation}

In words, the covariance is the expected value of the product of the
deviations of $X$ and $Y$ from their expected values. If, when $X$ takes on
values greater than its expected value, $Y$ also tends to take on values
greater than its expected value, then $\cov{X, Y} > 0$. Conversely, if, when
$X$ has positive deviations, $Y$ has negative deviations, then $\cov{X, Y} <
0$. If $X$ and $Y$ are independent, then $\cov{X, Y} = 0$.

\subsection{Correlation}

Rather than covariance, you're probably more familiar with the term
\emph{correlation}, especially the familiar Pearson's correlation coefficient,
usually written $\rho$ or $r$. The correlation coefficient is a function of
two random variables that gives a number between $-1$ and $1$:
\begin{gather*}
\rho : (\text{random variables}) \times (\text{random variables}) \to [-1, 1] \\
\rho[X, Y] \defeq \frac{\cov{X, Y}}{\sqrt{\var{X} \var{Y}}},
\end{gather*}
In other words, the correlation between $X$ and $Y$ is their covariance
``normalized'' by their variances. The reason for dividing by the two
variances was so that $\rho$ is between $-1$ and $1$. The math isn't hard, but
it's not particularly useful for us.

% swo: figure, or integrate this paragraph

To see how covariance and correlation are related, think about a correlated
$X$ and $Y$, like a simple bivariate normal distribution. \hl{figure} In
general, the points for which $X$ is greater than its mean value are the same
points for which $Y$ is greater than its mean value. Similarly, when one
deviation is negative, the other tends to be negative. This way, the overall
\emph{product} of the two deviations tends to be positive. In constrast, if
two variables are \emph{anticorrelated}, then when one is higher than average
the other tends to be lower than average, so the product of deviations tends
to be negative.

%!TEX root=main

\chapter{Estimators}

\section{A statistic is a function of data}

In common speech, we use the terms like expected value, variance, covariance, and
correlation in reference to data, that is, to sets of fixed numbers, rather than
to mathematical objects like random variables. In this chapter, we will bridge the
gap between the probability theory we have considered so far and the practice of
statistics.

The term ``statistics'' has two meanings. As a singular noun, a \emph{statistic},
is a function that maps data, that is, an ordered list of numbers, to a real
number. We use $t$ as a stand-in to mean ``any old statistic'':
\begin{equation*}
    t : \mathbb{R}^n \to \mathbb{R}
\end{equation*}
To study the properties of a statistic, we will define a random variable that is
analogous to that statistic. A random variable that is analogous to a statistic is
called an \emph{estimator}, often written $T$.

For example, the arithmetic mean maps a list of numbers $x_i$ to a single number, and
so is a statistic, which I will write as $t$, such that $t \defeq (1/n) \sum_i x_i$.
There is an analogous estimator $T \defeq t(X) = (1/n) \sum_i X_i$. As I emphasized in the
previous chapter, although the notation in these two equations is almost identical,
just replacing some lowercase letters with uppercase ones, it is critical to remember
that adding random variables is not the same thing as adding numbers. The function
$t$ maps numbers to a number, so $t(X)$ is, strictly speaking, nonsense. But in the
same way that we use the plus sign when adding numbers, like in $x + y$, and for
``adding'' random variables, like in $X + Y$, we use $t$ to refer to an operation on
numbers as well as to the analogous operation on random variables.

The word ``statistics'' ending in \textit{s} could either be the plural of statistic,
that is, many functions of the data, or it can refer to the field of study of
estimators. Thus it is syntactically and semantically correct to say ``statistics
\textit{is} fun''.

Up to this point, we have discussed random variables in a deductive way:
given the distribution of the random variable, you ask
what its expected value or variance is. Estimators are the starting point of
statistics (the field of study), which goes the opposite way: given some data
points, what can we say about the distribution of the random variables that gave
rise to that data?

\section{Parameters, populations, and samples}

Some people refer to statistics (the functions) as \emph{sample statistics} to
emphasize that they are functions of a sample of data that was drawn from some
larger \emph{population}, which has some fixed an unknowable \emph{parameters}
that describe it. For example, if you draw many data points $x_i$ from a distribution
and compute the mean of the drawn data points, you do not expect that the
\emph{sample} mean that you compute will be exactly equal to the true
\emph{population} mean.

In mathematical terms, we say that a random variable $X$ has some expected value
$\mathbb{E}[X]$ that is fixed. A random variable is a function, and the expected
value is another function that links the random variable with a single number.
There is nothing ``random'' about this so far. The analogy to the random sample
are iid random variables $X_i$, so the sample mean is analogous to the estimator
$t(X)$.

Ideally, the estimator $t(X)$ will actually reflect the parameter $\mathbb{E}[X]$
that we wanted to learn about. The rest of this section will discuss what makes
a ``good'' estimator.

\section{A useful example: the ``German tank'' problem}

To examine the properties of estimators and what makes an estimator a ``good'' one,
I will use an example that is mathematically tractable but just unfamiliar enough
to make you think.

During World War II, it was important to the Allies to know how many tanks
Germany was producing. The traditional approach was to use spies and aerial
reconnaissance. The new approach was to use statistics on serial numbers,
which turned out to be much more accurate.

A serial number is a unique number written on a manufactured part. Serial numbers
are usually assigned in sequential order, so that older parts have lower
serial numbers and newer parts have higher numbers. German tanks had serial
numbers. When
the Allies captured German tanks, they took note of those numbers, which gave
them a clue about how may thanks there were. For example, if you captured three
tanks and found serial numbers 1, 3, and 5, you know there are at least 5 tanks
total, and there probably aren't more than 10 or so. If you find serial numbers
100, 300, and 500, then you know there are at least 500 tanks, and there are
probably more like 1,000.

The Allies had a fairly complex problem, because they wanted to estimate the
rate of production, and they had many serial numbers on many different parts
of the tank, some of which were not
exactly sequential. Let's instead consider an abstracted, simplified version of
this problem. Say you drew $n$ numbers from \emph{uniform distribution} ranging
from $A$ to some unknown upper limit $B$. You want to estimate $B$, which is
analogous to the number of German tanks out there.

Mathematically, we are interested in iid random variables $X$ that follow the
uniform distribution:
\begin{gather*}
    f_X(x) = \frac{1}{B-A} \text{ for } A \leq x \leq B \\
    F_X(x) = \frac{x-A}{B-A} \text{ for } A \leq x \leq B
\end{gather*}
For simplicity, let's say that $A=0$ so that $f_X(x) = 1/B$ and $f_X(x) = x/B$
for $0 \leq x \leq B$.

Our challenge is to create a statistic, a function of the data, that estimates
the parameter $B$ and then examine the mathematical properties of the corresponding estimator. It's typical to denote estimators with a
``hat'', so we will write our estimators like $\hat{B}$. It's also typical to denote
statistics with lower-case letters reminiscent of the true value, so we'll
use names like $b$. To be clear, $b$ is a function of numbers, although we also
write $\hat{B} = b(X)$, where now ``$b$'' means some manipulation of a set of iid
random variables to create a new random variable.

\section{Consistent estimators}

Let me start with what will seem like a very inane choice of statistic.
Maybe my favorite number is 3, so I will say that, regardless of what data I
collect, I will just always guess that $B$ is 3:
\begin{equation*}
    b(x_1, \ldots, x_n) \defeq 3.
\end{equation*}
The corresponding estimator $\hat{B}$ is very simple. It takes on the
value 3 with 100\% probability:
\begin{equation*}
    f_{\hat{B}}(x) = \begin{cases}
        1 &\text{if $x = 3$} \\
        0 &\text{otherwise}
    \end{cases}
\end{equation*}

This is clearly a bad estimator. No matter how much data I accumulate, my
estimate doesn't improve. I will only be ``right'' if $B$, by some miracle,
happens to be exactly $3$. By contrast, my expectation would be that, in
the limit of collecting a lot of data, the statistic $b$ would be almost
guaranteed to be very close to $B$. This requirement is called \emph{consistency}.

Mathematically, I want the probability that $\hat{B}$ takes on a value far from $B$ to get close to zero as I collect more and more data. This requires defining what a ``limit'' is for a series of random variables. For a sequence of numbers, a limit means that, for any \textit{a priori}, fixed threshold $\varepsilon > 0$, there is some integer $n$ such that every value in the sequence after the $n$-th one is within $\varepsilon$ of the true value:
\begin{equation*}
    \lim_{n\to\infty} a_n = L \text{ means that for any $\varepsilon > 0$ there exists some $n$ such that } |a_i - L| < \varepsilon \text{ for all $i \geq n$}.
\end{equation*}
For example, given the geometric series $1, \tfrac{1}{2}, \tfrac{1}{4}, \ldots$,
if you pick the threshold $\varepsilon = \tfrac{1}{100}$, I can pick $n=7$, since the terms in the series are equal to $\tfrac{1}{2^n}$, and $\tfrac{1}{2^7} = \tfrac{1}{128}$. All subsequent terms in the series will be even closer to zero than that term.

In probability, if you give me $\varepsilon$, in most cases I cannot pick an
$n$ such that the estimator maps all values outside that threshold to zero probability.
Casually speaking, even after a trillion data points, there is no guarantee that
a series of random data points won't stray from their limit.
Instead, in probability, we say that a sequence of random variables $X_n$
\emph{converges} toward a number $a$ if
\begin{equation}
\lim_{n \to \infty} \prob{|X_n - a| > \varepsilon} = 0,
\end{equation}
that is, if the probability that $X_n$
takes on a value more than $\varepsilon$ away from $a$ approaches zero as $n$
increases. An estimator is \emph{consistent} if the series of random variables
corresponding to collecting more and more data converge toward the true value.

In our example, $\hat{B}_1$ corresponds to the estimator when we only collect
1 data point, $\hat{B}_2$ when we collect 2, and so on. In the dumb example
where $\hat{B}$ only takes on the value 3, the $\hat{B}_n$ converge to 3, but not
to $B$, except in the unlikely case that $B$ just happens to be 3.

For our second guess, let's use a more reasonable statistic. Say that $b$
is the maximum of whatever data points we collected:
\begin{equation*}
    b(x_1, \ldots, x_n) = \max_i x_i.
\end{equation*}
I know I can write out an analogous equation to define the estimator:
\begin{equation*}
    \hat{B} = \max_i X_i.
\end{equation*}
But what does it mean to take the maximum of two random variables? Recall that,
we computed the distribution of a sum of two random variables $Z = X + Y$ as
$f_Z(z) = \int f_X(x) f_Y(z-x) \,dx$. For a maximum, it is easier to express the
new variable's distribution using its cdf. To say $Z = \max[X, Y]$ means that,
if $X$ and $Y$ are independent,
\begin{align*}
    F_Z(z) &= \prob{Z \leq z} \\
    &= \prob{(X \leq x) \cap (Y \leq y)} \\
    &= \prob{X \leq x} \, \prob{Y \leq y} \\
    &= F_X(x) F_Y(y)
\end{align*}
So if $F_X(x) = x/B$, then $F_{\hat{B}_n} = (x/B)^n$. Note that $\hat{B}_n$ can
only take on values between $0$ and $B$, so if it diverges from $B$, it will do
so by falling short:
\begin{align*}
    \prob{|\hat{B}_n - B| > \varepsilon}
    &= \prob{\hat{B}_n \leq B - \varepsilon} \\
    &= F_{\hat{B}_n}(B - \varepsilon) \\
    &= \left(\frac{B - \varepsilon}{B}\right)^n \\
    &= \left(1 - \frac{\varepsilon}{B}\right)^n.
\end{align*}
For $\varepsilon > 0$, this limit of this number, as $n \to \infty$, is zero,
and thus $\hat{B}_n$ converges to $B$, so $\hat{B}$ is a consistent estimator.

\section{Unbiased estimators}

It is a nice thing that our estimator $\hat{B}$ is consistent: as we get more
and more data, the statistic $b$ will end up closer and closer to the true value
$B$. It is disappointing, though, that $b$ is always an underestimate. If the
data points $x_i$ are randomly distributed between $0$ and $B$, and we take
$b = \max_i x_i$, it is necessarily the case that $b < B$. Because $b$ is not
``centered around'' $B$, we say that $\hat{B}$ is a \emph{biased} estimator.

Consider the extreme case where $n=1$: we draw only one data point $x$, which is
somewhere between $0$ and $B$. Intuitively, I might expect $x$ to be about
halfway between $0$ and $B$, that is, that $x \approx \tfrac{1}{2}B$, so that
$b$ underestimates $B$ by a factor of $\tfrac{1}{2}$. And if
$n=2$, then I might expect those points, roughly speaking, to land somewhere
like $\tfrac{1}{3}B$ and $\tfrac{2}{3}B$, so that $b$ underestimates $B$ by
a factor of about $\tfrac{1}{3}$. These factors appear to shrink as $n$ increases,
which makes sense, since $\hat{B}$ is a consistent estimator. But can we do
something to figure out, mathematically, what that factor should be?

We do this by examining the expected value of the estimator:
\begin{equation*}
    \expect{\hat{B}} = \int_0^B x \, f_{\hat{B}}(x) \, dx.
\end{equation*}
We said above that $F_{\hat{B}} = (x/B)^n$, from which it follows that
\begin{equation*}
    f_{\hat{B}}(x) = \frac{d}{dx} F_{\hat{B}} = \frac{n}{B^n} x^{n-1}.
\end{equation*}
Plugging this definition of $f_{\hat{B}}$ into the integral gives
\begin{equation*}
    \expect{\hat{B}} = \int_0^B \frac{n}{B^n} x^n \, dx = \frac{n}{n+1} B.
\end{equation*}
This result accords exactly with our intuition above: for $n=1$, the expected
value of $\hat{B}$ is $\tfrac{1}{2}B$. For $n=2$, it is $\tfrac{2}{3}B$, and
so forth. Based on this finding, we can define a new estimator:
\begin{equation}
\hat{B}_\mathrm{unbiased} \defeq \frac{n+1}{n} \max_i X_i.
\end{equation}
This ensures that our estimates for $B$ are ``centered'' around $B$, because
$\expect{\hat{B}_\mathrm{unbiased}} = B$. More generally, we say that an estimator
is \emph{unbiased} if its expected value is equal to the parameter it is estimating.

Note the subtlety in what we did: without actually knowing what $B$ is, we found
a way to adjust the way that we estimate $B$ to ensure that we come up with a more
accurate estimate. I was mystified in introductory statistics that, while we computed
variance as $\tfrac{1}{n} \sum_i (x_i - \mu)^2$, we computed \emph{sample variance}
as $\tfrac{1}{n-1} \sum_i (x_i - \mu)^2$. The standard explanation has something to
do with ``degrees of freedom'', which I never understood. There is, I think, a
more straightforward explanation, which you can now understand, which is that the
$n-1$ is there in the sample variance, instead of $n$, because it corrects for a
bias.

\section{Efficient estimators}

Consistency ensures that, in the limit of infinite data, our estimate approaches
the true value. How can we be sure that we are getting the best estimate for our
data, given the fact that we can't collect infinite data? This question relates
to the concept of \emph{efficiency}. One unbiased estimator $\hat{X}_1$ is 
more \emph{efficient} than another
unbiased estimator $\hat{X}_2$ if it has lower variance:
\begin{equation*}
    \var{\hat{X}_1} < \var{\hat{X}_2}.
\end{equation*}
As we will see, confidence intervals are related to the variance of an estimator,
so a more efficient estimator means that we can get more narrow confidence intervals
for the same amount of data, which is clearly a desirable thing. I don't want to
appear more ignorant than I have to be, just because I picked a poor estimator!

It turns out that there is a theoretical lower limit to the variance of estimators
called the \emph{Cram\'{e}r-Rao bound}, and many estimators actually hit this limit.
Therefore, rather than saying that one estimator is \emph{more} efficient than
another, we usually just say that an estimator is ``efficient'', meaning that it
hits the lower bound on variance and is maximally efficient.

The math behind efficiency is somewhat complex, so I will simply tell you that
a whole class of estimators, \emph{maximum likelihood} estimators, are efficient.

\section{Maximum likelihood estimators}

The German tank problem has a useful toy example, but it's hard to imagine deriving
estimators by hand for the kind of complex data analysis that most scientists do.
Fortunately, modern computing means that we can address all sorts of complex data
using a \emph{maximum likelihood} (ML) approach. In a maximum likelihood approach, you
specify a theoretical model for how your data are generated, and then you use optimization to estimate values for the parameters that best ``fit'' the data.

The word \emph{likelihood} is confusing because, in common speech, ``probability''
and ``likelihood'' are synonymous. In statistics jargon, the likelihood function
is actually related to the probability density function, only it emphasizes the
parameters over the data. Say we have some data $x$ that was drawn from a probability 
distribution with density function $f_X$. We often think about families of probability
distributions and name them by their parameters. For example, we say that a normal
distribution is defined by two parameters, its mean and its variance. So say that
the density function $f_X$ changes depending on some parameter (or parameters) $\theta$.
We write the density function as $f_X(x; \theta)$ to emphasize that, while the
parameters $\theta$ are relevant, we think of them as fixed. By contrast, we write
the likelihood function $L$ as $L(\theta; x) \defeq f_X(x; \theta)$. For likelihood,
we consider the data fixed and vary the parameters.

To see how this plays out, let's return to the German tank problem. Given some data
points $x_i$, what is the probability density of drawing that data, for some varying
parameter $B$? We look for the value of $B$ that maximizes the probability density
function for all the measured data points $x_i$. This kind of optimization is called \emph{argmax}. While \emph{max} asks for the maximum value of some function $f(x)$ when varying $x$, argmax asks for the value of $x$ that leads to $f(x)$ being maximized. Thus:

\begin{align*}
    \hat{B}_\mathrm{ML}
    &= \underset{B}{\operatorname{argmax}} \prod_i f_X(x_i; B) \\
    &= \underset{B}{\operatorname{argmax}} \prod_i \begin{cases}
      1/B &\text{ if } 0 \leq x_i \leq B \\
      0 &\text{ if } x_i > B
    \end{cases} \\
    &= \underset{B}{\operatorname{argmax}} \begin{cases}
      (1/B)^n \&text{ if } x_i > B \text{ for any $i$} \\
      0 &\text{ otherwise}
     \end{cases} \\
    &= \max_i x_i
\end{align*}
Note that, in the maximum likelihood approach, when computing $f_X(x_i; B)$, we assume that we know $B$, because it is the parameter we are optimizing over. This makes the maximum likelihood computation relatively simple: we cannot choose $\hat{B}_\mathrm{ML}$ to be smaller than the maximum of the observed values $x_i$, because that would make the observed data impossible, and we set $\hat{B}_\mathrm{ML}$ at exactly $\max_i x_i$ because choosing any greater value would make the observed data seem unusual. If $B$ were actually much, much higher than $\max_i x_i$, the observed data would be rare, being clustered very close to zero, relative to that very large $B$.

Note also that the maximum likelihood approach for the German tank problem gave us a consistent estimator, but not an unbiased one. Given certain mathematical necessities that are probably true for your applications, maximum likelihood estimators are consistent, but they are in general not biased.

Again, a key practical advantage of maximum likelihood estimation is that it requires only a computer and an articulation of how the data are generated.


\section{The arithmetic mean as an estimator}

The German tank problem has simple math, but it is not a familiar application. Let us consider instead a very familiar estimator, the arithmetic mean, that is, the simple average of numbers. Just as the mean of some set of numbers $x_i$ is $(1/n) \sum_i x_i$, we analogously define the estimator $\hat{\mathbb{E}}_X \defeq (1/n) \sum_i X_i$ of the mean of i.i.d. random variables $X_i$.

It is immediately clear that $\hat{\mathbb{E}}_X$ is an unbiased estimator of $\expect{X}$:
\begin{equation*}
    \expect{\hat{\mathbb{E}}_X} = \expect{\frac{1}{n} \sum_i X_i} = \frac{1}{n} \expect{X_i} = \expect{X}.
\end{equation*}
But is $\hat{\mathbb{E}}_X$ as consistent estimator of $\expect{X}$? The mathematician Jacob Bernoulli published the first proof of this fact for a special case of random variables (i.e., those following the Bernoulli distribution) in 1713. It took him 20 years to develop a rigorous proof for what he proudly called the ``Golden Theorem''. Now the proof is considerably simpler and broader, and we call it the \emph{law of large numbers}.

The modern proof of the law of large numbers follows from an observation about the variance of the estimator $\hat{\mathbb{E}}_X$ that was surprising to many early statisticians:
\begin{equation*}
    \var{\hat{\mathbb{E}}_X} = \var{\frac{1}{n} \sum_i X_i} = \frac{1}{n^2} \sum_i \var{X_i} = \frac{1}{n} \var{X}
\end{equation*}
In words, the variance of the estimator is $1/n$ of the variance of the underlying random variable it is estimating. Intuitively, this means that, as you take more data, you are ever more likely to get an estimate near to the true mean of the underlying population you are sampling. This result makes it clear that $\hat{\mathbb{E}}_X$ is a consistent estimator, because for arbitrarily high $n$, it has arbitrarily low variance.

Because the \emph{standard deviation}, the square root of the variance, is more intuitive, one can say that the standard deviation of the estimator decreases with the square root of $n$. To distinguish the standard deviation of the estimator from the standard deviation of the underlying population, $\var{\hat{\mathbb{E}}_X}$ is also called the \emph{standard error of the mean}. With more data points $n$, you can get an ever more precise estimate of the mean, so the standard error of the mean declines, but the standard error of the underlying population $\var{X}$ is fixed.

This result was surprising because it was expected that the standard error would scale like $1/n$ rather than $1/\sqrt{n}$. In other words, it was expected that, if you collected double the data, you would get double the precision, when in fact, you get only a $\sqrt{2}$ improvement, that is, a 41\% improvement. More starkly, if you take one hundred times the data, you get only a ten-fold increase in precision.



\chapter{copied material}

\section{Things to include}\label{things-to-include}

\begin{itemize}
\item
  LMM as model for when biology becomes stats: you're doing GWAS.
\item
  Multiple hypothesis correction. Start by looking at the distribution
  of \(p\)-values.
\item
  Likelihood of data vs. \emph{probability} of data.
\item
  Start with Bayesian as a conceptual thing, then go to frequentist for
  most of the stats you will encounter, and then go back to Bayesian for
  the more hardcore stuff.

  \begin{itemize}
  \item
    Introduce priors with the binomial and beta prior. I think that's
    really intuitive.
  \item
    Confidence intervals as random variables with the parameter fixed,
    versus credible interval as a fixed thing that arises from the
    posterior of the random variable of the parameter.
  \end{itemize}
\item
  Visualizing pdf as equiprobable bins. Uniform distribution as dice
  rolls on a d10, for why the probability of rolling exactly
  0.5000\ldots{} is (nearly) zero.
\item
  Math on credible intervals: they align with confidence intervals in
  very particular cases. Even the normal distribution with unknown
  variance breaks that.
\item
  Normal approximation to the binomial, but then also to the Poisson, or
  whatever you want.
\item
  ``Regression to the mean'' means a slope less than one
\item
  MCMC for complex SIR-type models: you know the observed outcomes
  (e.g., occupancy of each compartment) comes from the input data, but
  it's not clear how the outcomes relate to the inputs
\item
  ABC for something similar, where we're interested in the parameters
  that generate the data, but there's no obvious mathematical
  relationship between the parameters and the observed data.
\item
  Poisson as horse kicks. Or as drops with cells (a la Weitz, etc.)
\item
  Simulation in general
\item
  Bayesian as the mystical: the only way to say the probability about
  the truth is to assert what you think is the truth
\item
  Frequentist as a weird subset of the Bayesian
\item
  Variance (shape) as more important than mean (location). If you have
  variance, it's pretty easy to measure the mean and get a sense of
  things. If you have the other way, it's a lot more murky.
\item
  Inference vs.~decision-making under uncertainty. How are these two
  things different? (e.g., what's the ``cost'' of making an incorrect
  scientific conclusion?)
\item
  Simpson's paradox: why two-way associations are confusing
\item
  Random number generation
\item
  Most textbooks are boring, and they just present a parade of stuff. I
  want to use those things that are probably familiar as entry points
  for learning. E.g., the difference between descriptive statistics and
  inferential statistics is what those statistics are used for. In one,
  they are used a estimators of population parameters. In the other,
  they are summed over the ask about the likelihood of data.
\item
  ABC as super hacky thing!
\item
  Finite population correction:
  \(\sigma_{\overline{X}} = \frac{\sigma}{\sqrt{n}} \sqrt{\frac{N - n}{n - 1}}\)
\item
  Wilk's theorem for likelihood ratio tests!
\item
  Cramer-Rao bound: inverse of Fisher information matrix is the lower
  bound for the variance of unbiased estimators. An estimator that hits
  this bound is minimum variance unbiased (MVU). (BLUE as a weird name,
  since B and MV mean the same thing?) Not \emph{a priori} clear that
  some estimator is an MVU.
\item
  Start with Bayesian, because that's how scientists are used to
  thinking, and then go onto frequentist, because that's how it's easier
  to do the math, and then return to Bayes.
\item
  Gauss: a minimum-variance, mean-unbiased estimator minimizes the
  squared-error loss function. Laplace: among median-unbiased
  estimators, a minimum-average-absolute-deviation estimator minimizes
  the absolute loss function. Maybe it's better to allow some bias so
  you can get less variance. That's the domain of statistical theory.
\item
  Linear mixed models using relationship-matrix?
\item
  Fisher's crazy sum test is the same thing as is used in \emph{TRANSIT}
  (DeJesus \emph{et al}.): they treat TA sites in the same gene as
  independent; the statistic is the difference in the sum of the
  (normalized) number of insertions in two treatments; the null
  distribution is generated by shuffling the values across the two
  datasets. OK, it's not \emph{exactly} like Fisher's test, since it's
  not paired, but it's pretty close. Fisher probably wouldn't have
  wanted to to the \(\binom{n}{2}\) options, compared to the \(2^n\)
  that he did.
\end{itemize}

\section{GEE notes}\label{gee-notes}

The estimator for the covariance matrix of the estimator for the
parameters \(\bm{\theta}\) has typical element \[
\widehat{\mathrm{Cov}}[\hat{\bm{\theta}}]_{i,j} =
  \left[ \left( -\frac{\partial^2 \mathcal{L}}{\partial \theta_u \, \partial \theta_v} \right) \right]^{-1}_{i,j}.
\] The inverse is a matrix inverse. This is weird notation.

For example, consider the easy case with just one parameter, so that \[
\widehat{\mathrm{Var}}[\hat{\theta}] =
  -\left( \frac{\partial^2 \mathcal{L}}{\partial \theta^2} \right)^{-1}.
\]

Then try the easy example where we're looking at the sample mean, which
is the estimator \(\hat{\mu}\) for some population with unknown mean
\(\mu\) and known variance \(\sigma^2\). Then the log likelihood is \[
\mathcal{L}(\theta) = \sum_i \left\{ -\frac{1}{2} \log (2 \pi \sigma^2) -
  \frac{(x_i - \mu)^2}{2 \sigma^2} \right\}
\] and the second derivative is just \[
\frac{\partial^2 \mathcal{L}}{\partial \mu^2} = -\frac{n}{\sigma^2},
\] from which it's clear that
\(\widehat{\mathrm{Var}}[\hat{\mu}] = \sigma^2/n\). This is just the
result that we got from our more straightforward approach, where we
asked about the variance of the estimator, before we had called it that.

Note, however, that this estimate depends on \(\sigma^2\), the
\emph{true} population variance. In an exercise, you'll compute the
estimator covariance matrix for the estimator of
\(\bm{\theta} = (\mu, \sigma^2)\). That will show that there is some
estimated covariance between \(\hat{\mu}\) and \(\hat{\sigma}^2\), but
you find that each of \(\widehat{\mathrm{Var}}[\hat{\mu}]\),
\(\widehat{\mathrm{Var}}[\hat{\sigma}^2]\), and
\(\widehat{\mathrm{Cov}}[\hat{\mu}, \hat{\sigma}^2]\) depend on the true
values, not on the estimates themselves. This isn't unreasonable: each
of these thing is a well-defined random variable that will have a real,
honest, true distribution, and we're deriving its properties. Clearly
the true properties of these random variables should depend on the true
values, but it leaves us in a pickle when we're doing estimation: when
computing the standard errors on our estimators, we need to use our
points estimates as the true values!

\textbf{Relationship between ``sampling distribution'' and estimator.
``Standard error'' as SD of sampling distribution OR estimate of
standard deviation.}

\section{Summary statistics}\label{summary-statistics}

The mean is also called the \emph{expectation value}, which gives a
contemporary English speaker a strange idea of what it is. The
expectation value is \emph{not} ``a value that you might expect''. That
is the \emph{mode}. In certain cases, the mean and the mode are the
same, but this is often not true. This is why it is so critical to look
at the distribution of your data \emph{before} selecting summary
statistics.

\subsection{The (arithmetic) mean}\label{the-arithmetic-mean}

Everyone knows the average: take your numbers, sum them, and divide by
the number of numbers.

Why is this a good idea? Early thinkers in statistics---who thought
about it before it had a name---were curious about the same thing. The
arithmetic mean was computationally simple (i.e., you could do it will
quill and parchment, OK, pen and paper) and it seemed to ``work'', but
why?

It turns out that the arithmetic mean has some nice properties for, you
guessed it, the normal distribution. In fact, we'll show that it is a
certain kind of ``best way'' to guess the true, population mean from our
sample mean. (It's the maximum likelihood estimator.)

\subsection{The ``average''}

In typical speech, we say something like, ``The average family has two
children'', by which we mean that a typical family has two children.
This is confusing because we also use the word ``average'' to refer to
the arithmetic mean. These two things are similar only in special cases,
including (you guessed it!) the normal distribution.

The arithmetic mean of number of children in families is somewhere
between 2 and 3 (let's say 2.5), but it's clear that no ``average
family'' has 2.5 children. Similarly, it's confusing that the mean
salary in the US is whatever, even thought that is above what whatever
percent of people get paid.

\subsection{Mode, median, and
percentiles}\label{mode-median-and-percentiles}

Typical can mean ``common'', so we might say, in a sense, that an
``average'' family has two children. Of the number of children you can
have (zero, one, two, etc.), two is the most common, so that's
``average''. Technically, the most common value is the \emph{mode}. No
one really talks about the mode except in stats textbooks.

A more interesting number is the middle point. We might say that an
average American makes whatever because half of people make more and
half of people make less. This is the \emph{median} (from Latin
\emph{medius}, meaning ``middle'').

It even feels natural to combine the median and mode ideas (have our
median cake and eat it a la mode?). We want a sense of what are
``middling'' numbers, and we want a sense of what are ``common''
numbers. So you might ask what range is covered by the most middle half
of numbers.

Children are often measured against a growth chart, which shows
\emph{percentiles}: if you are in the 25th percentile, you are taller
than 50\% of people. (The median is just another name for the 50th
percentile.) The range between the 25th and 75th percentiles, which
covers the middle half of people, is the \emph{interquartile range},
because the 25th, 50th, and 75th percentiles are also called the
\emph{quartiles}, because they divide the numbers into four quarters
(bottom quarter, bottom-middle, top-middle, and top).

\subsection{Mean, median, and mode are usually
different}\label{mean-median-and-mode-are-usually-different}

Roll a dice many times. What are the mean, median, and mode? Mean is
easy: each number is equally likely to come up, so we just take the mean
of 1, 2, 3, 4, 5, 6, which is 3.5. The median is a little tricky here:
it's between 3 and 4, and when we hit this situation we normally define
the median as the arithmetic mean of the two middle values. So the
median is also 3.5 here, but only because of some convention. The mode
is also confusing, since all numbers are equally likely, which means
that they are all equally the mode.

For the normal distribution, the mean, median, and mode are all the
same. So it's only in this very potentially unusual case that our
intuition is correct that that the ``average'' value, in the sense of
something common (i.e., the mode), is the same as the ``average'' value,
in the sense of something middling (i.e., the median), is the same as
the arithmetic mean.

Historical box for Quetelet: the idea of an ``average person'' is
directly traceable to a particular proto-statistician, who is remarkable
for having believed that almost \emph{everything} was normally
distributed. At the time, they only had very rudimentary methods for
determining if something was normally distributed (basically, eyeballing
it), and there were very appealing aesthetic/philosophical reasons to
believe that almost everything was normally distributed, so he believed
that too.

\section{$t$-test}

\subsection{Equal variance}\label{equal-variance}

The (old school) \emph{t}-test is two sample, assuming equal variances.
We're interested in the difference in the means between the two
populations.

The null hypothesis is that we're drawing \(n_1 + n_2\) samples from a
population that has this equal variance, and that the labels on the two
``populations'' are just fictitious.

Our estimator \(s_p^2\) for the pooled variance is just the average of
the variances of the two ``populations'', weighted by \(n_i - 1\) (which
is a better estimator than weighting by just \(n_i\)): \[
s_p^2 = \frac{(n_1 - 1) s_1^2 + (n_2 - 1) s_2^2}{n_1 + n_2 - 2}.
\]

The thing we're observing is the difference between the mean of \(n_1\)
samples from a (potentially ficitious) variable \(X_1\) and \(n_2\) from
\(X_2\): \[
\overline{X}_1 - \overline{X}_2 = \frac{1}{n_1} \sum_{i=1}^{n_1} X_{1i} - \frac{1}{n_2} \sum_{i=1}^{n_2} X_{2i}.
\] It would be nice if our statistic was distributed like
\(\mathcal{N}(0, 1)\), so we compute the variance of this observation:
\[
\begin{aligned}
\mathrm{Var}\left[ \overline{X}_1 - \overline{X}_2 \right]
  &= \frac{1}{n_1^2} \sum_i \mathrm{Var}[X_1] + \frac{1}{n_2^2} \sum_i \mathrm{Var}[X_2] \\
  &= \frac{1}{n_1^2} n_1 s_p^2 + \frac{1}{n_2^2} n_2 s_p^2 \\
  &= \left( \frac{1}{n_1} + \frac{1}{n_2} \right) s_p^2.
\end{aligned}
\]

So the statistic for this test is just the observation over its
variance: \[
t = \frac{\overline{X}_1 - \overline{X}_2}{s_p \sqrt{\frac{1}{n_1} + \frac{1}{n_2}}}.
\]

The confusing thing is that \(\overline{X}_1\), \(\overline{X}_2\), and
\(s_p\) are all random variables. We know how to take the sum (or
difference) of two random variables (i.e., how to figure out the
distribution of the numerator), but it's not immediately obvious how to
find the distribution of the whole thing, which has a different random
variable in its denominator.

\subsubsection{Computational approach}\label{computational-approach}

\begin{itemize}
\tightlist
\item
  Compute the observed \(t\) statistic
\item
  Compute the observed sizes, means, and standard deviations for the two
  sample populations
\item
  Many times, generate two sets of random variates. One set of variates
  is drawn from a normal distribution with the first sample mean and
  variance.
\item
  For each iteration, compute the simulated \(t\) statistic.
\item
  The empirical \(p\)-value is the fraction (not true! need \(r+1/n+1\))
  of simulated statistics that are greater than the observed statistic.
\end{itemize}

\subsection{Unequal variance (Welch's)}\label{unequal-variance-welchs}

This is the Behrens-Fisher problem. It stumped Fisher! He came up with a
weird statistic with a weird distribution (Behrens-Fisher), but it
didn't really stick, since he couldn't calculate confidence intervals
(?).

Instead, people went for the Welch-Satterthwaite equation, which
approximates the interesting distribution using a more handy one by
matching the first and second moments. (Maybe worth discussing those? Or
just say mean and variance?)

\section{anova}\label{anova}

Say you have some (equally-sized) groups. Each group was drawn from a
normal distribution (all with the same variance). Are the data
consistent with the model in which all those groups have the same mean?

The statistic is \(F \equiv \frac{\mathrm{MS}_B}{\mathrm{MS}_W}\), where
\(MS_B\) is the mean of the squares of the residuals(?) between the
group means and the grand mean (``between'') and \(MS_W\) is the mean of
the squares of the residuals between the data points and the group means
(``within'').

Again, focus on what's the population we're sampling from. It's easy to
think about a finite population, where you can just do all the possible
combinations and compare their \(F\) statistics. Then move on to say
that, if you believe that the particular variances and means you
measured are the exact, true distribution that you're sampling from, ask
what happens when you sample from that infinite population.

\subsection{z-test example}\label{z-test-example}

What's the nonparametric equivalent of this? It's just saying what the
empirical cdf is! Then say, if you really truly believe your mean and
standard deviation, then you can do that.

In other words, you say you are absolutely sure what population you are
drawing from. The same is actually true of the \emph{t}-test, except
that the \emph{z}-test is asking about a single value, distributed like
\(N(\mu, \sigma^2)\), while the \emph{t}-test is about the mean of the
\(n\) points, which is distributed like \(N(\mu, \sigma^2/n)\).

\section{Paired differences}\label{paired-differences}

\subsection{Historical example and
motivation}\label{historical-example-and-motivation}

Darwin's thing with the pots, as described in Fisher's \emph{Design of
Experiments}

We'll make a tower of the kinds of assumptions made to test Darwin's
hypothesis.

\subsection{Sign test}\label{sign-test}

Assume that it's meaningful if a hybrid plant is taller than a
self-fertilized plant, but don't assign any meaning beyond that. Then
the data are effectively dichotomized: you get some number of cases in
which one is taller and some number of cases in which the other is
taller.

Better to say that we're sampling from any distribution that has zero
median. You could even say you're sampling from \emph{all}
distributions. That's confusing, mathematically, because there are
infinitely many distributions, and it's not obvious how you should
sample from that functional space, but it works out, because all those
distributions will have the same distribution of pluses and minuses.

This is now just a binomial test.

\subsection{Rank test (Mann-Whitney $U$)}

Assume that the \emph{ranks} of the differences are meaningful.

Now you're sampling from any distribution that is symmetric about zero.
That means it has zero median also.

\subsection{Fisher's weird sum test}\label{fishers-weird-sum-test}

Not sure if there's any name for this. Assume that the actual values of
the differences are meaningful.

\subsection{Welch's $t$-test}

Assume that the two populations are normally distributed and, and
therefore that that the variances of the populations are meaningful.
Then you can infer where this set of differences would stand in an
infinite set of such differences.

For early statisticians, this was really appealing, mostly from a
computational point of view: you could actually compute the mean and
standard deviation with pen and paper in a reasonable amount of time,
but you definitely couldn't do all \(2^n\) different ways of taking
sums. Fisher does it for one example in his book, and I'm sure it was
pretty crazy. He makes it clear that he went to great lengths to do it,
and his conclusion is that the results are basically the same, so you
should probably be doing the easier thing and not worry about it.
Nowadays it's gotten pretty easy to do the other thing!, so it's

\section{Wilcox test and Mann-Whitney
test}\label{wilcox-test-and-mann-whitney-test}

\textbf{Walsh averages and confidence intervals}, from
\href{http://www.stat.umn.edu/geyer/old03/5102/notes/rank.pdf}{here}

There a few different names for these things:

\begin{itemize}
\tightlist
\item
  One-sample test: is this distribution symmetric about zero (or
  whatever)?
\item
  Two-sample unpaired (independent; Mann-Whitney): does one of these
  distributions ``stochastically dominate'' the other (i.e., is it that
  a random value drawn from population \(A\) is more than 50\% probable
  to be greater than a random value from \(B\))?
\item
  Two-sample paired (dependent): are the differences between paired data
  points symmetric about zero?
\end{itemize}

\subsection{Wilcoxon}\label{wilcoxon}

\begin{enumerate}
\def\labelenumi{\arabic{enumi}.}
\tightlist
\item
  For each pair \(i\), compute the magnitude and sign \(s_i\) of the
  difference. Exclude tied pairs.
\item
  Order the pairs by the magnitude of their difference: \(i=1\) is the
  pair with the smallest magnitude. Now \(i\) is the rank.
\item
  Compute the \(W = \sum_i i s_i\).
\end{enumerate}

Thus, the bigger differences get more weight.

(There might be a way to do a visualization of this: as you walk along
the data points, you get a good bump for every rank that is in the
``high'' data set, and you get a bad hit for every rank that is not.
Then it settles out pretty quickly, and you want to know the meaning of
the final intercept.)

For small \(W\), the distribution has to be computed for each integer
\(W\). For larger values (\(\geq 50\)), a normal approximation works.

Compare the sign test, which does not use ranks, and which assumes the
median is zero, but not that the distributions are symmetric. That's
just a binomial test of the number of pluses or minuses you get. It's
like setting the weights, which in \(W\) are the ranks, all equal.

\subsection{Mann-Whitney}\label{mann-whitney}

\begin{enumerate}
\def\labelenumi{\arabic{enumi}.}
\tightlist
\item
  Assign ranks to every observation.
\item
  Compute \(R_1\), the sum ranks that belong to points for sample 1.
  Note that \(R_1 + R_2 = \sum_{i=1}^N = N(N+1)/2\).
\item
  Compute \(U_1 = R_1 - n_1(n_1+1)/2\) and \(U_2\). Use the smaller of
  \(U_1\) or \(U_2\) when looking at a table.
\end{enumerate}

At minimum \(U_1 = 0\), which means that sample 1 had ranks
\(1,2,\ldots,n_1\). Note that \(U_1 + U_2 = n_1 n_2\).

For large \(U\), there is a normal distribution approximation.

\subsubsection{Generation}\label{generation}

Say you have \(N\) total points and \(n_1\) in sample 1. Find all the
ways to draw \(n_1\) numbers from the sequence \(1, 2, \ldots N\).
Compute \(U\) for each of those. Voila.

Note that, if you fix \(n_1\), then you don't have to subtract the
\(n_1(n_1+1)/2\) to get the right \(p\)-value.

\section{Statistical power: Cochrane-Armitage
test}\label{statistical-power-cochrane-armitage-test}

We never want to run just any test: we want to use the test that is most
capable of distinguishing between the scenarios we're interested in.
Usually this is a matter of choosing the test that has the right
assumptions: the one-sample \emph{t}-test is more powerful than the
Wilcoxon test if the data come from a truly normally-distributed
population.

In other cases, you might have more flexibility. There's a somewhat
obscure test that is, I think, a great illustration of this.

Imagine that you have some data with a dichotomous outcome for some
categorical predictor value. One classic example is drug dosing: you
think that, as the dosage of the drug goes up, you have more good
outcomes than bad outcomes. Did a greater proportion of people on board
the Titanic survive as you go up from crew to Third Class to Second to
First? Did the proportion of some kind of event increase over years?
Technically, this means you have a \(2 \times k\) table of counts, with
two outcomes and \(k\) predictor categories.

\begin{longtable}[]{@{}llll@{}}
\toprule
Outcome & Dose 1 & Dose 2 & Dose 3\tabularnewline
\midrule
\endhead
Good & 1 & 5 & 9\tabularnewline
Bad & 9 & 4 & 1\tabularnewline
\bottomrule
\end{longtable}

You could use a \(\chi^2\) test with equal expected frequencies across
the columns. In other words, there might be more ``good'' than ``bad''
outcomes, but you don't expect that proportion to differ meaningfully
across categories. You would pool the data across categories, use the
observed proportion of good outcomes as you best guess of the true
proportion \(f\), and compare the actual data with you expectation that
a fraction \(f\) of the counts in each column are ``good''.

In our examples, we think the data have some \emph{particular} kind of
pattern. The \(\chi^2\) test doesn't look for any particular pattern; it
just looks for any deviation from the null. The test statistic for the
\(\chi^2\) distribution is based around the sum of the square deviations
from the expected values, usually written \(\sum_i (O_i - E_i)^2\), with
some stuff in the denominator to make the distribution of the statistic
easier to work with. If the sum of the squared deviations is too large,
then we have evidence that the observed values are not ``sticking to''
the expected frequencies.

The trick I'm going to show you is to keep the same null
hypothesis---that outcome doesn't depend on dose---but adjust the test
so that it's more sensitive to particular kinds of dependencies.

This is a fair approach because we're still just trying to say, ``OK,
say you (the nameless antagonist) were right, and there really was no
pattern in the data. Then I'm free to make up any test statistic, so
long as, if you're right, we can show that the observed data were likely
to have arisen by chance.''

To start constructing the test, think about each flip as a weighted
binomial trial. We'll use these weights to adjust the test statistic to
be more sensitive to what we suspect the true pattern in the data is,
but we'll need to derive the distribution of the test statistic so that
we can satisfy the nameless antagonist.

Say each flip \(y_i\), which is in some category \(x_i\), gets some
associated weight \(w_i\). A really simple statistic would be
\(\sum_i w_i y_i\), the sum of the weights of the ``successful'' trials.
It would be nice to have this be zero-centered: \[
\sum_i \left\{ w_i y_i - \mathbb{E}\left[ w_i y_i \right] \right\} = \sum_i w_i (y_i - \overline{p}),
\] where \(\overline{p} = (1/N)\sum_i y_i\).

It would also be nice for this to have variance 1, so we can divide by
the square root of \[
\mathrm{Var}\left[ \sum_i w_i (y_i - \overline{p}) \right] = \overline{p} (1-\overline{p}) \sum_i w_i^2
\] to produce the statistic \[
T = \frac{\sum_i w_i (y_i - \overline{p})}{\sqrt{\overline{p} (1-\overline{p}) \sum_i w_i^2}}.
\]

You could also conceive of this being a table with two rows and some
number of columns. We bin the trials by their weights: all trials with
the same weight are in the same column. Successes go in the top row;
failures in the bottom. Now write \(t_c\) as the weight of the trials in
the \(c\)-th column, \(n_{1c}\) is the number of successful trials with
weight \(t_c\) (i.e., in column \(c\)), and \(n_{2c}\) is the number of
failures. Then some math shows that you can rewrite \(T\) as \[
T = \frac{\sum_c w_c (n_{1c} n_{2\bullet} - n_{2c} n_{1\bullet})}{\sqrt{(n_{1\bullet} n_{2\bullet} / n_{\bullet\bullet}^2) \sum_c n_{\bullet c} w_c^2}}
\] where \(n_{r\bullet}\) are the row margins, \(n_{\bullet c}\) are the
column margins, and \(n_{\bullet\bullet}\) is the total number of
trials.

\emph{N.B.}: The wiki page gives a different answer, but I don't trust
it, since the variance formula doesn't assume independence of the
\(y_i\). A fact sheet about the PASS software that shows the formula in
terms of the \(y_i\) seems to make a mistake by using \(i\) as an index
both for individual trials and for the weight categories.

The confusing thing here is how to pick the weights. This test is mostly
used to look for linear trends: imagine that each \(y_i\) is associated
with some \(x_i\), so that the weights would be \(x_i\) or
\(x_i - \overline{x}\). Why you pick these exact weights has to do with
the \emph{sensitivity} of the test. There could, of course, be a
nonlinear trend, like a U-shape, that would lead to a zero expectation
for this statistic. The \(\chi^2\) test can find that, but
Cochrane-Armitage with these weights cannot.

To see why you use those weights for a linear test, imagine that
\(p_i \propto x_i\), and zero-center the \(x_i\) such that
\(p_i = m x_i + \overline{p}\).

Then the question is what \(w_i\) maximize \(\mathbb{E}[T]\)? You can
quickly see that this is equivalent to maximizing
\(\sum_i w_i x_i / \sqrt{\sum_i w_i^2}\), and taking a derivative with
respect to \(w_j\) shows that, at the extremum,
\(x_j \sum_i w_i^2 = w_j \sum_i w_i x_i\), which \(w_i = x_i\) for all
\(i\) satisfies. So those weights are the best way to get a large
statistic if you think that there actually is a linear test.

\section{Estimators}\label{estimators}

\subsection{Motivating example: uniform
distribution}\label{motivating-example-uniform-distribution}

Consider a uniform distribution from zero up to some number \(B\).
What's the maximum of this distribution? Well, if you know it, it's
easy: it's just \(B\). It's also easy if you can sample a zillion times
from the distribution: you are very likely at some point to get a value
very close to \(B\), so you can just use that as your estimate.

But what if you only drew one point? Then clearly the maximum of the
drawn values, just that point, is a pretty bad estimate for the upper
limit. If you draw 3 points, then the maximum is a little better, but
you can feel pretty sure that estimating the \emph{population} maximum
using your \emph{sample} maximum is biased: you're always going to
underestimate the population maximum.

When we use our data to make a guess about a population parameter,
that's an \emph{estimator}. (Good name.) We say that an estimator that
we expect to not be ``systematically''\footnote{Why the scare quotes?
  Like in all frequentist statistics, we're stuck with assuming that we
  know the truth, finding the best answer knowing the full truth, and
  then pretending we didn't know the truth in the first place. So
  ``systematically'' is going to mean: assume you know the true value,
  evaluate the behavior of the estimator knowing the truth, and then
  forget you know the truth.} above or below the true value is
\emph{unbiased}. Estimators are typically written with a hat (because
you use a cone hat to communicate with the mothership?), so \(\hat{B}\)
is an estimator for \(B\). Of course, there isn't just one estimator for
a value. I know plenty of people who are perfectly happy to guess that 3
is the right answer in statistics, so for them we might write
\(\hat{B} = 3\). This is clearly a bad estimator, but what does ``bad''
mean?

To answer that, let's consider some of the properties of different
estimators. To keep the different estimators straight, let's put
subscripts on them. I'll call the clearly bad estimator
\(\hat{B}_\mathrm{bad} = 3\) and I'll call our naive estimator \[
\hat{B}_1 = \max_i X_i,
\] where I called it ``1'' because it was our first guess. I also called
the variables we're drawing from the uniform distribution as \(X_i\).

\subsubsection{Consistent estimators}\label{consistent-estimators}

One reason that \(\hat{B}_\mathrm{bad}\) seems bad is that it doesn't
change. It won't ever get more ``right'', no matter how much data we
collect. On the other hand, \(\hat{B}_1\) will get better. In fact, the
more data we take, the closer we expect \(\hat{B}_1\) to get to the true
maximum.

Estimators whose expected value approach the true value as the number of
points approaches infinity is called \emph{consistent}. Call \(M\) the
true value and \(n\) the number of points we drew. Then \(\hat{B}_1\) is
consistent because \[
\lim_{n \to \infty} \hat{B}_1 = M.
\] In other words, if you specify an error threshold (e.g., you want an
estimate within 10\% of the true value), then I can tell you what \(n\)
you need to pick in order to get an estimate within that threshold (with
some finite probability). In contrast, with \(\hat{B}_\mathrm{bad}\),
there's no \(n\) you can pick that will get you arbitrarily close to the
true value.

\subsubsection{Unbiased estimators}\label{unbiased-estimators}

Clearly, though, \(\hat{B}_1\) is not perfect. No matter how many \(n\)
we draw, \(\hat{B}_1\) will always be less than \(M\). We therefore say
that \(\hat{B}_1\) is \emph{biased}. Mathematically, we might write
\(\mathbb{E}[\hat{B}_1] < B\). Because it feels natural to expect that
the expected value of your estimator will be the true value, an
estimator \(\hat{B}\) for which \(\mathbb{E}[\hat{B}] = B\) is called
\emph{unbiased}.\footnote{Another standard for ``good'' estimators is
  the \emph{best linear unbiased} estimator (BLUE). ``Linear'' means
  that the estimator is a linear function of the data. (You could maybe
  imagine that there is some wacky, convoluted function that's a better
  estimator, but we stick with linear.) The final qualifier, ``best'',
  means that the variance of the estimator is the minimum of all other
  linear unbiased estimators. Here ``variance of the estimator'' is
  computed in the same sense that the expected value of the estimator is
  computed for determining bias. In other words, ``best unbiased'' means
  its centered on the true value with the smallest variance possible.}

Let's try to salvage our naive estimator \(\hat{B}_1\) and make an
unbiased estimator \(\hat{B}_\mathrm{ub}\) instead.

Where to start? It's likely that there are many unbiased operators, so
we need to pick the form of the operator we want to look for. It seems
likely that all our observed \(X_i\) only have two interesting things
about them: their maximum, and how many of them there are. For example,
if my first draw was \(1.0\), then I don't learn that much if I draw
\(0.1\) the next five times, because I know that \(M > 1.0\), and,
because it's a uniform distribution, I was just as likely to draw
\(0.1\) as \(1.0\). If I draw many values below \(1.0\), I only learn
that the maximum is probably not far above \(1.0\), which is already
encoded in the number of points \(n\).

So I want \(\hat{B}_\mathrm{ub}\) to be some function of \(n\), which is
fixed, and \(M\), which is a random variable{[}\^{}wtf{]}, such that
\(\mathbb{E}[\hat{B}_\mathrm{ub}] = M\). That will require knowing the
distribution of \(M\).

\textbf{swo} I mixed a lot of this up. I called \(B\) the true value,
which got confusing. I should just use \(M\) and \(\hat{M}\).

The cumulative distribution function, the probability that the observed
maximum is less than some \(m\), is just the probability that there is
no point that is above \(m\): \[
F_M(m) = \mathbb{P}[M \leq m] = \mathbb{P}[X_i \leq m \text{ for all } i] = \left( \frac{m}{B} \right)^n.
\] Then we compute the probability density function: \[
f_M(m) = \frac{d}{dm} F_M(m) = \frac{n}{B} \left( \frac{m}{B} \right)^{n-1},
\] from which we compute the expected value of \(M\): \[
\mathbb{E}[M] = \int_0^B m \, f_M(m) \, \mathrm{d}m = \frac{n}{n+1} B.
\]

Interesting, the expected value of \(M\) is just a multiple of \(B\).
Consider that if \(n=1\), you just draw one point, the expected value of
the maximum, which is just that value you drew, is \(\tfrac{1}{2} B\).
Makes sense: you're likely to get something right in the middle.

If you draw 2 points, the expected maximum if \(\tfrac{2}{3}B\),
somewhat closer to \(B\). It's easy to see that as \(n\) increases,
\(\mathbb{E}[B]\) approaches \(B\), just as we reasoned about before
doing any math.

Now, by linearity of expectation, I know that
\(\mathbb{E}\left[\tfrac{n+1}{n} M \right] = B\), so lo! I have found an
unbiased estimator: \[
\hat{B}_\mathrm{ub} = \frac{n+1}{n} M.
\]

\subsubsection{Maximum likelihood
estimator}\label{maximum-likelihood-estimator}

It's nice that we found an unbiased estimator, which we think is
``good'' in the sense that we're not systematically off in one direction
or the other. In one way this is comforting, but it isn't foolproof.
Imagine you're working with a bimodal distribution: it tends to give
small numbers or big numbers. Then an unbiased estimator splits the
difference, potentially estimating values that aren't even possible!

For this reason, it's useful to think about \emph{maximum likelihood}
estimators (or MLEs, if you're into that sort of acronym). ``Maximum
likelihood'' means that, given the \emph{observed} value, you find the
\emph{true} value that, if it were true, would be most likely, out of
all the other true values, to give rise to your data.\footnote{Did that
  sound crazy? If so, get excited about the Bayesian section of this
  book. In there we'll show that there's a mathematically precise way to
  show that this kind of thinking---saying that the most likely state of
  nature given our data is the state of nature that, of all states of
  nature, was most likely to produce the data we observed---is coherent
  only under some pretty strong assumptions about the universe.}

In other words, given the observed \(M\), what \(B\) would have given
rise to that \(M\) with the greatest probability? Mathematically we
write \[
\hat{B}_\mathrm{ML} = \max_B \mathbb{P}[M | B],
\] where ``MLE'' stands for ``maximum likelihood''.\footnote{The
  promised Bayesian result is that the \(B\) that maximizes
  \(\mathbb{P}[M | B]\), the probability of the data given the state of
  nature, is also the \(B\) for which \(\mathbb{P}[B | M]\), the
  ``probability of the state of nature'' given the data, is maximum,
  although that's only true under a specific, and pretty
  hard-to-believe, criterion. I put more scare quotes here because, as
  mentioned before, if probability is a frequency, it doesn't make sense
  to talk about the probability of a state of nature.} This should be a
piece of cake, since we already found the probability density function
for \(M\), which is a function of \(B\). It's easy to see, just by
inspection, that the value between zero and \(B\) for which \(m\)
maximum \(f_M(m)\) is just \(B\). In other words,
\(\hat{B}_\mathrm{ML} = M\).

You may have noticed that this is the same estimator that I previous
denigrated as ``naive'' and ``biased''. To be clear, this is the maximum
likelihood estimator: you know that \(\hat{B}_\mathrm{ML}\) can't be any
less than \(M\), since the upper bound must be at least as high as the
biggest data point we observed, and some careful thought shows that
\(\hat{B}_\mathrm{ML}\) can't be any greater than \(M\), since
hypothetically increasing \(B\) above \(M\) just decreases the
probability that you would have observed \(M\) as the maximum
value.\footnote{Imagine hypothetically increasing \(B\) to a zillion
  times \(M\). Clearly, observing that small of an \(M\) is given that
  large of a \(B\) is unlikely, and there's no ``cut point'' so that an
  infinitesimal increase is reasonable but a zillion-fold increase is
  not.}

So this example tells an interesting lesson: it's not the case that an
estimator that is ``good'' in one sense (e.g., unbiased) will be
``good'' in another sense (e.g., maximum likelihood). We'll show that
there's a special case in which that's true. (Can you guess what it is?
No really, try. Or, don't try, and just say the first probability
distribution that comes to mind.)

\textbf{German tank problem} as an exercise.

\subsection{The tyranny of the normal, part
XLIV}\label{the-tyranny-of-the-normal-part-xliv}

You get some data points from a normal distribution (I hope you guessed
it), and you're interested in the mean and standard deviation of that
distribution (because those are the only things to know about).

What is the maximum likehood estimator for the mean? In other words,
given your data points \(x_i\), what \(\mu\), among all the other
\(\mu\), would have given rise to these particular data with the
greatest probability? \[
\begin{aligned}
\hat{\mu}_\mathrm{ML}
  &= \mathrm{argmax}_\mu \, \prod_i \mathbb{P}[X_i = x_i] \\
  &= \mathrm{argmax}_\mu \, \prod_i \frac{1}{\sqrt{2 \pi \sigma^2}} \exp\left\{ -\frac{(x_i - \mu)}{2\sigma^2} \right\} \\
  &= \mathrm{argmax}_\mu \, \left(2\pi\sigma^2\right)^{-\frac{n}{2}} \exp\left\{ \sum_i -\frac{(x_i-\mu)^2}{2\sigma^2} \right\}
\end{aligned}
\]

At this point, it's nice to pull a little trick: the value \(x\) that
maximum a function \(f(x)\) is the same value that maximizes
\(\log f(x)\). In other words: \[
\mathrm{argmax}_x f(x) = \mathrm{argmax}_x \log f(x).
\] This turns the nasty exponent into a sum: \[
\hat{\mu}_\mathrm{ML} = \mathrm{argmax}_\mu \left\{ -\frac{n}{2} \left(2 \pi \sigma^2\right) - \frac{1}{2\sigma^2} \sum_i (x_i-\mu)^2 \right\}.
\] It's pretty clear that we'll need to do something about \(\sigma\) at
some point: if \(\sigma\) is too big, then the first term blows up, and
the whole thing goes negative, which is not great for the argmax. If
\(\sigma\) is too small, then second term blows up. Whatever we pick
\(\sigma\) to be, it's clear that that choice is independent of the one
we should make for \(\mu\), since we should just pick a \(\mu\) that
minimizes \(\sum_i (x_i - \mu)^2\).

Is this starting to look familiar? The \(\mu\) that minimizes this sum
is the one that solves: \[
0 = \frac{d}{d\mu} \sum_i (x_i - \mu)^2 = \sum_i -2 (x_i - \mu) = -2 \left( \sum_i x_i - \sum_i \mu \right),
\] which implies that \(\sum_i x_i = n\mu\), that is,
\(\mu = \tfrac{1}{n} \sum_i x_i\). It's just our old, dear friend, the
arithmetic mean.

\subsubsection{Gauss-Laplace synthesis}\label{gauss-laplace-synthesis}

If you're not shocked by what just happened, you have no heart. If you
\emph{are} shocked, then you're in good company: the realization that
the normal distribution is the distribution for which the arithmetic
mean is the maximum likelihood estimator was part of the ``Gauss-Laplace
synthesis''. In 1809, Gauss published a book about ``least squares
estimation'', a method for picking a ``best fit'' estimate to data by
minimizing the squares of the deviations from the arithmetic mean (i.e.,
minimizing \(\sum_i (x_i - \mu)^2\)), was a ``good'' guess for the
``true'' value of the mean, where ``good'' meant that it was relatively
easy to do my hand and seemed to work right.

In the same book, Gauss worked with the normal distribution, which had
previously been cooked up by Laplace, and showed how you could use it as
a model for measurement errors. (Gauss's work with the normal
distribution so overshadowed Laplace's original description that we now
call the normal distribution ``Gaussian'', and ``Laplacian'' refers to a
different distribution.)

In 1810, Laplace showed that the normal distribution arises as the sum
of many random variables \textbf{see somewhere}. Later that year, he
read Gauss's book from 1809, and he was \textbf{get the quote}. In that
moment, Laplace realized that the reason that the arithmetic mean always
seemed so nice, and the reason that least squares seemed to work so
well, was because the normal distribution's fundamental property causes
it to arise in so many places, and \textbf{clean up this section}.

\subsubsection{End of historical note}\label{end-of-historical-note}

As you might guess, the arithmetic mean is also the unbiased estimator
for the population mean: \[
\mathbb{E}[\mathrm{\mu}] = \mathbb{E}\left[\frac{1}{n} \sum_i X_i \right] = \frac{1}{n} \sum_i \mathbb{E}[X_i] = \mu.
\]

\textbf{Is this really all that special, then?}

\subsection{Variance}\label{variance}

One estimator is used so often that it's often not even explained as an
estimator, or even called an estimator, which is very confusing. Imagine
if someone said that the definition of a maximum of a set of points was
\(\tfrac{n+1}{n}\) times the biggest value. Crazy, right? If you agree,
then you're well prepared.

The definition of sample variance is the average square deviation from
the mean: \[
s^2 \equiv \frac{1}{n} \sum_i (x_i - \overline{x})^2,
\] where \(\overline{x}\) is just the normal arithmetic mean. Because
we're used to estimators, it's a nice question to ask, is this a
``good'' estimator of the true variance \(\sigma^2\)?

\textbf{Some math} will show that
\(\mathbb{E}[s^2] = \tfrac{n-1}{n} \sigma^2\), that is, that the sample
variance is a biased (underestimating) estimator for the true variance.
Happily, it requires merely multiplying by \(\tfrac{n}{n-1}\) (called
``Bessel's correction''), so that the unbiased estimator is: \[
\hat{\sigma^2} = \frac{1}{n-1} \sum_i (x_i - \overline{x})^2.
\]

This estimator is so commonly used that it is often simply referred to
as ``sample variance'', leaving students to wonder why the variance is
one thing for ``true distributions'' and another thing for samples. To
be clear, the sum of square deviations divided by \(n-1\) is an unbiased
\emph{estimator} for the population variance.

Unfortunately, the words ``sample variance'' are used to refer to the
actual sample variance (sum of square deviations divided by \(n\)) as
well as to this estimator. There's often no way to tell. I'm sorry. You
should probably guess that people mean the \(n-1\) denominator
\emph{estimator}, and you should probably guess that most people won't
know the difference.\footnote{Comfortingly, as \(n\) grows, the
  difference between the sample variance and the ``sample variance''
  becomes negligible.}

\subsection{Estimators about
estimators}\label{estimators-about-estimators}

\subsubsection{Jackknife}\label{jackknife}

You have \(n\) data points and compute an estimator \(\hat{\theta}\) for
some population parameter \(\theta\). If you don't know how the
population is structured, then it's not clear what you expect the
variance of \(\hat{\theta}\) to be. How sure can you be of this value?
In terms of inference, can you make any inference with it?

Compute the \emph{jackknife replicates}\footnote{The ``jackknife''
  method is so called because Tukey compared the method, which is
  ``rough-and-ready'', to another rough-and-ready tool, the pocket
  knife, also known as a jackknife. Although this name has the
  disadvantage of giving you no clue what it is about, it had the
  advantage of having more brevity and vivacity than ``delete-1
  resampling'', which is probably the more accurate name.}
\(\hat{\theta}_j\), which are the estimators computed using all the data
points except the \(j\)-th one.

That seems like a weird thing to have done, but you can use them to
compute two handy things:

\begin{enumerate}
\def\labelenumi{\arabic{enumi}.}
\tightlist
\item
  An estimate of the variance of the estimator. This can help you for
  description---by giving a confidence interval(?)---and for
  inference---by giving you a sense of the ``random'' ranges you would
  expect from two samples.
\item
  An estimate of the bias in the estimator. This is helpful if you don't
  want want your estimator to be biased but you don't know how to fix
  it.
\end{enumerate}

\paragraph{Jackknife variance
estimator}\label{jackknife-variance-estimator}

The variance estimator is \[
\widehat{\mathrm{Var}}_\mathrm{jk}[\hat{\theta}] := \frac{n-1}{n}  \sum_j \left( \hat{\theta}_j - \hat{\theta}_{(\cdot)} \right)^2,
\] where \(\hat{\theta}_{(\cdot)}\) is the average of the jackknife
replicates: \[
\hat{\theta}_{(\cdot)} := \frac{1}{n} \sum_j \hat{\theta}_j.
\] In other words, it's the variance of the jackknife replicates with
some rescaling: \[
\mathrm{Var}[\hat{\theta}_j] = \frac{1}{n-1} \sum_j \left( \hat{\theta}_j - \hat{\theta}_{(\cdot)} \right)^2 \implies
  \widehat{\mathrm{Var}}_\mathrm{jk}[\hat{\theta}] = \frac{(n-1)^2}{n} \mathrm{Var}[\hat{\theta}_j].
\]

The reason for that scaling factor is beyond the scope of this book
(Efron \& Stein 1981?), but the exercise gives you a sense of why it has
to be true for a specific case.

Some other work, also beyond the scope of this book, shows that the
jackknife estimate of variance is biased: it tends to overestimate the
true variance. This makes the jackknife a conservative tool.

\textbf{Exercise}. Let \(\theta\) be the mean. Show that the scaling
factor is what we think. Hints:

\begin{itemize}
\tightlist
\item
  Show that \(\hat{\theta}_{(\cdot)}\) is the sample mean.
\item
  Show that
  \(\hat{\theta}_j - \hat{\theta}_{(\cdot)} = (n \overline{x} - x_j) / (n - 1)\).
\item
  Show that that value is equal to \((\overline{x} - x_j) / (n - 1)\).
\end{itemize}

That exercise is from McIntosh's bioRxiv about jackknife resampling.

\paragraph{Jackknife bias estimator}\label{jackknife-bias-estimator}

The jackknife estimate of bias is
\((n-1) \left( \hat{\theta}_{(\cdot)} - \theta \right)\). This is the
sum of the deviations of the jackknife replicates from the observed
value \(\hat{\theta}\). Again, the reason that you would take the
average deviation and scale it up to the sum is beyond the scope.

However, if you have an expectation about the bias in an estimator, you
can make an unbiased estimator by subtracting out that bias: \[
\hat{\theta}_\mathrm{jk} := \hat{\theta} - \widehat{\mathrm{Bias}}_\mathrm{jk}[\theta].
\]

\textbf{Exercise}. Show that the jackknife estimate of bias for the
variance gives you the familiar unbiased variance estimator.

\textbf{Exercise}. Something about the maximum estimator?

\paragraph{Pros and cons of the
jackknife}\label{pros-and-cons-of-the-jackknife}

It's a piece of cake to implement. There are only \(n\) replicates to
do, so it's tractable. Those replicates are deterministic, so you only
run it once.

The cons are that it doesn't always work. For example, a jackknife
estimate of the variance of a median (\textbf{swo check Knight}) is not
consistent. It's also overly conservative: it's biased toward higher
variances. You can rescue some properties if you move to a delete-\(d\)
resampling and pick \(d\) from the correct range.

\subsubsection{Bootstrap}\label{bootstrap}

What do you do when you want to compute the variance of some statistic
that's not easy to compute? Or you don't know what distribution you're
sampling from? Then you permute your own data. How do we relate:

\begin{itemize}
\tightlist
\item
  permutation tests
\item
  bootstrapping (and jack-knifing, etc.)
\item
  nonparametric (which I know is different)
\end{itemize}

\section{\texorpdfstring{What does it mean to
``sample''?}{What does it mean to sample?}}\label{what-does-it-mean-to-sample}

Does it make sense to compute a confidence interval when you're sampled
all the 50 United States?

\textbf{Finite correction factor} to point out that there's a difference
between simple random sampling and something else. Then need to explain
what simple random sampling is!

\section{Confidence intervals}\label{confidence-intervals}

Confidence intervals are very slippery things. It's tempting to say that
``I am 95\% confident'' that the true value of a quantity lies within
the 95\% confidence interval. In frequentist statistics, ``I am
\emph{X}\% confident'' has no meaning. The probability that the true
value lies within an interval is either 0 or 1, since if you repeat the
world-experiment many times, the true value will always be the same. In
other words, ``confidence'' is a Bayesian notion of probability. The
interval that you are 95\% confident that something falls in is
therefore a Bayesian concept (and it gets the confusing name of
``credible interval''). So forget that ``confidence interval'' has
anything to do with confidence.

Here's how things actually work: before you collect any data, you
develop a \emph{method} for generating the upper and lower confidence
intervals, which are a pair of statistics, that is, functions of the
data. This method has the property that the statistics it generates an
interval that, in 95\% of cases, contains the true value.

Here's the slippery part: the confidence interval is generated so that,
in 95\% of cases, they are ``correct'' in that they contain the true
value. In the other 5\% of cases, they don't include the true value.
Strictly speaking, they can't tell you anything about the probability
that the value is in the range.

\textbf{What's the easy way to explain the Bayesian link?} The typical
frequentist answer is really pedantic.

\subsection{Binomial test example}\label{binomial-test-example}

We set \(N\) and observe \(X\) to guess \(p\): \[
P[p \in \mathrm{CI} | X] \propto P[X | p \in \mathrm{CI}] \times P[p \in \mathrm{CI}].
\]

Confidence intervals don't actually do any of these. E.g.,
Clopper-Pearson guarantees that, given any \(p\), 95\% of the outcomes
will include the true value. The actual ranges included depend on \(p\)!
This isn't the case for the normal distribution, because it has all
these amazing properties about scaling and so forth. I think this is the
best example.

For example, start with the super dorpy confidence interval \([0, 1]\).
This is always true, but it's way too conservative. Then say something
else dorpy like a constant range, and show that this won't work if \(N\)
is too small.

Let's look at a real example. The Clopper-Pearson interval are the
limits of the range of values of \(p\) such that
\(P[x < X | p] > 2.5\%\) and \(P[x > X | p] > 2.5\%\). The first
inequality is fulfilled by smaller values of \(p\) (e.g., if \(p\) is
zero, then \(x\) has to be zero), so the \emph{upper} confidence limit
is determined by the greatest \(p\) that satisfies that inequality. The
second inequality is fulfilled by larger \(p\) (e.g., if \(p\) is 1 then
\(x=N\)), so the \emph{lower} confidence limit is determined by the
smallest \(p\) that satisfies this inequality.

How can we tell that this is a confidence interval? Given any true \(p\)
and \(N\), if I make draws \(x\) from \(\mathrm{Bin}(p, N)\), will this
confidence interval include \(p\) in 95\% of cases?

\subsection{\texorpdfstring{Comparison to
\(p\)-value}{Comparison to p-value}}\label{comparison-to-p-value}

Given \(N\), and having observed \(X\), we want to test the hypothesis
that \(p\) equals some \(p_0\) (typically \(\tfrac{1}{2}\)).

The nonrejection interval is the range of \(x\) such that \(x N\) is
close enough to \(p_0\) that the probability that \(x\) arose from
\(p_0\) is above some threshold. The lower limit is the lowest \(x\)
that would have arisen from \(p_0\) with some probability \(\alpha/2\),
that is, the smallest \(x\) such that \(P[x | p_0] > (1-\alpha)/2\).
This means that we take the state of nature \(p_0\) as given and scan
over the possible data values, comparing our data to those values.

In contrast, in a confidence interval, we take the data as given and
scan over the possible states of nature \textbf{in some way}. For the
binomial, it's obvious that the confidence intervals and the
nonrejection intervals are not the same thing, since it the one is
discrete and the other is continuous. Only in the normal (\textbf{???})
can we assume they are the same thing.

\subsection{Returning}\label{returning}

\begin{enumerate}
\def\labelenumi{\arabic{enumi}.}
\tightlist
\item
  You get some data that has some parameters (e.g., a sample mean and
  sample variance).
\item
  You \emph{guess} that your observed sample mean and variance are the
  \emph{true} mean and variance.
\item
  You ask, ``If someone else sampled from this true distribution many
  times, and they got all kinds of sample means and variances, what
  method could they use to construct, from the values they observed, an
  interval that would, in 95\% of cases, include the true value?''
\item
  Then you forget that you are omniscient and know the true value and
  instead use the methodology that these ignorant people would use, but
  you use it on your own data.
\end{enumerate}

Now the crazy Bayesian switch comes in: you conflate the frequency of
cases with your confidence that you are in the 95\% of cases.

\emph{N.B.}: For the \emph{t}-distribution, there's this ``pivotal
quantity'' thing, which means that the true \(\mu\) and \(\sigma\) drop
out, which is very luck, and it means that we \emph{don't} need to make
a parametric assumption about how things work.

\subsection{\texorpdfstring{\emph{t}-distribution}{t-distribution}}\label{t-distribution}

Let's think about how to construct that method. Say you knew the true
variance \(\sigma^2\). Then we know that the sample means are drawn from
\(\mathcal{N}(0, \sigma^2/n)\). So it's pretty easy to see that
\((\overline{x} - \mu) / (\sigma^2) \sim \mathcal{N}(0, 1)\), from which
the familiar \(1.96\), etc. come.

What if you \emph{don't} know the true variance? The means are still
drawn from \(\mathcal{N}(0, \sigma^2/n)\), but now the sample variance
is also a random variable.

We know the confidence interval is some function of the sample mean and
variance, and let's guess that it's symmetric about the sample mean and
is some linear function of sample variance: \[
\mathrm{CI}_\pm(\overline{x}, s) = \overline{x} \pm A s.
\] We want to find \(A\) such that \[
\mathbb{P}\left[ \mathrm{CI}_- < \mu < \mathrm{CI}_+ \right] = 95\%,
\] or, if we're willing to trust in symmetry, \[
2.5\% = \mathbb{P}\left[ \mathrm{CI}_- > \mu \right] = \mathbb{P}\left[ \frac{\overline{x} - \mu}{A} - s > 0 \right].
\] We know the distribution of the first thing: \[
(\overline{x}-\mu)/A \sim \mathcal{N}\left(0, \frac{\sigma^2}{n A^2}\right).
\] Some math shows that \[
\frac{(n-1) s^2}{\sigma^2} \sim \chi^2(n-1).
\]

Call the first thing \(K\) and the second \(L\). We're interested in the
distribution of \(M \equiv K - L\): \[
f_M(m) = \int_0^\infty f_K(m + l) f_L(l) \,\mathrm{d}l,
\]

where the limits come from the fact that variance is positive. You're
probably not excited to do this integral, which was considered a major
achievement (well, it was the thought leading up to the integral, which
we've just outlined, but whatever). This major achievement was made by
William Sealy Gosset, who made it while he was a researcher for Guinness
ensuring the quality of their beer. Guinness had a policy of not
allowing its employee to publish their results, so Gosset signed his
paper ``a student'', so the result of that integral is now called
Student's \emph{t}-distribution: \[
f_t(x; \nu) = \frac{\Gamma(\frac{\nu+1}{2})}{\sqrt{\nu\pi} \Gamma\left(\frac{\nu}{2}\right)}
  \left(1+ \frac{x^2}{\nu}\right)^{-\frac{\nu+1}{2}},
\] where the (badly named) ``degrees of freedom'' \(\nu\) is \(n-1\) for
our purposes. I write this out fully because it is one of the things we
will \emph{not} derive in this book.

\section{Contingency tables}\label{contingency-tables}

These are nice examples for how to do statistical thinking.

\subsection{Barnard's test}\label{barnards-test}

The classic example is whether a certain treatment causes more of the
outcome of interest than just doing nothing. In medicine, that means
splitting your participants into a placebo group and a treatment group
and asking what fraction of each gets well. In a biology experiment, you
might split your mice into a treatment group and a control group and ask
what proportion of the mice in each group get cancer.

In statistics jargon, this is called a \(2 \times 2\) contingency table:

\begin{longtable}[]{@{}llll@{}}
\toprule
Group & Outcome \(p\) & Outcome not-\(p\) & Row sums\tabularnewline
\midrule
\endhead
A & \(a\) & \(c\) & \(m\)\tabularnewline
B & \(b\) & \(d\) & \(n\)\tabularnewline
Column sums & \(r\) & \(s\) & \(N\)\tabularnewline
\bottomrule
\end{longtable}

Because we picked \(m\) and \(n\), the sizes of the two groups, those
are fixed parameters. The question is whether the way that \(m\) gets
distributed into \(a\) and \(c\) (and that way that the \(n\) get put
into the \(b\) and \(d\)) is consistent with there being a common
probability \(p\) of the outcome of interest.

So we might say that \(a\) is distributed like a binomial distribution
with \(m\) draws and probability \(p_a\) of success, and \(b\) is
distributed like a binomial with \(n\) draws and a probability of
\(p_b\) of success. The null hypothesis is that \(p_a = p_b\). What's
the likelihood of the data given the null?

If we didn't assume the null, and gave the two binomials their own
probabilities, the likelihood of the data would be: \[
P(a, b | p_a, p_b) = \mathrm{Bin}(a; m, p_a) \times \mathrm{Bin}(b; n, p_b).
\] But, given that the probabilities are the same, we can collapse it:
\[
\begin{aligned}
\mathcal{P}[a, b | p_a = p_b = p] &= \mathrm{Bin}(a; m, p) \times \mathrm{Bin}(b; n, p) \\
  &= \binom{m}{a} p^a (1-p)^{m-a} \times \binom{n}{b} p^b (1-p)^{n-b} \\
  &= \binom{m}{a} \binom{n}{b} p^{a+b} (1-p)^{m+n-(a+b)} \\
  &= \frac{m! \, n!}{a! \, b! \, c! \, d!} p^r (1-p)^s.
\end{aligned}
\]

This result is a little confusing\footnote{I trotted out this test
  because these two confusions are actually great learning
  opportunities.}, for two reasons:

\begin{enumerate}
\def\labelenumi{\arabic{enumi}.}
\tightlist
\item
  The probability \(p\) of the outcome of interest might be interesting
  to design a later experiment, but it's \emph{not} interesting for
  designing a test. We certainly don't want to deliver a result like,
  ``Well, if the null hypothesis is true, \emph{and} \(p\) happens to be
  exactly such-and-such, then your \(p\)-value is so-and-so.'' The value
  \(p\) is called a \emph{nuisance parameter} since we don't actually
  care about its value.
\item
  We're usually not interested in the likelihood of exactly this data,
  but rather in the likelihood of data \emph{at least this extreme}. We
  usually measure ``extremeness'' using a statistic---a single
  number---so it's clear that ``more extreme'' means ``bigger'' (or
  ``smaller'' or ``bigger or smaller'', depending on if it's a one-sided
  or two-sided test). Here, we have two numbers, \(a\) and \(b\), so
  there aren't two ``sides'' to the distribution: there are four!
\end{enumerate}

To resolve the first point, we say that the null hypothesis
\(p_a = p_b = p\) doesn't restrict us to a particular value of \(p\). In
other words, the null hypothesis, which functions as a sort of Annoying
Skeptic, is free to pick \(p\) to make our results as uninteresting as
possible. Mathematically, this means that, when computing the
\(p\)-value, we should optimize over all values of \(p\), choosing the
one that makes our results as uninteresting as possible (i.e., which
maximizes the \(p\)-value).

We can't really ``resolve'' the second point, since it demonstrates that
our previous way of thinking about extremeness was not sufficient for
all cases. As Barnard notes in his original paper\footnote{Barnard
  conceived of the \((a, b)\) as points ``in a plane lattice diagram of
  points with integer co-ordinates'', that is, that \(a\) is like the
  \(x\)-axis and \(b\) is like the \(y\)-axis. Then the possible
  outcomes of the experiment are the points in the rectangle bounded by
  the horizontal lines \(a = 0\) and \(a = m\) and the vertical lines
  \(b = 0\) and \(b = n\). He then said that you should pick the
  non-extremal points (i.e., the values of \((a, b)\) for which you
  would not reject the null) such that they ``consist of as many points
  as possible, and should like away from that diagonal of the rectangle
  which passes through the origin. Formulated mathematically, these
  latter requirements mean that the {[}points for which you would reject
  the null{]} must in a certain sense be convex, symmetrical and
  minimal.''}, there are actually many ways to choose the pairs
\((a, b)\) that produce a \(p\)-value more than our threshold. This gets
into some fancy footwork to articulate exactly how you should pick this
area, but the basic results are pretty intuitive: when \(a/m\) and
\(b/n\) are similar, you tend to be under the rejection threshold; when
they are far apart, you tend to be over.

The interesting point here is that, whatever fancy footwork you pick to
choose that region, and no matter how ``reasonable'' your footwork is,
it's still footwork that doesn't obviously follow from the simple
definition of a hypothesis test. We'll encounter this problem again in
Bayesian statistics, when we find that the Bayesian analog of a
confidence interval is not unique: there are many ranges of values that
are compatible with our ignorance.

\subsection{Fisher's test to the
rescue(?)}\label{fishers-test-to-the-rescue}

If you've worked with contingency tables, you're probably saying, ``I've
never heard of this crazy Bernard's test, with its weird multi-sided
rejection space and its requirement to maximize over \(p\). We have
Fisher's exact test, which is the exactly right test to use here!''

Looking at the same contingency table, Fisher's test asks, given the row
marginals \(m\) and \(n\), the first column marginal \(r\), and the
grand total \(N\), what is the probability of a table at least this
extreme?

This is just a combinatoric problem: if you're as likely to assign items
in \(m\) to \(a\) as to \(c\) (and, analogously, to assign items from
\(n\) to \(b\) or \(d\)), then ``what's the probability of this table''
is equivalent to asking ``given the marginals, how many ways are there
to choose this table?''. More specifically, how many ways are there to
choose \(a\) items from a bank of \(m\) items and \(b\) items from a
bank of \(n\), given that we chose \(r = a + b\) items from the total
\(N\)? Mathematically: \[
\mathbb{P}[a | m, n, r, s] = \frac{\binom{m}{a} \binom{n}{b}}{\binom{N}{r}} = \frac{m! \, n! \, r! \, s!}{N! \, a! \, b! \, c! \, d!}.
\]

Computing the \(p\)-value is easier here than with Barnard's test
because we need to keep the row \emph{and column} marginals the same. In
Barnard's test, we just kept the row marginals constant, because we
considered those as fixed parameters, corresponding to things like the
number of patients we assigned to each of the placebo and treatment
groups. It doesn't make sense to allow the Annoying Skeptic to fiddle
with those values.

In Banard's test, we \emph{did} allow the Annoying Skeptic to fiddle
with the column marginals, since it wasn't clear, before the experiment
began, that \(r\) would have the outcome of interest. In other words, we
didn't know that \(r\) people in both the placebo and treatment groups
would get well.

Fisher's test, however, \emph{does} keep the column marginal constant.
This makes it a lot easier to compute the \(p\)-value. First, the
nuisance parameter \(p\) doesn't appear in the likelihood, so we don't
need to do the weird maximization. Second, we only need to vary one
value, \(a\) (or, equivalently, \(b\)), since, if you know the
marginals, there is only one axis along which to change the values in
the table. In other words, if you know \[
\begin{aligned}
a + c &= m \\
b + d &= n \\
a + b &= r,
\end{aligned}
\] then that's three equations with four unknowns (\(a\), \(b\), \(c\),
\(d\)), so specifying any one of \(a\), \(b\), \(c\), or \(d\) specifies
all the others. (You might be looking for a fourth equation
\(c + d = s\), but you can get that by adding the first two equations
and subtracting the third.)

Here's an example:

\begin{longtable}[]{@{}llll@{}}
\toprule
Group & Success & Failure & Row sums\tabularnewline
\midrule
\endhead
A & 1 & 9 & 10\tabularnewline
B & 11 & 3 & 14\tabularnewline
Column sums & 12 & 12 & 14\tabularnewline
\bottomrule
\end{longtable}

There's only one way to make this table more ``extreme'' without
changing the marginals: you can take the one group A success and make it
a group A failure and simultaneously make a group B failure into a group
B success. Similarly, there's only one way to make this table less
extreme: turn a group A failure into success, and turn a group B success
into failure.

So keeping the column sums constant made it way easier to compute the
\(p\)-value: count this table and all the tables with a more extreme
upper-left or bottom-right and see if your summed probability hits the
rejection threshold.

However, this simplicity came at a cost, which you may have noticed:
does it make sense to keep the columns constant? Experimentally, this
means that you're restricting the Annoying Skeptic to only consider
cases in which, say, the number of patients who got well \emph{in both
groups} is equal to the experimentally observed value. This is a little
weird. It suggest that your experimental design was like this:

\begin{enumerate}
\def\labelenumi{\arabic{enumi}.}
\tightlist
\item
  Pick \(m\), \(n\), and \(r\).
\item
  Assign \(m\) patients to placebo and \(n\) to treatment.
\item
  Wait until \(r\) patients \emph{across both groups} have gotten well.
\item
  Stop the experiment.
\end{enumerate}

This is almost certainly not reflective of how typical experiments are
run\footnote{It is, however, the way the famous ``lady tea tasting''
  experiment was designed. The myth is that Fisher didn't believe it
  when a high-class lady told him that she could detect whether tea was
  added to a cup with milk in it or whether the milk was added to the
  tea. He designed an experiment with \(m\) cups prepared one way, \(n\)
  prepared the other, and told her to detect the \(r = m\) cups that
  were prepared the first way. A Barnard-style experiment, in which the
  same \(m\) and \(n\) cups}.

\textbf{Fisherian small data}

\textbf{What happens if I use the ``wrong'' test? Chi-square as an
example of wrongness}

\section{Regression}\label{regression}

Hardin pages 56-57 talks about the difference between the normal
(subject-specific) and the generalized (population-average) estimating
equations. E.g., second-hand smoking: what's the odds of a kid having a
respiratory illness given that mom smokes? SS parameters give the OR for
\emph{each individual child} having the illness, so it's what we would
expect would happen if particular moms stopped smoking. PA parameters
give the OR \emph{across the population}, so it explains the difference
in prevalence we expect in the two groups. The first one is: \[
\mathrm{OR}^\mathrm{SS} = \frac{P(Y_{it}=1 | X_{it}=1, \nu_i) / P(Y_{it}=0 | X_{it}=1, \nu_i)}{P(Y_{it}=1 | X_{it}=0, \nu_i) / P(Y_{it}=0 | X_{it}=0, \nu_i)}
\] and the second is \[
\mathrm{OR}^\mathrm{PA} = \frac{P(Y_{it}=1 | X_{it}=1) / P(Y_{it}=0 | X_{it}=1)}{P(Y_{it}=1 | X_{it}=0) / P(Y_{it}=0 | X_{it}=0)}.
\] The difference, in case you didn't catch it, is whether you condition
on the random effect \(\nu_i\). The SS estimate is for these particular
people; the PA estimate marginalizes over the random effects.

\section{Bayesian}\label{bayesian}

One of the biggest sticking points about a Bayesian analysis is that it
requires specification of a prior. It can be thought of as an advantage
or a disadvantage, but I think it's better to think of it as a
responsibility. Let me tell you an allegory.

Once, a young statistician lived in her parents' house. She paid no rent
and simply never considered her orientation in the world. Some year
later, she left the home and had to do statistics in the wide world.
Where would she live? How would she pay rent? These decisions brought
power, since she was free to do things she couldn't do at home. She
could live a life that was more accurate to the real world. This
allegory is too long and rambly. But there's something in here.

Frequentist statistics is correct so long as you use it exactly for what
it is designed to do. The trouble is that we \emph{want} statistics to
answer the kinds of questions that \emph{only} Bayesian statistics can
answer. For example, how likely is it that this hypothesis is true? If
you perfectly adhere to the frequentist interpretation, then you are in
good shape. But if you deviate, if you start to say, ``Oh, the p-value
is kind-of like the probability my hypothesis is false.'' Then you have
SCREWED UP son.

\section{Appendix}\label{appendix}

\subsection{Random number generation}\label{random-number-generation}

In many places in this book, we've relied on the ability to generate
``random'' numbers. However, computers (in the sense of logical
machines) have no way to generate truly random numbers. Instead, we have
clever methods that get us something that's a pretty good approximation
of random numbers.

It's worth noting that a lot of the historical, conceptual directions in
statistics are due to the fact that doing any kind of Monte Carlo
methodology without computers is really onerous. Before we had today's
technology (pseudorandom number generators, to be discussed below), we
had tables of random numbers, the most notable being the hefty \emph{A
Million Random Digits with 100,000 Normal Deviates} (where ``normal
deviate'' means ``random number drawn from a standard normal
distribution), published by the (coincidentally-named) RAND
Corporation.\footnote{You can still get this book
  \href{https://www.rand.org/pubs/monograph_reports/MR1418.html}{on
  RAND's website} or
  \href{https://www.amazon.com/Million-Random-Digits-Normal-Deviates/dp/0833030477/}{in
  paperback}. The numbers were generated using an electronic device,
  specifically designed to shuffle a sort-of random table of numbers,
  attached to a computer. See the text about hardware random number
  generators.} RAND published this book because they had a lot of
engineers and researchers using Monte Carlo methods. Before the tables
of random numbers, you had to generate the random numbers yourself,
which was basically infeasible.\footnote{In 1777, Georges-Louis Leclerc,
  Comte de Buffon posed a math problem that included probability and
  geometry: if you throw needles (or matchsticks) onto a surface with
  parallel stripes whose widths are equal to the length of the needles,
  what fraction of the needles touch two stripes? It turns out that the
  answer has \(\pi\) in it. So in 1901, Mario Lazzarini published that
  he had tossed a needle 3,408 times and, using the analytical solution,
  back-calculated \(\pi\) as \(\tfrac{355}{113}\). This estimate of
  \(\pi\) was already known, and the fact that Lazzarini came up with
  exactly that value is taken as strong evidence that he faked the
  experiment. In other words, we have a pretty strong prior against
  believing that someone in 1901 even bothered to throw a needle 3,000
  times, much less the many more times than that that would be required
  for random number generation for more interesting Monte Carlo
  problems!}

As mentioned, now we have \emph{pseudorandom number
generators}\footnote{There is such a thing as a ``hardware'' random
  number generator, which is some kind of device that measures something
  that we think is truly noisy in the real world, like thermal noise or
  (what we believe are truly random) quantum phenomena like
  beamsplitting.}. These rely on some input \emph{seed}, which is the
(hopefully) truly random thing, and from that seed it generates a
deterministic list of numbers that, in the absence of knowing the seed,
appear random.\footnote{It may seem weird that, given a seed, you get a
  deterministic set of numbers. Most software with (pseudo)random number
  generators pick a seed using whatever entropy they have access to when
  you boot up the program, so you never notice that the seed is
  different each time you run a simulation. You can, however, always
  \emph{pick} the seed, which is nice, because it lets you repeat code
  with a Monte Carlo method in it and always get the same result, which
  is nice for testing and debugging.} Usually this seed comes from one
of the many ``entropy sources'' that a computer has access to, things
like the time between keystrokes, the time at which a process was
started, time between network pings, etc.

The pseudorandom number generator in most software now is the Mersenne
Twister. This algorithm is remarkable for having a long \emph{period} of
\(2^{19937} - 1\). (All pseudorandom number generators, started with
with some seed, will eventually end up repeating their output. The
period is the number of outputs you get before closing the loop.) The
random things produced by the generators are typically mapped into
uniformly distribution varibles over \([0, 1]\).

\subsubsection{Generating non-uniformly-distributed
numbers}\label{generating-non-uniformly-distributed-numbers}

Drawing numbers from \([0, 1]\) usually isn't that interesting. We want
to draw numbers from other distributions. There are two main approaches:

\begin{enumerate}
\def\labelenumi{\arabic{enumi}.}
\tightlist
\item
  Clever transformations
\item
  Various forms of \emph{rejection sampling}
\end{enumerate}

The idea with clever transformations is to generate random numbers from
the uniform distribution and somehow turn them into random numbers
distributed according to some other distribution.

Rejection sampling is a big class of approaches, including such notables
as ``Metropolis-Hastings'' and ``Markov chain Monte Carlo'' (MCMC). They
are very useful for the practical scientist.

\textbf{inverse transform, ziggurat, rejection, Metropolis-Hastings and
other Monte carlo mcmc stuff}

Clever transformations are nice when you can do them, but it's unlikely
you'll derive one for yourself. Basically, if a run-of-the-mill random
number generation function in some software purports to be able to
sample numbers from some distribution, it's doing this transformation. I
don't think there's anything really practical to be gained from knowing
these transformations, but they're fun, so I put them here.

\paragraph{Clever transformations}\label{clever-transformations}

If you can write the cdf of your distribution of interest, say
\(F_X(x)\) and you invert it (i.e., solve for \(x\) in terms of
\(F_X\)), then you can use a nice trick called \emph{inverse transform
sampling}.

\subparagraph{Normal distribution: Box-Muller
transformation}\label{normal-distribution-box-muller-transformation}

To generate normally-distributed numbers from uniformly distributed
numbers, consider this trick.

Think about a pair of independent, normally-distributed variables
\(Z_1\) and \(Z_2\). Their joint pdf will be \[
\begin{aligned}
f_{Z_1,Z_2}(z_1, z_2) &=
  \frac{1}{\sqrt{2\pi}} \exp\left\{ -\frac{z_1^2}{2} \right\} \times
  \text{same thing for $z_2$} \\
  &= \frac{1}{2\pi} \exp\left\{ -\frac{1}{2} \left( z_1^2 + z_2^2 \right) \right\}.
\end{aligned}
\] The trick is to think of \(z_1\) and \(z_2\) as Cartesian coordinates
like \(x\) and \(y\), from which it's very natural to replace
\(z_1^2 + z_2^2\) with \(r^2\) and define some \(\theta\) such that
\(z_1 = r \sin \theta\) and \(z_2 = r \cos \theta\). My claim is that
we'll be able to generate \(r\) and \(\theta\) from independent, uniform
random variables.

Because \(f_{Z_1,Z_2}\) is symmetric with respect to \(z_1\) and \(z_2\)
(i.e., you could swap that subscripts and come out with the same
expression), it must be that there isn't anything special about having
sine versus cosine. In other words, there mustn't be anything special
about \(\theta = 0\) versus \(\theta = \pi\). The origin can't matter.
Thus, it must be that \(\theta\) is uniformly distributed over
\([0, 2\pi]\). Any other distribution would end up treating \(z_1\) and
\(z_2\) differently, which would break their independence.

Generating \(r\) is a little more tricky. Let's look at the cumulative
distribution function of the random variable \(R\): \[
\begin{aligned}
\mathbb{P}[R < r] &= \int_0^{2\pi} \int_0^r f_{Z_1, Z_2}
    \,\mathrm{d}z_1 \, \mathrm{d}z_2 \\
  &= \int_0^{2\pi} \int_0^r \frac{1}{2\pi} \exp\left\{-\frac{1}{2} r'^2\right\}
    r' \,\mathrm{d}{r'} \,\mathrm{d}\theta \\
  &= \int_0^r \exp\left\{-\frac{1}{2} r'^2\right\}
    r' \,\mathrm{d}{r'} \,\mathrm{d}\theta \\
  &= \int_0^{\tfrac{1}{2} r^2} \exp\left\{-s\right\} \,\mathrm{d}s,
    \text{where $s = \tfrac{1}{2} r'^2$} \\
  &= 1 - \exp\left\{ -\frac{1}{2} r^2 \right\}.
\end{aligned}
\] If we define \(r = \sqrt{-2 \log u}\), then
\(\mathbb{P}[R < r] = 1 - u\), which is just the cdf for a uniformly
distributed variable \(U\) on \([0, 1]\). So we generate \(r\) using
that formula.

It may seem a little strange that we can generate independent \(z_1\)
and \(z_2\) using \(r\) and \(\theta\). You might think that if I know
\(z_1\), then I can guess something about \(r\) or \(\theta\) and use
that information to make a guess about the value of \(z_2\). However,
because the Cartesian and polar coordinate systems encode exactly the
same information, that argument is like saying that, because I told you
\(x\), you might be able to guess \(y\), which is clearly impossible.

\section{Unplaced}\label{unplaced}

\subsection{Chebyshev's inequality}\label{chebyshevs-inequality}

What's the relationship between confidence intervals and variance? We
all know the relationship for the normal distribution.

As a lemma, consider a random variable \(X\) that only takes on
nonnegative values. Then \[
\mathbb{E}[X] = \sum_{k=0}^\infty k \, f_X(k) \geq \sum_{k=1}^\infty f_X(k) = \mathbb{P}[X \geq 1].
\] (To see the middle inequality, note that you can drop \(k=0\), and
then, for all the \(k \geq 1\), you can replace \(k\) with \(1\), which
makes that term in the sum smaller than it might be.) We'll use the
reversed version: \(\mathbb{P}[X \geq 1] \leq \mathbb{E}[X]\).

Now, Chebyshev's inequality is easy. Use the \[
\begin{aligned}
\mathbb{P}\left[|X - \mu| \geq k \sigma\right]
  &= \mathbb{P}\left[ \frac{(X - \mu)^2}{k^2 \sigma^2} \geq 1 \right] \\
  &\leq \mathbb{E}\left[ \frac{(X - \mu)^2}{k^2 \sigma^2} \right]
    \quad \text{(by the lemma)} \\
  &= \frac{1}{k^2 \sigma^2} \mathbb{E}\left[(X-\mu)^2\right] \\
  &= \frac{1}{k^2}. \quad \text{(since that expected value is $\sigma^2$ by definition)}
\end{aligned}
\]

For \(k=1\), we result is trivial: at most 100\% of values fall outside
1 standard deviation from the mean. (An upper bound of 100\% tells us
nothing.) For \(k=2\), at most \(1/4 = 25\%\) of values fall outside 2
standard deviations. That is, 75\% fall inside. For \(k=3\), 89\% fall
inside. These are much more conservative results than for the normal
distribution, for which 95\% of values fall within 2 standard deviations
and 99.7\% fall within 3.

I'm not sure of the practical utility of this inequality. It requires
knowing the true variance, which already requires a whole bunch of data.

\subsection{\texorpdfstring{Visualiztion of how \(\hat{p}\) changes with
\(n\)}{Visualiztion of how \textbackslash{}hat\{p\} changes with n}}\label{visualiztion-of-how-hatp-changes-with-n}

\begin{verbatim}
n = 1000
p = 0.3
x = rbinom(n, 1, p)
cumx = cumsum(x)
cumn = cumsum(rep(1, n))
phat = cumx / cumn
cil = mapply(function(x, n) binom.test(x, n)$conf.int[1], cumx, cumn)
ciu = mapply(function(x, n) binom.test(x, n)$conf.int[2], cumx, cumn)

data_frame(flip=x, phat, cil, ciu) %>%
  mutate(x=1:n()) %>%
  ggplot(aes(x, phat)) +
  geom_ribbon(aes(ymin=cil, ymax=ciu), fill='grey80') +
  geom_line() +
  geom_hline(yintercept=p, linetype=2)
\end{verbatim}

\section{$\chi^2$ test}

Say you have $k$ iid standard normal random variables:
\begin{equation}
X_i \stackrel{\text{iid}}{\sim} \mathcal{N}(0, 1).
\end{equation}
Then $Y = \sum_{i=1}^k$ is $\chi^2$-distributed with $k$ degrees of freedom.

Let's start with a simple case where you have a table with two cells with
expected probabilities $p_1$ or $p_2 = 1-p_1$. We got $n$ total observations,
with $O_1$ in the first cell and $O_2 = n - O_1$ in the second. You probably
remember how to compute the test statistic from Stats 101:
\begin{equation}
\chi^2 = \sum_{k=1}^2 \frac{(O_i - E_i)^2}{E_i} = \frac{(O_1 - np_1)^2}{np_1} + \frac{(O_2 - np_2)^2}{np_2},
\end{equation}
where $E_i$ is the ``expected'' number of counts in each cell.

Consider the numerator of the second term:
\begin{equation*}
(O_2 - np_2)^2 = \left[(n - O_1) - n(1 - p_1)\right]^2 = (-O_1 + np_1)^2 = (O_1 - np_1)^2.
\end{equation*}
Handy, that's the same as numerator of the first term! That means we can re-write things:
\begin{equation*}
\chi^2 = \frac{(O_1 - np_1)^2}{n}\left( \frac{1}{p_1} + \frac{1}{p_2}\right).
\end{equation*}
A little algebra shows that $1/p_1 + 1/p_2 = 1/p_1(1-p_1)$, so that
\begin{equation}
\chi^2 = \frac{(O_1 - np_1)^2}{np_1(1-p_1)} = \left( \frac{O_1-np_1}{\sqrt{np_1(1-p_1)}} \right)^2.
\end{equation}
That might look terrible, but it's actually pretty cool. Here's why: $O_1$ is the observed value, $np_1$ is the expected mean, and $\sqrt{np_1(1-p_1)}$ is the standard deviation of the binomial distribution. I'll re-write that last equation with more suggestive notation:
\begin{equation}
\chi^2 = \left( \frac{x_1 - \mu_1}{\sigma_1} \right)^2
\end{equation}

This certainly \emph{looks} like a $\mathcal{N}(0, 1)$ variable, although we
said previously that the counts in the two cells follow a binomial
distribution. This is where the central limit theorem comes in: the sum of any
large set of (well-behaved) iid random variables approaches a normal
distribution. The binomial distribution approaches the normal distribution
particularly quickly such that (if the distribution is not highly skewed) you
only need about 5 counts for the normal approximation to be pretty
good.\footnote{The normal approximation to the binomial was proved long before
the central limit theorem. This special case, called the \emph{de Moive-Laplace 
theorem}, was first published by de Moivre in 1738. Laplace published
the reverse result, that the binomial approximates the normal, 75 years later,
in 1812. The general central limit theorem was proven, more than 150 years
after de Moivre's original result only, in 1901 by Lyapunov.}

So, so long as each cell has (ish) 5 or more counts, then we can approximate
the binomial variables with normal variables, which means that the test
statistic $\chi^2$ that I wrote is actually just the square of a single,
standard normal variable, which happens to be $\chi^2$-square distributed with
1 degree of freedom. Two cells in the table ($k=2$) meant $k-1=1$ degrees of
for the $\chi^2$ distribution.

The same result holds, that the sum of the $(O_i - E_i)^2/E_i$ values follows
a $\chi^2$ distribution with $k-1$ degrees of freedom, for $k>2$. The math is
a lot more involved because the $k$ cells in the table are distributed
according to a multinomial distribution. In other words, conditioned on the
total number $n$ of counts, the values in the different cells are not
independent: if cell 1 has a lot of counts, cells 2, 3, etc. can't have that
many cells. Like we've seen before, covariance makes the calculations hard!
Nevertheless, the same restrictions apply: you can only count on the normal
approximation working if you have enough counts in every cell.


\end{document}
