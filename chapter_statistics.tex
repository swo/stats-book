%!TEX root=main.tex

\chapter{Just enough statistics theory}

``Statistics'' is confusing because it means two things. It's a plural noun that
refers to multiple things, each of which is called a ``statistic''. It's also a
singular noun that refers to the study of these mathematical objects.

A mathematical approach will focus entirely on the behavior of the
mathematical objects. This is the route to take if you want to develop new, 
academically-interesting statistical methods. The cookbook,
which-test-should-I-use approach focuses entirely on ``statistics'' as a vague
bag of tools, often with little understanding of the underlying mathematical
objects.

I aim for a middle road, so we can use statistics intelligently and correctly,
focusing on statistics as a tool but understanding their underlying behavior.

\section{Statistics as a way of thinking}

I'm not a historian of statistics, so take this with a grain of salt.
Statistics as we understand it today arose to solve two problems, both of which should be familiar to practicing scientists.

The first goal is to prevent the over-interpretation of data. Someone does an experiment, and they get some results, and they say they're interesting and important. Statistics provides tools to investigate, in a quantitative way, whether and how those results are interesting.

The second goal is to improve experimental design so that experiments are more likely to deliver interpretable data. In pilot projects, experimental design can be based on rules of thumb and simple guesswork. As projects become larger, more costly, or more complex, it is important to ensure your experiment can deliver the information you want it to. Statistics provides tools to do this design quantitatively.

In either case, my central point is not that statistics is more than a bag of tools that you pull out, hoping that you'll have a tool that fits exactly your situation. That kind of thinking only works when you happen to ask questions that have well-worn intellectual answers. Even if there is a ``right tool'' for the job, you're more likely to do something foolish because you don't understand what the tool is doing.

\section{Statistics as mathematical objects}

To get concrete, a \emph{statistic} is a function of the \emph{sample data}, the numbers you collect during an experiment.
Although the field of statistics is often described as consisting of
``descriptive statistics'' and ``inferential statistics'' in this book I want to emphasize that,
for both purposes, we are interested in the properties of statistics.

The most interesting statistics are the ones that are used to estimate some
property of the population that the data were drawn from. These kinds of statistics are sensibly called
\emph{estimators}. Most of the study of statistics comes down to figuring out things
about estimators. One of the most critical is understanding the variance of
estimators, which is essential for both descriptive statistics ---so that you
can put error bars on your measurements--- and inferential statistics ---so you
can make a guess about how probable it is that your data arose under some null
hypothesis. All of this is about estimators and their variance.

The mathematical behavior of statistics arises from probability theory, which
is why the first part of this book is about math.
