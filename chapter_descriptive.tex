%!TEX root=main

\chapter{Confidence intervals}

\section{Definition}

So far we have worked with \emph{point estimates}, that is, guesses for
population parameters that are just a single number. This presents a problem,
because statistics and probability are all about uncertainty, and we know that
the point estimate we make for, say, $B$ in the German tank problem will likely
never be exactly correct, even if the estimator is consistent, unbiased, and
efficient.

In the frequentist framework, the approach for expressing uncertainty is using
\emph{confidence intervals}. A confidence interval for a parameter $\theta$ with estimator $\hat{\theta}$
at \emph{level} $1-\alpha$ is a pair of estimators $\hat{\theta}_-$ and $\hat{\theta}_+$
defined such that, for any parameter value $\theta$, it holds that:
\begin{equation}
  \prob{\hat{\theta}_- \leq \theta \leq \hat{\theta}_+} \geq 1 - \alpha.
\end{equation}
A typical value for $\alpha$ is 5\%, or $0.05$, which yields 95\% confidence intervals.

In other words, in the frequentist approach, we develop two statistics ---functions
of the observed data--- such that their corresponding estimators will, with some
probability, enclose the true value, for every possible true value. Strictly speaking,
you must select a method for constructing confidence intervals such that, for any
parameter $\theta$ I choose, the proportion of infinitely many repeated trials will
produce realizations of the confidence interval that enclose $\theta$.

\section{Meaning and interpretation}

Confidence intervals are a very new concept. They were first introduced in the statistical literature in 1937 and become commonplace in scientific work only in the later 20th century.
It is perhaps no surprise, given its youth, that the concept of the confidence interval
is very confusing.

The most common misconception about confidence intervals is that they represent
\emph{confidence}. (One might argue that ``confidence'' was a poor word to use to name this
thing!) In this misconception, we can have 95\% confidence that the true parameter $\theta$
lies inside the confidence interval constructed after the experimental data has been gathered.
Strictly speaking, this is incorrect. In a frequentist framework, the true
value is either inside the realized confidence interval, or it is not; our ignorance of
the true value has nothing to do with probability. In fact, the interval that encloses
the true value with some ``confidence'' is a Bayesian concept, the \emph{credible interval}.
``Confidence'' is part of the Bayesian definition of probability; it is foreign to the
frequentist construction.

To a practicing scientist, this distinction might appear entirely semantic. Regardless of
what they specifically mean, it is useful to have some quantification of uncertainty.

\section{Constructing intervals}

To show how confidence intervals are constructed, consider the German
tank problem again. Our unbiased point estimate is $\hat{B} = \tfrac{n+1}{n} \max_i X_i$.
We found that the cdf for $\max_i X_i$ was $(x/B)^n$, so the cdf for the
unbiased point estimate is:
\begin{equation*}
    F_{\hat{B}}(x) = \frac{n+1}{n} \left(\frac{x}{B}\right)^n
\end{equation*}

\subsection{Validity}

\subsection{Profile method}

\section{Jackknife}

\section{Bootstrapping}