\chapter{Preface}

\section*{Who this book is for}

I finished my PhD at MIT in Biological Engineering in 2016. While I was a
graduate student, and afterward when I was a postdoc, I found that graduate
students, especially in the life sciences, have a fair amount of quantitative
training, often including statistics, and are capable of hacking together
fairly sensible quantitative methods that answer their scientific questions. I
found, though, that there was often a steep drop-off when it came time to apply
statistical rigor to these hacky methods. I say this, in part, because it is my
own story too.

This book is for people who are experienced quantitative and scientific
thinkers, have decent algebra and maybesome rusty stats, and who want to learn
about how to use statistical thinking to improve their scientific thinking.

I also tried to have fun when I wrote this book, so there are lots of things in
it that I think are fun, like history and footnotes.

\section*{What this book is for}

As I tried to fill gaps in my statistics knowedge, I found that books about
statistics tended to come in two varities: utterly simplistic and
overwhelmingly complex. It was a similar feeling as trying to learn a new
language as an adult: I have the mind of an adult, and am interested in
grown-up things, but I can only read my target language at the level of an
elementary schooler. Books about grown-up things have overwhelming vocabulary,
while books are childish things are way below my level.

I intend this book to be conceptually advanced but technically manageable. Very
often there is sophisticated mathematical machinery behind something that can
be summarized, from a scientific view, very quickly. On the other hand, there
are some times that a little algebra goes a long way to illustrating a
statistical concept.

For example, "maximum likelihood" is a key concept in statistics. Rather than
show you how you can use linear algebra to compute maximum likelihood values
for linear regression, I will just tell you that there is such software as an
optimizer that finds maxima. On the other hand, I won't shy away from
explaining to you why you divide by $n$ when you compute the mean but you
divide by $n-1$ when you compute the standard deviation, which is pretty easy
to derive algebraically.

\section*{What this book is not}

Maybe it's just as easy to say that this \emph{is} a book for people who want
to \emph{use} rigorous statistics to do science, and maybe even to develop some
new statistical methods, but it is \emph{not} a book for people who want to
push the boundaries of what statisticians think are interesting mathematical
problems.

This is not a cookbook that tells you what steps to do to get a $p$-value. This
is more like Harold McGee's \textit{On Food and Cooking}, that tries to help
give you the principles by which you can understand what the tests are doing.

I also do not intend to fill this book with examples, because I find that it's
very easy to think that your particular scientific question is very
itneresting, and that the statistics that motivate are interesting, but that
it's hard to get motivated by hearing about someone \emph{else's} statistical
woes. I try to keep it general so you can imagine you are the hero of the story.

\section*{What you should be able to do after reading this book}

I basically want you to be able to articulate you own statistical tests, modify
and critique existing methodologies, and develop a healthy skepticism of the
idea of statistical inference in general.

\section*{What else should I read}

My principal sources of motivation for this book were Allen Downey's
\textit{Think Stats} and \textit{Think Bayes}, Miran Lipva\v{c}a's
\textit{Learn You a Haskell for Great Good}, and Stephen Stigler's
\textit{History of Statistics}. That order is probably the most profitable.
