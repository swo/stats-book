%!TEX root=main.tex

\chapter{Statistical inference}

The typical approach in statistics is to teach frequentist inference ---things like $t$-tests--- before Bayesian statistics. The reason is that Bayesian statistics is mathematically more complex, it was developed after frequentist statistics, and is more confusing. However, the reason Bayesian statistics is confusing is because inference is confusing, and pretending that it is not leads to some critical misunderstandings about statistical inference.

Rather than hide the fact that inference is confusing, I introduce the concept of inference in this chapter, contrasting frequentist and Bayesian inference, and showing very little math.

Having dealt with inference's limitations, I will go on to the nuts and bolts of ``doing'' statistical inference. But read this chapter first, so you don't hold on to some critical misconceptions.

\section{Inference is philosophically confusing}

\emph{Inference} means taking some concrete cases and making a broader
conclusion. Inference is always contrasted with deduction, which does the
opposite, reasoning from broad rules to particular circumstances. The classic
example of deduction is: all humans are mortal, Socrates is human, therefore
Socrates is mortal. We reason from a general rule about mortality to
a particular case of Socrates's mortality.

Inference goes from particular to general. For most of humanity's history, we
did this without worrying. Plato felt confident asserting that all humans are
mortal because he had never heard a counterexample: because every human he
observed or had heard about was mortal, he reasoned that \emph{every} human ---every one who ever lived and every one which will live--- is mortal.

\hl{inverse probability}

Thousands of years after Plato, the philosopher David Hume questioned this kind of logic. He said, ``Sure, every human \emph{so far} has been mortal. But that's not proof positive that \emph{no} human will ever be immortal.'' We today are very comfortable to say that ``correlation does not prove causation'', but Hume took that logic to its extreme: if correlation does not prove causation, then seeing that every single one of billions of human who lived is mortal does not prove that the next human is guaranteed to be mortal. If we can't even prove that humans are mortal, then how do we conclude anything?\footnote{This argument is usually couched as the \hl{``sunrise problem''}: given that the sun has risen every day, what's the probability it will rise tomorrow? Is that probability \emph{exactly} 1?}

This may seem abstract, but I want to point out that inference is philosophically confusing so that you will not be surprised to hear that inferential statistics is confusing!

\section{Null hypotheses are our imperfect approach to frequentist inference}

\hl{1. Freq. stats cannot adress states of nature. 2. But we want $P[\theta|X]$. Waht to do? 3. Trick is to use $P[X|\theta]$ to make p-values and supplement that with CIs.}

We resolve the paradox of using frequentist statistics to make inferences about the character of the universe ---things which do not have frequencies--- using a trick called a \emph{null hypothesis}. Rather than asking, ``Given the data, what's the probability that my hypothesis is correct?'', we construct an uninteresting, null hypothesis and then reverse the question, ``Imagine this null hypothesis were true. Then what's the probability of getting data like mine?''

For example, if I think you're cheating in a card game, the natural question is to ask, ``Given the cards I've seen you draw, what's the probability that you're cheating?'' The point of a null hypothesis is to flip it around: ``Imagine that you were playing fair. What's the probability you would be winning this month?''

To emphasize how these two questions are different, I'll introduce the typical
mathematical symbols for these concepts: $\theta$ is the event that the null
hypothesis is correct, and $X$ is the event that I observed the data I did. Our
natural inclination is to ask about $\prob{\theta | X}$. But in frequentist
statistics, $\prob{\theta}$ is nonsense, either you are cheating or you are not.
So we flip the question, instead asking about $\prob{X | \theta}$.


In the previous section, we replaced an induction question with a deduction question: $\prob{\theta | X}$ reasons from the specific $X$ to the general $\theta$, while $\prob{X | \theta}$ reasons from the general $\theta$ to the specific $X$. In other words, we replaced the hard, inference question (are you cheating? are all human mortal?) with the easier, deduction question: if you were not cheating, how likely are you to have gotten such lucky hands?

It's tempting to equate $\prob{\theta | X}$ and $\prob{X | \theta}$. This is called the \emph{prosecutor's fallacy}. Imagine there was a crime, and the perpetrator left type AB negative blood at the scene. A person with type AB negative blood is accused of the crime. The prosecutor will say, ``Only 1\% of Americans have type AB negative blood. So if this person were innocent, there is only a 1\% chance that type AB negative blood would be found at the crime site. Thus, there is a 1\% chance that this person is innocent.''

The prosecutor did a bait-and-switch. The null hypothesis $\theta$ is that the accused is innocent, and some random American committed the crime. The data $X$ is having found AB negative blood. The prosecutor accurately concluded that $\prob{X | \theta} = 1\%$. But then they fallaciously concluded that $\prob{\theta | X} = 1\%$.

If it's not obvious that this argument is fallacious, consider the more general proposition that $\prob{A | B}$ equals $\prob{B | A}$. For example, if it is snowing, what's the probability that the temperature is below freezing? Basically 100\%. But if it is below freezing, what is the probability that it is snowing? Clearly much less than 100\%. But clearly the two concepts have something to do with one another, which we will examine in the next section.

\section{Bayes's theorem is the only way to ask about whether your hypothesis is correct}

\hl{Move this to a Part IV}

The relation between $\prob{X | \theta}$ and $\prob{\theta | X}$ is laid out in \emph{Bayes's theorem}:
\begin{equation}
\prob{\theta | X} = \frac{\prob{X | \theta} \prob{\theta}}{\prob{X}}.
\end{equation}
This result follows almost immediately from the definition of conditional probability, laid out in chapter \ref{chapter:probability}.

Note that there are four terms in Bayes's theorem:
\begin{itemize}
\item $\prob{\theta | X}$: Given the data, what's the probability that hypothesis is true? This is what the court is trying to figure out: given the evidence, what's the probability that the accused is innocent (or guilty)?
\item $\prob{X | \theta}$: Given the hypothesis, or null hypothesis, what's the probability of observing the data we did?\footnote{You may recognize that this wording is the definition of the $p$-value. In a way, the point of this chapter is to conceptually prepare you to be critical of the $p$-value, and to understand that is is \emph{not} $\prob{\theta | X}$.}
\item $\prob{X}$: What's the probability of observing this data? In this case, this is essentially the same as $\prob{X | \theta}$.
\item $\prob{\theta}$: What's the probability, not considering the evidence, that this person is innocent?
\end{itemize}

In the courtroom example, $\prob{X}$ is the probability that, if we walked into a random courtroom trying a case with blood evidence, that the blood would be of type AB negative. If 1\% of Americans have type AB negative blood, and people of different blood types commit crimes at the same rates, then $\prob{X} = 1\%$.

In this example, $\prob{X | \theta}$ is very similar to $\prob{X}$. Say there are 300 million Americans, of whom 1\%, or 3 million, have type AB negative blood. If the accused is innocent, there are 299,999,999 type AB negative people left, so the probability of observing type AB negative blood is $(3\text{ million} - 1) / (300\text{ million}) = 1\% - 1 / (300\text{ million})$, or just slightly smaller than 1\%.

So, plugging in $\prob{X}$ and $\prob{X | \theta}$, which we computed based on our knowledge about blood types, we can conclude that $\prob{\theta | X}$ is just slightly smaller than $\prob{\theta}$. In other words, the probability that the accused is innocent given the blood evidence is only very slightly smaller than the probability that they are innocent when not considering that evidence. Without any evidence, the probability that some random accused person happens to be the perpetrator is small, say 1 in 300 million. In other words, the prosecutor, if they have only the blood evidence, has a very bad case.

\section{Bayes's theorem is about updating, not defining, probabilities}

In the example from the previous section, I showed that we can use data to compute the relationship between $\prob{\theta}$, the probability that a hypothesis is correct not considering the data, and $\prob{\theta | X}$, the probability that a hypothesis is correct given the data. In the language of Bayesian statistics, $\prob{\theta}$ is called the \emph{prior probability} of $\theta$, and $\prob{\theta | X}$ is the \emph{posterior probability}. In other words, the data ``updated'' the prior, pre-data-collection probability.

The blood type example worked because the choice of $\prob{\theta}$ was pretty clear: if we picked an Amercian at random, what's the chance they are innocent of this particular crime? (Or, conversely, what's the chance that a randomly-selected person is the perpretrator?)

In other cases, when $\theta$ is a hypothesis, $\prob{\theta}$ becomes extremely problematic. Take the Socrates example: what's the probability, if you hadn't observed the lives and deaths of any human, that all humans are mortal? Or take a science example: if Einstein had showed you his general theory of relativity, but neither you nor he had ever observed any experimental evidence or astronomical data, what would you say is the prior, pre-data-collection probability that his data is correct?

As I discussed above, it only makes sense to assign probability to hypotheses in a Bayesian context. It doesn't make sense to ask on what fraction of replicates of Earth humans are mortal; instead you ask how confident you are that humans are mortal.

Again, Bayes's theorem, is a perfectly mathematically sound theorem, but it only \emph{updates} probabilities. Given a prior probability, and the data, it gives a posterior probability. But it can never tell you the prior probability. This is why people don't like Bayesian statistics. How can you put a prior probability on a hypothesis? Do you make a guess? ``Well, Einstein was smart, so that's a plus, but his theory sounds complicated, so that's a minus, and so I think maybe fifty-fifty?''

There's no mathematically sound answer to this question because there is no philosophically sound answer to this question. In other words, frequentist statistics and Bayesian statistics are both rigorous mathematical systems, and there is clear link between experimental science and the proportion-probabilities of frequentist statistics, but the link betwen Bayesian degree-of-confidence probabilities and what we have in our heads is not so clear.

Having filled your mind with this philosophical cloud, let's get down to the nuts and bolts of statistical inference.
